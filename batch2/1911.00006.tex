\documentclass[12pt]{amsart}




\input{header_basic.tex}
\input{header_article.tex}
\input{header_subtle.tex}

\usepackage{enumitem}
\usepackage{color}
\usepackage{subfig, caption}
\usepackage{wrapfig}
\captionsetup{margin=0pt,font=small}
\usepackage{pdflscape}
\usepackage{array}
\usepackage{MnSymbol}



\newcommand{\DSS}{\text{DSS}}
\newcommand{\trace}{\operatorname{trace}}
\newcommand{\sing}{\operatorname{sing}}
\newcommand{\dev}{\operatorname{dev}}
\newcommand{\carr}{\prec}
\newcommand{\capR}{\textsc{r}}
\newcommand{\prep}{\textrm{prep}}
\newcommand{\tw}{\textrm{tw}}
\newcommand{\straight}{\textrm{straight}}
\newcommand{\acw}{\rcurvearrowup}

\newsavebox{\FigEightSnappy}
\newsavebox{\FigEightVeer}
\newsavebox{\FigEightSisSnappy}
\newsavebox{\FigEightSisVeer}
\newsavebox{\FourTetExSnappy}
\newsavebox{\FourTetExVeer}
\newsavebox{\NonFibredSnappy}
\newsavebox{\NonFibredVeer}
\newsavebox{\BigExSnappy}
\newsavebox{\BigExVeer}

\begin{lrbox}{\FigEightSnappy}
\verb+m004+
\end{lrbox}
\begin{lrbox}{\FigEightVeer}
\verb+cPcbbbiht_12+
\end{lrbox}
\begin{lrbox}{\FigEightSisSnappy}
\verb+m003+
\end{lrbox}
\begin{lrbox}{\FigEightSisVeer}
\verb+cPcbbbdxm_10+
\end{lrbox}
\begin{lrbox}{\FourTetExSnappy}
\verb+m203+
\end{lrbox}
\begin{lrbox}{\FourTetExVeer}
\verb+eLMkbcddddedde_2100+
\end{lrbox}
\begin{lrbox}{\NonFibredSnappy}
\verb+s227+
\end{lrbox}
\begin{lrbox}{\NonFibredVeer}
\verb+gLLAQbecdfffhhnkqnc_120012+
\end{lrbox}
\begin{lrbox}{\BigExSnappy}
\verb+m115+
\end{lrbox}
\begin{lrbox}{\BigExVeer}
\verb+fLLQccecddehqrwjj_20102+
\end{lrbox}

\title[There and back again]{From veering triangulations to link spaces and back again}
\author{Saul Schleimer and Henry Segerman} 
\date{\today}

\begin{document}

\begin{abstract}
Agol introduced veering triangulations of mapping tori as a tool for understanding the surgery parents of pseudo-Anosov mapping tori.  Gu\'eritaud gave a new construction of veering triangulations of mapping tori using the orbit spaces of their suspension flows.  Generalizing this, Agol and Gu\'eritaud announced a method that, given a closed manifold with a pseudo-Anosov flow (without perfect fits), produces a veering triangulation.

Here we begin the proof of the converse.  We first find, canonically associated to a given transverse veering triangulation, a circular order on the cusps of the universal cover invariant under the action of the fundamental group.  Using this we build the veering circle and the link space.  The similarities between the latter and the orbit space allow us to recover the veering triangulation from its link space, even when the manifold is not fibred.  Along the way we prove several results of independent interest. 
\end{abstract}





\maketitle

\section{Introduction}

Inspired by~\cite{FarbLeiningerMargalit11}, Agol introduced veering triangulations in order to better understand the surgery parents of the mapping tori of pseudo-Anosov maps with bounded normalised dilatation~\cite{Agol11}. His innovation, following ideas of Hamenst\"adt~\cite{Hamenstadt09}, was to produce a canonical train track splitting sequence for the monodromy.  In particular, he showed that the veering triangulation of the drilled mapping torus is a complete conjugacy invariant.  More generally, veering triangulations have wide applications in geometric topology~\cite{Bell15, Gueritaud16, MinskyTaylor17, Strenner18, Landry18}, as well as in dynamics~\cite{Frankel18}, and their properties have been a subject of study by numerous authors~\cite{HRST11, FuterGueritaud13, Kozai13, HodgsonIssaSegerman16, Sakata16, Worden18, FuterTaylorWorden18}.





All work on veering triangulations to date (other than~\cite{HRST11, FuterGueritaud13}) has additionally assumed that they are layered.  Agol asks his readers (Section~7, third question) to ponder the possibility and meaning of non-layered examples.  The first such was found via computer search by Hodgson, Rubinstein, Tillmann and the second author~\cite{HRST11}.  In other work~\cite{GSS19}, we give the census of all transverse veering triangulations with up to 16 tetrahedra; there are 87,047 of these. The evidence strongly suggests that non-layered veering triangulations dominate. 



The first hint that Agol's question has a general answer comes from the work of Gu\'eritaud~\cite[Theorem~1.1]{Gueritaud16}.  He gives an alternate construction of the veering triangulation, starting from a singular euclidean structure on the fibre of the given mapping torus.  Agol and Gu\'eritaud announced an important extension of this: given a manifold equipped with a pseudo-Anosov flows without perfect fits, there is an associated veering triangulation on its surgery parent~\cite{Agol15}. 

This paper is the first in a sequence aimed at proving the converse.  In fact, our construction and that of Agol--Gu\'eritaud are inverses. Thus veering triangulations are a perfect combinatorialisation of (topological) pseudo-Anosov flows without perfect fits. This fulfils part of Smale's program for dynamical systems, outlined in his 1967 Bulletin article~\cite{Smale67}; one first finds the appropriate notion of stability for flows (pseudo-Anosov without perfect fits), and then combinatorially classifies such systems (veering triangulations with filling slopes). 

\subsection{Outline}

In this paper, our main goal is to build the \emph{link space} associated to a given veering triangulation.
The link space is similar in some ways to the singular euclidean structure on a fibre of a pseudo-Anosov mapping torus.  
However, we must build the link space using only the combinatorics of the veering triangulation, which need not be layered.
We then show, motivated by Gu\'eritaud's construction, that the veering triangulation is recovered from the link space. 

\makeatletter
\g@addto@macro\@parboxrestore{\lineskiplimit\normallineskip}
\makeatother

\begin{figure}[htbp]
\centering
\labellist
\footnotesize\hair 2pt
\pinlabel {\parbox{3.5cm}{\begin{center} veering triangulation\\ $(M, \calT, \alpha)$ \refdef{Veering} \end{center}}} at 100 457

\pinlabel \refprop{VeerImpliesExhaust} [r] at 60 418
\pinlabel {\parbox{4.5cm}{\begin{center} continental exhaustion\\ $\{C_n\}$ of universal cover \\ \refdef{ContinentalExhaustion} \end{center}}} at 60 384

\pinlabel {\parbox{2.2cm}{\begin{center} \cite[Main\\ Construction]{Agol11} \end{center}}} [l] at 135 400 
\pinlabel {\parbox{4.4cm}{\begin{center} train tracks $\tau^f, \tau_f$\\ on a face $f$\\ \refdef{UpperLowerTrackFace}\end{center}}} at 135 359 

\pinlabel \refprop{ExhaustImpliesLayered}  [r] at 60 359
\pinlabel {\parbox{4.5cm}{\begin{center} layering $\calK = \{ K_i\}$\\ of universal cover\\ \refdef{Layered} \end{center}}} at 60 330

\pinlabel \reflem{LayeredImpliesUnique} [r] at 24 288
\pinlabel {\parbox{4.4cm}{\begin{center} unique compatible\\ circular order $\calO_\alpha$ \\ Definitions~\ref{Def:CircularOrder} and~\ref{Def:Compatible} \end{center}}} at 23 259

\pinlabel {\parbox{3.4cm}{\begin{center} branched surfaces \\ $\calB^\alpha$, $\calB_\alpha$, \refsec{UpperLowerSurfaces} \end{center}}} at 170 274

\pinlabel {\parbox{4.4cm}{\begin{center} train tracks $\tau^K$, $\tau_K$\\ on a layer $K$\\ \refdef{UpperTrack}\end{center}}} at 100 223 

\pinlabel \refthm{VeeringCircle} [r] at 24 234
\pinlabel {\parbox{4.4cm}{\begin{center} veering circle $S^1(\alpha)$ \\ \refdef{VeeringCircle} \end{center}}} at 23 205

\pinlabel {\parbox{3.4cm}{\begin{center} branch line \\ $S$, \refsec{BranchLines} \end{center}}} at 170 204

\pinlabel \refthm{Laminations} [r] at 97 161
\pinlabel {\parbox{4.4cm}{\begin{center} laminations $\Lambda^\alpha$, $\Lambda_\alpha$ in $S^1(\alpha)$\\ Definitions \ref{Def:LaminationInS1} and \ref{Def:UpperLamination}\end{center}}} at 100 133

\pinlabel \refthm{VeeringSphere} [r] at 24 80
\pinlabel {\parbox{4.4cm}{\begin{center} veering sphere $S^2(\alpha)$ \\ \refdef{VeeringSphere} \end{center}}} at 24 51

\pinlabel \refthm{Laminations}(\ref{Itm:LaminationsInM},\ref{Itm:LaminationsUnique}) [l] at 170 80
\pinlabel {\parbox{3.4cm}{\begin{center} essential laminations\\ $\Sigma^\alpha$, $\Sigma_\alpha$ in $M$ \\ \refsec{SuspendingDescending} \end{center}}} at 170 51

\pinlabel \refthm{LinkSpace} [r] at 97 80
\pinlabel {\parbox{4.4cm}{\begin{center} link space $\calL(\alpha)$\\ and its foliations $\calF^\alpha$, $\calF_\alpha$\\ Definitions \ref{Def:LinkSpace} and \ref{Def:UpperFoliation} \end{center}}} at 100 15

\pinlabel \refthm{Gueritaud} [l] at 243 204
\endlabellist
\includegraphics[width=0.85\textwidth]{Figures/paper_flowchart3}
\caption{The main objects and constructions in this paper.}
\label{Fig:PaperFlowChart}
\end{figure}

We now outline the structure of the paper.  The flowchart shown in \reffig{PaperFlowChart} illustrates the interrelations between the main objects as well as their constructions.  The examples given in \refsec{Examples} are not veering, and so do not appear in the flowchart.
In \refsec{OrdersTriangulations} we review the definitions of \emph{circular orders}, \emph{transverse taut ideal} triangulations, and \emph{layered} triangulations.  We also define what it means for a circular order to be \emph{compatible} with a given transverse taut structure.  
In \refsec{Examples} we show that layered triangulations are \emph{rigid}: that is, they admit a unique compatible circular order (see \refexa{SurfaceBundlesTwo}).  
In a striking contrast, we also give a non-layered taut ideal triangulation which is far from rigid -- it admits uncountably many compatible circular orders (see \refexa{FlatDSS}).  
On the other hand, layered or not, veering triangulations are rigid. 

\begin{restate}{Theorem}{Thm:VeerImpliesUnique}
Suppose that $(M, \calT, \alpha)$ is a transverse veering ideally triangulated three-manifold.  
Then there is a unique compatible circular order $\calO_\alpha$ on the cusps of $\cover{M}$.  Furthermore, $\calO_\alpha$ is dense and $\pi_1(M)$--invariant.
\end{restate}







\noindent
In \refsec{Geography} we quickly review the terminology of \emph{train tracks}.  We then introduce the new tools needed in the proof of \refthm{VeerImpliesUnique}.  The most important of these is the idea of a \emph{continental exhaustion}, introduced in \refdef{ContinentalExhaustion}.  This is a combinatorial version of a developing map into $\HH^2$.   The heart of the proof of \refthm{VeerImpliesUnique} is a delicate induction that promotes a transverse veering triangulation to a continental exhaustion.  

In \refsec{SurfacesAndLines} we construct various \emph{branched surfaces}.  Using these and the techniques developed for the proof of \refthm{VeerImpliesUnique}, we promote the circular order $\calO_\alpha$ on the cusps of $\cover{M}$ to the \emph{veering circle} $S^1(\alpha)$.  

\begin{restate}{Theorem}{Thm:VeeringCircle}
Suppose that $(M, \calT, \alpha)$ is a transverse veering ideally triangulated three-manifold.  Then the order completion of $(\Delta_M, \calO_\alpha)$ is a circle $S^1(\alpha)$ with the following properties.
\begin{enumerate}
\item 
The action of $\pi_1(M)$ on $\Delta_M$ extends to give a continuous, faithful, orientation-preserving action on  $S^1(\alpha)$.  
\item 
Furthermore, all orbits are dense.
\end{enumerate}
\end{restate}

The veering circle $S^1(\alpha)$ depends only on $(M, \calT, \alpha)$, and not on any other choices. This is in striking contrast with Thurston's \emph{universal circle} for a given closed manifold equipped with a taut foliation; many choices are required for his construction. See~\cite[Remark~6.27]{CalegariDunfield03}.  Our situation is more similar to Fenley's \emph{ideal circle boundary} for a closed three-manifold equipped with a pseudo-Anosov flow without perfect fits~\cite[Theorem A]{Fenley12}.  There the circle is unique.  We stress however that our results require only a finite amount of combinatorial data, while \cite{CalegariDunfield03, Fenley12} both require substantial topological or dynamical inputs.


\begin{remark}
Since our manifold $M$ comes equipped with a taut ideal triangulation, it necessarily has cusps.  Thus the fundamental group $\pi_1(M)$ surjects $\ZZ$.  We deduce that $\pi_1(M)$ has \emph{left orders} and so has \emph{left circular orders}. See~\cite[Definitions~2.26 and~2.40]{Calegari07}.  
We note that \refthm{VeeringCircle} gives a new left circular order on $\pi_1(M)$, by inserting gaps at the cusps and, in these gaps, adding left orders on the peripheral groups.   
\end{remark}

We next build the \emph{upper} and \emph{lower laminations}.  

\begin{restate}{Theorem}{Thm:Laminations}
Suppose that $(M, \calT, \alpha)$ is a transverse veering ideally triangulated three-manifold.  
Then there is a lamination $\Lambda^\alpha$ in $S^1(\alpha)$ with the following properties.
\begin{enumerate}
\item 
The upper lamination  $\Lambda^\alpha$ is $\pi_1(M)$--invariant.
\item 
The upper lamination $\Lambda^\alpha$ suspends to give a $\pi_1(M)$--invariant lamination $\cover{\Sigma}^\alpha$ in $\cover{M}$; this descends to $M$ to give a lamination $\Sigma^\alpha$ which 
\begin{enumerate}
\item
is carried by the upper branched surface $\calB^\alpha$, 
\item
has only plane, annulus and M\"obius band leaves, and
\item
is essential.
\end{enumerate}
\item 
Suppose that $\Sigma$ is a lamination carried by $\calB^\alpha$.  Then after collapsing parallel leaves, $\Sigma$ is tie-isotopic to $\Sigma^\alpha$.
\end{enumerate}
There is also a lamination $\Lambda_\alpha$ with the same properties with respect to $\calB_\alpha$. 
\end{restate}

\refthm{Laminations}\refitm{LaminationsUnique} is surprising; in \refcor{Surprise} we use this to show that any two laminations carried by the stable branched surface associated to a pseudo-Anosov homeomorphism are tie-isotopic (after collapsing parallel leaves).

Using the upper and lower laminations, we build the \emph{link space} with its upper and lower foliations.  We then prove the following.

\begin{restate}{Theorem}{Thm:LinkSpace}
Suppose that $(M, \calT, \alpha)$ is a transverse veering ideally triangulated three-manifold.  
\begin{enumerate}
\item
The link space $\calL(\alpha)$ is homeomorphic to $\RR^2$.  
\item
$\calF^\alpha$ and $\calF_\alpha$ are transverse foliations of $\calL(\alpha)$.  
\item
Non-singular leaves are dense in each of $\calF^\alpha$ and $\calF_\alpha$. 
\item
The natural action of $\pi_1(M)$ on $\calL(\alpha)$ is continuous, faithful, and orientation preserving.
\end{enumerate}
\end{restate}

The proof of \refthm{LinkSpace} is lengthy and quite involved.  In essence, we must collapse a pair of transverse laminations in a circle to obtain a pair of foliations in a plane.  This process is due to Thurston; there are various expositions in the literature.  See \cite[Chapter~6]{CassonBleiler88} and \cite[Section~11.9]{Kapovich09}.  However, we do not have any of the usual hypotheses: 
we do not have a surface group action,
we do not have invariant measures, 
and we do not have compactness (of the original three-manifold). 


We also show, in \refthm{Gueritaud}, that we can recover the veering triangulation from the link space.

\subsection*{Acknowledgements}
We thank Ian Agol for many interesting discussions about veering triangulations, and for posing the question that led to this work. These conversations happened at the Institute for Advanced Study during the \emph{Geometric Structures on 3-manifolds program}, supported by the National Science Foundation under Grant No. DMS-1128155.

We thank Lee Mosher and S\'ergio Fenley for teaching us about pseudo-Anosov flows; this directly motivated our work on the link space. 

We thank Yair Minsky for helping us understand the connection between the veering hypothesis and Fenley's notion of \emph{perfect fits}, and we thank Marc Culler for suggesting the name \emph{continents}. These conversations happened at the Mathematical Sciences Research Institute during the \emph{Geometric Group Theory} program, supported by the National Science Foundation under Grant No. DMS-1440140. 

We thank Danny Calegari and Steven Frankel for their excellent talks at the workshop \emph{Contact structures, laminations and foliations}, at the Mathematical Institute of the LMU in Munich; these inspired the material in \refsec{LaminationsAlone}.  

We thank Sam Taylor for pointing out the relation between the link space and the Guirardel core.

We thank Jason Manning for many interesting discussions about the veering two-sphere. 

We thank the Institute for Computational and Experimental Research in Mathematics in Providence, RI, for their hospitality during the \emph{Illustrating Mathematics} program as we finished up this paper.  The program was supported by the National Science Foundation under Grant No. DMS-1439786 and the Alfred P. Sloan Foundation award G-2019-11406. The second author was supported in part by Australian Research Council grant DP1095760 and National Science Foundation grants DMS-1308767 and DMS-1708239.

\section{Circular orders and taut ideal triangulations}
\label{Sec:OrdersTriangulations}

\subsection{Circular orders}
References for this material include~\cite[Section~2]{Thurston98}, \cite[Section~2.6]{Calegari07}, and~\cite[Chapter~3]{Frankel13}.

\begin{definition}
\label{Def:CircularOrder}
Suppose that $\Delta$ is a set.  A function $\calO \from \Delta^3 \to \{-1, 0, 1\}$ is a \emph{circular order on $\Delta$} if for all $a$, $b$, $c$, and $d$ in $\Delta$ we have the following.
\begin{itemize}
\item
$\calO(a, b, c) \neq 0$ if and only if all of $a$, $b$, and $c$ are distinct.
\item
$\calO(a, b, c) = \calO(c, a, b) = -\calO(b, a, c)$.
\item
If $\calO(a, b, d) = \calO(b, c, d) = 1$ then $\calO(a, c, d) = 1$. \qedhere
\end{itemize}
\end{definition}

\begin{example}
The canonical example is the unit circle $S^1$ equipped with the anti-clockwise circular order. 
\end{example}

\begin{definition}
\label{Def:Dense}
We say that a circular order $\calO$ on $\Delta$ is \emph{dense} if, for all $a, b, c \in \Delta$, if $\calO(a, b, c) = 1$, then there is a point $d \in \Delta$ so that $\calO(d, c, b) = 1$. 
\end{definition}

\begin{definition}
Now suppose that $\Gamma$ is a group acting on $\Delta$.  A circular order $\calO$ on $\Delta$ is \emph{$\Gamma$--invariant} if for all $a, b, c \in \Delta$ and for all $\gamma \in \Gamma$ we have $\calO(\gamma a, \gamma b, \gamma c) = \calO(a, b, c)$.
\end{definition}

\begin{example}
A second example comes from taking $F$ to be a compact, connected, oriented surface with non-empty boundary and with $\chi(F) < 0$.
Let $\cover{F}$ be the universal cover of $F$.  Let $\Delta_F$ be the set of boundary components of $\cover{F}$; these are the \emph{cusps} of $\cover{F}$.  Note that the orientation on $F$ gives a dense $\pi_1(F)$--invariant circular order on $\Delta_F$.  See \reffig{Farey}. 
\end{example}

\begin{figure}[htbp]
\includegraphics[width=0.5\textwidth]{Figures/farey_2kx2k}
\caption{Equipping $F$ with an ideal triangulation and lifting to $\cover{F}$ yields the Farey tessellation.  The cusps of $\cover{F}$ correspond to the rational numbers $\QP^1$.  Figure by Roice Nelson.}
\label{Fig:Farey}
\end{figure}

For our next example we go up another dimension.  Suppose that $M$ is a compact, connected, oriented three-manifold, with non-empty boundary, and where all boundary components are tori.  Let $\cover{M}$ be the universal cover of $M$.  Let $\Delta_M$ be the set of boundary components of $\cover{M}$; these are the \emph{cusps} of $\cover{M}$.  We want to find a dense $\pi_1(M)$--invariant circular order on $\Delta_M$.

\begin{example}
\label{Exa:SurfaceBundlesOne}
Suppose that $F$ is a compact, connected, oriented surface with negative Euler characteristic and with non-empty boundary.  Suppose that $f \from F \to F$ is an orientation-preserving homeomorphism.  
Define the \emph{surface bundle} $M = M_f$ with \emph{monodromy} $f$ to be the three-manifold obtained by forming $F \cross [0,1]$ and then identifying the point $(x, 1)$ with $(f(x), 0)$, for all $x \in F$.  The surfaces $F_t = F \cross \{t\}$ are the \emph{fibres} of the bundle $M$.  The intervals $\{x\} \cross [0,1]$ glue together to give an oriented flow $\Phi$ transverse to the fibres.  We choose the orientation on $M$ so that the orientation of $F_0$, followed by the orientation of $\Phi$, makes a right-handed frame.  One simple example is $M_{\Id} \homeo F \cross S^1$; more interesting is \refexa{SurfaceBundlesTwo}, below. 

We now identify $F$ with the fibre $F_0$.  This induces a homeomophism between $\cover{F} \cross \RR$ and $\cover{M}$ and thus a bijection between $\Delta_F$ and $\Delta_M$.  The orientation of $\cover{F}$ gives us a dense circular order $\calO_F$ on $\Delta_F$ and thus on $\Delta_M$.  Note that $\calO_F$ is $\pi_1(M)$--invariant.
This completes \refexa{SurfaceBundlesOne}. 
\end{example}

\begin{remark}
\label{Rem:BundleRigidity}
The circular order $\calO_F$ of \refexa{SurfaceBundlesOne} possesses a certain kind of rigidity.  Suppose that $F'$ is a properly embedded, connected, oriented surface in $M = M_f$.  Suppose that $F'$ is transverse to the flow $\Phi$, meets every flow-line of $\Phi$, and the orientations of $F'$ and $\Phi$, in that order, form a right-handed frame in $M$.
Then $F'$ is a fibre of some, possibly quite different, surface bundle structure on $M$.  
Thus any such $F'$ gives a circular order $\calO_{F'}$ on the cusps of $M$.  However, both $\cover{F}$ and $\cover{F}'$ are naturally homeomorphic to the leaf-space of $\cover{\Phi}$; we deduce that $\calO_F = \calO_{F'}$.
\end{remark}

\subsection{Taut ideal triangulations}
\label{Sec:TautIdealTriangulation}

To discuss more general examples, we will replace bundle structures by ideal triangulations equipped in various ways.  From now on, we will always assume that M is a compact, connected, oriented three-manifold, with non-empty boundary, and where all boundary components are tori.  Suppose that $\calT$ is a three-dimensional triangulation and that $|\calT|$ is its realisation space. We say $\calT$ is an \emph{ideal triangulation}~\cite[Section~4.2]{thurston_notes} of $M$ if $|\calT|$, minus a small open regular neighbourhood of the zero-skeleton of $\calT$, is homeomorphic to $M$.  We refer to these as the \emph{compact models} of $M$ and $\calT$; here the truncated faces of $\calT$ are hexagons.  In the \emph{cusped model} the faces of $\calT$ are ideal triangles.  This is far easier to draw, and all of our pictures will be of the cusped model. 

We use $\cover{\calT}$ to denote the induced ideal triangulation of the universal cover, $\cover{M}$. 


A \emph{taut angle structure}~\cite[Definition~1.1]{HRST11}, here denoted by $\alpha$, on an ideal triangulation $\calT$ is an assignment of dihedral angles as follows.
\begin{itemize}
\item
Every model edge of every model tetrahedron has dihedral angle zero or $\pi$.
\item
For every edge, the angle sum of its models is $2\pi$.
\item
For each model vertex, the angle sum of the three adjacent model edges is $\pi$.
\end{itemize}
A model tetrahedron with such dihedral angles is called a \emph{taut} tetrahedron.  See \reffig{TransverseTet}.  

A \emph{transverse} taut angle structure $\alpha$ on $\calT$~\cite[Definition~1.2]{HRST11} has, in addition to the dihedral angles, a co-orientation on the faces of $\calT$.  These are arranged so that, for any pair of model faces $f, f'$ of any model tetrahedron $t$ with common model edge $e$, we have the following.
\begin{itemize}
\item 
The dihedral angle of $e$ inside of $t$ is zero if and only if exactly one of the co-orientations on $f, f'$ points into $t$.
\end{itemize}
We extend the co-orientation to the edges of $\calT$ so that, along an edge $e$, it points into the tetrahedron above $e$.  The co-orientation on the faces of a model taut tetrahedron are shown in \reffig{TransverseTet}.  Note that here we are following the terminology of~\cite{HRST11} which is a slight variant of the original definition~\cite[page~370]{Lackenby00}.

\begin{figure}[htbp]
\centering
\subfloat[Co-orientations and angles in a transverse taut tetrahedron.]{
\labellist
\small\hair 2pt
\pinlabel 0 at 20 130
\pinlabel 0 at 240 120
\pinlabel 0 at 135 27
\pinlabel 0 at 135 217
\pinlabel $\pi$ at 125 140
\pinlabel $\pi$ at 125 87
\endlabellist
\includegraphics[width=0.35\textwidth]{Figures/TransverseTet}
\label{Fig:TransverseTet}
}
\qquad \qquad
\subfloat[Co-orientations around an edge.]{
\includegraphics[width=0.4\textwidth]{Figures/TransverseEdge}
\label{Fig:TransverseEdge}
}
\caption{}
\label{Fig:Transverse}
\end{figure}

\subsection{Horizontal branched surface}

See~\cite[Section~6.3]{Calegari07} for a reference on branched surfaces.  Generally, suppose that $\calB$ is a branched surface.  The one-skeleton $\calB^{(1)}$ is here called the \emph{branch locus}.  
The branch locus decomposes as a union of \emph{branch components}: these are connected one-manifolds immersed into $M$ meeting (and possibly self-intersecting) only at the vertices of $\calB^{(0)}$.  The components of $\calB - \calB^{(1)}$ are called the \emph{sectors} of $\calB$.  

Suppose that $\calT$ is an ideal triangulation of $M$.  A transverse taut angle structure $\alpha$ on $\calT$ will be called simply a \emph{transverse taut structure}.  Given $\alpha$, we may isotope the two-skeleton of $\calT$ near each edge, as shown in \reffig{TransverseEdge}, to obtain $\calB(\alpha)$: a co-oriented, non-generic branched surface without vertices~\cite[page~371]{Lackenby00}.  The branched surface $\calB(\alpha)$ is \emph{taut}~\cite[page~374]{Lackenby00}, and this gives taut ideal triangulations their name.  Note that the sectors of $\calB(\alpha)$ are exactly the original triangular faces of $\calT$. 

Let $\calB(\cover{\alpha})$ be the lift of $\calB(\alpha)$ to the universal cover $\cover{M}$.  Hodgson, Rubinstein, Tillmann, and the second author give a thin position argument for the following~\cite[Theorem~6.1]{Hodgson15}.

\begin{theorem}
\label{Thm:ThreeDistinct}
Suppose that $f$ is a face of $\calB(\cover{\alpha})$.  Then $f$ meets three distinct cusps of $\cover{M}$.  \qed
\end{theorem}

\subsection{Layered triangulations}

Suppose that $\calB$ is a general branched surface. Recall that a \emph{tie-neighbourhood} $N(\calB)$ of $\calB$ is a regular neighbourhood of $\calB$ equipped with a foliation by intervals, called \emph{ties}. See~\cite[Figure~1.2]{Mosher96}. A surface $F$, properly embedded in $M$, is \emph{carried} by $\calB$ if it is isotopic into $N(\calB)$ and is there transverse to the ties.  Properties of $\calB$ are generally inherited by its carried surfaces: decomposition into sectors, co-orientations, and so on. 

\begin{definition}
\label{Def:Layered}
We say that a transverse taut structure $(M, \calT, \alpha)$ is \emph{layered} if there is a collection $\calK = \{K_i\}$ of pairwise disjoint surfaces, all carried by $\calB(\alpha)$, so that
\begin{itemize}
\item 
$K_i$ and $K_{i+1}$ cobound a single tetrahedron (above $K_i$ and below $K_{i+1}$) and
\item 
for every tetrahedron $t$ there is an index $i$ so that $t$ is the unique tetrahedron cobounded by $K_i$ and $K_{i+1}$.
\end{itemize}
If $M$ has a finite number $n$ of tetrahedra then the indices are recorded modulo $n$.  Otherwise the indices lie in $\ZZ$.  We call $\calK$ a \emph{layering} of $M$ and the surfaces $K_i$ \emph{layers} of $\calK$. 
\end{definition}


Recall that $\cover{M}$ is the universal cover of $M$.  We use $\cover{\alpha}$ to denote the corresponding structure on $\cover{\calT}$.  Typically we fix $\calT$ and $\alpha$ explicitly in $M$ and then only refer to $\cover{\calT}$ and $\cover{\alpha}$ implicitly in $\cover{M}$.  When we do this, we will simply use the notation $\cover{M}$ to refer to the lifted structures.

\begin{lemma}
\label{Lem:LayeredImpliesLayered}
Suppose that $(M, \calT, \alpha)$ is a transverse taut ideally triangulated three-manifold.  
If $(M, \calT, \alpha)$ is layered then so is $\cover{M}$. 
\end{lemma}

\begin{proof}
Suppose that $\calF = \{F_i\}$ is the given finite layering of $M$ and $t_i$ is the unique tetrahedron between $F_{i}$ and $F_{i+1}$.  Applying \cite[Corollary~1.1]{SchleimerSegerman19}, every component of the lift of the $F_i$ to $\cover{M}$ is combinatorially a copy of the Farey tessellation (as shown in \reffig{Farey}).
These surfaces do not give a layering of $\cover{M}$;  an adjacent pair of connected lifts of $F_i$ and $F_{i+1}$ cobound infinitely many tetrahedra (in fact all lifts of $t_i$).  We must rearrange the order of attachment.

Fix $G$, a connected component of the full preimage of $F_0$.  Choose a linear order $\{t^i\}_{i \in \NN}$ on the tetrahedra of $\cover{M}$ lying \emph{above} $G$.  
We now build a layering onto $G$.  By induction suppose that we have attached all of the tetrahedra $\{t^i\}_{i = 0}^{n-1}$, plus perhaps finitely more, to $G$.  Now consider the tetrahedron $t^n$.  If it is already attached, there is nothing to prove.  If it is not we proceed as follows.  Consider all possible paths $\gamma$ in $\cover{M}$ so that
\begin{itemize}
\item 
$\gamma$ starts on $G$ and ends in $t^n$,
\item 
$\gamma$ is transverse to the two-skeleton, and
\item 
$\gamma$ crosses each face $f$ in the direction of its co-orientation.
\end{itemize} 
Note that each such $\gamma$ crosses the lifts of the $F_i$ the same number of times; we call this number, $h(t^n)$, the \emph{height} of $t^n$.
Thus each $\gamma$ meets at most $h(t^n)$ tetrahedra.  Also, since a tetrahedron has only two lower faces, there are at most $2^{h(t^n)}$ such paths.  So, we attach all of the finitely many tetrahedra met by any of these paths, in height order.  The last one attached is $t^n$, completing the inductive step. 

To complete the proof we perform the corresponding process below $G$.
\end{proof}

\subsection{Compatibility}
\label{Sec:Compatibility}

Suppose that $(M, \calT, \alpha)$ is a transverse taut ideally triangulated three-manifold.  Suppose that $f$ is a face of $\cover{\calT}$.  Let $\Delta_f \subset \Delta_M$ be the cusps of $f$: by \refthm{ThreeDistinct} there are exactly three of these.  The co-orientation of $f$ and the orientation of $\cover{M}$ picks out a unique circular order $\calO_f$ on $\Delta_f$ (say, anti-clockwise as viewed from above).  

\begin{definition}
\label{Def:Compatible}
Suppose that $\calO$ is a circular order on $\Delta_M$, the cusps of $M$.  We say that $\calO$ is \emph{compatible} with $(M, \calT, \alpha)$ if, for every face $f \in \cover{\calT}$, we have $\calO | \Delta_f = \calO_f$.  
\end{definition}

\begin{lemma}
\label{Lem:LayeredImpliesUnique}
Suppose that $(M, \calT, \alpha)$ is a transverse taut ideally triangulated three-manifold.
If $\cover{M}$ is layered then there is a unique compatible circular order $\calO$ on $\Delta_M$, the cusps of $\cover{M}$.  Furthermore, $\calO$ is dense and $\pi_1(M)$--invariant. 
\end{lemma}
 
\begin{proof}
We are given a sequence of carried surfaces $\{F_i\}_{i \in \ZZ}$.  Applying \cite[Corollary~1.1]{SchleimerSegerman19}, each $F_i$ is a copy of the Farey tessellation.  Each is obtained from the previous by a single flip across a tetrahedron.  So $F_i$ and $F_{i+1}$ meet the same subset of cusps of $\cover{M}$.  But every cusp meets some $F_j$.  Thus every $F_i$ meets all cusps of $\cover{M}$.

Let $\calO$ be the circular order on $\Delta_M$ coming from $F_0$.  Picking an oriented edge of $F_0$ determines an order isomorphism of $(\Delta_M, \calO)$ with $\QP^1$, the rational points of $S^1$.  Thus the circular order $\calO$ is dense.  Since each $F_i$ differs from $F_0$ by a finite number of flips, all of the $F_i$ give the same circular order.  Also, every face of $\cover{\calT}$ lies in some $F_i$.  So $\calO$ is compatible with $(M, \calT, \alpha)$.  

Suppose that $\calO'$ is another compatible circular order on $\Delta_M$.  Then $\calO'$ and $\calO$ agree on all faces of $F_0$; we deduce that $\calO' = \calO$.   
Thus $\calO$ is the unique compatible circular order.

Finally, fix any $\gamma \in \pi_1(M)$ and define $\calO_\gamma$ via 
\[
\calO_\gamma(a, b, c) = \calO(\gamma a, \gamma b, \gamma c).
\]
The action of $\gamma$ on $\Delta_M$ is bijective, so $\calO_\gamma$ is a circular order on $\Delta_M$.  Now, if $f$ is a face of $\cover{M}$ with vertices $x$, $y$, and $z$ then $\calO_\gamma(x, y, z) = \calO(\gamma x, \gamma y, \gamma z)$.  Note that $\gamma x$, $\gamma y$, and $\gamma z$ are the vertices of the face $\gamma f$.  Also, $\gamma$ sends the co-orientation of $f$ to that of $\gamma f$; also $\gamma$ preserves the orientation of $M$.  It follows that $\calO_\gamma$ is compatible.  Thus, by the previous paragraph $\calO_\gamma = \calO$. 
\end{proof}

\section{Examples}
\label{Sec:Examples}

We now give non-trivial examples of circular orders coming from transverse taut structures.

\begin{example}
\label{Exa:SurfaceBundlesTwo}
Following the notation of \refexa{SurfaceBundlesOne} suppose $M = M_f$ is an oriented surface bundle with fibre $F$ and monodromy $f$.  Let $(\calT, \alpha)$ be a layered ideal triangulation of $M_f$.  The best-known example is shown in \reffig{VeerFigEight}; it is the canonical triangulation of the figure-eight knot complement as a punctured torus bundle.  In general, since $M_f$ is layered, \reflem{LayeredImpliesLayered} implies that $\cover{M}_f$ is layered.  Thus, by \reflem{LayeredImpliesUnique} the cusps of $\cover{M}_f$ admit a unique compatible circular order $\calO_F$ which is furthermore $\pi_1(M_f)$--invariant.  Since the layering of $\cover{M}_f$ has $\cover{F}$ as one of its layers, the circular order $\calO_F$ is the same as the one constructed in \refexa{SurfaceBundlesOne}.  This completes \refexa{SurfaceBundlesTwo}. 
\end{example}

\begin{figure}[htbp]
\includegraphics[width=0.6\textwidth]{Figures/veering_fig_8}
\caption{A veering structure on the canonical triangulation for the figure-eight knot complement.  This manifold is \usebox{\FigEightSnappy} in the SnapPy census~\cite{snappy}.  This veering structure is \usebox{\FigEightVeer} in the veering census~\cite{GSS19}.  See \refsec{Veering} for the definition of veering; in this section we only use the transverse taut structure.  The $\pi$--edges of the transverse taut structure are the diagonals of the squares.  The model edges on the sides of the squares all have dihedral angle zero.}
\label{Fig:VeerFigEight}
\end{figure}

Before we give our next family of examples, there is a bit of necessary background.  Useful references include~\cite[Chapter~4]{thurston_notes} and~\cite[Section~2]{Tillmann12}.  Take $\calP = \CC - \{0, 1\}$.  
Suppose that $M$ is a compact, connected, oriented three-manifold with boundary being a single torus.  Suppose that $\calT = \{t_i\}_{i = 0}^{n-1}$ is an ideal triangulation of $M$.  We say that $\calT$ admits a \emph{geometric structure} if there is a tuple of \emph{shapes} $\rho_\infty = (z_i) \in \calP^n$ as follows.
\begin{itemize}
\item
The tuple $\rho_\infty$ solves the Thurston gluing and holonomy equations (called the completeness equations in ~\cite[page~800]{Tillmann12}). 
\item
Each $z_i \in \rho_\infty$ has positive imaginary part. 
\end{itemize}
It follows that $\rho_\infty$ gives the interior of $M$ a complete, finite volume, hyperbolic metric.  The tuples of $\calP^n$ that solve the gluing equations (ignoring the holonomy and positivity) make up the \emph{shape variety} $\calS(M, \calT)$.  
Let $\calS_\infty$ be the irreducible component of $\calS(M, \calT)$ which contains $\rho_\infty$.  Our assumptions on $M$ and the existence of $\rho_\infty$ imply that $\calS_\infty$ is a complex curve~\cite[page~314]{NeumannZagier85}.  

Fix a triangle $f$ of $\cover{\calT}$ and a circular order on ideal vertices of $f$.  This choice allows us to define, for every $\rho \in \calS_\infty$, a \emph{developing map} 
\[
\dev_{\rho} \, \from \,\, \cover{M} \,\, \to \,\, \HH^3
\]
We extend $\dev_\rho$ to give a function from $\Delta_M$ (the cusps of $M$) to $\CP^1 = \bdy_\infty \HH^3$.  Our choices ensure that the vertices of $f$ are sent to $0$, $1$, and $\infty$, respectively. 

\begin{definition}
\label{Def:Collide}
If $u$ and $v$ are distinct cusps of $\cover{M}$ then we say that $u$ and $v$ \emph{collide} at $\rho \in \calS_\infty$ if $\dev_\rho(u) = \dev_\rho(v)$. 
\end{definition}

\begin{lemma}
\label{Lem:FiniteCollisions}
A pair of distinct cusps $u$ and $v$ of $\cover{M}$ collide only finitely many times. 
\end{lemma}


\begin{proof}
For any cusp $w \in \Delta_M$ we define a function $\dev(w) \from \calS_\infty \to \CP^1$ by taking $\rho \mapsto \dev_\rho(w)$.  Note that the coordinate functions, restricted to $\calS_\infty$, are meromorphic.  Also, $\dev(w)$ can be written as a rational function in terms of the coordinates.  Thus $\dev(w)$ is also meromorphic. 

Since $u$ and $v$ are distinct, the functions $\dev(u)$ and $\dev(v)$ disagree at $\rho_\infty$.  Thus their difference is not identically zero. 
\end{proof}

\begin{figure}[htbp]
\labellist
\small\hair 2pt
\pinlabel $z$ at 53 53
\pinlabel $z$ at 144.5 53
\pinlabel $z$ at 53 106
\pinlabel $z$ at 144.5 106
\pinlabel $\frac{z-1}{z}$ at 85 15
\pinlabel $\frac{z-1}{z}$ at 176.5 15
\pinlabel $\frac{z-1}{z}$ at 85 68
\pinlabel $\frac{z-1}{z}$ at 176.5 68
\pinlabel $\frac{z-1}{z}$ at 85 121
\pinlabel $\frac{z-1}{z}$ at 176.5 121
\pinlabel $\frac{1}{1-z}$ at 85 39
\pinlabel $\frac{1}{1-z}$ at 176.5 39
\pinlabel $\frac{1}{1-z}$ at 85 92
\pinlabel $\frac{1}{1-z}$ at 176.5 92

\pinlabel $z$ at 42 46
\pinlabel $z$ at 133.5 46
\pinlabel $z$ at 225 46
\pinlabel $z$ at 42 99
\pinlabel $z$ at 133.5 99
\pinlabel $z$ at 225 99
\pinlabel $\frac{z-1}{z}$ at 39 11
\pinlabel $\frac{z-1}{z}$ at 130.5 11
\pinlabel $\frac{z-1}{z}$ at 222 11
\pinlabel $\frac{z-1}{z}$ at 39 64
\pinlabel $\frac{z-1}{z}$ at 130.5 64
\pinlabel $\frac{z-1}{z}$ at 222 64
\pinlabel $\frac{z-1}{z}$ at 39 117
\pinlabel $\frac{z-1}{z}$ at 130.5 117
\pinlabel $\frac{z-1}{z}$ at 222 117
\pinlabel $\frac{1}{1-z}$ at 13.5 27
\pinlabel $\frac{1}{1-z}$ at 105 27
\pinlabel $\frac{1}{1-z}$ at 196.5 27
\pinlabel $\frac{1}{1-z}$ at 13.5 80
\pinlabel $\frac{1}{1-z}$ at 105 80
\pinlabel $\frac{1}{1-z}$ at 196.5 80

\pinlabel $w$ at 37 52.5
\pinlabel $w$ at 128.5 52.5
\pinlabel $w$ at 220 52.5
\pinlabel $w$ at 37 105.5
\pinlabel $w$ at 128.5 105.5
\pinlabel $w$ at 220 105.5
\pinlabel $\frac{w-1}{w}$ at 53 40
\pinlabel $\frac{w-1}{w}$ at 144.5 40
\pinlabel $\frac{w-1}{w}$ at 53 93
\pinlabel $\frac{w-1}{w}$ at 144.5 93
\pinlabel $\frac{1}{1-w}$ at 77 27
\pinlabel $\frac{1}{1-w}$ at 168.5 27
\pinlabel $\frac{1}{1-w}$ at 77 80
\pinlabel $\frac{1}{1-w}$ at 168.5 80

\pinlabel $w$ at 50.5 8
\pinlabel $w$ at 142 8
\pinlabel $w$ at 50.5 61
\pinlabel $w$ at 142 61
\pinlabel $w$ at 50.5 114
\pinlabel $w$ at 142 114
\pinlabel $\frac{w-1}{w}$ at 7.5 38
\pinlabel $\frac{w-1}{w}$ at 99 38
\pinlabel $\frac{w-1}{w}$ at 190.5 38
\pinlabel $\frac{w-1}{w}$ at 7.5 91
\pinlabel $\frac{w-1}{w}$ at 99 91
\pinlabel $\frac{w-1}{w}$ at 190.5 91
\pinlabel $\frac{1}{1-w}$ at 7.5 15
\pinlabel $\frac{1}{1-w}$ at 99 15
\pinlabel $\frac{1}{1-w}$ at 190.5 15
\pinlabel $\frac{1}{1-w}$ at 7.5 68
\pinlabel $\frac{1}{1-w}$ at 99 68
\pinlabel $\frac{1}{1-w}$ at 190.5 68
\pinlabel $\frac{1}{1-w}$ at 7.5 121
\pinlabel $\frac{1}{1-w}$ at 99 121
\pinlabel $\frac{1}{1-w}$ at 190.5 121
\endlabellist
\includegraphics[width=\textwidth]{Figures/fig8_complete_shapes2}
\caption{The triangulation of the universal cover of the cusp torus of the figure-eight knot complement, as induced by the canonical triangulation (shown in \reffig{VeerFigEight}).  The tetrahedron shapes are all regular ($z = w = e^{\pi i/3}$).  The dashed rectangle is a fundamental domain for the tiling, with the meridian $m$ drawn vertically and the longitude $l$ drawn horizontally.  For more general shapes $z$ and $w$ we obtain the complex dihedral angles shown, following the usual conventions~\cite[page~47]{thurston_notes}.}
\label{Fig:CuspTriangulationComplete}
\end{figure}

We are now ready for the next example.

\begin{example}
\label{Exa:FlatDSS}
Let $M$ be the figure-eight knot complement.  Let $\calT$ be the canonical triangulation of $M$, shown in \reffig{VeerFigEight}, there equipped with a taut structure.  The triangulation $\calT$ admits two more taut structures, each of which is again transverse.  These \emph{exotic} structures are generated by \emph{leading-trailing deformations} of the veering triangulation~\cite[Proposition 6.8]{FuterGueritaud13}.
We will show: 
\begin{quote}
\emph{each of the exotic taut structures on $(M, \calT)$ admits uncountably many compatible circular orders.}
\end{quote}
\noindent
The triangulation $\calT$ induces a triangulation of the boundary torus $\bdy M$.  Lifting, we obtain a triangulation of the universal cover of the cusp torus; this is shown in \reffig{CuspTriangulationComplete} along with other details.  We think of this as looking into $\cover{M}$ from a fixed cusp $c_\infty$.  The cusps connected to $c_\infty$ by an edge of $\cover{\calT}$ form a combinatorial copy of the lattice $\ZZ \oplus \ZZ e^{\pi i/3}$.  Let $c_{0,0}$ be any fixed cusp in this lattice.  Let $m$ and $l$ be the usual meridian and longitude of the figure-eight knot.  We define a sublattice by taking $c_{p,q} = m^p l^q (c_{0,0})$.  

As shown in \reffig{CuspTriangulationComplete}, we label the corners of the cusp triangles with the corresponding complex dihedral angles of the ideal tetrahedra.  We can now derive Thurston's gluing equations for this triangulation.  The equations for the two ideal edges are identical: namely 
\[
z(z-1)w(w-1) = 1.
\]
As discussed above, the solutions $\rho = (z, w) \in \calP^2$ form the shape variety $\calS(M, \calT)$.  Again consulting \reffig{CuspTriangulationComplete}, we compute the holonomies for $m$ and $l$ and obtain
\[
H(m) = 1/(w(1-z)) \quad \mbox{and} \quad H(l) = z^2(1-z)^2.
\] 
Recall that Thurston's hyperbolic Dehn surgery equation~\cite[page~57]{thurston_notes}, with real \emph{surgery coordinates} $\mu$ and $\lambda$, is
\[
\mu \log H(m) + \lambda \log H(l) = 2 \pi i.
\]
(Care must be taken to choose the correct branch of the logarithm.)  Suppose we have found shapes $z$ and $w$ with positive imaginary part solving the gluing equations, as well as $\mu$ and $\lambda$ solving the surgery equation.  Then the tetrahedra with those shapes fit together to give an incomplete hyperbolic structure on $M$.  

We now concentrate on the case where $-4 < \mu < 4$ and $\lambda = 1$.  This is the top edge of the hourglass shown in \reffig{m004_canonical_dss}.  These ``surgeries" are on the ``boundary" of Dehn surgery space: the tetrahedron shapes $z$ and $w$ are real and their volume has decreased to zero.
The shapes along the top edge give the triangulation a fixed transverse taut structure $\alpha$: one of the two exotic structures from~\cite[Proposition 6.8]{FuterGueritaud13}.  The other exotic structure corresponds to the bottom edge of the hourglass. Again, see \reffig{m004_canonical_dss}.

\begin{figure}[htbp]
\includegraphics[width=\textwidth]{Figures/m004_canonical_dss.png}
\caption{The Dehn surgery space of the figure-eight knot complement.  We call the white region the ``hourglass".  The complement of the hourglass is the component of the positive volume locus which contains the shapes giving the complete, finite volume, hyperbolic structure.  The origin $(0, 0)$ is at the centre, the point $(0, 1)$  is the midpoint of the top of the hourglass, and the point $(4,1)$ is the top right corner of the hourglass.  For a discussion of the colour scheme and more see \url{https://math.okstate.edu/people/segerman/dehn_surgery_images.html}.}
\label{Fig:m004_canonical_dss}
\end{figure}

Let $\calI \subset \calS_\infty$ be the set of (pairs of) shapes corresponding to the top edge of the hourglass.  So, for each $\mu \in (-4, 4)$ we have shapes $\rho_\mu = (z_\mu, w_\mu) \in \calI$ and a map $\dev_\mu \from \Delta_M \to \bdy \HH^2$.  We adapt the conventions that $\bdy \HH^2 = \RR \cup \infty$ and that $\dev_\mu(c_\infty) = \infty$. 

\reffig{fig8_(-1_1)_shapes2} is a cartoon of the image of the developing map for the shapes $(\mu, \lambda) = (-1,1)$.  The vertical positions of features are purely combinatorial; all of the cusps of the original lattice $\ZZ \oplus \ZZ e^{\pi i/3}$ map into a single horizontal line.  For some $\mu$ (for example, those at the rational points) $\dev_\mu$ is not one-to-one and we do not get a circular order.  However, by \reflem{FiniteCollisions}, a pair of cusps collide at only a finite number of points of $\calI$.   The collection of pairs of distinct cusps is countable; thus the set of $\rho_\mu \in \calI$ having some collision is only countable.  The uncountable remainder give circular orders; these are denoted by $\calO_\mu$.  The definition of $\dev_\mu$ implies that $\calO_\mu$ is compatible with $\alpha$.  The deck group $\pi_1(M)$ acts on cusps $\Delta_M$ directly and acts on $\bdy \HH^2 = \RR \cup \{\infty\}$ via the holonomy representation to $\PSL(2,\RR)$.  The circular orders $\calO_\mu$ are obtained by pulling back from $\bdy \HH^2$ and so are invariant.

Finally, we show that all of the circular orders $\calO_\mu$ are distinct.  As an example, \reffig{fig8_(-1p1_1)_shapes} shows the tetrahedron shapes at Dehn surgery coordinates $(-1.1,1)$.  The coordinates are not integral and so, unlike \reffig{CuspTriangulationComplete}, the developing map does not factor through a tiling.  Instead, we have a cut along the negative real axis where the ends of the longitude and meridian do not line up.  The cusps associated to the end of the longitude have slid to the left relative to the cusps associated to the end of the meridian.  However, if we go eleven times backwards along the meridian and ten times forwards along the longitude, the corresponding holonomy is again the identity, and again the cusps line up.  In general, if $\mu = p/q$ is rational (and in lowest terms) then the holonomy of $m^p l^q$ is trivial.  We deduce that $\dev_\mu(c_{p,q}) = \dev_\mu(c_{0,0})$.  If $\mu < p/q$ then $\dev_\mu(c_{p,q})$ is to the left of $\dev_\mu(c_{0,0})$ in our picture.  If $\mu > p/q$ then $\dev_\mu(c_{p,q})$ is to the right of $\dev_\mu(c_{0,0})$ in our picture.

Therefore, if $\rho_\mu \in \calI$ has no collisions, then the circular order $\calO_\mu$ satisfies 
\[
\calO_\mu(\dev_\mu(c_\infty), \dev_\mu(c_{0,0}), \dev_\mu(c_{p,q})) = \pm 1
\]
as $\mu < p/q$ or $p/q < \mu$, respectively.  
Now, for any two circular orders $\calO_\mu$ and $\calO_{\mu'}$, there is a rational number $p/q$ between $\mu$ and $\mu'$; therefore $\calO_\mu$ and $\calO_{\mu'}$ are distinct.  This completes \refexa{FlatDSS}. 
\end{example}

\begin{figure}[htbp]
\labellist
\small\hair 2pt
\pinlabel $z$ at 303 31
\pinlabel $z$ at 303 131
\pinlabel $w$ at 260 89
\pinlabel $w$ at 102 148
\pinlabel $z$ at 109 248
\pinlabel $z$ at 189 302
\pinlabel $w$ at 310 172
\pinlabel $w$ at 213 237

\tiny
\pinlabel $z$ at 256 112
\pinlabel $z$ at 256 149
\pinlabel $w$ at 242 128.5
\pinlabel $w$ at 182.5 151

\pinlabel $z$ at 187 192
\pinlabel $z$ at 218 213
\pinlabel $w$ at 261 170
\pinlabel $w$ at 224.5 194.5

\pinlabel $0$ at 102 344
\pinlabel $1$ at 182 344

\endlabellist
\includegraphics[width=0.8\textwidth]{Figures/fig8__-1_1__shapes2}
\caption{A cartoon of the image of the developing map, as applied to the triangulation of the universal cover of the boundary torus of the figure-eight knot complement.  Here the tetrahedron shapes are $z \approx 1.8731617275018602$ and $w \approx -0.428119859615$ (calculated by SnapPy~\cite{snappy}).  These correspond to the surgery coefficients $(\mu, \lambda) = (-1, 1)$.  Since the shapes are real, the tetrahedra are flat.  Thus the cusp triangles degenerate to intervals.  The vertical scale in this figure has no geometric significance; we draw the triangles as rectangles instead of as intervals in order to see their gluings.  The vertical sides of each rectangle give the vertices (of the cusp triangle) with dihedral angle zero; the small black arrow in each rectangle points at the vertex with dihedral angle $\pi$.}
\label{Fig:fig8_(-1_1)_shapes2}
\end{figure}

\begin{figure}[htbp]
\labellist
\small\hair 2pt
\pinlabel $z$ at 335 31
\pinlabel $z$ at 335 131
\pinlabel $w$ at 290 89
\pinlabel $w$ at 115 148
\pinlabel $z$ at 137 248
\pinlabel $z$ at 218 302
\pinlabel $w$ at 342 172
\pinlabel $w$ at 241 237

\tiny
\pinlabel $w$ at 210 150.5
\pinlabel $z$ at 216 192

\pinlabel $0$ at 129.5 344
\pinlabel $1$ at 210.5 344

\endlabellist
\includegraphics[width=0.8\textwidth]{Figures/fig8__-1p1_1__shapes}
\caption{Another cartoon, as in \reffig{fig8_(-1_1)_shapes2}.  Here the tetrahedron shapes are $z \approx 1.90447711038$ and $w \approx -0.411335649452$ (calculated by SnapPy~\cite{snappy}).  These correspond to the surgery coefficients $(\mu, \lambda) = (-1.1, 1)$.}
\label{Fig:fig8_(-1p1_1)_shapes}
\end{figure}

\begin{question}
\label{Que:RigidVersusNotSo}
As discussed in \refrem{BundleRigidity} the transverse taut structures coming from layered triangulations of surface bundles (as in \refexa{SurfaceBundlesTwo}) have unique compatible circular orders.  As we shall show in \refthm{VeerImpliesUnique}, veering triangulations also have unique compatible circular orders.  

On the one hand, most layered triangulations have no veering structure.  On the other hand, Hodgson, Rubinstein, Tillmann, and the second author~\cite[Section~4]{HRST11} give a veering triangulation for the SnapPy manifold s227; they also show that s227 is not fibred.  Thus \refthm{VeerImpliesUnique} gives examples of circular orders that do not come from fibrations.  We give many more non-fibred manifolds with veering triangulations in~\cite[\texttt{veering\_census\_with\_data.txt}]{GSS19}.  We also use \emph{veering Dehn surgery} to give an infinite family of such manifolds~\cite{veering_dehn_surgery}. 

However, the exotic transverse taut structures in \refexa{FlatDSS} each have uncountably many compatible circular orders.  This raises several questions:  Suppose that $(M, \calT, \alpha)$ is a transverse taut triangulation.  What are necessary and sufficient conditions on $\alpha$ to ensure there is a unique compatible order?  Is this property decidable?  Is there a transverse taut ideal triangulation $(M, \calT, \alpha)$ that admits \emph{no} compatible orders?
\end{question}

\section{Geography of taut triangulations}
\label{Sec:Geography}

In this section, after reviewing material on \emph{train tracks}, we introduce the new concepts of \emph{landscapes}, \emph{continents}, and \emph{continental exhaustions}. 
 
\subsection{Train tracks}
\label{Sec:TrainTracks}

Here we will closely follow the drawing style, and thus the imposed definitions, introduced in~\cite[Figure~11]{Agol11}.  A \emph{pre-track} $\tau$ in a surface $F$ is a locally finite, properly embedded, smooth graph.  The vertices of $\tau$ are called \emph{switches} while the edges of $\tau$ are called \emph{branches}.  Every switch of $\tau$ is equipped with a tangent line.  
We call $\tau$ a \emph{train track} if 
\begin{itemize}
\item 
every switch in the interior of $F$ has at least one branch entering on each of its two sides and
\item 
every switch in $\bdy F$ has its tangent perpendicular to $\bdy F$ and has valence at least one. 
\end{itemize}
This is a variant of the usual definition (see, for example,~\cite[Definition~8.9.1]{thurston_notes}); we are omitting the Euler characteristic condition on the components of $F - \tau$. 


Suppose that $s \in \tau$ is a switch in the interior of $F$.  We say $s$ is \emph{large} if $s$ has exactly two branches entering on each side.  We say $s$ is \emph{small} if $s$ has exactly one branch entering on each side.  We say $s$ is \emph{mixed} if one branch enters on one side of $s$ and two enter on the other.  See \reffig{Tracks}.

\begin{figure}[htbp]
\subfloat[A large switch.]{
\includegraphics[width=0.28\textwidth]{Figures/switch_large}
\label{Fig:SwitchLarge}
}
\quad
\subfloat[A small switch.]{
\includegraphics[width=0.28\textwidth]{Figures/switch_small}
\label{Fig:SwitchSmall}
}
\quad
\subfloat[A mixed switch.]{
\includegraphics[width=0.28\textwidth]{Figures/switch_mixed}
\label{Fig:SwitchMixed}
}
\caption{Switches.}
\label{Fig:Tracks}
\end{figure}

We may \emph{split} (either to the right or to the left) the track $\tau$ along a large switch to obtain a new track $\tau' \subset F$.  See \reffig{SwitchSplit}.  After a split the components of $F - \tau'$ are homeomorphic to those of $F - \tau'$.  The reverse of a split is called a \emph{fold}.

\begin{figure}[htbp]
\includegraphics[width=.935\textwidth]{Figures/switch_large_split}
\caption{A large switch can split either to the left or to the right.}
\label{Fig:SwitchSplit}
\end{figure}

A \emph{train route} in a train track $\tau$ is a smooth embedding of $[0,1]$, of $S^1$, of $\RR_{\geq 0}$, or of $\RR$ into $\tau$.  Such a route is called, respectively, a \emph{train interval}, a \emph{train loop}, a \emph{train ray}, or a \emph{train line}.  


\subsection{Landscapes}

Suppose that $(M, \calT, \alpha)$ is a transverse taut ideally triangulated three-manifold.  
Let $\calB = \calB(\alpha)$ be the associated taut branched surface, made from the faces of $\calT$.   Let $\cover{\calB}$ be the preimage of $\calB$ in the universal cover $\cover{M}$.

\begin{definition}
A \emph{landscape} $L \subset \cover{\calB}$ is a connected embedded carried surface, which is a union of triangles of $\cover{\calB}$.  
\end{definition}

\noindent
A landscape $L$ inherits an ideal triangulation and a co-orientation from $\calB$.  Boundary components of $L$ (if any) are necessarily edges of $\cover{\calT}$.  We now restate \cite[Corollary~1.1]{SchleimerSegerman19} in terms of landscapes.

\begin{lemma}
\label{Lem:Disk}
The interior of a landscape $L$ is an open disk. \qed
\end{lemma}

Thus any landscape $L$ induces a circular order on the cusps of $\cover{M}$ which it meets; this circular order is compatible with the triangles making up $L$. 

Simple examples of landscapes include: any single face of $\cover{\calT}$, the upper or lower boundary of any single tetrahedron of $\cover{\calT}$, or any connected component of any lift of any surface carried by $\calB$.  

\begin{definition}
\label{Def:UpperLowerTrackFace}
Suppose that $f$ is a face of $\cover{\calT}$.  Suppose that $t$ is the tetrahedron attached to $f$, above $f$.  Let $e_0$, $e_1$, and $e_2$ be the edges of $f$ and suppose that the lower $\pi$--edge of $t$ is attached to $e_0$.  We define $\tau^f$, the \emph{upper track} for $f$ as follows.  The track $\tau^f$ has three switches; we place these at the midpoints of the edges $e_i$.  The track $\tau^f$ has two branches, running from $e_1$ to $e_0$ and from $e_2$ to $e_0$.  The \emph{lower track} $\tau_f$ is defined in the same way, using the tetrahedron attached to, and immediately below, $f$.  See \reffig{UpperLowerTracks}. 
\end{definition}

In this section, it sometimes is convenient to think of water flowing along the branches of $\tau^f$; in the notation of the previous paragraph, the water flows away from $e_1$ and $e_2$ and flows towards $e_0$.  

\begin{definition}
\label{Def:TrackCusp}
We define the \emph{track-cusp} of $\tau^f$ by taking the switch of $\tau^f$ and adding a small neighbourhood of it in the region of $f - \tau^f$ between the two branches.  See \reffig{UpperLowerTracks}.  We define the track-cusps of $\tau_f$ similarly. 
\end{definition}

\begin{definition}
\label{Def:UpperTrack}
Suppose that $L$ is a landscape on $\cover{\calB}$.  We define the \emph{upper track} $\tau^L$ to be the union of $\tau^f$ as $f$ runs over the ideal triangles of $L$.  We define $\tau_L$ similarly.  See \reffig{Tracks} for simple examples.  
\end{definition}

\begin{figure}[htbp]
\subfloat[Two taut tetrahedra $t^f$ and $t_f$ above and below a face $f$.]{
\labellist
\small
\hair 2pt
\pinlabel $f$ at 88 71
\endlabellist
\includegraphics[height=1in]{Figures/two_taut_tetrahedra}
\label{Fig:TwoTautTetrahedra}
}
\quad
\subfloat[The upper train track $\tau^f$ in $f$.]{
\labellist
\small
\hair 2pt
\pinlabel $\tau^f$ [tr] at 65 42
\endlabellist
\includegraphics[height=1in]{Figures/upper_track}
\label{Fig:UpperTrack}
}
\quad
\subfloat[The lower train track $\tau_f$ in $f$.]{
\labellist
\small
\hair 2pt
\pinlabel $\tau_f$ [tl] at 77 37
\endlabellist
\includegraphics[height=1in]{Figures/lower_track}
\label{Fig:LowerTrack}
}
\caption{  The upper track $\tau^f$ points at the bottom $\pi$--edge of $t^f$ while the lower track $\tau_f$ points at the top $\pi$--edge of $t_f$. The track-cusps in $t^f$ and $t_f$ are shaded darker grey.}
\label{Fig:UpperLowerTracks}
\end{figure}

Suppose $e \subset L$ is an interior edge.  Drawing on the water analogy above, we call $e$ a \emph{sink} for $\tau^L$ if $e \cap \tau^L$ is a large switch (that is, the flows in the two adjacent triangles flow into $e$).  We call $e$ a \emph{fall} for $\tau^L$ if $e \cap \tau^L$ is a mixed switch (that is, the flows cross $e$).  We call $e$ a \emph{watershed} for $\tau^L$ if $e \cap \tau^L$ is a small switch (that is, the flows both flow out of $e$).  Again, see \reffig{Tracks}.

\begin{remark}
\label{Rem:Sink}
Note that if $e \subset L$ is a sink for $\tau^L$ then there is a tetrahedron, immediately above $L$, which shares two faces with $L$ and whose lower $\pi$--edge is equal to $e$.  There is a similar statement for $\tau_L$. 
\end{remark}

\subsection{Rivers and continents}

\begin{definition}
\label{Def:River}
A landscape $R$ is an \emph{upper river} if 
\begin{itemize}
\item
every triangle of $R$ meets at most two others and 
\item
every interior edge of $R$ is a fall for $\tau^R$. 
\end{itemize}
For an example, see \reffig{River}(0).  Note that the flows equip $\tau^R$ with a consistent orientation.
We call the first triangle of $R$ the \emph{source} of the river.  We call the final edge of $R$ the \emph{mouth} of the river.  The number of triangles in $R$ is its \emph{length} and is denoted $\ell(R)$.  Lower rivers are defined using $\tau_L$.
\end{definition}

\begin{definition}
\label{Def:Continent}
A finite union of tetrahedra $C \subset \cover{\calT}$ is a \emph{continent} if 
\begin{itemize}
\item
any pair of tetrahedra in $C$ are connected by a path in $C$ transverse to the two-skeleton of $C$ and 
\item
$\bdy C$ is the union of a pair of landscapes $L$ and $L'$, called \emph{upper} and \emph{lower} respectively, meeting only along edges with dihedral angle zero in $C$.  
\end{itemize}
The last property says that $\bdy L = \bdy L'$.  We call this common boundary the \emph{coast} of $C$.  Our convention is that $L$ is above $C$ and $L'$ is below. 
\end{definition}

\noindent
For example, a single tetrahedron in $\cover{\calT}$ is a continent.  See \reffig{Continent} for a cartoon of a more complicated continent.  

\begin{figure}[htb]
\centering
\includegraphics[width=0.5\textwidth]{Figures/continent}
\caption{A continent.}
\label{Fig:Continent}
\end{figure}

Suppose that $C \subset \cover{M}$ is a continent with upper landscape $L$.  Suppose that $f$ is any face of $L$.  Let $R \subset L$ be the maximal river with source $f$.  Let $e$ be the mouth of $R$.  Thus $e$ is either a sink for $\tau^L$ or a coastal edge of $C$.  We can now obtain a new continent $C'$ from $C$.  Consider the tetrahedron $t$ with lower $\pi$--edge equal to $e$.  Thus $t$ is above $L$; the tetrahedron shares two faces or one face with $L$ as $e$ is a sink or is coastal.  In either case $C' = C \cup t$ is again a continent.  Also, $C'$ either has the same cusps as $C$ or has one more cusp than $C$, exactly as $e$ is a sink or is coastal.  We call $t$ respectively an \emph{interior landfill} (or simply an \emph{in-fill}) or a \emph{coastal landfill} of $C$.  Finally, we can do the same procedure underneath $C$, attaching $t$ up to $L'$. 

\begin{remark}
\label{Rem:Fill}
Landfilling is the only way to extend a continent by a single tetrahedron and again have a continent.  Attaching any other tetrahedron along a face causes an overhang: the boundary of the resulting union of tetrahedra consists of more than two landscapes.
\end{remark}

Consider a continent $C \subset \cover{M}$.  A landscape $K \subset C$ \emph{spans} $C$ if $K$ contains all coastal edges, and thus all cusps, of $C$.  
For example, the upper and lower boundaries of $C$ span $C$.  A continent $C$ is \emph{layered} if it contains a sequence $\{L_i\}$ of landscapes with $L_0 = L'$, with $L_N = L$, and with $L_{i}$ and $L_{i+1}$ cobounding a single tetrahedron. 

\begin{lemma}
\label{Lem:ContinentsAreLayered}
Suppose that $(M, \calT, \alpha)$ is a transverse taut ideally triangulated three-manifold.  
Suppose that $C \subset \cover{M}$ is a continent.  Then $C$ is layered.
\end{lemma}

\begin{proof}
Take $L_0 = L'$.  Suppose that we have found, by induction, landscapes $L_i \subset C$, for $i \leq k$, each separated from the next by a single tetrahedron.  Since $L_0$ contains all coastal edges, the same holds for the $L_i$; thus each $L_i$ spans $C$.  If $L_k = L$ we are done.  Suppose not.  Let $t$ be any tetrahedron in $C$, above $L_k$ and meeting $L_k$, say along the face $f$.  Let $\tau^k$ be the upper track for $L_k$.  

We now apply the \emph{river argument}, as follows.  Let $R \subset L_k$ be the maximal river with source $f$.  Since $t$ is attached to $f$ we may appeal to \refrem{Fill}: there are tetrahedra of $C$ attached to every triangle of $R$.  Note that the mouth of $R$ is either a sink for $\tau^k$ or a coastal edge of $C$.  If it is a sink, then we in-fill the corresponding tetrahedron and form $L_{k+1}$.  If $R$ flows to the coast then the coastal landfill finds a new cusp of $C$.  Thus $L_k$ did not span $C$, a contradiction. 
\end{proof}

We deduce that any continent $C$ is topologically a ball.  Also, every landscape spanning $C$ gives the same circular order to the cusps of $C$.

\subsection{Continental exhaustions}
Here is a simple, but important, combinatorial tool.

\begin{definition}
\label{Def:ContinentalExhaustion}
A \emph{continental exhaustion} of $\cover{M}$ is a sequence of continents $\{C_n\}_{n \in \NN}$ so that $C_n \subset C_{n+1}$ and so that $\cover{M} = \cup C_n$. 
\end{definition}

\begin{remark}
\label{Rem:OneAtATime}
Any continental exhaustion can be refined so that $C_{n+1}$ is obtained from $C_n$ by adding exactly one tetrahedron.  To see this suppose that $C \subset C'$ are continents.  If there is a tetrahedron $t$ of $C'$ attached to $C$ along a boundary face $f$ then we apply the river argument (as in the proof of \reflem{ContinentsAreLayered}) to find a tetrahedron $t'$ of $C'$ so that $C \cup t'$ is again a continent, possibly with one more cusp. 
\end{remark}

\begin{proposition}
\label{Prop:ExhaustImpliesLayered}
Suppose that $(M, T, \alpha)$ is a transverse taut ideally triangulated three-manifold.  If $\cover{M}$ admits a continental exhaustion then $\cover{M}$ is layered. 
\end{proposition}

\begin{remark}
\label{Rem:LayeredImpliesExhaust}
The converse of \refprop{ExhaustImpliesLayered} also holds; the proof is similar to that of \reflem{LayeredImpliesLayered}.  The converse is not needed for \refthm{VeerImpliesUnique} so we omit it. 
\end{remark}


\begin{remark}
\label{Rem:BadExample}
Note that \refprop{ExhaustImpliesLayered} together with \reflem{LayeredImpliesUnique} imply that the transverse taut ideal triangulations of \refexa{FlatDSS} 
do not have layerings and thus do not have continental exhaustions.
\end{remark}

\begin{proof}[Proof of \refprop{ExhaustImpliesLayered}]
Suppose that $\cover{M}$ has a continental exhaustion $\{C_n\}$.  By \refrem{OneAtATime} we may assume that $C_{n+1}$ is obtained from $C_n$ by adding exactly one tetrahedron.  Our induction hypothesis is that $C_n$ is layered by $\calK_n = \{K_n^i\}_{i = p(n)}^{q(n)}$, a sequence of $n+1$ spanning landscapes.  That is,
\begin{itemize}
\item
$K_n^{p(n)}$ is the lower landscape for $C_n$,
\item
$K_n^{q(n)}$ is the upper landscape for $C_n$,
\item
$K_n^{i+1}$ is obtained from $K_n^i$ by a single in-fill tetrahedron, and
\item
$q(n) - p(n) = n$.
\end{itemize}
We further assume that for any $m < n$ and for any $i$ between $p(m)$ and $q(m)$ we have
\begin{itemize}
\item
$K_n^i \cap C_m = K_m^i$.
\end{itemize}

Suppose now that $C_{n+1}$ is obtained from $C_n$ by attaching the tetrahedron $t$.  There are two cases: either $C_{n+1}$ has the same cusps as $C_n$, or it has one more cusp than $C_n$. 

If $C_n$ and $C_{n+1}$ have the same cusps, then $t$ is attached above or below $C_n$.  If above, then we set $p(n+1) = p(n)$ and $q(n+1) = q(n) + 1$, we set $K_{n+1}^{q(n+1)}$ to be the upper landscape of $C_{n+1}$, and we set $K_{n+1}^i = K_n^i$ for all $i$ between $p(n)$ and $q(n)$.  If $t$ is below $C_n$ then the argument is similar. 

If $C_{n+1}$ has one more cusp than $C_n$ then there is a face $f$, say on the bottom of $t$, which meets the new cusp.  Note that $f$ is not contained in $C_n$ and meets exactly one coastal edge of $C_n$.  For $i$ between $p(n)$ and $q(n)$ we take $K_{n+1}^i = K_n^i \cup f$ and note that $K_{n+1}^i$ spans $C_{n+1}$.  Note that $t$ is layered onto $K_{n+1}^{q(n)}$.  So set $p(n+1) = p(n)$, set $q(n+1) = q(n) + 1$, and define $K_{n+1}^{q(n+1)}$ to be the upper landscape for $C_{n+1}$.  

Thus $\calK_{n+1} = \{K_n^i\}_{i = p(n+1)}^{q(n+1)}$ is the desired collection of landscapes spanning $C_{n+1}$.

Finally, we define $K^i = \cup_{n \in \ZZ} K_n^i$.  Note that $K^i$ has no boundary edges and that $K^{i+1}$ is obtained from $K^i$ by a single in-fill.  Thus $\{K^i\}_{i \in \ZZ}$ is the desired layering of $\cover{M}$. 
\end{proof}

\section{Veering triangulations}
\label{Sec:Veering}

We now give the definition of \emph{veering}, following~\cite{Agol11, HRST11}.  Recall that $M$ is given with an orientation. 

\begin{definition}
\label{Def:Veering}
A \emph{veering structure} $\alpha$ on an ideal triangulation $\calT$ is a taut angle structure together with an edge colouring.  Each edge is coloured red or blue as follows. 
Suppose that $t$ is a model taut tetrahedron and $f \subset t$ is a face.  Suppose $e_0$, $e_1$, and $e_2$ are the edges of $f$, ordered anti-clockwise as viewed from the outside of $t$, with $e_0$ having dihedral angle $\pi$ inside of $t$.  Then $e_1$ is red and $e_2$ is blue.  
\end{definition}

\reffig{VeeringTet} shows a model veering tetrahedron.  If $\calT$ has a veering structure $\alpha$ whose taut angle structure is transverse then we will say that $\alpha$ is a \emph{transverse veering structure}.

\begin{figure}[htbp]
\labellist
\small\hair 2pt
\pinlabel 0 at -5 40
\pinlabel 0 at 87 40
\pinlabel 0 at 40 -7
\pinlabel 0 at 40 87
\pinlabel $\pi$ at 65 55
\pinlabel $\pi$ at 65 25
\endlabellist
\includegraphics[width=0.2\textwidth]{Figures/veering_on_tetrahedron}
\caption{A model veering tetrahedron.  The dotted edges are red; the dashed edges are blue. The two edges with angle $\pi$ may have either colour.  See \reffig{UpperGluingAutomaton} for the four possible veering tetrahedra.}
\label{Fig:VeeringTet}
\end{figure}

We can now state the main result of this section.  Recall that $\Delta_M$ is the set of cusps of $\cover{M}$.  

\begin{theorem}
\label{Thm:VeerImpliesUnique}
Suppose that $(M, \calT, \alpha)$ is a transverse veering ideally triangulated three-manifold.  Then there is a unique compatible circular order $\calO_\alpha$ on the cusps of $\cover{M}$.  Furthermore, $\calO_\alpha$ is dense and $\pi_1(M)$--invariant.
\end{theorem}

Given the work we have already done, to prove \refthm{VeerImpliesUnique} it suffices to prove the following.

\begin{proposition}
\label{Prop:VeerImpliesExhaust}
Suppose that $(M, \calT, \alpha)$ is a transverse veering ideally triangulated three-manifold.  
Then $\cover{M}$ admits a continental exhaustion.
\end{proposition}

\refprop{VeerImpliesExhaust} and \refprop{ExhaustImpliesLayered} give the following. 

\begin{corollary}
\label{Cor:VeerImpliesLayered}
Suppose that $(M, \calT, \alpha)$ is a transverse veering ideally triangulated three-manifold.  
Then $\cover{M}$ admits a layering.  \qed
\end{corollary}

\begin{proof}[Proof of \refthm{VeerImpliesUnique}]
\refcor{VeerImpliesLayered} tells us that $\cover{M}$ admits a layering.  \reflem{LayeredImpliesUnique} now gives the desired unique, compatible, dense, $\pi_1(M)$--invariant circular order. 
\end{proof}

\subsection{Combinatorics of veering triangulations}

We begin by setting down some of the combinatorics of veering triangulations.  Suppose that $(M, \calT, \alpha)$ is transverse veering.  Suppose that $f$ is a face of $\cover{\calT}$.  So $f$ has at least one blue edge and at least one red edge.  Ordering the cusps of $f$ as in \refsec{Compatibility} -- that is, looking from above, in anti-clockwise order -- there is a unique cusp $u(f)$ of $f$ where the colours switch from blue to red.   Consulting \reffig{VeeringTet} we see that the branches of the upper track $\tau^f$ flow away from $u(f)$.  (Swapping colours gives a corresponding statement for $\tau_f$.)  This implies that, in the presence of a veering structure, our upper track is the same as the train track used by Agol~\cite[Main~Construction]{Agol11} to define veering triangulations of mapping tori.  See \reffig{VeeringTriangles}.

\begin{figure}[htb]
\centering
\subfloat[Upper tracks.]{
\includegraphics[width=0.40\textwidth]{Figures/veering_triangles_upper_track}
\label{Fig:VeeringTrianglesUpperTrack}
}
\qquad
\subfloat[Lower tracks.]{
\includegraphics[width=0.40\textwidth]{Figures/veering_triangles_lower_track}
\label{Fig:VeeringTrianglesLowerTrack}
}
\caption{The two possible faces in a veering triangulation, as viewed from above.  We draw the upper track in green, and the lower track in purple.}  
\label{Fig:VeeringTriangles}
\end{figure}

\begin{definition}
\label{Def:FanToggle}
Suppose that $t$ is a model veering tetrahedron.
\begin{itemize}
\item
If $t$ has three red and three blue edges then we call $t$ a \emph{toggle tetrahedron}.  
\item
If $t$ has more red than blue edges (more blue than red edges) then we call $t$ a red (blue) \emph{fan tetrahedron}.   \qedhere
\end{itemize}
\end{definition}

The four possible model veering tetrahedra are shown in \reffig{GluingAutomaton}; we also show all possible face gluings.  In \reffig{EdgeNeighbourhood} we show one possibility for the tetrahedra on the two sides of an edge. 

\begin{figure}[htb]
\centering
\subfloat[Upper tracks.]{
\includegraphics[width=0.47\textwidth]{Figures/upper_gluing_automaton}
\label{Fig:UpperGluingAutomaton}
}
\thinspace
\subfloat[Lower tracks.]{
\includegraphics[width=0.47\textwidth]{Figures/lower_gluing_automaton}
\label{Fig:LowerGluingAutomaton}
}
\caption{In each diagram: moving anti-clockwise from the top we have a toggle tetrahedron (blue on top), a blue fan tetrahedron, the other toggle tetrahedron (red on top), and a red fan tetrahedron.  An arrow points from one tetrahedron to another if the second tetrahedron can be glued on top of the first.  In \reffig{UpperGluingAutomaton} (\reffig{LowerGluingAutomaton}) we draw in each face the corresponding upper (lower) track; the tracks on the lower faces are more faintly drawn.}
\label{Fig:GluingAutomaton}
\end{figure}

\begin{figure}[htbp]
\includegraphics[width=0.35\textwidth]{Figures/edge_neighbourhood}
\caption{One possible neighbourhood of a blue edge, $e$.  To the right of $e$ there is a single blue fan tetrahedron.  To its left there are two toggle tetrahedra, one of each type, and a stack of two red fan tetrahedra.  We have not drawn the tetrahedron above $e$ (in front of the page) or the one below $e$ (behind the page).  See also~\cite[Figure~12]{Agol11}.}
\label{Fig:EdgeNeighbourhood}
\end{figure}

\begin{remark}
In other literature the toggles are called \emph{hinge tetrahedra}; see~\cite[page~1247]{Gueritaud06} or~\cite[page~211]{FuterGueritaud13}. 
\end{remark}


We record, in \reffig{PossibleTwoTriangles}, the upper tracks in all possible two-triangle veering landscapes.  Note that watersheds and sinks fit together to give the boundaries of veering tetrahedra.  

\begin{figure}[htbp]
\subfloat[A left watershed.]{
\includegraphics[width=0.28\textwidth]{Figures/left_watershed}
\label{Fig:LeftWatershed}
}
\quad
\subfloat[A right watershed.]{
\includegraphics[width=0.28\textwidth]{Figures/right_watershed}
\label{Fig:RightWatershed}
}

\subfloat[A left fall.]{
\includegraphics[width=0.28\textwidth]{Figures/left_fall}
\label{Fig:LeftFall}
}
\quad
\subfloat[A right fall.]{
\includegraphics[width=0.28\textwidth]{Figures/right_fall}
\label{Fig:RightFall}
}
\quad
\subfloat[A sink.]{
\includegraphics[width=0.28\textwidth]{Figures/sink}
\label{Fig:Sink}
}
\caption{Upper tracks on all possible two-triangle veering landscapes.  Black edges can be either red or blue.}
\label{Fig:PossibleTwoTriangles}
\end{figure}

Veering triangulations are far more rigid that transverse taut ideal triangulations.  \reflem{EdgeNeighbourhood} is the first piece of evidence of that; the proofs follow from the veering hypothesis.  See \reffig{EdgeNeighbourhood}.

\begin{lemma}
\label{Lem:EdgeNeighbourhood}
Suppose that $(M, \calT, \alpha)$ is a transverse veering ideally triangulated three-manifold.  
Suppose that $e$ is an oriented blue edge in $\cover{\calT}$.  Then the following hold.  
\begin{enumerate}
\item
There is exactly one tetrahedron above, and exactly one below, $e$.  
\item
There is at least one tetrahedron to the right of $e$, and similarly to its left.  
\label{Itm:HalfEdgeDegreeAtLeastOne}
\item
If there is exactly one tetrahedron to the right of $e$, then it is a blue fan tetrahedron.  If there is more than one then in the resulting stack of tetrahedra contains, from bottom to top, a red-on-top toggle, some number of red fans, and finally a blue-on-top toggle.  The same holds to the left of $e$. 
\item
There are exactly four faces meeting $e$ that have more blue edges than red.
\end{enumerate}
Similar statements hold when $e$ is red instead of blue. \qed
\end{lemma}

Here is a more long-range restriction on the combinatorics of veering triangulations; we will not need this until the proof of \refthm{LinkSpace}.

\begin{lemma}
\label{Lem:NoParallelEdges}
Two cusps of $\cover{\calT}$ are connected by at most one edge.
\end{lemma}

This follows from the existence of a strict angle structure~\cite{HRST11, FuterGueritaud13} on $(\calT, \alpha)$ and a combinatorial area argument.  Here is an argument more in the spirit of this paper. 

\begin{proof}[Proof of \reflem{NoParallelEdges}]
Fix cusps $c, d \in \Delta_M$ and suppose that $e$ is an edge of $\cover{\calT}$ connecting them.  By \refcor{VeerImpliesLayered} there is a layering $\calK = \{K_i\}$ of $\cover{M}$.  We choose the indexing so that $K_0$ is the lowest layer containing $e$.  

Let $\tau^i$ be the upper track in $K_i$.  We deduce that $\tau^0$ has a sink at $e$.  Applying \reflem{Disk}, we define $P_i \subset K_i$ to be the strip of triangles connecting $c$ and $d$.  Consulting \reffig{UpperGluingAutomaton} and applying induction we find that, for all $i < 0$, the upper train track $\tau^i$ in $P_i$ contains a train interval between the track-cusps in the first and last triangles of $P_i$.  Furthermore, as $i$ tends to negative infinity (as we descend in $\calK$), the length of these train intervals increase (non-strictly) monotonically.  The same holds above $e$ replacing upper train tracks with lower.
\end{proof}

\subsection{New continents from old}

Our aim is to build a continental exhaustion for the universal cover of a veering triangulation. Given a continent $C$ and a tetrahedron $t$, we must enlarge $C$ to include $t$. Continents do not have overhangs, so we can only add tetrahedra in certain locations; these are given by following rivers downstream to sinks or the coast. Adding such tetrahedra is always possible. However, \refexa{FlatDSS} shows that in the absence of veering, this procedure may never arrive at $t$. 

\begin{lemma}
\label{Lem:NoNewSinks}
Suppose that $(M, \calT, \alpha)$ is a transverse veering ideally triangulated three-manifold.  
Suppose that $C \subset \cover{M}$ is a continent.  Suppose that the lower track on the bottom of $C$ has no sinks.  Then this again holds after a single coastal landfill on top of $C$.  The corresponding statement holds after switching top and bottom. 
\end{lemma}

\begin{proof}
Suppose that $t$ is the coastal landfill tetrahedron above $C$.  Suppose $e$ is the lower $\pi$--edge of $t$.  Let $f$ be the free lower face of $t$.  Suppose that $t'$ is the tetrahedron having $e$ as its upper $\pi$--edge.  From \reflem{EdgeNeighbourhood}\refitm{HalfEdgeDegreeAtLeastOne} we deduce that $t'$ cannot be glued to $t$ along $f$. 
\end{proof}

\begin{definition}
\label{Def:Convex}
Suppose that $C \subset \cover{M}$ is a continent with upper and lower landscapes $L$ and $L'$.  We say that $C$ is \emph{convex} if neither $\tau^L$ nor $\tau_{L'}$ have sinks. 
\end{definition}

For example, a continent containing only one tetrahedron is convex. 

\begin{lemma}
\label{Lem:FiniteInFill}
Suppose that $(M, \calT, \alpha)$ is a transverse veering ideally triangulated three-manifold.  
Suppose that $C \subset \cover{M}$ is a continent.  Then after a finite number of in-fills above and below $C$ we obtain a convex continent $C'$ having the same cusps as $C$. 
\end{lemma}

\begin{proof}
We only consider the top of $C$; the argument below is similar.  Set $C_0 = C$ and $L_0 = L$, the upper landscape of $C$.  Let $\tau^0$ be the upper track for $L_0$.  If $\tau^0$ has no sinks then we are done.  Otherwise, suppose that we have built $C_k$ with upper landscape $L_k$.  Suppose that $t_k = t$ is any infill tetrahedron attached to $L_k$ from above.  By \reflem{Disk}, the landscape $L_k$ is a triangulated disk.  Let $\tau^k$ be the upper track for $L_k$; thus $\tau^k$ is a tree, with all leaves on coastal edges.  The track $\tau^{k+1}$ is obtained from $\tau^k$ by doing a single split.  See \reffig{UpperGluingAutomaton}.  This gives a one-to-one correspondence between the track-cusps of $\tau^k$ and those of $\tau^{k+1}$. 


A track-cusp $s$ of $\tau^k$ in the interior of $L_k$ meets some edge $e$; the flows behind $s$ give $e$ a co-orientation in $L_k$.  The orientation divides the coastal edges into two collections: those in front of $e$ and those behind.  When we attach $t_k$, the track-cusp $s$ moves forward to a new track-cusp $s'$ and a new edge $e'$, both in $L_{k+1}$. Again, see \reffig{UpperGluingAutomaton}.  Recall that that $L_{k+1}$ and $L_k$ have the same coastal edges.
Now, at least one coastal edge goes from being in front of $e$ to being behind $e'$, and none move forward.  Any such sequence $s, s', s'', \ldots$ of track-cusps eventually reaches the coast or otherwise halts.

Let $n = |L_0|$ be the number of triangles in $L_0$, and thus in $L_k$.  Thus any sequence of track-cusps $s, s', s'', \ldots$ as above has length at most $n - 1$.  Also, the number of track-cusps of $\tau^0$ equals $n$.  Thus after adding at most $n(n-1)/2$ in-fill tetrahedra we arrive at an upper track $\tau^k$ without sinks.
\end{proof}

Suppose that $R \subset \cover{M}$ is a river: that is a landscape which is a strip of triangles so that all interior edges are falls for $\tau^R$.  We oriented $R$ in the direction of the flow.  Recall that $\ell(R)$ is the length of $R$.  Let $e_i$ be the $i^\thsup$ fall, counting from the source. We define $h_i$ to be the \emph{height} of the fall $e_i$; that is, $h_i$ is the degree of $e_i$ in $\cover{M}$ minus the degree of $e_i$ \emph{below} the river $R$.  

\begin{definition}
\label{Def:Complexity}
The \emph{complexity} of a river $R$ is the list 
\[
c(R) = (\ell(R), h_1, h_2, \ldots, h_{\ell-1})
\]
We order complexities lexicographically.
\end{definition}

\begin{figure}[htb]
\centering
\labellist
\small\hair 2pt
\pinlabel $f$ [r] at 365 1195
\pinlabel $(0)$ at 310 1145
\pinlabel $(1)$ at 15 940
\pinlabel $(1')$ at 1190 940
\pinlabel $(2')$ at 15 760
\pinlabel $(2)$ at 1190 760
\pinlabel $(3')$ at 15 577
\pinlabel $(3)$ at 1190 577
\pinlabel $(8)$ at 15 207
\pinlabel $(8')$ at 1190 207
\pinlabel (A) at 15 27
\pinlabel (B) at 1190 27
\endlabellist
\includegraphics[width=\textwidth]{Figures/river}
\caption{Some possible steps in a channelisation process. The figures $(0)$, $(1)$, $\ldots$, $(8)$, (A), and (B) all show rivers: $(0)$ is the initial river and the others arise from landfilling.  The figures $(1')$, $(2')$, $\ldots$ , $(8')$ are forked rivers.  The primed indices count the total number of triangles in the two distributaries after the sink.  Riverbanks are shaded brown.  We place a subfigure on the left or right according to the colour of the upper edge of the most recently added tetrahedron.}
\label{Fig:River}
\end{figure}

\begin{definition}
\label{Def:Channelisation}
Suppose $C$ is a convex continent; thus all rivers in $\bdy C$ flow to the coast.  Suppose that $f$ is a face of the upper boundary of $C$. Let $R$ be the maximal river with $f$ as its source.  Let $f'$ be the last triangle of $R$ and let $e'$ be the mouth of $R$. See \reffig{River}(0) for one possible example. Let $t'$ be the tetrahedron above $f'$. Note that $e'$ is the lower edge of $t'$. 

We now landfill $t'$ onto $C$.  By \reflem{NoNewSinks}, the union $C \cup t'$ is again a continent, with no new sinks in its lower boundary.  We now appeal to \reflem{FiniteInFill}: after in-filling the top of $C \cup t'$ a finite number of times we obtain a convex continent $C'$.  We say that $C'$ is the \emph{channelisation} of $(C, f)$. 
\end{definition}



Note that in general, a single channelisation process will not cover the face $f$. The heart of the proof of \refprop{VeerImpliesExhaust} is in showing that a finite number of channelisations suffices to cover $f$.

\begin{proof}[Proof of \refprop{VeerImpliesExhaust}]
We must build a continental exhaustion.  We begin by choosing an enumeration $\{t_m\}_{m \in \NN}$ of the tetrahedra of $\cover{\calT}$.  
Our first continent is $C_0 = t_0$; note that $C_0$ is convex.  Now suppose that we have obtained a convex continent $C_n$.  Let $t_{m(n)}$ be the smallest tetrahedron in our enumeration which is not contained in $C_n$ and which meets $C_n$ along a face $f$.  It will suffice to find a convex continent $C_{n+1}$ containing $C_n$ and $t_{m(n)}$. (For suppose that $t'$ is any tetrahedron.  Let $\{t_{p(k)}\}_{k = 0}^K$ be a sequence of tetrahedra so that $t_{p(0)} = t_0$, so that $t_{p(K)} = t'$, and the tetrahedra $t_{p(k)}$ and $t_{p(k+1)}$ share a face.  Induction proves that $t_{p(k)}$ belongs to $C_{p(k)}$.)

Set $t = t_{m(n)}$.  Breaking the symmetry of the situation, we assume that $t$ is above $C_n$.  Thus $f$ is a face of the upper landscape of $C_n$.  (If $t$ is attached below, the analysis is similar.)  Set $C_n^0 = C_n$.  Suppose that we have constructed $C_n^k$, convex.  
If $t = t_{m(n)}$ is contained in $C_n^k$ then we set $C_{n+1} = C_n^k$ and we are done.  
If not, then $f$ is still a face of the upper boundary of $C_n^k$.  Then we define $C_n^{k+1}$ to be the result of channelising the pair $(C_n^k, f)$.  Note that $C_n^{k+1}$ is again convex.  Let $R^k$ be the maximal river in the upper boundary of $C_n^k$ with $f$ as its source.

\begin{lemma}
\label{Lem:RiversSimplify}
Suppose that $f$ is a face of the upper boundary of both $C_n^k$ and $C_n^{k+1}$.  Then $c(R^{k+1})$ is strictly smaller than $c(R^k)$.  
\end{lemma}

\noindent
Note that \reflem{RiversSimplify} completes the proof of \refprop{VeerImpliesExhaust}; since complexities are well-ordered, eventually $f$ is not a face of some upper landscape.  At that stage $t$ is contained in the continent, as desired. 

Before giving the proof of \reflem{RiversSimplify} we require an expanded notion of a river. We give this in the next definition.  We also give an extended example to illustrate how the proof works. 

\begin{definition}
\label{Def:Forked}
Suppose that $C$ is a continent with upper boundary $L$.  Let $f$ be a face of $L$.  We define the \emph{forked river} $R = R^{(C, f)}$ as follows. 

Let $R^f$ be the maximal river in $L$ with source $f$.  If $R^f$ ends at the coast then we take $R = R^f$; in this case the forked river is simply a river.  Suppose instead that $R^f$ ends at a sink $e$ for $\tau^L$.  Let $f_0$ be the triangle in $L - R^f$ adjacent to $e$.  Let $e_0$ and $e_1$ be the edges of $f_0 - e$.  Let $R^{e_0}$ and $R^{e_1}$ be the maximal rivers starting at $e_0$ and $e_1$, respectively, and flowing away from $f_0$.  These are the \emph{distributaries} flowing from $f_0$; they may contain zero triangles.  Finally, the desired forked river is the landscape 
\[
R = R^f \cup f_0 \cup R^{e_0} \cup R^{e_1}
\]
If $R^{e_i}$ contains no triangles then we take the mouth of $R^{e_i}$ to be $e_i$.   We define $\ell(R)$ to be the \emph{length} of $R$: the number of triangles in $R$.  In the case that a forked river $R$ contains a sink, then we say that the \emph{mouths} of $R$ are the mouths of its distributaries.   The \emph{riverbank edges} of $R$ are the boundary edges of $R$ other than the mouth(s).
\end{definition}

For examples of forked rivers, with distributaries of various sizes (including zero), see subfigures $(1')$, $(2')$, $(3')$, and $(8')$ of \reffig{River}.  

\begin{example}
\label{Exa:RiversSimplify}
We now discuss channelisation, illustrated in \reffig{River}, in more detail.  Subfigure~\ref{Fig:River}(0) shows a possible river $R^k$, with source face $f$.  Suppose that $t'$ is the coastal landfill tetrahedron.  Note that attaching $t'$ fills the mouth of $R^k$ and adds one new cusp and two new coastal edges.  After the landfill there are two possibilities according to the colour, blue or red, of the top edge of $t'$.  These are shown immediately below $(0)$ on the left and right in subfigures $(1)$ and $(1')$, respectively.

In $(1)$ we see a river $R^{k+1}$.  This is isomorphic (as a landscape) to $R^k$; however the last fall of $R^{k+1}$ is less high.  Thus, following \refdef{Complexity}, we have $c(R^{k+1}) < c(R^k)$.  In the remainder of the $(k+1)^\stsup$ channelisation procedure we in-fill all sinks on the top of $C_n^k \cup t'$.  There are none in $R^{k+1}$.  There are none outside of the subfigure (1) because $C_n^k$ was convex.  Thus the only possible in-fill is at the track-cusp pointing out of the top of the subfigure (1).  This may lead to further sinks, which we in-fill in their turn.  However, these in-fills do not alter $R^{k+1}$ because the only place to fill $R^{k+1}$ is at the coast.  

In $(1')$ we have a forked river $R^k_1$ as in \refdef{Forked}.  Note that $R^k_1$ has one more coastal edge than the original river $R^k$ but has the \emph{same} riverbanks.  Thus the only in-fill on the top (or bottom) of the current continent is the unique sink in $R^k_1$.  In $(1')$ the distributary on the left riverbank has length one while the distributary to the right has length zero.  

When we in-fill the sink of $R^k_1$, there are two possibilities, shown in $(2')$ and $(2)$, as the top edge of the landfill tetrahedron is blue or red.  In example $(2)$ we obtain a river $R^{k+1}$ which has been routed along the distributary (to the left) in $R^k_1$.  Again we claim that $c(R^{k+1}) < c(R^k)$.  Firstly, in-fills disjoint from $R^{k+1}$ do not alter it.  Also, while $R^k$ and $R^{k+1}$ have the same length, the height of the second-to-last fall of $R^{k+1}$ is smaller than that of $R^k$.  Since we do not control the change in height of the last fall, our complexity is lexicographic. 


In example $(2')$ we obtain a forked river $R^k_2$.  Note that the sink is further upstream while the total length of the distributaries has increased by one.  Again the only in-fill on the boundary of the continent is the sink of $R^k_2$.  At that point we will either obtain $(3)$, a river with lower complexity, or $(3')$, a forked river where the sink has moved upstream and one of the distributaries is one triangle larger. 

Continuing in this way, if we do not make a river, then eventually the two distributaries contain all of the riverbank edges.  This is shown in the forked river of subfigure~$(8')$.  Here the sink has reached the source and again there is a unique in-fill.  Depending on the colour of the upper edge of this in-fill, we arrive at one of the two rivers shown in subfigures (A) or (B).  In this case our continent now contains $t$. The channelisation process may continue beyond (A) or (B), but by \reflem{FiniteInFill} it eventually terminates and gives a convex continent.
\end{example}

We now consider the general case.  

\begin{proof}[Proof of \reflem{RiversSimplify}]
Recall that $f$ is the upper face of $C_n^0 = C_n$ which we must layer onto.  Recall that $C^{k+1}_n$ is the result of channelisation of the pair $(C^k_n, f)$.  By hypothesis, the face $f$ is contained in the upper boundary of both $C^k_n$ and $C^{k+1}_n$.  Thus $R^k$ and $R^{k+1}$ are both defined. 

Set $C^k_{n,0} = C^k_{n}$ and let $C^k_{n,i}$ be the continent obtained after attaching the first $i$ landfills of this $(k+1)^\stsup$ channelisation.  Only the first landfill is coastal; the rest are in-fill. 
We define $R^k_i$ to be the forked river for the pair $(C^k_{n,i}, f)$.  

\begin{claim}
\label{Clm:ChannelisationInduction}
\mbox{}
\begin{enumerate}
\item
\label{Itm:CoastalEdges}
The mouth(s) of $R^k_i$ are coastal edges. 
\item
\label{Itm:ForkedRiverInduction}
Suppose that $R^k_i$ contains a sink.  Then $\ell(R^k_i) = \ell(R^k) + 1$.  Also, the riverbanks of $R^k_i$ are the same as those of $R^k$.  Also, $R^k \cap R^k_i$ is the sub-river of $R^k$ with source $f$ that ends at the sink of $R^k_i$.  
\item
\label{Itm:RiverInduction}
Suppose $i > 0$ and $R^k_i$ does not contain a sink.  Then $\ell(R^k_i) \leq \ell(R^k)$.  
\end{enumerate}
\end{claim}

\refclm{ChannelisationInduction} proves \reflem{RiversSimplify}, as follows.  By \refitm{RiverInduction} the length of $R^{k+1}$ is no greater than that of $R^k$.  If it is shorter we are done; if it is the same length then by \refitm{ForkedRiverInduction} the in-fill giving $R^{k+1}$ has reduced the height of one fall without altering the heights further upstream.

\begin{proof}[Proof of \refclm{ChannelisationInduction}]
If $i = 0$ then we only need check \refitm{CoastalEdges} for $R^k_0 = R^k$.  This holds because $C^k_n$ is convex.  We now induct on $i$, starting with the base case of $i = 1$.  

So suppose that $i = 1$.  Let $f'$ be the coastal triangle of $R^k_0$.  Let $t'$ be the tetrahedron immediately above $f'$.  Thus $C^k_{n,1} = C^k_{n,0} \cup t'$.  If $f' = f$ then $f$ is not a face of the upper boundary of $C^{k+1}_n$, contrary to hypothesis.  Thus $f' \neq f$.  Thus there is an edge $e'$ of $f'$ that is neither on the riverbank nor coastal.   There are four cases to check as the riverbank edge of $f'$ is either blue or red and as the upper edge of $t'$ is either blue or red.  These are very similar to the general case discussed immediately below, so we omit them.  (See \reffig{Moves}, but delete the triangle attached to $f'$ in the first row of figures.)

Suppose now that $i > 1$.  If $R^k_i$ contains no sink then $R^{k+1} = R^k_{i+1} = R^k_i$ and there is nothing to prove.  Suppose instead that $R^k_i$ contains a sink.  We define $f'$ to be the triangle of $R^k_i$ immediately before the sink edge.  Let $t'$ be the tetrahedron immediately above $f'$.  As before, if $f' = f$ then $f$ does not appear in the upper boundary of the next continent, contrary to hypothesis.  So $f' \neq f$.   Let $e'$ be the edge of $f'$ that is neither on the riverbank nor the sink.  There are again four possibilities according to the colours of the riverbank edge of $f'$ and the upper edge of $t'$. See \reffig{Moves}.

\begin{figure}[htb]
\centering
\subfloat[Landfilling along the blue river bank.]{
\labellist
\small\hair 2pt
\pinlabel (1) [r] at 115 175
\pinlabel $(1')$ at 250 180
\pinlabel $f'$ at 132 310
\pinlabel $e'$ [r] at 106 280
\endlabellist
\includegraphics[width=0.46\textwidth]{Figures/landfill_moves_blue}
\label{Fig:landfill_moves_blue}
}
\quad
\subfloat[Landfilling along the red river bank.]{
\labellist
\small\hair 2pt
\pinlabel $(2')$ [r] at 115 170
\pinlabel (2) at 255 170
\pinlabel $f'$ at 132 250
\pinlabel $e'$ [r] at 106 280
\endlabellist
\includegraphics[width=0.46\textwidth]{Figures/landfill_moves_red}
\label{Fig:landfill_moves_red}
}
\caption{The two possible results of attaching $t'$ to $f'$, depending on the colour of its uppermost edge.  The river bank is shown shaded in brown.  The sink, covered by the landfill tetrahedron, can be of either colour.  Since the results are the same in each case, we here colour that edge black.}
\label{Fig:Moves}
\end{figure}

Suppose that $f'$ has an edge on the blue riverbank and the upper edge of $t'$ is blue.  Then, as shown in \reffig{landfill_moves_blue}(1), after attaching $t'$ we have that $R^k_{i+1}$ is a river.  We deduce that the mouth of $R^k_{i+1}$ is the mouth of the distributary meeting the red riverbank of $R^k_i$, which is coastal by induction.  This proves \refitm{CoastalEdges} in this case.  Since $R^k_{i+1}$ is a river, \refitm{ForkedRiverInduction} holds vacuously.  Note that exactly one of the two upper faces of $t'$ are contained in $R^k_{i+1}$.  Thus $R^k_{i+1}$ is one triangle smaller than $R^k_i$. Applying \refitm{ForkedRiverInduction} to $R^k_i$ we deduce that $\ell(R^k_{i+1}) \leq \ell(R^k)$.  This proves \refitm{RiverInduction} in this case. 

Suppose that $f'$ has an edge on the blue riverbank and the upper edge of $t'$ is red.  Then, as shown in \reffig{landfill_moves_blue}(1), after attaching $t'$ we have that $R^k_{i+1}$ is again a forked river.  Both lower faces of $t'$ belonged to $R^k_i$ and both upper faces of $t'$ belong to $R^k_{i+1}$.  Thus these forked rivers have the same number of triangles.  Likewise they have the same boundaries, and so have the same riverbanks and mouths.  The distributary meeting the blue riverbank has grown by one triangle and the sink of $R^k_{i+1}$ is one triangle further upstream.  This proves the induction hypotheses in this case. 

The remaining two cases (where $f'$ has a red riverbank edge) are the same, switching sides and colours correctly. 

This completes the proof of \refclm{ChannelisationInduction}.
\end{proof}
This completes the proof of \reflem{RiversSimplify}.
\end{proof}
This completes the proof of \refprop{VeerImpliesExhaust}.
\end{proof}

\section{Branched surfaces and branch lines}
\label{Sec:SurfacesAndLines}

Here, following ideas of Agol, given a veering triangulation $(M, \calT, \alpha)$ we construct the upper and lower branched surfaces $\calB^\alpha$ and $\calB_\alpha$.  The \emph{branch lines} of these play an important role in \refsec{LaminationsAlone}.  

\subsection{Branched surfaces}
\label{Sec:UpperLowerSurfaces}

Again, see~\cite[Section~6.3]{Calegari07} for a reference on branched surfaces.  

We construct the \emph{upper branched surface} $\calB^\alpha$ in two stages.  First, fix a tetrahedron $t$ in $\cover{\calT}$.  Let $L$ be the two-triangle landscape above $t$ and $L'$ the two-triangle landscape below.  Note that $\tau^L$, the upper track for $L$, is obtained from the upper track for $L'$ by doing a single split.  If we perform this split continuously then it sweeps out a two-complex $\calB^t$, shown in \reffig{NormalUpperBranchedSurface}.  

The two-complex $\calB^t$ is not quite a branched surface; at the unique vertex the two one-cells are tangent rather than transverse.  We abuse terminology and nonetheless call $\calB^t$ the \emph{upper} branched surface, in $t$, in \emph{normal position}.  We use this terminology because the sectors of $\calB^t$ are three \emph{normal disks}: two triangles and a quadrilateral.  The track-cusps sweep along the two branch components of  $\calB^t$; each branch intervals is a normal arc in a lower face of $t$.  The three sectors and two branch intervals meet at the midpoint of the bottom edge of $t$; this is the unique vertex of $\calB^t$.  Again, see \reffig{NormalUpperBranchedSurface}.

We now remedy the failure of $\calB^t$ to be a branched surface.  We fold $\calB^t$ in a small regular neighbourhood of its branch intervals.  We call the result the \emph{dual position} of $\calB^t$ because, after folding, $\calB^t$ is isotopic to the dual two-skeleton for $t$.  See \reffig{DualUpperBranchedSurface}; note how the branch locus is no longer contained in the lower faces of $t$.   

\begin{figure}[htb]
\centering
\subfloat[Normal position.]{
\label{Fig:NormalUpperBranchedSurface}
\includegraphics[width=0.47\textwidth]{Figures/branched_surf_in_taut_tet_normal}
}
\subfloat[Dual position.]{
\label{Fig:DualUpperBranchedSurface}
\includegraphics[width=0.47\textwidth]{Figures/branched_surf_in_taut_tet_dual}
}
\caption{Two positions of the upper branched surface in a tetrahedron.}
\label{Fig:UpperBranchedSurface}
\end{figure}

We now push $\calB^t$ down to $M$.  We define $\calB^\alpha \subset M$, the \emph{upper branched surface} in \emph{normal position} or in \emph{dual position}, to be the union $\cup_{t \in \calT} \calB^t$ where the latter are also in normal or dual position.  

\begin{remark}
\label{Rem:Shift}
We justify giving a single name, $\calB^\alpha$, to the two positions with the following claim: the normal and dual positions of $\calB^\alpha$ are isotopic in $M$.  To see this, start with $\calB^\alpha$ in normal position.  Now isotope $\calB^\alpha$ slightly upwards.  This makes the branch locus transverse to $\calB(\alpha)$.  We may then isotope a bit further, to make the branch locus self-transverse.  The result is $\calB^\alpha$ in dual position.  Again, see \reffig{DualUpperBranchedSurface}.
\end{remark}

We will almost always draw $\calB^\alpha$ in its simpler, normal, position.  We define $\calB_\alpha$ similarly, in both its normal and dual position, by using the lower tracks $\tau_L$ and $\tau_{L'}$ to build $\calB_t$, and so on. 

\begin{remark}
\label{Rem:Dual}
Note that $\calB^\alpha$ and $\calB_\alpha$ (in dual position) are both isotopic to the dual two-skeleton of $\calT$.  Thus $\calB^\alpha$ is isotopic to $\calB_\alpha$.  Again, see \reffig{DualUpperBranchedSurface}.
\end{remark}

\subsection{Sectors}
\label{Sec:Sectors}

We now consider the various ways the normal quadrilaterals and triangles can meet. See \reffig{UpperGluingAutomaton}.  Recall that $\alpha$ endows the faces and edges of $\calB(\alpha)$ with a co-orientation. 

\begin{remark}
\label{Rem:Sector}
Suppose that $D$ is a sector of $\calB^\alpha$ (in normal position).  Then $D$ is a disk which 
\begin{itemize}
\item
contains exactly one normal quadrilateral, 
\item
contains two (possibly empty) collections of normal triangles above the normal quadrilateral (to its left and right), 
\item
has exactly one \emph{upper} (\emph{lower}) \emph{vertex} where $\alpha$ points out of (into) $D$, at the top (bottom) of the normal quadrilateral, 
\item
has two \emph{cusp vertices} where $\bdy D$ has a cusp with respect to $\alpha$, and 
\item
has one \emph{under-side vertex} where $\alpha$ points into $D$, at the bottoms of each normal triangle. \qedhere
\end{itemize}
\end{remark}

See \reffig{Sector} for a fairly generic example of a sector of $\calB^\alpha$.  

We refer to~\cite[Definitions~6.9 and~6.14, page~215]{Calegari07} for the definitions of \emph{surface laminations} in three-manifolds, their \emph{leaves}, \emph{essential laminations}, and how branched surfaces \emph{carry} laminations.
Further investigation of the upper and lower branched surfaces show that they are \emph{laminar} in the sense of Tao~Li~\cite[Definition~1.4]{Li02}.  Thus they carry essential laminations.  We reprove this result in our setting to get a combinatorial understanding of the laminations and their uniqueness properties.  See Sections \ref{Sec:SuspendingDescending} and \ref{Sec:Uniqueness}.

\begin{figure}[htbp]
\includegraphics[height = 3.5 cm]{Figures/green_sector_blue_top}
\caption{Any sector of $\calB^\alpha$ (in normal position) contains exactly one normal quadrilateral and some (perhaps zero) normal triangles.  The boundary of the sector is drawn in black, while internal boundaries between normal disks are drawn in green.}
\label{Fig:Sector}
\end{figure}

Any leaf of any lamination carried by $\calB^\alpha$ inherits a decomposition into sectors.  We have drawn a small portion of one such leaf in \reffig{SectorsForFig8KnotComplement}.  It is carried by the upper branched surface for the veering triangulation of the figure-eight knot complement, given in \reffig{VeerFigEight}.

\begin{figure}[htbp]
\includegraphics[width = 0.8\textwidth]{Figures/fig_8_knot_green_leaf}
\caption{The decomposition of a leaf into sectors, and then into normal disks.  Here we draw the normal disks in a polygonal style, without the smoothing coming from $\calB(\alpha)$.  This leaf is carried by the upper branched surface for the veering triangulation of the figure-eight knot complement given in \reffig{VeerFigEight}.  There are two kinds of sector, here coloured light blue and light red according to the colour of the dual edge, as in \refrem{Dual}.}
\label{Fig:SectorsForFig8KnotComplement}
\end{figure}

\subsection{Branch lines}
\label{Sec:BranchLines}
The branched surface $\calB^\alpha \subset M$ cuts $M$ into a disjoint union of \emph{upper cusp neighbourhoods}.  Each of these is homeomorphic to a torus crossed with a ray.  
Suppose that $\check{c}$ is a cusp of $M$ and $\check{N}$ is the corresponding upper cusp neighbourhood.  We will abuse notation slightly and identify $\check{N}$ with its closure in the induced path metric; this makes $\check{N}$ homeomorphic to a torus cross an interval.  The inner boundary of $\check{N}$ receives a piecewise smooth structure from $\calB^\alpha$.  The locus of non-smooth points in $\bdy \check{N}$ is the union of \emph{upper branch loops} for $\check{N}$.  The name comes from the fact that the union of all branch loops for all upper cusp neighbourhoods surjects the one-skeleton of $\calB^\alpha$. 


Suppose that $\check{S}$ is an upper branch loop in $\check{N}$.  Let $c$ be a lift of $\check{c}$ to $\cover{M}$. Let $N^c$ be the corresponding lift of $\check{N}$.  Let $S \subset N^c$ be a lift of $\check{S}$.  We call $S$ an \emph{upper branch line}.  

\begin{lemma}
\label{Lem:BranchLinesAreLines}
Branch lines are in fact lines -- not intervals, rays, or loops.  
\end{lemma}

\begin{proof}
With notation as immediately above: $S$ cannot be an interval or a ray as $\check{S}$ is a loop and so has no boundary.  Consulting 
\reffig{NormalUpperBranchedSurface} we find that we may orient $\check{S}$ to everywhere agree with the co-orientation given by $\alpha$.  Thus $\check{S}$ is \emph{vertical} in sense of~\cite[Definition~2.2]{SchleimerSegerman19}.  Applying~\cite[Theorem~3.2]{SchleimerSegerman19} we find that $\check{S}$ is essential in $\pi_1(M)$.  
\end{proof}

\begin{definition}
\label{Def:Adjacent}
We say two upper branch lines $S$ and $T$ in $N^c$ are \emph{adjacent} if they are not separated by a third branch line in $N^c$.  
\end{definition}

\begin{figure}[htb]
\centering
\includegraphics[width=0.5\textwidth]{Figures/cusp_neighbourhood}
\caption{The boundary of an upper cusp neighbourhood $N^c$: the component of $\cover{M} - \cover{\calB}^\alpha$ containing the cusp $c$.  The black lines represent the branch lines in $N^c$.}
\label{Fig:CuspNeighbourhood}
\end{figure}

We now show that \reffig{CuspNeighbourhood} and~\ref{Fig:NeighbourhoodLayer} is an accurate depiction of an upper cusp neighbourhood. 


\begin{figure}[htb]
\centering
\labellist
\small\hair 2pt
\pinlabel $c$ [t] at 137 90
\endlabellist
\includegraphics[width=0.5\textwidth]{Figures/neighbourhood_layer}
\caption{A portion of an upper cusp neighbourhood $N^c$ as it intersects a layer $K$.  The track-cusps in $K$ associated to $c$ are shaded grey.}
\label{Fig:NeighbourhoodLayer}
\end{figure}

\begin{lemma}
\label{Lem:BranchLines}
Suppose that $c$ is a cusp of $\cover{M}$.  Let $\{S_i\}$ be the branch lines on the boundary of $N = N^c$.  Fix $K$ a layer of some layering $\calK$ of $\cover{M}$. 
\begin{enumerate}
\item
\label{Itm:Cyclic}
The upper branch lines in $N$, equipped with the adjacency relation, are naturally indexed by $\ZZ$.  
\item
\label{Itm:BranchLinesLayer}
For each $i$, there is a unique track-cusp $s_i$ of $S_i$ meeting $K$.
\item
\label{Itm:NeighbourhoodLayer}
The intersection $N \cap K$ is contained in the faces of $K$ incident to $c$. The track-cusp $s_i$ points away from $c$, in the face that contains $s_i$ and meets $c$. 
\item
\label{Itm:BranchStrip}
The strip in $\bdy N$ cobounded by $S_i$ and $S_{i+1}$ meets $K$ in a train interval running from $s_i$ to $s_{i+1}$.
\item
\label{Itm:CuspNeighbourhood}
The cusp neighbourhood $N$ is homeomorphic to $(N \cap K) \cross \RR$.

\end{enumerate}
\end{lemma}

\begin{proof}
In \reflem{BranchLinesAreLines} we showed that the upper branch loops give essential loops when projected to their boundary torus.  Since they are also disjoint, they are parallel, giving \refitm{Cyclic}.  See \reffig{CuspNeighbourhood}.

Let $s$ and $s'$ be adjacent track-cusps of $S_i$, with $s'$ above $s$.  Let $f$ and $f'$ be the faces of $\cover{\calB}$ that contain $s$ and $s'$ respectively. 

Let $L$ be the highest layer of $\calK$ containing $f$; let $L'$ be the lowest layer that contains $f'$.  Thus $L$ and $L'$ are consecutive layers and there is a single tetrahedron, $t$, between them.  Consulting \reffig{NormalUpperBranchedSurface}, we see that that $f$ is a lower face of $t$ while $f'$ is an upper face.  Thus $s$ is not contained in $L'$ and $s'$ is not contained in $L$.  A similar argument, below $f$, proves that any layer containing $f$ meets $S_i$ exactly at $s$.  Thus \refitm{BranchLinesLayer} now follows by induction on the number of layers between $L$ and $K$. 

The intersection $N \cap K$ is a component of $K - \tau^K$. For any face $f$, the train track $\tau^f$ cuts $f$ into three pieces, each containing exactly one cusp of $f$ and disjoint from the opposite edge. See \reffig{NeighbourhoodLayer}.  This proves \refitm{NeighbourhoodLayer}. 

The strip in $\bdy N$ cobounded by $S_i$ and $S_{i+1}$ meets $K$ transversely within $\tau^K$. The endpoints are the switches contained in $s_i$ and $s_{i+1}$, giving \refitm{BranchStrip}.  Again, see \reffig{NeighbourhoodLayer}.

Suppose that $K'$ is the layer of $\calK$ immediately above $K$.  Move $K$ and $K'$ slightly apart, keeping both carried by the branched surface $\cover{\calB}$.  Let $t$ be the tetrahedron between $K$ and $K'$.  If $t \cap N$ is empty then $K$ and $K'$ cobound a product region $P \homeo (N \cap K) \cross [0,1]$.  The product $P$ is in turn carried by $\cover{\calB}$.  Suppose instead that $t \cap N$ is non-empty.  Away from $t$ the layers again cobound a product region $P_0$.  Let $Q_t$ be a small closed neighbourhood of $t$.  Inside of $Q_t \cap N$ the layers $K$ and $K'$ cobound a small neighbourhood $P_t$ of $t \cap N$.  Consulting \reffig{NormalUpperBranchedSurface} we see that $P_t$ is again a product region.  We form $P = P_0 \cup P_t$.  Again $P \homeo (N \cap K) \cross [0,1]$ and we have proven \refitm{CuspNeighbourhood}.  See \reffig{CuspNeighbourhood}.
\end{proof}

\begin{remark}
\reflem{BranchLines}\refitm{Cyclic} is a version of \cite[Observation~2.3]{FuterGueritaud13} in our context.  The branch loops in $M$ equip every component of $\bdy M$ with a \emph{branch slope}; these are called the \emph{ladderpole slopes} in~\cite[Observation~2.4]{FuterGueritaud13}.  
These are closely related to the \emph{sutures} of \emph{sutured manifolds}~\cite[Definition~2.6]{Gabai83}, the \emph{parabolic locus} of \emph{pared manifolds}~\cite[Definition~4.8]{Morgan84}, the \emph{degeneracy slopes} of~\cite[page~62]{GabaiOertel89}, 
and the \emph{maw loops} of~\cite[page~27]{Mosher96}.
\end{remark}

\begin{definition}
Let $K$ be a layer of a layering. A train ray in $\tau^K$ travels through the triangles of $K$, entering through one edge of each triangle and exiting through the edge either to the left or the right.  We refer to the former as a \emph{left turn} and the latter as a \emph{right turn}. 
\end{definition}

We now have the following corollaries of \reflem{BranchLines}.

\begin{corollary}
\label{Cor:CannotTurnLeftForever}
Suppose that $K$ is a layer of a layering and $\rho$ is a train ray in the upper track $\tau^K$ on $K$.  Then $\rho$ turns both left and right infinitely often.  The same is true for a train ray in the lower track $\tau_K$. 
\end{corollary}
\begin{proof}
Suppose that $\rho$ is a train ray that only turns left. Then it turns around a cusp, which cannot have track-cusps pointing away from it beyond the start of $\rho$. This contradicts \reflem{BranchLines}.
\end{proof}

\begin{corollary}
\label{Cor:BranchLinesToggle}
Suppose that $S$ is an upper branch line of $\cover{\calB}^\alpha$ in $\cover{M}$.  Then there are track-cusps of $S$ contained in the lower faces of toggle tetrahedra.  In particular $S$ meets both red and blue edges periodically. 
\end{corollary}

\begin{proof}
\reflem{BranchLines}\refitm{BranchLinesLayer} tells us that the upper branch line $S$ passes through a bi-infinite vertical sequence $\{s_i\}_{i \in \ZZ}$ of track-cusps.  
Let $f_i$ be the face containing $s_i$.  Let $t_i$ be the tetrahedron immediately above $f_i$.  Suppose, for a contradiction, that $t_i$ is always a fan tetrahedron.  It follows that all of the $t_i$ are same colour and attached to a single edge of the opposite colour.  See \reffig{GluingAutomaton}.  But this contradicts the finiteness of edge degrees. 
\end{proof}

\section{The veering circle}
\label{Sec:VeeringCircle}

In this section, from the data $(M, \calT, \alpha)$, we build the \emph{veering circle} $S^1(\alpha)$.  Here is the desired statement. 

\begin{theorem}
\label{Thm:VeeringCircle}
Suppose that $(M, \calT, \alpha)$ is a transverse veering ideally triangulated three-manifold.  Then the order completion of $(\Delta_M, \calO_\alpha)$ is a circle $S^1(\alpha)$ with the following properties.
\begin{enumerate}
\item 
\label{Itm:Acts}
The action of $\pi_1(M)$ on $\Delta_M$ extends to give a continuous, faithful, orientation-preserving action on  $S^1(\alpha)$.  
\item 
\label{Itm:Dense}
Furthermore, all orbits are dense.
\end{enumerate}
\end{theorem}

\subsection{Arcs at infinity}

We introduce several pieces of notation.  

\begin{definition}
\label{Def:Arcs}
Suppose that $(M, \calT, \alpha)$ is a transverse veering ideally triangulated three-manifold.  
Suppose that $a$ and $b$ are cusps in $\Delta_M$.  Define
\[
[a, b]_\Delta^\acw = \{ c \in \Delta_M \st \calO_\alpha(a, c, b) \geq 0 \}
\]
This is the \emph{arc} in $\Delta_M$ that is anti-clockwise of $a$ and clockwise of $b$. 

Now fix an edge $e$ of $\cover{\calT}$.  Orient $e$ by ordering the cusps $c$ and $c'$ at the ends of $e$.  This, together with the transverse taut structure $\cover{\alpha}$ determines a co-orientation of $e$ using the right-hand rule. 
We set $\Delta(e) = [c', c]_\Delta^\acw$.  Note that $e$ and $\Delta(e)$ have the same endpoints.   The co-orientation on $e$ points towards the arc $\Delta(e)$.  See \reffig{CoorientedEdge}.
\end{definition}

\begin{figure}[htbp]
\labellist
\small\hair 2pt
\pinlabel {$c$} at 13 450 
\pinlabel {$c'$} at 565 450 
\pinlabel {$e$} at 220 340 
\pinlabel {$\Delta(e)$} at 220 610
\endlabellist
\[
\begin{array}{c}
\includegraphics[height = 3.5 cm]{Figures/cooriented_edge}
\end{array}
\]
\caption{The arc $\Delta(e)$ (drawn dashed) in $\Delta_M$ corresponding to a co-oriented edge $e$.}
\label{Fig:CoorientedEdge}
\end{figure}

Let $K$ be a layer of a layering of $\cover{M}$.  Suppose that $\rho$ is a train ray in the upper track $\tau^K$.  Let $\{e_n\}_{n \in \NN}$ be the sequence of co-oriented edges crossed by $\rho$; here the co-orientation of $e_n$ agrees with the orientation of $\rho$.  Note that the arc $\Delta(e_{n+1})$ is contained in, and shares exactly one endpoint with, the arc $\Delta(e_n)$.  

\begin{lemma}
\label{Lem:RoutesShrink}
Let $\{e_n\}$ be the edges crossed by the train ray $\rho$.  Then the nested intersection $\cap \Delta(e_n)$ is empty. 
\end{lemma}

\begin{proof}
The edges in $K$ make it into a copy of the Farey tessellation (\reflem{Disk}).  The dual graph is an infinite trivalent tree $T$.  Suppose that $c$ is a cusp of $\cover{M}$.  Let $H$ be the line in $T$ adjacent to $c$.  Note that $\rho$ gives a path in $T$.  By \refcor{CannotTurnLeftForever}, $\rho$ turns left and right infinitely often.  Thus no sub-ray of $\rho$ is contained in $H$.  


Among the edges $\{e_n\}$ of $K$ crossed by $\rho$, let $e_m$ be the last one which is as close as possible to $H$.  Thus, either $e_m$ is the last edge (that $\rho$ crosses) that meets $c$, or $e_m$ is the second edge (that $\rho$ crosses) of a Farey triangle, the third edge of which separates $\rho$ from $c$.  In either case, $c$ is not in the arc $\Delta(e_{m+1})$.  
\end{proof}

Suppose that $S$ is an upper branch line. Let $\{s_n\}_{n \in \ZZ}$ be the track-cusps of $S$, where $s_{n+1}$ is the track-cusp immediately above $s_n$. Let $e_n$ be the edge meeting $s_n$. We co-orient $e_n$ away from $s_n$. Note that  $\Delta(e_{n+1})$ is contained in, and shares exactly one endpoint with, $\Delta(e_n)$.  See \reffig{UpperGluingAutomaton} for examples.  

\begin{lemma}
\label{Lem:BranchLinesShrink}
Let $\{e_n\}$ be the edges crossed by the branch line $S$.  Then the nested intersection $\cap \Delta(e_n)$ is empty. 
\end{lemma}

The proof of \reflem{BranchLinesShrink} is more difficult than that of \reflem{RoutesShrink}.  This is because a branch line, unlike a train ray, is never contained in a single layer of a layering, by \reflem{BranchLines}\refitm{BranchLinesLayer}.

\begin{proof}[Proof of \reflem{BranchLinesShrink}]
Fix a cusp $c$ of $\cover{M}$.  By \refprop{VeerImpliesExhaust} there is a continent $C$ which meets both $S$ and $c$.  By \reflem{FiniteInFill} we can assume that $C$ is convex.  We define a sequence of convex continents $\{C_n\}$ as follows. 

Let $C_0 = C$.  Let $L_n$ be the upper landscape of $C_n$, let $s_n$ be the track-cusp of $L_n$ meeting $S$, and let $f_n \subset L_n$ be the face containing $s_n$.  Note that the edge of $f_n$ meeting $s_n$ is one of the edges associated to $S$.  Choosing indices correctly, we arrange that $e_n$ (the edge crossed by $S$, defined above) lies in $f_n$.  Note that $e_n$ is co-oriented away from $f_n$.  

Let $\tau^n$ be the upper track for $L_n$ and let $R_n \subset L_n$ be the maximal river with $f_n$ as its source.  Since $C_n$ is convex, the mouth, $e'_n$ say, of $R_n$ is coastal.  We co-orient $e'_n$ away from $C_n$.   Let $c_n$ and $d_n$ be the cusps at the endpoints of $e'_n$ meeting the left and right riverbanks of $R_n$, respectively.  To obtain $C_{n+1}$ we channelise repeatedly until the face $f_n$ is covered; by \reflem{RiversSimplify} this process terminates.

The proof of the following claim is omitted; it is similar to the analysis of forked rivers in \reflem{RiversSimplify}.  See \reffig{River}.  


\begin{claim*}
For all $n$, the interval $\Delta(e'_{n+1})$ is strictly contained in $\Delta(e'_n)$.  For all $n > 0$, the interval $\Delta(e_{n})$ shares an endpoint with $\Delta(e'_n)$.  This shared endpoint is $c_n$ or $d_n$ as the edge $e_n$ is blue or red. \qed
\end{claim*}

By \refcor{BranchLinesToggle}, both colours appear amongst the edges $\{e_n\}$. Thus, repeated channelisation eventually removes the original cusp $c$ from $\cap \Delta(e_n)$.  
\end{proof} 

\subsection{Completing the cusps}
\label{Sec:BuildingVeeringCircle}
Again we refer to~\cite[Section~2]{Thurston98}, \cite[Section~2.6]{Calegari07}, and~\cite[Chapter~3]{Frankel13} as references for circular orders. 

Suppose that $\Delta$ is a countable set and $\calO$ is a dense circular order in the sense of \refdef{Dense}.  The \emph{order completion} of the pair $(\Delta, \calO)$ is homeomorphic to $S^1$.  The construction is essentially identical that of $\RR$ from the pair $(\QQ, <)$.  See~\cite[Proposition 2.1.7]{Thurston98} or~\cite[Theorem~2.47]{Calegari07}.  

\begin{definition}
\label{Def:VeeringCircle}
Suppose that $(M, \calT, \alpha)$ is a transverse veering ideally triangulated three-manifold.  Note that $\Delta_M$, the set of cusps of $\cover{M}$, is countable.  Let $\calO_\alpha$ be the dense circular order given by \refthm{VeerImpliesUnique}.  Thus the order completion of $(\Delta_M, \calO_\alpha)$ is a circle $S^1(\alpha)$; we call this the \emph{veering circle}.  
\end{definition}

For $x, y \in S^1(\alpha)$ we define $[x, y]^\acw$ to be the arc of $S^1(\alpha)$ anti-clockwise of $x$ and clockwise of $y$.  Note that if $x$ and $y$ are cusps then this arc is the closure of $[x, y]_\Delta^\acw$.  Similarly, we define $A(e) \subset S^1(\alpha)$ to be the closure of $\Delta(e)$.  

We have the following corollary of Lemmas~\ref{Lem:RoutesShrink} and~\ref{Lem:BranchLinesShrink}.

\begin{corollary}
\label{Cor:Irrational}
Suppose that $\{e_n\}$ is the sequence of edges of $\cover{\calT}$ crossed by either an (upper or lower) train ray $\rho$ or an (upper or lower) branch line $S$.  Then the intersection $\cap A(e_n)$ is a single non-cusp point of $S^1(\alpha)$. \qed
\end{corollary}

We call this point the \emph{endpoint at infinity} and it is denoted by $\bdy \rho$ or $\bdy S$, respectively.  
An oriented train line $\sigma$ has two endpoints at infinity, denoted $\bdy_\pm \sigma$.

\begin{remark}
\label{Rem:NeighbourhoodBasis}
Suppose that $K$ is a layer of a layering of $\cover{M}$.  Applying \reflem{Disk}, the triangulation of $K$ is a copy of the Farey tessellation.  Fix $x \in S^1(\alpha)$.  
\begin{itemize}
\item
If $x$ is not a cusp then the arcs $\{ A(e) \}$, for $e$ an edge of $K$ and co-oriented towards $x$, form a closed neighbourhood basis for $x$.  
\item
If $x$ is a cusp then the unions $\{ A(e) \cup A(e') \}$, for edges $e$ and $e'$ of $K$ both meeting $x$ and co-oriented away from each other, form a closed neighbourhood basis for $x$. \qedhere
\end{itemize}
\end{remark}


\begin{proof}[Proof of \refthm{VeeringCircle}\refitm{Acts}]
Fix $\delta \in \pi_1(M)$.  Suppose that $a, b \in \Delta_M$ are distinct cusps.  Since $\delta$ preserves the circular order on $\Delta_M$ (by \refthm{VeerImpliesUnique}) the image of $[a, b]^\acw \cap \Delta_M$ is $[\delta a, \delta b]^\acw \cap \Delta_M$.  This implies that $\delta$ is well-defined on the non-cusp points of $S^1(\alpha)$. 
and thus sends arcs to arcs.  Thus $\delta$ is continuous 
and orientation preserving, and so is $\delta^{-1}$.  Thus $\delta$ is a homeomorphism.  Finally, the action is faithful because $\Delta_M$ embeds into $S^1(\alpha)$.
\end{proof}

\subsection{Parabolics}
\label{Sec:Parabolics}

For the proof of \refthm{VeeringCircle}\refitm{Dense} we require one further tool.  Suppose that $\check{c}$ is a cusp of $M$.  Let $\check{N} \subset M$ be the upper cusp neighbourhood of $\check{c}$.  Fix a basepoint $\check{p} \in \check{N}$.  We lift to obtain a cusp $c \in \Delta_M$ and a basepoint $p \in N = N^c$.  This gives an isomorphism between $\pi_1(M)$ and the deck group of $\cover{M}$, and thus an isomorphism between $\Stab(c)$ and $\pi_1(\check{N})$.  We suppress this from our notation.   

\begin{definition}
\label{Def:BranchSlope}
Let $\beta \in \Stab(c)$ be the based homotopy class whose free homotopy class contains any (thus all) oriented branch loops $\check{S}$ in $\bdy \check{N}$.  In an abuse of notation we again call $\beta$ the \emph{branch slope}.
\end{definition}

Suppose that $K$ is a layer of a layering of $\cover{M}$.  By \reflem{BranchLines}\refitm{Cyclic} we may index the branch lines of $N = N^c$, say $\{S_i\}_{i \in \ZZ}$, so that $S_i$ and $S_{i+1}$ are adjacent.  By \reflem{BranchLines}\refitm{BranchLinesLayer} there is a unique track-cusp $s_i$ of $S_i$ contained in $K$.  We choose the indexing so that the $s_i$ march anti-clockwise around the disk $N \cap K$ as $i$ increases. By \reflem{BranchLines}\refitm{NeighbourhoodLayer}, each of the $s_i$ meets an edge $e_i$ which we co-orient away from $c$. See \reffig{NeighbourhoodLayer}.  By \refcor{Irrational} we deduce that $\bdy S_i \in A(e_i)$.   From \refrem{NeighbourhoodBasis} and \reflem{BranchLines}\refitm{BranchStrip} we deduce that the points $\bdy S_i$ tend to $c$ as $i$ tends to infinity.  See \reffig{TipsOfCrown}. We summarise this discussion as follows. 

\begin{figure}[htbp]
\labellist
\small\hair 2pt
\pinlabel {$c$} [t] at 205 0 
\pinlabel {$\bdy S_{-2}$} [tl] at 315 34
\pinlabel {$\bdy S_{-1}$} [l] at 405 162
\pinlabel {$s_{-1}$} [r] at 312 150
\pinlabel {$\bdy S_0$} [br] at 130 395
\pinlabel {$s_0$} [t] at 178 218  
\pinlabel {$\bdy S_1$} [r] at 5 172
\pinlabel {$s_1$} [l] at 98 150
\pinlabel {$\bdy S_2$} [tr] at 94 37
\pinlabel {$\bdy S_3$} [t] at 115 19
\endlabellist
\includegraphics[width = 0.6\textwidth]{Figures/tips_of_crown}
\caption{The points $\{\bdy S_i\}$ converge to $c$ as $i$ tends to infinity. Compare with \reffig{NeighbourhoodLayer}.}
\label{Fig:TipsOfCrown}
\end{figure}

\begin{lemma}
\label{Lem:Parabolics}
Suppose that $c \in \Delta_M$ is a cusp and $\{S_i\}$ are its upper branch lines.  Then we have the following. 
\begin{enumerate}
\item
\label{Itm:TipsOfCrown}
$S^1(\alpha) - \{ c \} = \bigcup_i \, [\bdy S_i, \bdy S_{i+1}]^\acw$  
\item
\label{Itm:BranchSlope}
The branch slope $\beta \in \Stab(c)$ fixes $\bdy S_i$ for all $i$.  
\item
\label{Itm:NonBranchSlope}
Suppose that $\gamma \in \Stab(c)$ is not a power of the branch slope $\beta$.  Then there is a non-zero integer $k$ so that $\gamma(\bdy S_i) = \bdy S_{i + k}$ for all $i$.  Also, if $x$ is any point of $S^1(\alpha)$ then $\gamma^n(x)$ converges to $c$ as $n$ tends to infinity. \qed
\end{enumerate}
\end{lemma}

\begin{proof}[Proof of \refthm{VeeringCircle}\refitm{Dense}]
Fix any point $x \in S^1(\alpha)$.  Also, fix a cusp $c \in \Delta_M$.  Suppose that $\gamma \in \Stab(c)$ is not a power of the branch slope.  By \reflem{Parabolics}\refitm{NonBranchSlope} the sequence $\gamma^n(x)$  converges to $c$.  Thus the orbit of $x$ accumulates on $\Delta_M$; the latter is dense in $S^1(\alpha)$ by construction.  
\end{proof}

\section{Laminations}
\label{Sec:LaminationsAlone}

In this section we build the \emph{upper} and \emph{lower laminations} $\Lambda^\alpha$ and $\Lambda_\alpha$ within $S^1(\alpha)$.  In fact, we will only do this for the upper lamination; the arguments go through without change for the lower.

\begin{theorem}
\label{Thm:Laminations}
Suppose that $(M, \calT, \alpha)$ is a transverse veering ideally triangulated three-manifold.  
Then there is a lamination $\Lambda^\alpha$ in $S^1(\alpha)$ with the following properties.
\begin{enumerate}
\item 
\label{Itm:LaminationsInS1}
The upper lamination  $\Lambda^\alpha$ is $\pi_1(M)$--invariant.
\item 
\label{Itm:LaminationsInM}
The upper lamination $\Lambda^\alpha$ suspends to give a $\pi_1(M)$--invariant lamination $\cover{\Sigma}^\alpha$ in $\cover{M}$; this descends to $M$ to give a lamination $\Sigma^\alpha$ which 
\begin{enumerate}
\item
\label{Itm:Carried}
is carried by the upper branched surface $\calB^\alpha$, 
\item
\label{Itm:Types}
has only plane, annulus and M\"obius band leaves, and
\item
\label{Itm:Essential}
is essential.
\end{enumerate}
\item 
\label{Itm:LaminationsUnique}
Suppose that $\Sigma$ is a lamination carried by $\calB^\alpha$.  Then after collapsing parallel leaves, $\Sigma$ is tie-isotopic to $\Sigma^\alpha$.
\end{enumerate}
There is also a lamination $\Lambda_\alpha$ with the same properties with respect to $\calB_\alpha$. 
\end{theorem}

\noindent
We prove this in parts, first giving the necessary definitions for each. 

\subsection{Laminations in the circle}
Following Thurston~\cite[page~187]{thurston_notes}, we define $\calM = \calM(S^1)$ to be the \emph{M\"obius band past infinity}; this is equal to $S^1 \cross S^1$, minus the diagonal and quotiented by the symmetry that interchanges the two factors.  
A \emph{leaf at infinity} $\lambda = \lambda(a, b)$ is the point of $\calM$ with \emph{endpoints} $a$ and $b$.  Suppose that $\mu = \lambda(c, d)$ is another leaf.  We say $\lambda$ and $\mu$ are \emph{asymptotic} if they share an endpoint.  

We say that $\lambda$ and $\mu$ are \emph{linked} if all of $a$, $b$, $c$, and $d$ are distinct and if $c$ and $d$ are contained in different components of $S^1 - \{a, b\}$.  In all other cases we say that $\lambda$ and $\mu$ are \emph{unlinked}.

\begin{definition}
\label{Def:LaminationInS1}
A \emph{lamination at infinity} $\Lambda$ in $S^1$ is a closed subset of $\calM$ for which no pair of leaves $\lambda, \mu \in \Lambda$ are linked.
\end{definition}

\noindent 
To explain the terminology, recall that $S^1$ is the Gromov boundary of the hyperbolic plane $\HH^2$ and that $\calM$ parametrises the space of unoriented bi-infinite geodesics in $\HH^2$. 


\subsection{From cusps to train lines}

Fix $c$ a cusp, set $N = N^c$ and let $S$ be an upper branch line of $N$.  Suppose that $s$ is a track-cusp of $S$; suppose that $f$ is the face of $\cover{\calT}$ containing $s$.  Note that $c$ is a vertex of $f$.  We define a \emph{connecting arc} $\ell^f(c, s) \subset f$ to be 
\begin{itemize}
\item 
a smooth arc in $f$, 
\item 
connecting $c$ to the switch in $s$, and 
\item 
meeting $\tau^f$ only at that switch.  
\end{itemize}
We take $\ell^f(s,c) = \ell^f(c,s)$.  If $K$ is a layer containing $f$, then we set $\ell^K(c,s) = \ell^f(c,s)$.  See \reffig{ConnectingArc}.

\begin{figure}[htbp]
\labellist
\small\hair 2pt
\pinlabel {$c$} at -5 0 
\pinlabel {$s$} at 73 37
\endlabellist
\includegraphics[height = 3.5 cm]{Figures/connecting_arc}
\caption{The connecting arc (drawn dotted) from the cusp $c$ to the track-cusp $s$.}
\label{Fig:ConnectingArc}
\end{figure}

\begin{definition}
\label{Def:ConnectingLine}
Suppose that $K$ is a layer of a layering.  
\reflem{BranchLines}\refitm{BranchLinesLayer} implies that the upper branch line $S$ meets $K$ in a track-cusp, say $s$.  We define $\ell^K(c, S) \subset K$ and $\ell^K(s, S) \subset K$, a \emph{cusp line} and its \emph{cusp train ray}, as follows.
\begin{itemize}
\item $\ell^K(c, S)$ is a smooth line properly embedded in $K$, containing $s$,
\item $\ell^K(c, S) = \ell^K(c, s) \cup \ell^K(s, S)$,
\item $\ell^K(c, s) \cap \ell^K(s, S)$ is the switch contained in $s$,
\item $\ell^K(s, S)$ is a train ray carried by $\tau^K$, and
\item $\bdy\ell^K(s, S) = \bdy S$.  \qedhere
\end{itemize}
\end{definition}

\begin{lemma}
\label{Lem:ConnectingLines}
Suppose that $K$ is a layer of a layering $\calK$.  Suppose that $S$ is an upper branch line in the boundary of $N^c$.  Then a cusp line $\ell^K(c, S)$ exists and is unique (up to isotopy of the connecting arc). 
\end{lemma}

\begin{proof}
Choose indices in $\calK = \{K_i\}$ so that $K_0 = K$.  We define $\tau^i$ to be the upper track for $K_i$.  Also, applying \reflem{BranchLines}\refitm{BranchLinesLayer}, we choose the indexing of the track-cusps of $S$ so that $s_i$ is the track-cusp of $S$ in $K_i$.  Note that here we allow $s_{i+1} = s_i$.  

Let $\Phi^i \from \tau^{i+1} \to \tau^i$ be the \emph{carrying map}; it is the identity everywhere except at one watershed which is folded by $\Phi^i$ to produce a sink.  See \reffig{UpperBranchedSurface}.  Note that $\Phi^i$ sends switches to switches. However, if $s_{i+1} \neq s_i$ then $\Phi^i(s_{i+1}) \neq s_i$. Again see \reffig{UpperBranchedSurface}.
 We extend $\Phi^i$ to obtain a map from $K_{i+1}$ to $K_i$.  It is a homeomorphism on all components of $K_{i+1} - \tau^{i+1}$ except for two.  In those two it crushes two track-cusps which are then carried by $\tau^i$. 

Fix $i < j$ and define $\Phi^{i,j} \from \tau^j \to \tau^i$ by composing carrying maps.  When $i = j$ we take $\Phi^{i,i}$ to be the identity map. 

For each $i$, fix a connecting arc $\ell^i = \ell^i(c, s_i) \subset K_i$.  We arrange matters so that $\Phi^{i, i+1}(\ell^{i+1})$ contains $\ell^i$.  Set $\ell^{i,j} = \Phi^{i,j}(\ell^j)$.  Thus, for any $i \leq j \leq k$ we have $\ell^{i,j} \subset \ell^{i,k}$.
Furthermore, the containment is proper whenever $s_k \neq s_j$.

We now set
\[
\ell^i(c, S) = \cup_{i \leq j} \ell^{i,j} \quad \text{and} \quad \ell^i(s_i,S) = \ell^i(c, S) - \ell^i(c, s_i)
\]
We now verify that these are a cusp line and its cusp train ray.  First, $\ell^i(c, S)$ is carried by $\tau^i$ so it is a one-manifold. Second, it contains $c$. Third, it is a properly embedded line in $K_i$ because it is an ascending union that does not stabilise. Fourth, it contains $\ell^i(c, s_i)$.  Also $\Phi^{i,j}(\ell^j(c, S)) = \ell^i(c, S)$.  Thus $\ell^0(c, S)$ and $\ell^i(c, S)$ are identical in $\cover{M}$ away from at most $i$ tetrahedra (those between $K_0$ and $K_i$).  We deduce that $\bdy \ell^0(c, S) = \bdy \ell^i(c, S)$ for all $i$. 


Take $e_i$ to be the edge pointed at by the track-cusp $s_i$. Recall that by \refcor{Irrational}, the unique point of $\cap A(e_i)$ is $\bdy S$. Since the cusp train ray $\ell^i(s_i, S)$ crosses $e_i$, we deduce that $\bdy \ell^i(s_i, S)$ lies in $A(e_i)$.  Thus $\bdy \ell^0(s_0, S) = \bdy S$.  We now set $\ell^K(c, S) = \ell^0(c, S)$ and the proof is complete. 
\end{proof}

With notation as in the previous lemma, we define $\lambda(c, \bdy S)$ to be an \emph{(upper) cusp leaf} associated to $S$.  We abbreviate the notation and usually write $\lambda(c, S)$ for $\lambda(c, \bdy S)$.



\begin{lemma}
\label{Lem:Unlinked}
Any pair of upper cusp leaves $\lambda(c, S)$ and $\lambda(d, T)$ are unlinked.  
\end{lemma}

\begin{proof}
If $c = d$ then $\lambda(c, S)$ and $\lambda(c, T)$ are asymptotic and thus unlinked.

Now assume that $c \neq d$.  Fix a layer $K$ of a layering.  Recall that $\ell^K(c,s)$ and $\ell^K(d,t)$ are the resulting connecting arcs.  These are disjoint.  
This also holds for all layers above $K$; applying the carrying maps we deduce that $\ell^K(c,S)$ does not cross $\ell^K(d,T)$. 
\end{proof}

\begin{lemma}
\label{Lem:BranchLinesNotAsymptotic}
Suppose that $S$ and $T$ are upper branch lines.  If $\bdy S = \bdy T$ then $S = T$.
\end{lemma}

\begin{proof}
Let $\{K_i\}$ be a layering of $\cover{M}$.  Let $\tau^i$ be the upper track for $K_i$.  Again, applying \reflem{BranchLines}\refitm{BranchLinesLayer}, we index the track-cusps of $S$ so that $s_i$ lies in $K_i$; we define $t_i$ similarly.  

Let $\ell^0(s_0,S)$ and $\ell^0(t_0,T)$ be the cusp train rays given by \reflem{ConnectingLines}. Since $\bdy S = \bdy T$, the cusp train rays eventually cross the same collection of edges in $K_0$.  

If neither of $\ell^0(s_0, S)$ and $\ell^0(t_0, T)$ is contained in the other, then their union is a sub-tree $Y \subset \tau^0$.  Note that $Y$ has a single trivalent vertex, smoothed according to $\tau^0$.  There is a track-cusp, say $u_0$, at this vertex. Let $U$ be the branch line running through $u_0$.  Let $a$, $b$ and $c$ be the cusps associated to $S$, $T$, and $U$.  Appealing to \reflem{Unlinked}, we find that the cusp leaf $\lambda(c, U)$ does not link either $\lambda(a, S)$ or $\lambda(b, T)$.  Thus $\bdy U = \bdy S = \bdy T$ and so $\ell^0(u_0, U) \subset \ell^0(t_0, T)$. 

Relabelling if necessary, we thus restrict to the case where $\ell^0(s_0, S) \subset \ell^0(t_0, T)$.  If they are equal, we are done.  For a contradiction, suppose that they are not equal.  Consulting \reffig{UpperGluingAutomaton}, we deduce that $\ell^i(s_i, S) \subset \ell^i(t_i, T)$ for all $i$.  Thus, the track-cusps $t_i$ follow behind the track-cusps $s_i$ forever.  Let  $\ell^i(t_i, s_i) = \ell^i(t_i, T) - \ell^i(s_i, S)$ be the train interval in $\tau^i$ connecting $t_i$ to $s_i$. 

For each train track $\tau^i$, we measure the \emph{trailing distance} $d_i$, as follows.  This is the number of track-cusps incident to $\ell^i(t_i, s_i)$ (including $s_i$ but excluding $t_i$) that point in the same direction as $s_i$.  \reffig{ChasingTrackCusps} shows an example where $d_i$ equals four. 

\begin{figure}[htbp]
\labellist
\small\hair 2pt
\pinlabel {$t_i$} at -5 30
\pinlabel {$1$} at 150 20
\pinlabel {$2$} at 53 40
\pinlabel {$3$} at 29 20
\pinlabel {$s_i$} at 184 20
\endlabellist
\includegraphics[width=0.5\textwidth]{Figures/chasing_track-cusps}
\caption{Counting the distance between track-cusps $t_i$ and $s_i$.}
\label{Fig:ChasingTrackCusps}
\end{figure}

We claim that $d_{i+1} \leq d_i$. To see this, note that as we move up through the layers, track-cusps move by splitting past each other.  Thus, track-cusps pointing in the same direction as $s_i$ may leave $\ell^i(t_i, s_i)$ but no such track-cusps may enter. 

Thus, above some layer, say $K_j$, the trailing distance $d_i$ becomes constant.  So for $i > j$ all of the track-cusps in $\ell^i(t_i, s_i)$ follow $s_i$ forever.  We now consider the branch line $V$ with track-cusps $v_i$ that follow immediately behind $s_i$ in $\ell^i(t_i, s_i)$; that is, at trailing distance one.

\begin{figure}[htbp]
\centering
\subfloat[A strip of triangles containing $\ell^i(v_i, s_i)$.]{
\labellist
\small\hair 2pt
\pinlabel {$v_i$} [r] at 90 70
\pinlabel {$s_i$} [r] at 612 70
\endlabellist
\includegraphics[width=0.98\textwidth]{Figures/immediate_follower_v3}
\label{Fig:ImmediateFollower}
}

\subfloat[Before landfilling a tetrahedron $t$ above $s_i$.]{
\labellist
\small\hair 2pt
\pinlabel {$s_{i}$} [br] at 55 85
\endlabellist
\includegraphics[width=0.2\textwidth]{Figures/immediate_follower_split_past_s}
\label{Fig:ImmediateFollowerSplit}
}
\qquad
\subfloat[If the upper edge of $t$ is blue.]{
\labellist
\small\hair 2pt
\pinlabel {$s_{i+1}$} [br] at 65 52
\endlabellist
\includegraphics[width=0.2\textwidth]{Figures/immediate_follower_split_past_s_opp}
\label{Fig:ImmediateFollowerSplitOpp}
}
\qquad
\subfloat[If the upper edge of $t$ is red.]{
\labellist
\small\hair 2pt
\pinlabel {$s_{i+1}$} [br] at 105 77
\endlabellist
\includegraphics[width=0.2\textwidth]{Figures/immediate_follower_split_past_s_same}
\label{Fig:ImmediateFollowerSplitSame}
}
\caption{}
\label{Fig:ImmediateFollowerChanges}
\end{figure}

Let $P_i$ be the minimal strip of triangles containing $\ell^i(v_i, s_i)$.  Breaking symmetry, we assume that $\ell^j(v_j, s_j)$ enters the face containing $s_j$ 
through a red edge.  \reffig{ImmediateFollower} shows an example of what $P_j$ could look like.   

For $i \geq j$, as we move from layer $K_i$ to $K_{i+1}$ there are three possibilities; the tetrahedron $t$ between them either:
\begin{itemize}
\item misses $P_i$, 
\item is layered on the edge that $s_i$ points at, or 
\item is layered on the edge that $v_i$ points at. 
\end{itemize}
The tetrahedron $t$ cannot meet any other edges of $P_i$ as they are not sinks. 

If we layer onto the edge meeting $s_i$, as shown in \reffig{ImmediateFollowerSplit}, then there are two possibilities, depending on the colour of the upper edge of $t$. These are shown in Figures \ref{Fig:ImmediateFollowerSplitOpp} and \ref{Fig:ImmediateFollowerSplitSame}.  Either the length of $P_{i+1}$ equals that of $P_i$ or goes up by one, with the new triangle being a majority red triangle.  In this case, $P_{i+1}$ has one more internal red edge than $P_i$. 

If we layer onto the edge meeting $v_i$, then the track-cusp $v_i$ splits past the first track-cusp pointing backwards along $P_i$.  Thus $P_{i+1}$ is one triangle shorter than $P_i$.  Note then, that there is no way for $P_{i+1}$ to have more internal blue edges than $P_i$ has.  Thus the branch line $V$ visits only a finite number of blue edges.  This contradicts \refcor{BranchLinesToggle}.
\end{proof}

\subsection{Building the laminations}

Suppose that $S$ and $T$ are adjacent upper branch lines lying in $N = N^c$.  Fix a layer $K$ of a layering; let $s$ and $t$ be the upper track-cusps in $K$ given by $S$ and $T$ respectively (\reflem{BranchLines}\refitm{BranchLinesLayer}).  Since $S$ and $T$ are adjacent, there is a train interval $\ell^K(s, t) \subset \tau^K \cap \bdy N$ connecting $s$ to $t$ (\reflem{BranchLines}\refitm{BranchStrip}).  See \reffig{NeighbourhoodLayer}.  We define the \emph{boundary train line} in $K$ between $S$ and $T$ to be 
\[
\ell^K(S, T) = \ell^K(S, s) \cup \ell^K(s, t) \cup \ell^K(t, T)
\]
We define the \emph{(upper) boundary leaf} to be $\lambda(S, T) = \lambda(\bdy S, \bdy T)$. Suppose now that $R$ is the other branch line in $N$ adjacent to $S$. Then we say that $\lambda(R,S)$ is \emph{adjacent} to $\lambda(S, T)$.

\begin{definition}
\label{Def:UpperLamination}
We define $\Lambda^\alpha$, the \emph{upper lamination} for $(M, \calT, \alpha)$, to be the closure, taken in the M\"obius band $\calM$, of the union of upper boundary leaves:
\[
\Lambda^\alpha = \closure{\cup \lambda(S, T)}
\]
Here the pairs $(S, T)$ range over all adjacent upper branch lines. We give $\Lambda^\alpha \subset \calM$ the subspace topology. Non-boundary leaves are called \emph{interior leaves}.  Similarly, we define $\Lambda_\alpha$ using the lower boundary leaves. 
\end{definition}



\begin{remark}
We justify the names \emph{boundary} and \emph{interior} for leaves of $\Lambda^\alpha$ in \reflem{Approach}.
\end{remark}

\begin{definition}
Any leaf $\lambda \in \Lambda^\alpha$ separates $S^1(\alpha)$ into two components, which we call the two \emph{sides} of $\lambda$. 
\end{definition}

Here we gather together the basic properties of $\Lambda^\alpha$.

\begin{lemma}
\label{Lem:Laminations}
The upper lamination $\Lambda^\alpha$ has the following properties. 
\begin{enumerate}
\item
\label{Itm:NoLinking}
It is a lamination.
\item
\label{Itm:PiOneInvariant}
It is $\pi_1(M)$--invariant.  
\item 
\label{Itm:LeavesAreCarried}
For any leaf $\lambda \in \Lambda^\alpha$ and for any layer $K$ of any layering, there is a train line $\ell$ carried by $\tau^K$ with the same endpoints as $\lambda$.  
\item
\label{Itm:Irrational}
No endpoint of any leaf is a cusp.
\item
\label{Itm:Asymptotic}
A pair of distinct leaves $\lambda, \lambda' \in \Lambda^\alpha$ share an endpoint if and only if they are adjacent boundary leaves.
\end{enumerate}
Replacing upper by lower, we obtain the same properties for $\Lambda_\alpha$.
\end{lemma}

\begin{proof}
\noindent
\begin{enumerate}
\item 
Being linked is an open property of pairs of leaves in $\calM$, so it suffices to prove that any pair of upper boundary leaves $\lambda(S, T)$ and $\lambda(U, V)$ are unlinked.  We fix a layer $K$ and argue that the associated boundary train lines $\ell^K(S, T)$ and $\ell^K(U, V)$ are unlinked.  Suppose that $S, T \subset N^c$ and $U, V \subset N^d$.  We consider the unions $\ell^K(S, c) \cup \ell^K(c, T)$ and $\ell^K(U, d) \cup \ell^K(d, V)$.  If $c = d$ then we are done by the definition of adjacency and by \reflem{BranchLines}\refitm{Cyclic}.  If $c \neq d$ then, by \reflem{Unlinked}, the set $\{ \bdy U, d, \bdy V\}$ is contained in exactly one of the components of $S^1(\alpha) - \{ \bdy S, c, \bdy T\}$. 
\item 
Suppose that $\gamma \in \pi_1(M)$ is a deck transformation.  Suppose that $S$ and $T$ are upper branch lines.  Then $S$ and $T$ are adjacent if and only if $\gamma(S)$ and $\gamma(T)$ are adjacent.  Thus the collection of upper boundary leaves is $\pi_1(M)$--invariant and hence so is its closure. 
\item
If $\lambda = \lambda(S, T)$ is a boundary leaf, then we take $\ell = \ell^K(S, T)$ to be the associated boundary train line.  

Suppose instead that $\lambda$ is an interior leaf.  Let $x$ and $x'$ be the endpoints of $\lambda$ and orient $\lambda$ from $x$ to $x'$.  Fix a sequence $\{ \lambda_n \}$ of oriented upper boundary leaves converging to $\lambda$.  If $x'$ is a cusp then we pass to a subsequence of the $\lambda_n$ to force their positive endpoints to converge to $x'$ monotonically and from one side.  Applying \refitm{NoLinking}, we deduce that the negative endpoints of the $\lambda_n$ converge to $x$ monotonically and from the same side.  Let $\ell_n \subset \tau^K$ be the boundary train line associated to $\lambda_n$.  

We now apply \refrem{NeighbourhoodBasis}: pick nested neighbourhood bases of $x$ and $x'$ respectively.  If either point is a cusp then its bases require a pair of edges; however we will only keep those edges on the same side as the $\lambda_n$.  This gives us two collections of nested edges $\{ e_i \}$ and $\{ e'_j \}$.  We pass to subsequences repeatedly (both of the $\lambda_n$ and of the $e_i$) to arrange that $\lambda_j$ links both $e_j$ and $e'_j$.  Recall that $\tau^K$ is a tree.  Thus the train line $\ell_j$ crosses both of $e_j$ and $e'_j$ and runs along a unique train interval $L_j$ that runs from $e_j$ to $e'_j$.  Note that $L_j$ is a subinterval of all $\ell_k$ for $k \geq j$.  Also, since the two collections of edges are each nested, we have $L_j \subset L_{j+1}$.  Thus $\ell = \cup_j L_j$ is the desired train line. 

\item
This follows from property \refitm{LeavesAreCarried} and \refcor{Irrational}. 

\item
The backwards direction is immediate.  For the forward direction, suppose that $\lambda$ and $\lambda'$ are distinct leaves of $\Lambda^\alpha$ that share a single common endpoint, $x$.  Let $y$ and $y'$ be the other endpoints of $\lambda$ and $\lambda'$, respectively. 

Suppose that both $\lambda$ and $\lambda'$ are boundary leaves.  \reflem{BranchLinesNotAsymptotic} implies that there is a branch line $S$ so that $x=\bdy S$. Thus $\lambda$ and $\lambda'$ are adjacent.  

Suppose instead that $\lambda$ is an interior leaf.  Let $K$ be a layer of a layering of $\cover{M}$.  By \refitm{LeavesAreCarried}, there are train lines $\ell$ and $\ell'$ carried by $\tau^K$ with the same endpoints as $\lambda$ and $\lambda'$ respectively.  Thus the union of $\ell$ and $\ell'$ forms a sub-tree $Y \subset \tau^K$ with a single trivalent vertex, smoothed according to $\tau^K$.  See \reffig{NonAsymptotic}.

\begin{figure}[htbp]
\labellist
\small\hair 2pt
\pinlabel {$y$} at -5 50
\pinlabel {$\bdy S'$} at 15 15
\pinlabel {$t$} at 12 43
\pinlabel {$s$} at 118 43
\pinlabel {$c$} at 109 -5
\pinlabel {$y'$} at 126 -5
\pinlabel {$x$} [l] at 175 50
\endlabellist
\includegraphics[width=0.5\textwidth]{Figures/non_asymptotic}
\caption{One possible picture for an asymptotic pair of train lines.}
\label{Fig:NonAsymptotic}
\end{figure}

There is a track-cusp $s$ at the vertex of $Y$. Let $S$ be the branch line containing $s$ and let $c$ be its associated cusp.  The points $y$ and $y'$ are on opposite sides of $\ell^K(c,S)$.  Let $S'$ be the branch line adjacent to $S$ on the side containing $y$.  Then $\ell^K(S, S')$ is a boundary train line. By \refitm{NoLinking}, it does not link either $\ell$ or $\ell'$. Thus $\ell^K(S, S')$ has one endpoint at $x = \bdy S$ and the other at $\bdy S'$.  Since $\lambda$ is not a boundary leaf, we have that $\ell \neq \ell^K(S, S')$.  Thus, the $\bdy S'$ is not equal to $y$. We deduce that in $S^1(\alpha)$, the point $\bdy S'$ is separated from $x$ by $\{y,y'\}$. 

As in the above argument, the union of $\ell$ and $\ell^K(S, S')$ forms a sub-tree giving a new track-cusp, $t$. Let $T$ be the branch line passing through $t$.  Again, $x = \bdy T$.  This implies that $\bdy S = \bdy T$.  However, $s \neq t$ 
so \reflem{BranchLines}\refitm{BranchLinesLayer} implies that $S \neq T$.  This contradicts \reflem{BranchLinesNotAsymptotic}. \qedhere
\end{enumerate}
\end{proof}

\begin{proof}[Proof of \refthm{Laminations}\refitm{LaminationsInS1}]
Parts~\refitm{NoLinking} and~\refitm{PiOneInvariant} of \reflem{Laminations} give the desired statement.
\end{proof} 

\begin{lemma}
\label{Lem:Approach}
For either side of an interior leaf $\lambda \in \Lambda^\alpha$, there is a sequence of boundary leaves $\lambda_k$ that converge to $\lambda$ from that side. A boundary leaf $\lambda \in \Lambda^\alpha$ is also a limit of boundary leaves $\lambda_k$, but only on the side not containing its cusp. 
\end{lemma}

\begin{proof}
Let $K$ be a layer of a layering.  Let $\ell$ be the oriented train line carried by $\tau^K$ with the same endpoints as $\lambda$. Suppose that $s_0$ is a track-cusp on the side of $\ell$ from which we wish to approach. This exists by \refcor{CannotTurnLeftForever}.  See \reffig{Approach}.

\begin{figure}[htbp]
\labellist
\small\hair 2pt
\pinlabel {$s_2$} at 14 31
\pinlabel {$\bdy S_1$} at 16 2
\pinlabel {$\bdy S'_0$} at 47 2
\pinlabel {$s_0$} at 93 31
\pinlabel {$\bdy S_0$} at 160 2
\pinlabel {$\bdy S'_1$} at 177 2
\pinlabel {$s_1$} at 160 31
\pinlabel {$\bdy S_2$} at 226 2
\pinlabel {$s_3$} at 226 31
\endlabellist
\includegraphics[width=0.9\textwidth]{Figures/approach}
\caption{Approaching a leaf by a sequence of boundary leaves.}
\label{Fig:Approach}
\end{figure}

The track-cusp $s_0$ is part of a branch line $S_0$. 
As in the proof of \reflem{Laminations}\refitm{Asymptotic}, the track-cusp $s_0$ gives us a pair of asymptotic boundary train lines both ending at $\bdy S_0$.  Let $\ell_0$ be the one of these that separates $s_0$ from $\ell$.  Note that $\ell_0$ is not equal to $\ell$, either because $\lambda$ is not a boundary leaf or because $\lambda$ is a boundary leaf whose cusp is on the other side from $s_0$.  Since the train lines $\ell$ and $\ell_0$ both run past $s_0$, their intersection is non-empty.  By \reflem{Laminations}\refitm{Asymptotic} the intersection is a train interval.  Let $s_1$ be the track-cusp at the end $\ell \cap \ell_0$ pointing in the opposite direction to $s_0$.  

Suppose that $k>0$.  By induction, we have the following:
\begin{itemize}
\item a track-cusp $s_k$ (in a branch line $S_k$) meeting $\ell$ on the desired side and
\item a boundary train line $\ell_{k-1}$ that passes through the switch of $s_k$
\end{itemize}
such that
\begin{itemize}
\item if $k$ is even, $s_k$ points in the same direction as $s_0$; if $k$ is odd, $s_k$ points in the opposite direction,
\item the boundary train line $\ell_{k-1}$ does not separate $s_k$ from $\ell$, and
\item the boundary train line $\ell_{k-1}$ is not asymptotic to $\bdy S_k$.
\end{itemize}

For the induction step, let $\ell_k$ be the boundary train line that ends at $\bdy S_k$ and that separates $s_k$ from $\ell$. By \reflem{Laminations}\refitm{NoLinking}, the new boundary train line separates $\ell_{k-1}$ from $\ell$. Thus we have $\ell \cap \ell_{k-1} \subset \ell \cap \ell_k$, and the latter is at least one branch of $\tau^K$ longer than the former, near $s_k$. Let $s_{k+1}$ be the track-cusp at the end of the train interval $\ell \cap \ell_k$ pointing in the opposite direction to $s_k$.  
This completes the construction of $s_{k+1}$ and $\ell_k$. 

Since $s_{k+1}$ points in the opposite direction to $s_k$, the first inductive property holds. The second property holds by construction. The third property holds by \reflem{Laminations}\refitm{Asymptotic}.

Let $\lambda_k$ be the leaf with the same endpoints as $\ell_k$. Since the train intervals $\ell \cap \ell_k$ grow in both directions, and end on the midpoints of edges of the Farey triangulation of $K$, by \refrem{NeighbourhoodBasis} the leaves $\lambda_k$ converge to $\lambda$. 

We finally claim that a boundary leaf $\lambda(S, T)$ cannot be approached from the side containing its cusp $c$. To see this, note that if $\lambda'$ is another boundary leaf, then $\lambda'$ is contained in one of the three components of $S^1(\alpha) - \{\bdy S, c, \bdy T\}$.
\end{proof}



\subsection{Transverse Cantor sets}
\label{Sec:Cantor}

We now discuss the transverse structure of the upper lamination; similar properties hold for the lower.

\begin{definition}
\label{Def:LeavesThatLink}
Fix distinct cusps $c \neq d$.  Let $\Lambda^{(c, d)}$ be the leaves of $\Lambda^\alpha$ that separate $c$ from $d$.  We equip $\Lambda^{(c, d)}$ with the subspace topology.  We define a total order on $\Lambda^{(c, d)}$ by taking $\lambda <^{(c,d)} \lambda'$ if the endpoints of $\lambda$ separate $c$ from some endpoint of $\lambda'$.  See \reffig{TotalOrder}. 
Thus $\lambda'$ separates some endpoint of $\lambda$ from $d$. 
\end{definition}

\begin{figure}[htb]
\centering
\labellist
\small\hair 2pt
\pinlabel $c$ [t] at 199 0
\pinlabel $\bdy S'$ [tl] at 286 24
\pinlabel $\bdy T$ [bl] at 299 372
\pinlabel $d$ [b] at 199 400
\pinlabel $\bdy T'$ [br] at 132 384
\pinlabel $\bdy S$ [tr] at 59 61
\endlabellist
\includegraphics[width=0.6\textwidth]{Figures/transverse_cantor}
\caption{Some of the boundary leaves (solid) in $\Lambda^{(c, d)}$.  The corresponding cusp leaves are dotted.}
\label{Fig:TotalOrder}
\end{figure}

Recall that $[x, y]^\acw$ is the closed arc in $S^1(\alpha)$ anti-clockwise of $x$ and clockwise of $y$.  Let $\calC \subset [0,1]$ be the middle-thirds Cantor set equipped with its usual topology and total order. 

\begin{lemma}
\label{Lem:Cantor}
The subspace $\Lambda^{(c, d)}$ is order isomorphic (and thus homeomorphic) to the Cantor set $\calC$.  Thus interior leaves are dense in $\Lambda^\alpha$. 
\end{lemma}

\begin{proof}
By \reflem{Parabolics}\refitm{TipsOfCrown} there is exactly one boundary leaf $\lambda(S, S')$ associated to $c$ that lies in $\Lambda^{(c, d)}$.
Similarly, there is exactly one boundary leaf $\lambda(T, T')$ associated to $d$ that lies in $\Lambda^{(c, d)}$.  These then are the minimum and maximum points of the order $<^{(c, d)}$.  Breaking the symmetry of the situation we assume that the order of these points in $S^1(\alpha)$ is 
\[
c, \bdy S', \bdy T, d, \bdy T', \bdy S
\]
See \reffig{TotalOrder}.

We deduce that any leaf of $\Lambda^{(c, d)}$ has one endpoint in $[\bdy S', \bdy T]^\acw$ and the other in $[\bdy T', \bdy S]^\acw$. 
Thus $\Lambda^{(c, d)}$ is closed. Also, $\Lambda^{(c, d)}$ is contained in the quotient of $[\bdy S', \bdy T]^\acw \cross [\bdy T', \bdy S]^\acw$ lying in the M\"obius band $\calM$.  Thus $\Lambda^{(c, d)}$ is compact.  

We now recursively construct ternary codes for the leaves of $\Lambda^{(c, d)}$.  By \reflem{Approach} there is a pair of asymptotic upper boundary leaves $\lambda$ and $\lambda'$ in $\Lambda^{(c, d)}$.  Break symmetry and assume that $\lambda <^{(c,d)} \lambda'$.  Define
\[
I_0 = \{ \mu \in \Lambda^{(c, d)} \st \mu \leq^{(c,d)} \lambda \} 
\quad
\mbox{and}
\quad
I_2 = \{ \mu \in \Lambda^{(c, d)} \st \lambda' \leq^{(c,d)} \mu \} 
\]
In general, suppose $\omega$ is a finite ternary string, without ones, and $I_\omega \subset \Lambda^{(c, d)}$ has been defined and is clopen.  Apply \reflem{Approach} again to obtain a pair of asymptotic upper boundary leaves in $I_\omega$, say $\lambda_\omega <^{(c,d)} \lambda'_\omega$.  For the ternary strings $\omega 0$ and $\omega 2$ we define 
\[
I_{\omega 0} = \{ \mu \in \Lambda^{(c, d)} \st \mu \leq^{(c,d)} \lambda_\omega \} \\
\quad
\mbox{and}
\quad
I_{\omega 2} = \{ \mu \in \Lambda^{(c, d)} \st \lambda'_\omega \leq^{(c,d)} \mu \} 
\]
Each of these is closed, so each is clopen and compact.

Now, for any leaf $\lambda \in \Lambda^{(c, d)}$ we send it to the real number whose ternary expansion agrees with the subscripts of the nested sets $I_\omega$ containing it.  Note that $\lambda(S, S')$ and $\lambda(T', T)$ are sent to $\bar{0}$ and $\bar{2}$ -- zero repeating and two repeating -- respectively.  Similarly, the asymptotic pair of boundary leaves splitting $I_\omega$ receive the codes $\omega \bar{0}$ and $\omega \bar{2}$.  Conversely, for any ternary expansion $\omega$ let $\omega^n$ be the prefix of length $n$.  Then the nested intersection $\cap I_{\omega^n}$ is a singleton by \reflem{Approach}.  Thus the coding is a continuous order preserving injection, homeomorphic onto its image. 


Since interior points are dense in the Cantor set $\calC$ we deduce that interior leaves are dense in $\Lambda^{(c, d)}$.  We deduce interior leaves are dense in $\Lambda^\alpha$ from this and \reflem{Approach}. 
\end{proof}

\subsection{Suspending and descending}
\label{Sec:SuspendingDescending}
 
For every edge $e$ of $\cover\calT$ we place a copy $\calC^e$ of the Cantor set $\Lambda^e$ on $e$. We arrange matters so that if $\gamma \in \pi_1(M)$ then $\gamma(\calC^e) = \calC^{\gamma(e)}$. Suppose that $f$ is a face of $\cover\calT$ with edges $e_0, e_1, e_2$. Breaking symmetry, suppose that $\Lambda^{e_0} = \Lambda^{e_1} \cup \Lambda^{e_2}$. Therefore connect the points of 
 $\calC^{e_0}$ to the corresponding points of $\calC^{e_1}$ or $\calC^{e_2}$ using disjoint normal arcs in $f$. Again we arrange matters so that the arcs, $\calC^f$ say, are invariant under the action of $\pi_1(M)$. Finally, inside of every tetrahedron $t \in \cover\calT$, we choose a collection of normal disks $\calC^t$ spanning the normal curves on the boundary of $t$. To prove that only normal disks are needed to span the curves in $\bdy t$, we fix a layering, consider the layers $K$ and $K'$ immediately below and above $t$, and apply \reflem{Laminations}\refitm{LeavesAreCarried}. See \reffig{NormalUpperBranchedSurface}. Again we do this invariantly.  Let $\cover{\Sigma}^\alpha$ be the union $\cup_t \calC^t$.  
 
\begin{proof}[Proof of \refthm{Laminations}\refitm{LaminationsInM}]
The lamination $\cover{\Sigma}^\alpha$ constructed above consists of planes and it is $\pi_1(M)$--invariant.  Thus it descends to give a $\pi_1(M)$--injective lamination $\Sigma^\alpha$ in $M$, carried by $\calB^\alpha$.  This proves \refthm{Laminations}\refitm{Carried}.

Suppose that $\sigma$ is leaf of $\Sigma^\alpha$. By \refrem{Sector}, the leaf is a union of sectors; see \reffig{Sector}.
Thus, if $\sigma$ is closed then it has Euler characteristic zero, so it is a torus or Klein bottle. 
However, by \refrem{Sector} the leaf $\sigma$ is a normal surface containing a normal quadrilateral.  Thus by~\cite[Theorem~1.5]{HRST11} or by \cite[Theorem~1.4]{FuterGueritaud13} the leaf $\sigma$ has negative Euler characteristic.  This is a contradiction.  We deduce that $\Sigma^\alpha$ has no closed leaves.

Suppose that $\gamma$ is an essential simple closed multicurve in the leaf $\sigma$.  Isotope $\gamma$ inside of $\sigma$ to be transverse to the one-skeleton of $\sigma$.  So $\gamma$ meets every sector in a collection of  \emph{maxima}, \emph{minima}, and \emph{ascenders}.  See \reffig{SectorArcs}.  The number of these is the \emph{combinatorial length} of $\gamma$.  The \emph{depth} of $\gamma$ is the minimal possible ascending combinatorial distance from a point of $\gamma$ to any under-side vertex of any normal triangle in $\sigma$.  (Note that this is finite by \refcor{BranchLinesToggle}; see \reffig{GluingAutomaton}.)

\begin{figure}[htb]
\centering
\labellist
\small\hair 2pt
\pinlabel {maxima} [r] at 105 42
\pinlabel {ascenders} [b] at 145 246
\pinlabel {minima} [b] at 142 304

\endlabellist
\includegraphics[width = 0.8\textwidth]{Figures/sector_arcs}
\caption{The various possible subarcs of $\gamma$ meeting a fixed sector.  Ascenders separate the cusp vertices of the sector; maxima and minima do not.}
\label{Fig:SectorArcs}
\end{figure}

\begin{claim*}
The multicurve $\gamma = \{\gamma_i\}$ can be isotoped to have no minima.
\end{claim*}
We sketch a proof.
\begin{proof}[Proof of Claim]
We define a lexicographical complexity for a single $\gamma_i$ of $\gamma$ by 
\[
( \mbox{combinatorial length of $\gamma_i$, depth of $\gamma_i$} )
\]
We will perform a sequence of \emph{bigon moves} on $\gamma_0$ to reduce its complexity.  These are ambient isotopies of $\gamma_0$ in $\sigma$ moving a (perhaps long) subarc of $\gamma_0$ (including exactly one minimum) upwards. 
If there is a minimum $\beta$ of $\gamma_0$ that does not separate the upper and lower vertices of its sector, then we simply push $\beta$ across an edge of $\sigma$ and reduce the length.  After removing all such minima, if there are minima remaining then we are left with two cases: the depth either equals, or is greater than, zero.  We find that there is either a (perhaps long) bigon move that reduces the length, or depth (respectively), of $\gamma_0$.

As long as $\gamma_0$ has minima there are bigon moves reducing its complexity.  Once $\gamma_0$ has no minima we can perform bigon moves on $\gamma_1$ without changing $\gamma_0$. 
\end{proof} 

Recall that $\sigma$ is not closed.  If $\sigma$ is not a plane, annulus, or M\"obius band then we may choose a multicurve $\gamma$ which cuts out of $\sigma$ a compact subsurface $\sigma'$ of negative Euler characteristic.  Applying the claim immediately above, we may assume that $\gamma$ has no minima.  Thus each component of $\sigma'$ is a union of \emph{cusped bigons}, \emph{boundary trigons}, and \emph{rectangles} (see~\cite{SchleimerSegerman19}).  Thus $\sigma'$ has Euler characteristic zero.  This is a contradiction.
This proves \refthm{Laminations}\refitm{Types}.

Finally, \reflem{BranchLines}\refitm{CuspNeighbourhood} implies all complementary regions to $\Sigma^\alpha$ are pared torus shells.  Thus $\Sigma^\alpha$ is an \emph{essential lamination}. 
(See~\cite[Definition~6.14]{Calegari07}.)  This proves \refthm{Laminations}\refitm{Essential}.
\end{proof}

\subsection{Uniqueness}
\label{Sec:Uniqueness}

Before proving \refthm{Laminations}\refitm{LaminationsUnique} we need a few tools. To simplify the combinatorics of ascending paths, we now relax the transversality assumption. That is, suppose that $\gamma \subset \calB^\alpha$ is an oriented path. In this section we say that $\gamma$ is an \emph{ascending path} if 
\begin{itemize}
\item
$\gamma$ enters every sector through a lower edge or a lower vertex and 
\item
$\gamma$ exits every sector through an upper edge or its upper vertex.
\end{itemize}
An ascending path $\gamma$ is \emph{steeply ascending} if it exits sectors only through the upper vertex of each quadrilateral it visits. 

\begin{remark}
\label{Rem:Ascend}
Suppose that $\sigma$ is a leaf carried by the upper branched surface $\calB^\alpha$. Suppose that $\gamma$ is an ascending path that ends in $\sigma$. Then after a small isotopy, $\gamma$ is completely contained in $\sigma$.
\end{remark}

\begin{lemma}
\label{Lem:Transitive}
Suppose that $D$ and $D'$ are sectors of $\calB^\alpha$.  Then there is an ascending path $\gamma \subset \calB^\alpha$ that starts in $D$ and ends in $D'$.  
\end{lemma}

\begin{figure}[htb]
\centering
\subfloat[]{
\includegraphics[width=0.3\textwidth]{Figures/branched_surface_front}
\label{Fig:branched_surface_front}
}
\subfloat[]{
\includegraphics[width=0.3\textwidth]{Figures/branched_surface_middle}
\label{Fig:branched_surface_middle}
}
\subfloat[]{
\includegraphics[width=0.3\textwidth]{Figures/branched_surface_back}
\label{Fig:branched_surface_back}
}
\caption{A leaf carried by the upper branched surface is patterned by the branch lines of the branched surface in one of these three ways in the vicinity of a vertex of the branched surface (compare with \reffig{DualUpperBranchedSurface}).}
\label{Fig:LambdasAndX}
\end{figure}

\begin{proof}
Note that the branch locus of $\calB^\alpha$ in dual position is a four-valent graph with natural crossings at the vertices.  Thus it naturally decomposes as a union of smooth circles, which are the images of branch lines under the covering map. 


By \refrem{Dual}, there is some oriented path $\gamma$ in $\calB^\alpha$ that starts and ends in the interiors of $D$ and $D'$ respectively.  We will progressively improve $\gamma$ to make it ascending.  Homotope an initial segment of $\gamma$, staying in $\calB^\alpha$, so that it initially exits $D$ through its uppermost vertex $v$. Similarly, we may make $\gamma$ enter $D'$ through its lowest vertex $v'$.  Let $\beta$ be the subpath of $\gamma$ between $v$ and $v'$.  Since the branch locus is isotopic to the dual one-skeleton (again by \refrem{Dual}) we may homotope $\beta$, relative to its endpoints, into the branch locus.  Tighten $\beta$ so that it is locally an embedding.  Consider a maximal arc $\beta'$ of $\beta$ which is contained in a smooth circle $C$ of the branch locus, and whose orientation disagrees with that of $\alpha$ (the transverse orientation).  We call $\beta'$ a \emph{descending} arc.  We replace $\beta'$ by its (ascending) complement in $C$.  We now alternate between tightening and replacing descending arcs with ascending arcs, until all arcs are ascending.  This produces the desired ascending path from $D$ to $D'$. 
\end{proof}


\begin{definition}
\label{Def:Parallel}
Fix $M$ a three-manifold and $\calB \subset M$ a branched surface.  Let $N = N(\calB)$ be a tie-neighbourhood of $\calB$.  A \emph{tie-isotopy} is an ambient isotopy of $M$ supported in $N$ and which preserves the ties of $N$. 

Suppose that $\Sigma$ is a lamination carried by $\calB$.  We say that leaves $\sigma$ and $\sigma'$ of $\Sigma$ are \emph{parallel} if they are tie-isotopic.
\end{definition}

Note that, by construction, $\Sigma^\alpha$ has no distinct parallel leaves.


\begin{lemma}
\label{Lem:Squeeze} 
Fix laminations $\Sigma$ and $\Sigma'$ carried by $\calB^\alpha$.  Lift $\Sigma$ and $\Sigma'$ to $\cover{M}$. For any leaf $\sigma$ of $\cover{\Sigma}$ there is a leaf $\sigma'_\infty$ of $\cover{\Sigma}'$ so that $\sigma$ and $\sigma'_\infty$ are equivariantly tie-isotopic in $N(\cover{\calB}^\alpha)$.
\end{lemma}

\begin{proof}
Fix $K$ a layer of a layering.
Let $\ell = \sigma \cap K$. Fix a sector $E$ of $\cover{\calB}^\alpha$ which meets $\ell$. 
 
Let $\sigma'$ be any leaf of $\cover{\Sigma}'$. 
We have the following claim.
\begin{claim*}
For every $n > 0$ there is a deck translation $\sigma'_n$ of $\sigma'$ so that 
\begin{itemize}
\item
$\sigma'_n$ crosses $E$ and
\item
the train lines $\ell$ and $\ell'_n = \sigma'_n \cap K$ fellow travel for at least $n$ triangles of $K$, to either side of $E \cap K$. 
\end{itemize}
\end{claim*}

\begin{proof}[Proof of Claim]
Let $D'$ be any sector crossed by the image of $\sigma'$. Fix $n > 0$.  Let $\epsilon_n \subset \sigma$ be the unique (up to isotopy) steeply ascending path which begins in $E$ and which crosses $n$ sectors.  See \reffig{LambdasAndX}.  Let $E_n$ be the final sector that $\epsilon_n$ meets.  

Let $D_n \subset \calB^\alpha$ be the image of $E_n$, under the universal covering map.  Apply \reflem{Transitive} to find an ascending path $\gamma_n$ from $D_n$ to $D'$.  Lift $\gamma_n$ to a path $\eta_n \subset \cover{M}$ starting at $E_n$.  Let $E'_n$ be the final sector visited by $\eta_n$.  Thus, some translate of $\sigma'$, say $\sigma'_n$, crosses $E_n'$. By \refrem{Ascend} applied to $\epsilon_n \cup \eta_n$, the leaf $\sigma'_n$ also crosses $E_n$ and $E$. 

By \reflem{BranchLines}\refitm{BranchLinesLayer}, all branch lines meet $K$.  This includes the branch lines $\{S_i\}$ that meet $\epsilon_n$. Let $s_i$ be the track-cusp of $S_i$ contained in $K$.  Note that for each $i$ the lower component of $S_i - \epsilon_n$ fellow travels $\sigma$. Thus the lower component of $S_i - \epsilon_n$ contains $s_i$. We deduce that $\ell$ crosses the edge of $K$ meeting $s_i$. Since all of the $s_i$ are distinct, and since there are $n$ of them to each side of $E$, the claim follows.
\end{proof}

Since $\cover{\Sigma}'$ is a lamination, and since all of the $\sigma'_n$ meet $E$, we pass to a convergent subsequence of the leaves $\sigma'_n$ (in the Gromov-Hausdorff topology).  
Thus $\cover{\Sigma}'$ contains the limiting leaf $\sigma'_\infty$.  Thus $\ell'_\infty = \sigma'_\infty \cap K$ has the same endpoints in $S^1(\alpha)$ as $\ell$. This holds for all layers of the layering. We deduce that $\sigma$ and $\sigma'_\infty$ are tie-isotopic in the tie-neighbourhood $N(\cover{\calB^\alpha})$.

This tie-isotopy can be promoted to an equivariant tie-isotopy by inducting on the skeleta of $\calB^\alpha$.
\end{proof}

\begin{definition}
\label{Def:Collapse}
Fix $M$ a three-manifold and $\calB \subset M$ a branched surface.  Suppose that $\sigma$ and $\sigma'$ are parallel leaves of a lamination $\Sigma$ carried by $\calB$.  Then we may delete all leaves between $\sigma$ and $\sigma'$ and tie-isotope $\sigma'$ to $\sigma$. This tie-isotopy extends to give a new lamination $\Sigma'$. We say that $\Sigma'$ is obtained by \emph{collapsing} the given parallel leaves.
\end{definition}

\begin{proof}[Proof of \refthm{Laminations}\refitm{LaminationsUnique}] 
Suppose that $\Sigma$ is a lamination carried by $\calB^\alpha$.  From \reflem{Squeeze} we deduce that every leaf of $\Sigma$ is tie-isotopic to some leaf of $\Sigma^\alpha$.  However distinct leaves of $\Sigma$ may be parallel; if so, this correspondence will not be one-to-one. So we collapse parallels in $\Sigma$.  This done we may tie-isotope $\Sigma$ to $\Sigma^\alpha$.  This completes the proof of \refthm{Laminations}\refitm{LaminationsUnique}.  
\end{proof}

\subsection{An application}

In the context of pseudo-Anosov homeomorphisms, \refthm{Laminations}\refitm{LaminationsUnique} specialises to the following. 

\begin{corollary}
\label{Cor:Surprise}
Suppose that $S$ is a compact, connected, oriented surface.  Suppose that $f \from S \to S$ is a pseudo-Anosov homeomorphism.  Let $\calB^f$ be the stable branched surface in the mapping torus $M_f$. Let $\Sigma^f$ be the suspension of the stable lamination for $f$. Then any lamination $\Sigma$ carried by $\calB^f$ is tie-isotopic to $\Sigma^f$, after collapsing parallel leaves of $\Sigma$.  
\end{corollary}

\begin{proof}
We remove the boundary of $M_f$ and drill out the singular orbits of the suspension flow $\Phi_f$.  By \cite[Main~Construction]{Agol11} the resulting manifold $M$ admits a veering triangulation $(\calT, \alpha)$; furthermore, $\calB^f$ is isotopic to $\calB^\alpha$ and $\Sigma^f$ is isotopic to $\Sigma^\alpha$.  Now apply \refthm{Laminations}\refitm{LaminationsUnique}.
\end{proof}

\section{Interaction between the upper and lower laminations}
\label{Sec:LaminationInteractions}

With the upper and lower laminations in hand we may explore their interactions.  The main results of this section are \refthm{LinkSpace} where we construct the \emph{link space} $\calL(\alpha)$ and \refthm{MaximalRectangle} where we characterise the \emph{maximal rectangles} appearing in $\calL(\alpha)$. 

\subsection{No shared endpoints}

We begin with the following. 

\begin{lemma}
\label{Lem:NoMixedTypeAsymptotics}
No leaf of $\Lambda^\alpha$ is asymptotic to a leaf of $\Lambda_\alpha$.
\end{lemma}

\begin{proof}
For a contradiction, suppose that $\lambda = \lambda(x, y)$ and $\mu = \lambda(y, z)$ are leaves of $\Lambda^\alpha$ and $\Lambda_\alpha$ respectively, which share a common endpoint $y \in S^1(\alpha)$.  

Fix $K$ a layer of some layering.  Let $\tau^K$ and $\tau_K$ be the upper and lower train tracks in $K$.  \reflem{Laminations}\refitm{LeavesAreCarried} gives us train lines $\ell$ in $\tau^K$ and $m$ in $\tau_K$ that have the same endpoints as $\lambda$ and $\mu$, respectively.  We orient $\ell$ and $m$ towards $y$.   Since $\bdy_+ \ell = y = \bdy_+ m$, and applying \reflem{Laminations}\refitm{Irrational}, there are subrays of $\ell$ and $m$ that cross an identical collection of edges in the triangulation of $K$.  This collection of edges determines an infinite strip $P \subset K$ of triangles.  There are only two kinds of veering triangle, both shown in \reffig{VeeringTriangles}; from this we deduce that the internal edges (and the initial edge) of $P$ are all the same colour.  The boundary edges (other than the initial edge) are the opposite colour.  Breaking symmetry, we assume that the interior edges of $P$ are red.  For an example, see \reffig{NoMixedTypeAsymptotics}.  

\begin{figure}[htb]
\centering
\subfloat[A strip of triangles $P_0$ with blue boundary edges and red interior edges.  This locally contains the fellow travelling upper and lower train lines.]{
\includegraphics[width=0.8\textwidth]{Figures/perfect_fit4}
\label{Fig:NoMixedTypeAsymptotics}
}

\subfloat[Layering a fan tetrahedron on top of $P_i$.]{
\includegraphics[width=0.8\textwidth]{Figures/perfect_fit_remains4}
\label{Fig:NoMixedTypeAsymptoticsRemains}
}

\subfloat[Layering a toggle tetrahedron on top of $P_i$ sends upper (green) train routes out of the strip.]{
\includegraphics[width=0.8\textwidth]{Figures/perfect_fit_exits4}
\label{Fig:NoMixedTypeAsymptoticsExits}
}
\caption{Layering a tetrahedron on top of a strip of triangles. The altered triangles are shaded in light grey. When the upper and lower tracks coincide we draw a thicker grey arc.}
\label{Fig:NoMixedTypeAsymptoticsFigs}
\end{figure}

\begin{claim*}
Any in-fill tetrahedron $t$ attached above $P$, to an interior edge of $P$, is a fan tetrahedron.
\end{claim*}

\begin{proof}
Let $K'$ be the landscape obtained from $K$ by removing the bottom two triangles of $t$ from $K$ and replacing it with the top two triangles of $t$.  Note that $K'$ is again a layer of some layering.  We define $P' \subset K'$ similarly.  Applying \reflem{Laminations}\refitm{LeavesAreCarried}, the leaf $\lambda$ determines a train line $\ell'$ in the track $\tau^{K'}$.  

Suppose that $t$ is a toggle.  Thus $P'$ has an internal edge which is blue.  Since $\ell'$ cannot cross this edge, we have reached the desired contradiction.  See \reffig{NoMixedTypeAsymptoticsExits}.  
\end{proof}

Suppose that $t$ is a in-fill fan tetrahedron attached above $P$, to an interior edge of $P$.  The resulting strip $P'$ again has red interior edges.  Also, $\tau^{P'}$ still carries $\ell'$.  See \reffig{NoMixedTypeAsymptoticsRemains}.  


By \refcor{CannotTurnLeftForever} the rays $\ell \cap P$ and $m \cap P$ both turn right and left infinitely often.  Thus there is an upper track-cusp $s$ meeting an interior edge of $P$ so that $s$ points in the same direction as $\ell$ and $m$.  By the claim above, the branch line $S$ passing through $s$ meets only fan tetrahedra.  This contradicts \refcor{BranchLinesToggle}.
\end{proof}

\subsection{Crowns}

\begin{definition}
\label{Def:Crown}
We define $\Lambda^c \subset \Lambda^\alpha$, the \emph{upper crown} of the cusp $c$, to be the union 
\[ 
\Lambda^c = \cup_{S, T} \lambda(S, T) 
\]
where $S$ and $T$ range over all adjacent upper branch lines in the upper cusp neighbourhood $N^c$.  We define the \emph{lower crown} $\Lambda_c$ similarly.  The endpoints of the boundary leaves of a crown are called its \emph{tips}.
\end{definition}

\begin{figure}[htbp]
\labellist
\small\hair 2pt
\pinlabel {$c$} [t] at 285 0
\pinlabel {$\bdy S'$} [tl] at 475 80
\pinlabel {$\bdy T$} [l] at 565 230
\pinlabel {$d$} [b] at 285 570
\pinlabel {$\bdy T'$} [br] at 97 492
\pinlabel {$\bdy S$} [r] at 5 341
\endlabellist
\includegraphics[width=0.4\textwidth]{Figures/upper_crowns}
\caption{Two upper crowns $\Lambda^c$ and $\Lambda^d$.  The solid lines are boundary leaves and the dotted lines are cusp leaves. Compare with \reffig{TotalOrder}.}
\label{Fig:UpperCrowns}
\end{figure}

By \reflem{BranchLines}\refitm{Cyclic} the boundary leaves of an upper crown $\Lambda^c$ are adjacent in pairs and ordered by $\ZZ$.  By \reflem{Parabolics}\refitm{TipsOfCrown} the tips of $\Lambda^c$ accumulate (only) on $c$.  See \reffig{UpperCrowns}. 

\begin{definition}
\label{Def:LinkedCrowns}
Suppose that $\Lambda^c$ and $\Lambda_d$ are upper and lower crowns. 
\begin{itemize}
\item 
They are \emph{unlinked} if no leaf of $\Lambda^c$ links any leaf of $\Lambda_d$. 
\item 
They \emph{cross} if there are leaves $\lambda(R,S)$ and $\lambda(S, T)$ in $\Lambda^c$ and leaves $\lambda(U,V)$ and $\lambda(V,W)$ in $\Lambda_d$ so that these leaves of the crowns link and no others do.  See \reffig{CrossingCrowns}.
\item 
They \emph{interleave} if $c = d$ and we may index the upper and lower branch lines $S_i$ and $V_i$ such that $\lambda(S_i,S_{i+1})$ links $\lambda(V_{i-1},V_i)$ and $\lambda(V_i,V_{i+1})$.  See \reffig{InterleavingCrowns}. \qedhere
\end{itemize}
\end{definition}

\begin{figure}[htb]
\centering
\subfloat[]{
\labellist
\small\hair 2pt
\pinlabel {$\bdy U$} at 590 200
\pinlabel {$\bdy V$} at 60 500
\pinlabel {$\bdy W$} at 370 -15
\pinlabel {$d$} at 500 70
\pinlabel {$\bdy R$} at -20 200
\pinlabel {$\bdy S$} at 520 500
\pinlabel {$\bdy T$} at 200 -15
\pinlabel {$c$} at 70 70
\endlabellist
\includegraphics[width=0.43\textwidth]{Figures/crossing_crowns}
\label{Fig:CrossingCrowns}
}
\qquad
\subfloat[]{
\includegraphics[width=0.43\textwidth]{Figures/interleaving_crowns}
\label{Fig:InterleavingCrowns}
}
\caption{Crossing and interleaving crowns.  We illustrate each leaf by drawing an arc through the unit disk joining the two points of the leaf.  We indicate linking pairs of leaves with black or yellow dots as the pair comes from interleaving crowns or otherwise.}
\label{Fig:LeafIdentifications}
\end{figure}

\begin{lemma}
\label{Lem:CrownsInterleave}
The upper and lower crowns $\Lambda^c$ and $\Lambda_c$ interleave.  
\end{lemma}

The analysis here is similar to, but more delicate than that in the proof of \reflem{BranchLines}\refitm{NeighbourhoodLayer}.

\begin{proof}[Proof of~\reflem{CrownsInterleave}]
Fix $K$ a layer of a layering.  We walk in $K$ anti-clockwise about the cusp $c$.  As we do so, we pass a sequence $\{e_i\}$ of edges meeting $c$.  Let $f_i$ be the triangle meeting $e_i$ and $e_{i+1}$.  If $e_i$ is red and $e_{i+1}$ is blue then $f_i$ contains an upper track-cusp pointing away from $c$.  If the colours are interchanged then $f_i$ contains an lower track-cusp pointing away from $c$.  If $e_i$ and $e_{i+1}$ have the same colour then $f_i$ does not contain a track-cusp pointing away from $c$.  See \reffig{VeeringTriangles}.  Thus the connecting arcs emanating from $c$ alternate in the correct fashion.  
By \reflem{Disk} the layer $K$ is a copy of the Farey tessellation.  Thus we deduce that the cusp lines emanating from $c$ also alternate.  Thus the upper and lower crowns interleave.  See \reffig{InterleavingCrowns}.
\end{proof}

Before dealing with crossing crowns, we require the following.

\begin{lemma}
\label{Lem:TriangleLeafCross}
Suppose that $c$ is a cusp and $R$ and $S$ are adjacent upper branch lines in $N^c$. Suppose that $\mu$ is a lower cusp leaf, a lower boundary leaf or a lower interior leaf. Then $\mu$ links at most one of the cusp leaves $\lambda(c,R)$ and $\lambda(c,S)$.  The same statement holds swapping upper and lower. 
\end{lemma}

\begin{proof}
Suppose for contradiction that $\mu$ links both $\lambda(c,R)$ and $\lambda(c,S)$. Thus $\mu$ is neither a cusp leaf based at $c$ nor a boundary leaf of the crown $\Lambda_c$. By \reflem{CrownsInterleave}, there is a lower branch line $U$ in $N_c$ such that the cusp leaf $\lambda(c,U)$ links the boundary leaf $\lambda(R,S)$. We deduce that $\mu$ also links the cusp leaf $\lambda(c,U)$. If $\mu$ is a cusp leaf then we have a contradiction to \reflem{Unlinked}. 
Thus by \reflem{Parabolics}\refitm{TipsOfCrown} the endpoints of $\mu$ lie in distinct components of $S^1(\alpha)$ minus the tips of the crown $\Lambda_c$. Thus, $\mu$ links exactly two boundary leaves of $\Lambda_c$. This contradicts \reflem{Laminations}\refitm{NoLinking}.
\end{proof}

\begin{figure}[htb]
\centering
\subfloat[]{
\labellist
\tiny\hair 2pt
\pinlabel {$\bdy U$} at 590 200
\pinlabel {$\bdy V$} at 60 500
\pinlabel {$\bdy R$} at -20 200
\pinlabel {$\bdy S$} at 520 500
\endlabellist
\includegraphics[width=0.28\textwidth]{Figures/build_crossing_crown_1}
\label{Fig:CrownsCross1}
}
\quad
\subfloat[]{
\labellist
\tiny\hair 2pt
\pinlabel {$\bdy U$} at 590 200
\pinlabel {$\bdy V$} at 60 500
\pinlabel {$\bdy R$} at -20 200
\pinlabel {$\bdy S$} at 520 500
\pinlabel {$c$} at 70 70
\endlabellist
\includegraphics[width=0.28\textwidth]{Figures/build_crossing_crown_2}
\label{Fig:CrownsCross2}
}
\quad
\subfloat[]{
\labellist
\tiny\hair 2pt
\pinlabel {$\bdy U$} at 590 200
\pinlabel {$\bdy V$} at 60 500
\pinlabel {$\bdy R$} at -20 200
\pinlabel {$\bdy S$} at 520 500
\pinlabel {$\bdy T$} at 200 -15
\pinlabel {$c$} at 70 70
\endlabellist
\includegraphics[width=0.28\textwidth]{Figures/build_crossing_crown_3}
\label{Fig:CrownsCross3}
}

\subfloat[]{
\labellist
\tiny\hair 2pt
\pinlabel {$\bdy U$} at 590 200
\pinlabel {$\bdy V$} at 60 500
\pinlabel {$d$} at 500 70
\pinlabel {$\bdy R$} at -20 200
\pinlabel {$\bdy S$} at 520 500
\pinlabel {$\bdy T$} at 200 -15
\pinlabel {$c$} at 70 70
\endlabellist
\includegraphics[width=0.28\textwidth]{Figures/build_crossing_crown_4}
\label{Fig:CrownsCross4}
}
\quad
\subfloat[]{
\labellist
\tiny\hair 2pt
\pinlabel {$\bdy U$} at 590 200
\pinlabel {$\bdy V$} at 60 500
\pinlabel {$\bdy W$} at 370 -15
\pinlabel {$d$} at 500 70
\pinlabel {$\bdy R$} at -20 200
\pinlabel {$\bdy S$} at 520 500
\pinlabel {$\bdy T$} at 200 -15
\pinlabel {$c$} at 70 70
\endlabellist
\includegraphics[width=0.28\textwidth]{Figures/build_crossing_crown_5}
\label{Fig:CrownsCross5}
}
\caption{Crowns with distinct cusps that have linking boundary leaves must cross. }
\label{Fig:CrownsCross}
\end{figure}

\begin{lemma}
\label{Lem:CrownsCross}
Suppose that $c$ and $d$ are distinct cusps. Then $\Lambda^c$ and $\Lambda_d$ are either unlinked or cross. 
\end{lemma}

Recall that $[x, y]^\acw$ is the closed arc in $S^1(\alpha)$ between $x$ and $y$ and anti-clockwise of $x$. 

\begin{proof}[Proof of \reflem{CrownsCross}]
By \refcor{Irrational}, the cusps $c$ and $d$ are distinct from the tips of both crowns.  Also, by \reflem{NoMixedTypeAsymptotics}, the tips of $\Lambda^c$ are all distinct from the tips of $\Lambda_d$. 

Suppose that there is a boundary leaf $\lambda(R,S)$ of $\Lambda^c$ that links some boundary leaf $\lambda(U,V)$ of $\Lambda_d$.  Breaking symmetry, we assume that the points $\bdy R, \bdy U, \bdy S, \bdy V$ lie on the circle in that anti-clockwise order.  See \reffig{CrownsCross1}.  The cusp $c$ is distinct from these points.  Breaking symmetry we will assume that $c$ lies in $[\bdy R, \bdy U]^\acw$.  See \reffig{CrownsCross2}.  Let $T$ be the branch line of $N^c$ that is adjacent to $S$ and not equal to $R$.  Note that $\bdy T$ is on the opposite side of $\lambda(c, S)$ from $\bdy R$.  \reflem{TriangleLeafCross} implies that $\bdy T$ cannot lie in $[\bdy U, \bdy S]^\acw$.  So $\bdy T$ lies in $[c, \bdy U]^\acw$.  See \reffig{CrownsCross3}.

We now consider the location of the cusp $d$.  Again by \reflem{TriangleLeafCross} the cusp $d$ cannot lie in $[\bdy R, \bdy T]^\acw$.  Breaking symmetry, there are two cases.  Either $d$ lies in $[\bdy T, \bdy U]^\acw$ or $[\bdy U, \bdy S]^\acw$.

Suppose that $d$ lies in $[\bdy T, \bdy U]^\acw$.  See \reffig{CrownsCross4}.  Let $W$ be the branch line of $N_d$ that is adjacent to $V$ and not equal to $U$.  Note that $\bdy W$ is on the opposite side of $\lambda(d, V)$ from $\bdy U$.  \reflem{TriangleLeafCross} implies that $\bdy W$ cannot lie in $[\bdy V, \bdy T]^\acw$.  So $\bdy W$ lies in $[\bdy T, d]^\acw$.  See \reffig{CrownsCross5}.  Thus the crowns $\Lambda^c$ and $\Lambda_d$ cross, as desired. 

The case where $d$ instead lies in $[\bdy U, \bdy S]^\acw$ is simpler and we omit it.
\end{proof}

\subsection{The veering sphere}
\label{Sec:VeeringSphere}

We take a brief detour before constructing the link space.

\begin{definition}
\label{Def:VeeringSphere}
We build an equivalence relation on $S^1(\alpha)$ by taking the smallest closed equivalence relation that contains the leaves of both $\Lambda^\alpha$ and $\Lambda_\alpha$. The quotient of $S^1(\alpha)$ by this equivalence relation is the \emph{veering sphere} $S^2(\alpha)$.
\end{definition}

\begin{theorem}
\label{Thm:VeeringSphere}
The veering sphere $S^2(\alpha)$ is a two-sphere.  The action of $\pi_1(M)$ on $S^2(\alpha)$ is continuous, faithful, and orientation-preserving.  Furthermore, all orbits are dense.
\end{theorem}

This theorem and its proof are inspired by Theorem~5.7 of~\cite{Thurston82}. In particular see Figure~10 of that paper.

\begin{proof}[Proof of \refthm{VeeringSphere}]
We take two copies of the closed unit disk, called the \emph{upper} and \emph{lower hemispheres}.  We glue these by the identity map on their boundary to obtain a copy of $S^2$.  We identify the equator with $S^1(\alpha)$.  Using this, we embed a copy of $\Lambda^\alpha$ in the upper hemisphere, and a copy of $\Lambda_\alpha$ in the lower hemisphere.  To be precise, given a leaf $\lambda = \lambda(x,y) \in \Lambda^\alpha$, we connect $x$ and $y$ by a geodesic in the upper hemisphere model of $\HH^2$.  We do the same for the lower hemisphere.  Note that in this model of $\Lambda^\alpha$ leaves accumulate on each other if and only they do so in $S^1(\alpha)$.

Fix a cusp $c$ and let $D^c$ be the interior of the crown $\Lambda^c$; that is, $D^c$ is the complementary component of $\Lambda^c$ in the upper hemisphere whose closure meets $c$.  We define $D_c$ similarly.

We define an equivalence relation on $S^2$ by giving the equivalence classes as subsets.
\begin{itemize}
\item For each cusp $c$, the closure in $S^2$ of $D^c \cup D_c$.
\item For each interior leaf $\lambda$ in $\Lambda^\alpha$, the closure in $S^2$ of $\lambda$.
\item For each interior leaf $\mu$ in $\Lambda_\alpha$, the closure in $S^2$ of $\mu$.
\item For each point $x$ of $S^1(\alpha)$ that is not a cusp or an endpoint of a leaf of $\Lambda^\alpha$ or $\Lambda_\alpha$, the singleton $\{x\}$.
\end{itemize}

Now we check that this is indeed an equivalence relation.  No intersection of classes can lie in $S^1(\alpha)$ by \reflem{Laminations}\refitm{Asymptotic} and \reflem{NoMixedTypeAsymptotics}.  No intersection can lie in the upper or lower hemispheres by construction.  We must also show that the classes cover $S^2$.  They cover $S^1(\alpha)$ by construction.  Suppose that $D$ is a complementary component of $\Lambda^\alpha$ in the upper hemisphere.  Since cusps are dense in $S^1(\alpha)$, the boundary of $D$ contains no intervals of $S^1(\alpha)$.  Suppose that $\lambda$ is a leaf of $\Lambda^\alpha$ on the boundary of $D$.  From \reflem{Approach} we deduce that $\lambda$ is a boundary leaf and also $D = D^c$ for the cusp $c$ associated to $\lambda$. 

The intersection of this equivalence relation with $S^1(\alpha)$ gives the equivalence relation in \refdef{VeeringSphere}. Also every equivalence class meets $S^1(\alpha)$. Thus the two quotients are homeomorphic.

We now check that every class is closed, connected, and non-separating in $S^2$.  That they are closed follows by construction.  Classes of the first type are connected because $D^c$ and $D_c$ are and their closures meet at $c$.  The others are either intervals or points.  Finally points and intervals cannot separate, the latter by the Jordan curve theorem.  The closure of $D^c \cup D_c$ is the one point union (at $c$) of closed disks that do not otherwise meet, by \reflem{NoMixedTypeAsymptotics}.  
Thus by a theorem of Moore~\cite[Theorem~25]{Moore25} (see also ~\cite[Theorem~10.18]{Calegari07}) the quotient space is a two-sphere.  

The action of $\pi_1(M)$ on $S^2(\alpha)$ is continuous because the action on $S^1(\alpha)$ is.  The action is faithful because $\Delta_M$ again embeds.  Again appealing to~\cite[Theorem~10.18]{Calegari07}, the quotient map between two-spheres is approximable by homeomorphisms.  Since the action before quotienting is orientation preserving, the action after is as well. The fact that all orbits in $S^2(\alpha)$ are dense follows from \refthm{VeeringCircle}\refitm{Dense}.
\end{proof}

\begin{conjecture}
The veering sphere $S^2(\alpha)$ is equivariantly homeomorphic to $\bdy_\infty \HH^3$, where the action on the latter is given by the discrete and faithful representation of $\pi_1(M)$.
\end{conjecture}

\begin{remark}
To prove this requires a delicate investigation of the topological properties of the action, which is beyond the scope of the current paper.  Theorems~1.2 and~1.3 of \cite{Gueritaud16} are relevant, as the dependence on the fibred hypothesis can almost surely be removed from that paper.  Obtaining the conjecture would give new families of Cannon-Thurston maps~\cite{CannonThurston07} in the cusped setting and not arising from surface subgroups. 
\end{remark}

\subsection{The link space}

We will now give a direct construction of the \emph{link space} $\calL(\alpha)$.

\begin{remark}
For the experts, an alternative construction of $\calL(\alpha)$ is outlined as follows. The upper and lower laminations $\Lambda^\alpha$ and $\Lambda_\alpha$ collapse to give a pair of dendrites. The fundamental group acts on their product.  Taking the \emph{Guirardel core}~\cite{Guirardel05} and removing $\Delta_M^2$ gives the link space. 
\end{remark}

We define the \emph{pair space} to be
\[
\calP(\alpha)  = \big\{ (\lambda, \mu) \in \Lambda^\alpha \cross \Lambda_\alpha \st \mbox{$\lambda$ and $\mu$ are linked} \big\}
\]
We give $\calP(\alpha)$ a topology by realising it as a subspace of a product.  We define an equivalence relation on points of $\calP(\alpha)$ as follows.  If $\lambda$ and $\lambda'$ are asymptotic upper leaves and if $\mu$ links both then $(\lambda, \mu) \sim (\lambda', \mu)$.  We do the same for asymptotic lower leaves.  Finally we take the transitive closure.  Let $[(\lambda,\mu)]$ be the equivalence class of $(\lambda,\mu)$.

\begin{lemma}
\label{Lem:EquivalenceInL}
An equivalence class $[(\lambda,\mu)]$ in $\calP(\alpha)$ has either one, two, four or infinitely many representatives. These correspond exactly to the cases that:
\begin{itemize}
\item the leaves $\lambda$ and $\mu$ are both interior,
\item one is interior and the other is boundary,
\item the leaves are boundary about different cusps, or
\item the leaves are boundary about a single cusp.
\end{itemize}
\end{lemma}

\begin{proof}
\mbox{}
\begin{itemize}
\item Suppose that $\lambda$ and $\mu$ are interior leaves. By \reflem{Laminations}\refitm{Asymptotic}, the leaf $\lambda$ has no asymptotic partner in $\Lambda^\alpha$. The same holds for $\mu$.

\item Breaking symmetry, suppose that $\mu$ is an interior leaf and $\lambda = \lambda(R,S)$ is a boundary leaf. Let $c$ be the cusp such that $N^c$ contains $R$ and $S$.  Breaking symmetry again, the leaf $\mu$ also links $\lambda(c, S)$. Let $T$ be the other branch line adjacent to $S$.  Applying \reflem{TriangleLeafCross} we have 
\[
[(\lambda, \mu)] = \{ (\lambda(R, S), \mu), (\lambda(S, T), \mu) \}
\] 

\item Suppose that both $\lambda$ and $\mu$ are boundary leaves, adjacent to distinct cusps $c$ and $d$.  Thus by \reflem{CrownsCross}, the crowns $\Lambda^c$ and $\Lambda_d$ cross, and $[(\lambda,\mu)]$ consists of four pairs.  See \reffig{CrossingCrowns}.

\item Suppose that both $\lambda$ and $\mu$ are boundary leaves, both adjacent to the cusp $c$. Thus by \reflem{CrownsInterleave},  the crowns $\Lambda^c$ and $\Lambda_c$ interleave, and $[(\lambda,\mu)]$ consists of a countable collection of pairs. See \reffig{InterleavingCrowns}. \qedhere
\end{itemize}
\end{proof}

We call the last type of equivalence class \emph{singular}.  

\begin{definition}
\label{Def:LinkSpace}
The \emph{link space} $\calL(\alpha)$ is the quotient of $\calP(\alpha)$, minus the singular classes.  That is: 
\[
\calL(\alpha) \,\,=\,\, \calP(\alpha)/{\sim} \,\,-\,\, \{\mbox{singular classes}\}
\]
We give $\calL(\alpha)$ a topology by realising it as a subspace of a quotient.  
\end{definition}

In an abuse of notation, similar to how ideal points of ideal triangles are treated in $\HH^2$, we will allow a sequence of points in $\calL(\alpha)$ to limit to a singular class $p$.  We will refer to such a $p$ as an \emph{ideal point} of $\calL(\alpha)$. 

\begin{definition}
\label{Def:UpperFoliation}
We now define the \emph{upper foliation} $\calF^\alpha$ of $\calL(\alpha)$. It consists of two kinds of leaves.  Suppose that $\lambda \in \Lambda^\alpha$ is an interior leaf.  Then we take 
\[
\ell^\lambda = \big\{ [(\lambda, \mu)] \in \calL(\alpha) \mbox{ for some $\mu \in \Lambda_\alpha$} \big\}
\]
This is a \emph{non-singular leaf} of $\calF^\alpha$.  On the other hand, suppose that $\lambda(c,S)$ is an upper cusp leaf.  Let $\lambda$ be one of $\lambda(R,S)$ or $\lambda(S, T)$.  Then we define 
\[
\ell^S = \big\{ [(\lambda, \mu)] \in \calL(\alpha) \mbox{ for those $\mu \in \Lambda_\alpha$ that also link $\lambda(c,S)$} \big\}
\]
This is a \emph{singular leaf} of $\calF^\alpha$.  (Note that applying \reflem{TriangleLeafCross} twice shows that $\ell^S$ does not depend on whether we chose $\lambda(R,S)$ or $\lambda(S, T)$ above.)  

We define the lower foliation $\calF_\alpha$ similarly.
\end{definition}
 
\begin{theorem}
\label{Thm:LinkSpace}
Suppose that $(M, \calT, \alpha)$ is a transverse veering ideally triangulated three-manifold.  
\begin{enumerate}
\item
\label{Itm:LinkSpacePlane}
The link space $\calL(\alpha)$ is homeomorphic to $\RR^2$.  
\item
\label{Itm:LinkSpaceTransverse}
$\calF^\alpha$ and $\calF_\alpha$ are transverse foliations of $\calL(\alpha)$.  
\item
\label{Itm:LinkSpaceDense}
Non-singular leaves are dense in each of $\calF^\alpha$ and $\calF_\alpha$. 
\item
\label{Itm:LinkSpaceAction}
The natural action of $\pi_1(M)$ on $\calL(\alpha)$ is continuous, faithful, and orientation preserving.
\end{enumerate}
\end{theorem}

Before proving the theorem, we develop some structure.   

\begin{definition}
\label{Def:Rectangle}
A \emph{rectangle} $R$ in $\calL(\alpha)$ is an embedding of $(0,1)^2$ into $\calL(\alpha)$ that sends line segments parallel to the $y$--axis ($x$--axis) to arcs of the upper (lower) foliations.  
\end{definition}



\begin{definition}
\label{Def:Boundary}
Suppose that $R$ is a rectangle in $\calL(\alpha)$.  Suppose that $\{\ell_i\}$ is a monotonic sequence of leaves of $R \cap \calF^\alpha$ (or of $R \cap \calF_\alpha$), exiting $R$.  Then the set of accumulation points of $\{\ell_i\}$ is one of the four \emph{sides} of $R$.  Any point of $\bdy R$ lying in two sides is called a \emph{corner} of $R$.  
\end{definition}

Note that the sides of a rectangle $R$ lie in $\bdy R$.  Note also that a side need not be connected.  As usual we denote the closure of $R$ in $\calL(\alpha)$ by $\closure{R} = R \cup \bdy R$.  

Suppose that $R$ is a rectangle.  Suppose that $p$ is an ideal point of $\calL(\alpha)$ which is an accumulation point of $R$.  Then we will say that $p$ \emph{lies in} $\bdy R$.  Again abusing notation we will say that $p$ is either in the \emph{interior} of a side of $R$ or is an \emph{ideal corner} of $R$.  The non-ideal corners of $R$ are called \emph{material corners}. 

We now take up the task of building rectangles in $\calL(\alpha)$. 

\begin{definition}
\label{Def:EdgeRectangle}
Fix an edge $e$ of $\cover{\calT}$.  Fix $(\lambda, \mu) \in \calP(\alpha)$.  We say that a pair $(\lambda, \mu) \in \calP(\alpha)$ \emph{links} $e$ if both $\lambda$ and $\mu$ link $e$.  The \emph{edge rectangle} $R(e)$ is defined to be
\[
R(e) = \{ p \in \calL(\alpha) \st \mbox{every representative $(\lambda,\mu)$ of $p$ links $e$} \} \qedhere
\]
\end{definition}

\begin{lemma}
\label{Lem:EdgeRectangle}
Suppose that $e$ is an edge of $\cover{\calT}$ with endpoints at the cusps $c$ and $d$.
\begin{enumerate}
\item
\label{Itm:EdgeRectRect}
The edge rectangle $R(e)$ is a rectangle in the sense of \refdef{Rectangle}.
\item
\label{Itm:EdgeRectBdy}
The boundary of $R(e)$ is contained in a union of four singular leaves that alternatingly lie in $\calF^\alpha$ and $\calF_\alpha$.  Each side connects an ideal corner to a material corner of $R(e)$.  The former are exactly $c$ and $d$. 
\end{enumerate}
\end{lemma}

\begin{figure}[htb]
\centering
\labellist
\small\hair 2pt
\pinlabel {$d$} [r] at 0 21
\pinlabel {$c$} [l] at 141 56
\pinlabel {$e$} [b] at 69 40
\pinlabel {$R(e)$} at 289 36
\pinlabel {$d$} [r] at 198 8
\pinlabel {$c$} [l] at 378 69
\endlabellist
\includegraphics[width=0.6\textwidth]{Figures/link_space_edge_rectangle}
\caption{Left: a red edge $e$.  Right: the edge rectangle $R(e)$.  The dotted, shrunken rectangle is intended to remind the reader that $R(e)$ does not include its boundary.  We have coloured the material corners yellow and the ideal corners black.}
\label{Fig:LinkSpaceEdge}
\end{figure}

\begin{proof}
Breaking symmetry, we suppose that $e$ is coloured red.  We orient $e$ from $c$ to $d$.   \reflem{CrownsInterleave} tells us that the crowns $\Lambda^c$ and $\Lambda_c$ interleave, as do the crowns $\Lambda^d$ and $\Lambda_d$.  

\begin{claim*}
There are boundary leaves 
\[
\lambda(S, S') \in \Lambda^c, \quad
\lambda(U, U') \in \Lambda_c, \quad
\lambda(T, T') \in \Lambda^d, \quad
\lambda(V, V') \in \Lambda_d
\]
so that each links the next, cyclically. 
\end{claim*}

\begin{proof}
Fix a layer $K$ of a layering, chosen so that $e$ lies in $K$.   Let $P$ be the maximal strip of majority red faces in $K$ to the right of $e$. See \reffig{FiniteFellowTravel}. When $P$ is finite, we add a single majority blue face $f'$ at the end of $P$.  In this case, let $b$ be the cusp of $f'$ not meeting a red edge of $P$.

By construction, the boundary of $P$ consists of one red edge, $e$, and the rest blue.  Recall that $\tau^P$ and $\tau_P$ are the upper and lower tracks in $P \subset K$.  Consider the blue edge $e' \subset \bdy P$ adjacent to $c$.  Since $e$ and all interior edges of $P$ are red, the face of $P$ meeting $e'$ contains a lower track-cusp $u$ that points away from $c$.  Again see \reffig{FiniteFellowTravel}.  Let $U$ be the branch line containing $u$.  
Similarly, let $e'' \subset \bdy P$ be the blue edge adjacent to $d$. The face of $P$ meeting $e''$ contains an upper track-cusp $t$ that points away from $d$. Let $T$ be the branch line containing $t$. 

Partition the blue edges of $P$ into two sets: those connected either to $c$ or to $d$ by a sequence of blue edges (not passing through $b$ in the case that $P$ is finite). Call these the $c$--edges and the $d$--edges.
Due to the colouring of the edges of $P$, the cusp line $\ell_K(c, U)$ cannot exit $P$ through a $c$--edge. Similarly, $\ell^K(d, T)$ cannot exit $P$ through a $d$--edge. By \reflem{NoMixedTypeAsymptotics}, at least one must exit $P$, and hence the cusp leaves $\lambda(c, U)$ and $\lambda(d, T)$ link.

\begin{figure}[htbp]
\labellist
\small\hair 2pt
\pinlabel {$c$} [tr] at 0 0
\pinlabel {$d$} [br] at 0 120
\pinlabel {$e$} [r] at 0 30
\pinlabel {$e'$} [tl] at 30 0
\pinlabel {$e''$} [bl] at 30 122
\pinlabel {$u$} [r] at 83 53
\pinlabel {$t$} [r] at 46 70
\pinlabel {$b$} [l] at 400 61
\endlabellist
\[
\begin{array}{c}
\includegraphics[width = 0.7\textwidth]{Figures/finite_fellow_travel}
\end{array}
\]
\caption{A possible (finite) strip of faces $P$ starting at the edge $e$.}
\label{Fig:FiniteFellowTravel}
\end{figure}



We make the same argument to the left of $e$.  This produces branch lines $S$ in $N^c$ and $V$ in $N_d$ so that $\lambda(d, V)$ links $\lambda(c, S)$.  Thus the points 
\[
c, \bdy T, \bdy U, d, \bdy S, \bdy V
\]
appear in $S^1(\alpha)$ in that anti-clockwise order.  Let $T'$ be the branch line in $N^d$, adjacent to $T$, chosen so that $c$ lies in $[\bdy T', \bdy T]^\acw$.  We define $S'$, $U'$, and $V'$ similarly.  Applying \reflem{CrownsInterleave} and \reflem{CrownsCross} we deduce that the points
\[
c, \bdy S', \bdy T, \bdy U, \bdy V', d, \bdy T', \bdy S, \bdy V, \bdy U'
\]
appear in $S^1(\alpha)$ in that anti-clockwise order. See \reffig{LeafIdentificationsEdge}.
\end{proof}

\begin{figure}[htb]
\centering
\labellist
\small\hair 2pt
\pinlabel {$c$} [t] at 285 0
\pinlabel {$\bdy S'$} [tl] at 475 80
\pinlabel {$\bdy T$} [l] at 565 230
\pinlabel {$\bdy U$} [l] at 565 341
\pinlabel {$\bdy V'$} [bl] at 475 492
\pinlabel {$d$} [b] at 285 570
\pinlabel {$\bdy T'$} [br] at 97 492
\pinlabel {$\bdy S$} [r] at 5 341
\pinlabel {$\bdy V$} [r] at 5 230
\pinlabel {$\bdy U'$} [tr] at 93 80
\endlabellist
\includegraphics[width=0.6\textwidth]{Figures/leaf_identifications_edge}
\caption{The upper and lower crowns for cusps at the two ends of an edge.}
\label{Fig:LeafIdentificationsEdge}
\end{figure}

Recall that $e$ meets the cusps $c$ and $d$.  Set $\Lambda^e = \Lambda^{(c, d)}$ as in \refdef{LeavesThatLink}.  Note that the leaves of $\Lambda^e$ are exactly those of $\Lambda^\alpha$ that link $e$.   We define $\Lambda_e$ in similar fashion.  By the claim immediately above, every $\lambda$ in $\Lambda^e$ links every $\mu$ in $\Lambda_e$.  Thus $\Lambda^e \cross \Lambda_e$ is a subset of $\calP(\alpha)$.  By \reflem{Cantor} this is a product of two copies of the Cantor set $\calC$.  Mapping this product to its quotient in $\calP(\alpha)/{\sim}$ is realised by the applying the Cantor function in each coordinate.  That is, we replace ternary expansions by binary by replacing all twos by ones.  Now, since $0.\bar{1} = 1.\bar{0}$ and applying \reflem{EquivalenceInL}, pairs and four-tuples of points are identified in the desired fashion.  Thus the image in $\calP(\alpha)/{\sim}$ is homeomorphic to the closed square $[0,1]^2$.  

Note, however, that $\lambda(S, S')$ is smallest in $(\Lambda^e,<^e)$.  So, its adjacent boundary leaf in $\Lambda^c$, sharing the endpoint $\bdy S$, does not link $e$.  Thus any point $p \in \calL(\alpha)$, having a representative of the form $(\lambda(S, S'), \mu)$, is not in $R(e)$.  The same holds for $\lambda(U', U)$ as well as $\lambda(T', T)$ and $\lambda(V, V')$ in $\Lambda_e$.  This proves that $R(e)$ is homeomorphic to the open square $(0,1)^2$.  Fixing one coordinate and applying the Cantor function to the other produces the desired foliations, obtaining \refitm{EdgeRectRect}. 

We deduce that $\bdy R(e)$ lies in the union of the singular leaves $\ell^S$, $\ell_U$, $\ell^T$, and $\ell_V$.  Note that $\ell^S \cap \ell_V$ and $\ell^T \cap \ell_U$ give points of $\calL(\alpha)$ which lie in $\bdy R(e)$.  On the other hand $\ell^S$ and $\ell_U$ have a common ideal point at $c$ while $\ell^T$ and $\ell_V$ have a common ideal point at $d$.  This gives \refitm{EdgeRectBdy}. 
\end{proof}

\begin{remark}
\label{Rem:BoundaryEdgeRectangle}
We have the following useful characterisation of points of $\bdy R(e)$: these are the classes where some (but not every) representative links $e$.  
\end{remark}



\begin{lemma}
\label{Lem:EdgesCover}
Suppose that $K$ is a layer of a layering.  Then
\[
\{ \closure{R}(e) \st \mbox{$e$ is an edge of $K$} \}
\]
is a cover of $\calL(\alpha)$. 
\end{lemma}

\begin{proof}
Fix any linking pair $(\lambda, \mu) \in \calP(\alpha)$.  Let $\ell$ and $m$ be the train lines in $\tau^K$ and $\tau_K$ given by \reflem{Laminations}\refitm{LeavesAreCarried}.  These cross.  Consulting \reffig{VeeringTriangles} we deduce that $\ell$ and $m$ are tangent to each other along some train interval, perhaps of length zero.  Some edge or edges are transverse to this interval; any such edge $e$ links both $\lambda$ and $\mu$.  Thus $[(\lambda, \mu)]$ lies in $\closure{R}(e)$.  
\end{proof}

\begin{definition}
\label{Def:FaceRectangle}
Fix a face $f$ in $\cover{\calT}$.  Suppose that $e$, $e'$, and $e''$ are the edges of $f$.  We say that a pair $(\lambda, \mu) \in \calP(\alpha)$ \emph{links} $f$ if the pair links at least one of $e$, $e'$, or $e''$.  The \emph{face rectangle} $R(f)$ is defined to be
\[
R(f) = \{ p \in \calL(\alpha) \st \mbox{every representative $(\lambda,\mu)$ of $p$ links $f$} \} \qedhere
\]
\end{definition}

From the definitions we deduce that the edge rectangles $R(e)$, $R(e')$, and $R(e'')$ are contained in $R(f)$.  However, $R(f)$ is not the union of the edge rectangles.  For example, there are points in $\calL(\alpha)$ with one representative linking only $e$, and another representative linking only $e'$, say.  See the right side of \reffig{LinkSpaceFace}.

\begin{figure}[htb]
\centering
\labellist
\tiny\hair 2pt
\pinlabel {$s$} at 77.5 91
\pinlabel {$u$} at 88 80
\small\hair 2pt
\pinlabel {$c$} [l] at 140 141
\pinlabel {$c'$} [r] at 3 107
\pinlabel {$c''$} [l] at 105 8
\pinlabel {$e$} [tr] at 50 50
\pinlabel {$e'$} [l] at 120 70
\pinlabel {$e''$} [b] at 63 130
\pinlabel {$c$} [l] at 378 141
\pinlabel {$c'$} [r] at 201 82
\pinlabel {$c''$} [t] at 304 3
\pinlabel {$R(e)$} at 250 42
\pinlabel {$R(e')$} at 337 50
\pinlabel {$R(e'')$} at 260 108
\endlabellist
\includegraphics[width=0.6\textwidth]{Figures/link_space_face_rectangle}
\caption{Left: a majority red triangle with upper and lower tracks shown.
Right: the three edges $e$, $e'$, $e''$ of a face $f$ of $\cover{\calT}$ give three edge rectangles, all subrectangles of the face rectangle $R(f)$.}
\label{Fig:LinkSpaceFace}
\end{figure}

\begin{lemma}
\label{Lem:FaceRectangle}
Suppose that $f$ is a face of $\cover{\calT}$ with edges $e$, $e'$, and $e''$ opposite cusps $c$, $c'$, and $c''$ respectively.  Suppose that $e'$ and $e''$ have the same colour.  
\begin{enumerate}
\item
\label{Itm:FaceRectRect}
The face rectangle $R(f)$ is a rectangle in the sense of \refdef{Rectangle}.
\item
\label{Itm:FaceRectBdy}
The boundary of $R(f)$ is contained in a union of six singular leaves.  Two sides of $R(f)$ are segments connecting the ideal corner $c$ to distinct material corners.  The two remaining sides meet two material corners and contain $c'$ and $c''$ respectively. 
\end{enumerate}
\end{lemma}

\begin{proof}
Breaking symmetry, suppose that $e'$ and $e''$ are red and $e$ is blue.  Note that $\tau^f$ has a track-cusp $s$, say, on $e''$.  Let $S$ be the upper branch line through $s$. Also note that $\tau_f$ has a track-cusp $u$, say, on $e'$. Let $U$ be the lower branch line through $u$. 

\begin{claim*}
The intersection $\closure{R}(e) \cap \closure{R}(e')$ is equal to $\ell^S \cap \bdy R(e)$. 
\end{claim*}

\begin{proof} 
Suppose that $p$ is a point of $\calL(\alpha)$.  Thus $p$ is not a singular class.  We claim that the following are equivalent.  
\begin{enumerate}[label=(\alph*)]
\item The point $p$ lies in $\closure{R}(e) \cap \closure{R}(e')$.
\item The point $p$ has representatives $(\lambda,\mu)$ and $(\lambda',\mu')$, where $(\lambda,\mu)$ links $e$, where $(\lambda',\mu')$ links $e'$, where $\lambda$ and $\lambda'$ are distinct and asymptotic, and where $\mu$ and $\mu'$ are asymptotic.
\item The point $p$ has representatives $(\lambda,\mu)$ and $(\lambda',\mu')$, where $(\lambda,\mu)$ links $e$, where $(\lambda',\mu')$ links $e'$, where $\lambda$ and $\lambda'$ are the distinct boundary leaves sharing the endpoint $\bdy S$, and where $\mu$ and $\mu'$ are asymptotic.
\item The point $p$ lies in $\ell^S \cap \bdy R(e)$.
\end{enumerate}

Statement (a) implies the first half of statement (b) by definition.  Consulting the left-hand side of \reffig{LinkSpaceFace}, we see that $\lambda$ does not link $e'$ and that $\lambda'$ does not link $e$, so $\lambda$ and $\lambda'$ are distinct.  The various leaves are asymptotic by \reflem{EquivalenceInL}.  Statement (b) implies statement (a) by definition.

Statement (b), and \reflem{Laminations}\refitm{Asymptotic}, imply that $\lambda$ and $\lambda'$ are boundary leaves.  They share an endpoint; examining the track $\tau^f$ this endpoint must be $\bdy S$.  Thus (b) implies (c); the converse direction follows from the definition of asymptotic.

The first part of statement (c) implies that $p$ lies on $\ell^S$. Since $(\lambda,\mu)$ links $e$, but $(\lambda',\mu')$ does not, \refrem{BoundaryEdgeRectangle} tells us that the point $p$ lies in $\bdy R(e)$. This gives (d).  To see the converse, since $p$ lies in $\ell^S$, we may choose two representatives, $(\lambda,\mu)$ and $(\lambda',\mu)$, so that $\lambda$ and $\lambda'$ are distinct asymptotic boundary leaves sharing the endpoint $\bdy S$ and $\lambda$ links $e$ while $\lambda'$ links $e'$.  If $\mu$ links $e$ then it also links $e'$ and we are done. If not, then as $p$ lies in $\bdy R(e)$, we deduce that $\mu$ is asymptotic to some $\mu'$ which links $e$.  Replacing $\mu$ with $\mu'$ proves the claim.
\end{proof}

Similarly, the intersection $\closure{R}(e) \cap \closure{R}(e'')$ is equal to $\ell_U \cap \bdy R(e)$.

We adopt the following notation.  Set $R_{00} = R(e)$.  From $R(e')$ we remove the open interval $\ell_U \cap R(e')$ to obtain a pair of open rectangles $R_{10}$ and $R_{11}$.  We choose these so that $\closure{R}_{00}$ and $\closure{R}_{10}$ intersect in an arc; thus $\closure{R}_{00}\cap\closure{R}_{10}$ is exactly $\ell^S \cap \bdy R(e)$.  Appealing to \reflem{EdgeRectangle}\refitm{EdgeRectBdy} this interval is homeomorphically embedded in the boundaries of $R_{00}$ and $R_{10}$.  We cut $R(e'')$ using the interior interval $\ell^S \cap R(e'')$ into a pair of rectangles $R_{01}$ and $R'_{11}$.  Again $R_{00}$ and $R_{01}$ are nicely glued along $\ell_U \cap \bdy R(e)$.  We claim that $R'_{11} = R_{11}$.  To see this, note that if $\lambda$ and $\mu$ link each other, link $e'$, and neither link $e$, then necessarily both link $e''$. 

We deduce that the closures of the four rectangles $R_{00}$, $R_{10}$, $R_{11}$, and $R_{01}$ meet cyclically along subintervals of $\ell^S$ and $\ell_U$.  The only point contained in all four closures is their common material corner $\ell^S \cap \ell_U$.  This proves \refitm{FaceRectRect}.  Conclusion \refitm{FaceRectBdy} follows from \refitm{FaceRectRect} and \reflem{EdgeRectangle}\refitm{EdgeRectBdy}.
\end{proof}

\begin{lemma}
\label{Lem:FacesCover}
Suppose that $K$ is a layer of a layering.  Then
\[
\{ R(f) \st \mbox{$f$ is a face of $K$} \}
\]
is an open cover of $\calL(\alpha)$.  This cover has no finite subcover.
\end{lemma}

\begin{figure}[htbp]
\centering
\labellist
\small\hair 2pt
\pinlabel {(a)} at -20 742
\pinlabel {(b)}  at 372 897
\pinlabel {(c)}  at 372 515
\pinlabel {(d)} at 920 897
\pinlabel {(e)} at 700 185
\pinlabel {(f)} at -20 295
\scriptsize
\pinlabel {$e$} [b] at 85 776
\pinlabel {$e$} [b] at 455 1017
\pinlabel {$e$} [b] at 455 635
\pinlabel {$e'$} [tr] at 440 570
\pinlabel {$e''$} [l] at 505 584
\pinlabel {$d$} [r] at 390 617
\pinlabel {$c$} [bl] at 525 652
\pinlabel {$e'''$} [l] at 1140 970
\pinlabel {$e'''$} [bl] at 893 274
\pinlabel {$e'''$} [bl] at 199 355
\tiny
\pinlabel {$u$} [r] at 483 596
\pinlabel {$s$} [b] at 867 234
\pinlabel {$s$} [b] at 181 320
\endlabellist
\includegraphics[width=\textwidth]{Figures/link_space_face_rectangle_overlaps3}
\caption{The logic of the proof of \reflem{FacesCover}. Note that there are two possible configurations for the rectangles corresponding to the two triangles in case (e).} 
\label{Fig:LinkSpaceFaceFlowChart}
\end{figure}

\begin{proof}
Fix a point $p$ in $\calL(\alpha)$.  By \reflem{EdgesCover} there is some edge $e$ in $K$ so that $p$ lies in $\closure{R}(e)$.  Suppose that $p$ lies in $R(e)$.  Then let $f$ be either of the two faces in $K$ incident to $e$.  Thus $R(e) \subset R(f)$ and we are done.  Suppose instead that $p$ lies in $\bdy R(e)$.  We now show that there is some face of $K$ so that $p$ lies in its face rectangle. 

Breaking symmetry, we suppose that $e$ is red.  See \reffig{LinkSpaceFaceFlowChart}(a).  Recall that the singular classes are not in $\calL(\alpha)$.  This and \reflem{EdgeRectangle}\refitm{EdgeRectBdy} implies that $\bdy R(e)$ has two components.  The point $p$ lies in one of these; we co-orient $e$ towards this component of $\bdy R(e)$.  Let $f$ be the face of $K$ incident to $e$ and pointed at by the co-orientation of $e$. 

There are now two cases as $f$ is majority blue or majority red.   In the former, $p$ lies in $R(f)$ and we are done.  See \reffig{LinkSpaceFaceFlowChart}(b). 

Suppose instead that $f$ is majority red.  Label the edges of $f$ as $e$, $e'$, and $e''$, ordered anti-clockwise, arranging matters so that $e'$ is the blue edge.  See \reffig{LinkSpaceFaceFlowChart}(c).  (The argument for the other case is similar, reversing $\calF^\alpha$ and $\calF_\alpha$.) 

Let $d$ be the cusp that $e$ shares with $e'$.  Let $c$ be the cusp that $e$ shares with $e''$.  Let $u$ be the track-cusp of $\tau_f$.  So $u$ is contained in the cusp neighbourhood $N_d$.  Let $U$ be the lower branch line containing $u$.  Let $\ell_U \in \calF_\alpha$ be the associated singular leaf.  If $p$ lies in the interior of $\ell_U \cap \bdy R(e)$ then $p$ lies in $R(f)$ and we are done.  See \reffig{LinkSpaceFaceFlowChart}(c). 

Suppose instead that $p$ lies in $\ell^S \cap \bdy R(e)$.  Here $\ell^S$ is the upper singular leaf meeting $\bdy R(e)$ and emanating from the cusp $c$.  Thus $p$ lies in $\bdy R(f)$, and in fact lies in the interior of $\ell^S \cap \bdy R(f)$.  Suppose that $f'$ is the face that meets $f$ along $e''$.  Let $e'''$ be the edge of $f'$ that meets $c$.  There are two cases as $e'''$ is red or blue.
\begin{itemize}
\item
If $e'''$ is red then $p$ and $\ell^S$ again lie in $\bdy R(f')$.  So, in this case, we replace $f$ by $f'$ and continue rotating about $c$ in the anti-clockwise direction. See \reffig{LinkSpaceFaceFlowChart}(d). By \refcor{CannotTurnLeftForever} we do not revisit this case forever, and must eventually find that $e'''$ is blue.
\item
If $e'''$ is blue then there are two cases, as the remaining edge of $f'$ is red or blue. In either case, $p$ lies in $R(f')$ and we are done. See \reffig{LinkSpaceFaceFlowChart}(e) and (f). 
\end{itemize}
This proves the first statement.

Finally, note that any finite union of face rectangles is incident to only finitely many singular classes.  Thus the open cover by face rectangles has no finite subcover. 
\end{proof}

\begin{proof}[Proof of \refthm{LinkSpace}]
Fix $K$, a layer of a layering. We fix one edge $e_0$ of $K$; we order the faces $\{f_i\}$ of $K$ by their combinatorial distance from $e_0$ and break ties arbitrarily.  Set $\calL_k = \cup_{i = 0}^k R(f_i)$.  

We claim that $\calL_k$ is homeomorphic to a closed disk with $k + 3$ points removed from its boundary.  The base case of $\calL_0 = R(f_0)$ follows from \reflem{FaceRectangle}\refitm{FaceRectBdy}.  To pass from $\calL_k$ to $\calL_{k+1}$ we note that $\calL_k \cap R(f_{k+1})$ is an edge rectangle.  

We now take $D_k \subset \calL_k$ to be a compact disk, obtained from $\calL_k$ by removing small neighbourhoods of the $k + 3$ singular classes.  By appropriately shrinking these neighbourhoods we arrange that $D_k \subset D_{k+1}$.  By \reflem{FacesCover}, the link space $\calL(\alpha)$ is the increasing union of the closed disks $D_k$.  Thus $\calL(\alpha)$ is a non-compact connected surface without boundary so that any compact subsurface is planar.  By Ker\'ekj\'art\'o's \textit{Hauptsatz der Fl\"achentopologie f\"ur offen Fl\"achen}~\cite[page~170]{Kerekjarto23}, we deduce that $\calL(\alpha)$ is homeomorphic to the plane, as desired.  (See also \cite[Theorem~1]{Richards63}.)  This gives \refitm{LinkSpacePlane}. 



The foliations $\calF^\alpha$ and $\calF_\alpha$ are transverse inside of every face rectangle.  Since these cover $\calL(\alpha)$, we obtain \refitm{LinkSpaceTransverse}. From \reflem{Cantor}, we obtain \refitm{LinkSpaceDense}.

The action of $\pi_1(M)$ on $S^1(\alpha)$ preserves the laminations, hence the induced action preserves the foliations.  The action of $\pi_1(M)$ on $\cover{T}$ sends faces to faces so the action on $\calL(\alpha)$ sends face rectangles to face rectangles; this implies that the action is continuous.  

Note that $\pi_1(M)$ acts faithfully on the edges of $\cover{\calT}$.  So fix $\gamma \in \pi_1(M)$ as well as distinct edges $e$ and $e'$ with $\gamma(e) = e'$.  If $R(e) = R(e')$ then the cusps at the ends of $e$ and $e'$ must agree, by \reflem{EdgeRectangle}\refitm{EdgeRectBdy}.  However, this contradicts \reflem{NoParallelEdges}.  We deduce that the action on $\calL(\alpha)$ is faithful. 

The action on $\cover{T}$ preserves the red and blue edges and the transverse orientation.  This together with the fact that it preserves the foliations in $\calL(\alpha)$ proves that the action is orientation preserving.  
This completes the proof of \refitm{LinkSpaceAction}.  
\end{proof}

We end this section with a result that will be useful in our future work.  It follows from \reflem{NoMixedTypeAsymptotics}.

\begin{proposition}
\label{Prop:PerfectFits}
Two leaves $\ell \in \calF^\alpha$ and $m \in \calF_\alpha$ share an ideal corner of a rectangle $R \subset \calL(\alpha)$ if and only if there are upper and lower branch lines $S$ and $U$ for a cusp $c$ so that $\ell = \ell^S$ and $m = \ell_U$ and $\bdy S$ and $\bdy U$ are adjacent tips of the crowns $\Lambda^c$ and $\Lambda_c$.  \qed
\end{proposition}

\begin{remark}
Given a pseudo-Anosov flow (without perfect fits) on a closed three-manifold, Fenley constructs an \emph{orbit space} and its boundary at infinity~\cite[Theorem A]{Fenley12}.  He proves that their union is compact and homeomorphic to a closed disk.  The synthetic geometries of his orbit space and of our link space are very similar.
Proving that $S^1(\alpha)$ is equivariantly homeomorphic to Fenley's boundary for $\calL(\alpha)$ would imply that the natural topology $\calL(\alpha) \cup S^1(\alpha)$ makes it homeomorphic to a closed disk.
\end{remark}

\subsection{Orienting the foliations}
\label{Sec:OrientingFoliations}

We must fix the orientations of our rectangles in $\calL(\alpha)$ to match the given orientations of our veering triangles.  We do so as follows. 

Applying \refthm{LinkSpace}\refitm{LinkSpacePlane} we deduce that $\calF^\alpha$ and $\calF_\alpha$ are orientable.  We fix an arbitrary orientation on $\calF_\alpha$.  Fix any point $[(\lambda, \mu)]$ of $\calL(\alpha)$ where both $\lambda$ and $\mu$ are interior leaves.  The orientation of $\calF_\alpha$ induces an orientation of $\ell_\mu$ and thus of $\mu$.  Let $\bdy_+ \mu$ be the endpoint of $\mu$ that this orientation points at; let $\bdy_- \mu$ be the other endpoint.  We orient $\lambda$ so that $\lambda$ points at its endpoint in $[\bdy_+ \mu, \bdy_- \mu]^\acw$.  This induces an orientation on $\ell^\lambda$ and thus of $\calF^\alpha$ which we now fix.  

Thus we can use the directions \emph{east, north, west,} and \emph{south} when working in a rectangle of $\calL(\alpha)$. 


\begin{lemma}
\label{Lem:PartialOrders} 
The relation \emph{to the east of} gives a partial order on $\calL(\alpha)$.  The same holds for the other three cardinal directions.
\end{lemma}

\begin{proof}
Transitivity follows by concatenating sequences of eastwards paths.  Antisymmetry follows because leaves of $\calF^\alpha$ and $\calF_\alpha$ are properly embedded lines, which thus separate $\calL(\alpha)$.
\end{proof}

For the remainder of the paper we require all rectangles $R \from (0, 1)^2 \to \calL(\alpha)$ to send oriented segments in the positive $y$--direction ($x$--direction) to oriented segments of $\calF^\alpha$ ($\calF_\alpha$).  

\begin{lemma}
\label{Lem:EdgeRectSlope}
With the choice of orientations of $\calF^\alpha$ and $\calF_\alpha$ as above we have the following.  An edge $e \in \cover{\calT}$ is red if and only if its edge rectangle $R(e)$ has ideal points in its northeast and southwest corners. 
\end{lemma}

\begin{proof}
We refer to the proof of \reflem{EdgeRectangle}, where $e$ was assumed to be red. With notation as in that lemma, we have that 
\[
c, \bdy U, \bdy V', d, \bdy T', \bdy S, 
\]
appear in $S^1(\alpha)$ in that anti-clockwise order. See \reffig{LeafIdentificationsEdge}. Fix $[(\lambda,\mu)] \in R(e)$ so that both $\lambda$ and $\mu$ are interior leaves. Breaking symmetry, we may assume that $\bdy_+ \mu$ lies in $[\bdy U, \bdy V']^\acw$. Since $\calL(\alpha)$ is connected, we deduce that $\bdy_+ \lambda$ lies in $[\bdy T', \bdy S]^\acw$. Thus $\ell_U$ and $\ell^S$ both point away from $c$. Thus $c$ lies in the southwest corner of $R(e)$.
\end{proof}

In this paper, all figures showing parts of $\calL(\alpha)$ are drawn using this choice of orientation on $\calF^\alpha$ and on $\calF_\alpha$.

\subsection{Maximal rectangles}
\label{Sec:MaximalRectangles}

\begin{definition}
\label{Def:MaxRectangle}
We call a rectangle $R \subset \calL(\alpha)$ \emph{maximal} if it is not properly contained in any other rectangle.  
\end{definition}

\begin{definition}
\label{Def:TetRectangle}
To each tetrahedron $t$ of $\cover{\calT}$ we associate an \emph{tetrahedron rectangle} $R(t) \subset \calL(\alpha)$ as follows.  If $f$ and $g$ are the upper faces of $t$ then we define $R(t)$ to be the union of the face rectangles $R(f)$ and $R(g)$.  
\end{definition}

\begin{lemma}
\label{Lem:TetRectangleLower}
Suppose that $t$ is a tetrahedron with lower faces $f'$ and $g'$.  Then $R(t) = R(f') \cup R(g')$.
\end{lemma}

\begin{proof}
Let $f$ and $g$ be the upper faces of $t$.  Let $e = f \cap g$ and $e' = f' \cap g'$ be the upper and lower (respectively) edges of $t$.  Note that the edges of $f \cup g$ other than $e$ are also edges of $f' \cup g'$.  We call these four edges the \emph{equator} of $t$. 

Let $(\lambda,\mu) \in \calP(\alpha)$ be a representative of a point of $R(t)$.  Then there is some edge of $f\cup g$ that links both $\lambda$ and $\mu$. If $(\lambda,\mu)$ links an equatorial edge then it lies in $R(f') \cup R(g')$ and we are done.  Suppose not.  We deduce that $\lambda$ links a pair of opposite equatorial edges and $\mu$ links the remaining pair of opposite equatorial edges.  Thus $(\lambda,\mu)$ links $e'$ and we are done.
\end{proof}

\begin{lemma}
\label{Lem:TetRectangle}
Suppose that $t$ is a tetrahedron of $\cover{\calT}$.
\begin{enumerate}
\item
\label{Itm:TetRectRect}
The tetrahedron rectangle $R(t)$ is a rectangle in the sense of \refdef{Rectangle}.
\item
\label{Itm:TetRectBdy}
The boundary of $R(t)$ is contained in a union of eight singular leaves.  There are four material corners. Each of the four sides contains exactly one ideal point, corresponding to one of the four cusps of $t$.
\item
\label{Itm:TetRectMax}
The tetrahedron rectangle $R(t)$ is maximal.
\end{enumerate}
\end{lemma}

Before proving the lemma, we introduce a simple definition.

\begin{definition}
\label{Def:Spans}
Let $Q$ and $R$ be rectangles. We say that $Q$ \emph{south-north (west-east) spans $R$} if $Q \cap R$ contains a leaf of $R \cap \calF^\alpha$ (of $R \cap \calF_\alpha$). 
\end{definition}



\begin{proof}[Proof of \reflem{TetRectangle}]
Breaking symmetry, suppose that the upper edge $e$ of $t$ is red.  Let $f$ and $g$ be the upper faces of $t$.  Note that $R(e)$ is a subrectangle of, and south-north spans, both $R(f)$ and $R(g)$.  See \reffig{LinkSpaceFaceFlowChart}(e).   Thus $R(t)$ is a rectangle, and we obtain \refitm{TetRectRect}.

Note that $R(f)$ and $R(g)$ share precisely two singular classes, namely those on the boundary of $R(e)$.  Thus there are four singular classes in $\bdy R(t)$.  No two can be in a single side as that would give a singular leaf meeting two singular classes, contradicting \refcor{Irrational}.  This gives \refitm{TetRectBdy}.  

Any rectangle properly containing $R(t)$ would have a singular class in its interior, a contradiction.  This proves \refitm{TetRectMax}.
\end{proof}

The next result is needed to relate the combinatorics of $\cover\calT$ to the combinatorics of rectangles in $\calL(\alpha)$. 

\begin{lemma}
\label{Lem:FaceInTet}
Suppose that $f$ and $t$ are, respectively, a face and a tetrahedron of $\cover{\calT}$.  Then $f \subset t$ if and only if $R(f) \subset R(t)$. 
\end{lemma}

\begin{proof}
The forwards direction follows from \refdef{TetRectangle} if $f$ lies above $t$ and from \reflem{TetRectangleLower} if $f$ lies below $t$. 

We prove the contrapositive of the backwards direction.  Suppose that $f$ is not a face of $t$.  Fix a layering $\calK$ of $\cover{\calT}$.  We break symmetry and suppose that $f$ is contained in a layer $K$, of $\calK$, that lies above $t$.  Furthermore, we take $K$ to be the lowest such layer in $\calK$.  

Suppose that the upper faces of $t$ lie in $K$.  Thus there is some edge, say $e$, of $f$ that separates $f$ (in $K$) from the upper faces of $t$.  Thus the cusp of $f$ not meeting $e$ cannot lie in a side of $R(t)$ and we are done.  See Figures~\ref{Fig:LinkSpaceFace} and~\ref{Fig:LinkSpaceFaceFlowChart}(e).

Suppose instead that the upper faces of $t$ do not lie in $K$.  In this case let $e$ be the edge that is removed from $K$ in order to obtain the immediately lower layer.  We deduce that $e$ is an edge of $f$.  Let $c$ and $d$ be the endpoints of $e$.  The faces of $t$ are all strictly below $K$, hence $e$ is not an edge of $t$.  Therefore by \reflem{NoParallelEdges}, no edge of $t$ connects $c$ and $d$.  Therefore at least one of $c$ or $d$ does not lie in any side of $R(t)$.  Therefore $R(f)$ is not a subset of $R(t)$ and we are done.
\end{proof}

We now consider the converse of \reflem{TetRectangle}\refitm{TetRectMax}.

\begin{theorem}
\label{Thm:MaximalRectangle}
If $R$ is a maximal rectangle in $\calL(\alpha)$ then there is a tetrahedron $t$ of $\cover{\calT}$ so that $R = R(t)$.
\end{theorem}

Before giving the proof we need several lemmas.

\begin{lemma}
\label{Lem:FinitelySpanned}
Let $K$ be a layer of a layering and $R$ a rectangle. Then $R$ is south-north spanned by at most finitely many edge rectangles $R(e)$ for $e$ in $K$.  The same holds west-east.
\end{lemma}

\begin{proof}
We consider only the south-north case. Applying the second statement of \reflem{Cantor}, we choose distinct non-singular leaves $m, m' \in \calF_\alpha$ which intersect $R$.  
For any edge $e$, if its rectangle $R(e)$ south-north spans $R$, then $m$ and $m'$ both cross $R(e)$.

Since $m$ is non-singular, by \refdef{UpperFoliation}, there is a unique leaf $\mu$ of $\Lambda_\alpha$ so that
\[
m = \big\{ [(\lambda, \mu)] \in \calL(\alpha) \mbox{ for some $\lambda \in \Lambda^\alpha$} \big\}
\]
Let $\mu'$ be the corresponding leaf for $m'$.  Thus $\mu$ and $\mu'$ are distinct interior leaves.  \reflem{Laminations}\refitm{LeavesAreCarried} provides train lines $m_K$ and $m'_K$ carried by the train track $\tau_K$ with the same endpoints as $\mu$ and $\mu'$ respectively.  Note that for any edge $e$ in $K$, if the leaf $m$ crosses $R(e)$ then the train line $m_K$ links $e$.

By \reflem{Laminations}\refitm{Asymptotic}, the train lines $m_K$ and $m'_K$ share no endpoints; thus they fellow travel for at most a finite collection of edges $E$ of $K$.  Thus, at most $|E|$ edge rectangles south-north span the subrectangle of $R$ between $m$ and $m'$.  Therefore, at most $|E|$ edge rectangles south-north span $R$.
\end{proof}

\begin{definition}
\label{Def:FaceRectangleMedian}
Let $f$ be a face with edges $e$, $e'$, and $e''$. The \emph{median} of $R(f)$ is the intersection $\closure{R}(e) \cap \closure{R}(e') \cap \closure{R}(e'')$.
\end{definition}

In the right-hand side of \reffig{LinkSpaceFace}, the median is the central yellow dot. 

\begin{lemma}
\label{Lem:MedianSpan}
Suppose that $f$ is a face with edges $e$, $e'$, and $e''$, where $e$ has the minority colour in $f$.  If $R$ is a rectangle that contains the median of $R(f)$ then $R$ is spanned by at least one of $R(e')$ or $R(e'')$.
\end{lemma}

\begin{proof}
Breaking symmetry, suppose that $e$ is blue while $e'$ and $e''$ are red.  For example, see \reffig{LinkSpaceFace}.  Since $R$ contains the median, it cannot extend out of $R(f)$ to the west or south; the singular leaves running through the median end at ideal points in the sides of $R(f)$.  The rectangle $R$ may extend out of $R(f)$ to either the east or north, but not both.  This is because of the northeastern ideal corner of $R(f)$. 

If $R$ extends out of $R(f)$ to the east then $R$ is south-north spanned by $R(e')$.  If it extends out the north then it is west-east spanned by $R(e'')$.  If $R$ is contained in $R(f)$ then it is spanned by both edge rectangles.
\end{proof}

\begin{lemma}
\label{Lem:SpannedSNOrWE}
Let $K$ be a layer of a layering and $R$ a rectangle. Then there is an edge $e$ in $K$ so that $R(e)$ either south-north or west-east spans $R$.
\end{lemma}

\begin{proof}
By \reflem{EdgesCover}, there is an edge $e_0$ in $K$ so that $R\cap \closure{R}(e_0)$ is non-empty.  Since $R$ is open, we also have that $R \cap R(e_0)$ is non-empty.  Breaking symmetry, suppose that $e_0$ is red.  If $R(e_0)$ does not span $R$ in either direction, then $R$ contains precisely one of the two material corners of $R(e_0)$.  Breaking symmetry again and applying \reflem{EdgeRectSlope}, we assume that $R$ contains $p_0$, the southeastern corner of $R(e_0)$.  

Consider now the maximal strip $P \subset K$ of edges and triangles $\{(f_i, e_i)\}_{i \geq 1}$ so that 
\begin{enumerate}
\item
all edges $e_i$ are red, 
\item
all triangles $f_i$ are majority red,
\item 
$e_0$ is contained in $f_1$, 
\item
$e_i$ is contained in $f_i$ (and in $f_{i+1}$ when $f_i$ is not the last triangle in $P$), 
\item
$R$ contains $p_i$, the southeastern material corner of $R(e_i)$. 
\end{enumerate}

Let $c_0$ and $d_0$ be the endpoints of $e_0$.  We label the other vertices of $P$ as follows.  Suppose that we have labelled the vertices of the edges $e_0, \ldots, e_{k-1}$.  Two of the vertices of $f_k$ are already labelled.  The remaining vertex of $f_k$ receives the label  $c_k$ if we turn left through $f_k$ when travelling from $e_{k-1}$ to $e_k$, or $d_k$ if we turn right.  For an example, see the left side of \reffig{RedStripRectangles}.

\begin{figure}[htbp]
\subfloat[The rectangle $R$ is not spanned by any of the edge rectangles $R(e_i)$.]{
\labellist
\small\hair 2pt
\pinlabel {$e_0$} [br] at 107 237
\pinlabel {$c_0$} [r] at 3 170
\pinlabel {$c_1$} [tr] at 67 129
\pinlabel {$c_2$} [tr] at 117 81
\pinlabel {$c_5$} [tr] at 154 29
\pinlabel {$d_0$} [bl] at 193 295
\pinlabel {$d_3$} [bl] at 270 249
\pinlabel {$d_4$} [bl] at 325 212
\pinlabel {$R$} [tl] at 660 209
\pinlabel {$c_0$} [r] at 430 148
\pinlabel {$c_1$} [tr] at 526 92
\pinlabel {$c_2$} [tr] at 580 60
\pinlabel {$c_5$} [t] at 620 23
\pinlabel {$p_0$} [l] at 728 151
\pinlabel {$d_0$} [b] at 728 348
\pinlabel {$d_3$} [bl] at 801 301
\pinlabel {$d_4$} [bl] at 854 263
\endlabellist
\includegraphics[width = 0.9\textwidth]{Figures/red_strip_rectangles1}
\label{Fig:RedStripRectangles1}
}

\subfloat[The rectangle $R$ is spanned by the edge rectangle $R(e')$.]{
\labellist
\small\hair 2pt
\pinlabel {$e_0$} [br] at 107 237
\pinlabel {$c_0$} [r] at 3 170
\pinlabel {$c_1$} [tr] at 67 129
\pinlabel {$c_2$} [tr] at 117 81
\pinlabel {$c_5$} [tr] at 154 29
\pinlabel {$d_0$} [bl] at 193 295
\pinlabel {$d_3$} [bl] at 270 249
\pinlabel {$d_4$} [bl] at 325 212
\pinlabel {$d$} [bl] at 365 159
\pinlabel {$e'$} [tl] at 263 100
\pinlabel {$R$} [tl] at 660 209
\pinlabel {$c_0$} [r] at 430 148
\pinlabel {$c_1$} [tr] at 526 92
\pinlabel {$c_2$} [tr] at 580 60
\pinlabel {$c_5$} [t] at 620 23
\pinlabel {$p_0$} [l] at 728 151
\pinlabel {$d_0$} [b] at 728 348
\pinlabel {$d_3$} [bl] at 801 301
\pinlabel {$d_4$} [bl] at 854 263
\pinlabel {$d$} [bl] at 943 123
\endlabellist
\includegraphics[width = 0.9\textwidth]{Figures/red_strip_rectangles2}
\label{Fig:RedStripRectangles2}
}
\caption{A possible strip $P$ and its corresponding edge rectangles.  A possible rectangle $R$ is drawn in black.}
\label{Fig:RedStripRectangles}
\end{figure}

\begin{figure}[htbp]
\subfloat[Northwest.]{
\includegraphics[width = 0.4\textwidth]{Figures/rightmost_face_blue}
\label{Fig:Northwest}
}
\qquad
\subfloat[Northeast.]{
\includegraphics[width = 0.4\textwidth]{Figures/leftmost_face_red}
\label{Fig:Northeast}
}

\subfloat[Southwest.]{
\includegraphics[width = 0.4\textwidth]{Figures/rightmost_face_red}
\label{Fig:Southwest}
}
\qquad
\subfloat[Southeast.]{
\includegraphics[width = 0.4\textwidth]{Figures/leftmost_face_blue}
\label{Fig:Southeast}
}

\caption{The four possible face rectangles, labelled by the intercardinal direction of the ideal corner.}
\label{Fig:FourFaces}
\end{figure}

\begin{claim*}
The closure of $R(e_k)$ contains $p_0$.
\end{claim*}

\begin{proof}
The base case of the induction is trivial.  Suppose that the claim holds for $e_k$.  Breaking symmetry, suppose that $f_{k+1}$ has an ideal point in its southwest corner.  See \reffig{Southwest}. By the inductive hypothesis, $p_0$ is in the closure of $R(e_k)$: the western half of $R(f_{k+1})$. Note that $R(e_{k+1})$ is the southern half; thus $R(e_{k})$ and $R(e_{k+1})$ intersect in the southwestern quadrant. Again see \reffig{Southwest}. Since $R$ contains $p_{k+1}$, the north side of $R$ is (perhaps non-strictly) south of $d_{k+1}$.  Since $R$ contains $p_0$, this and \reflem{PartialOrders} means that $p_0$ is strictly south of $d_{k+1}$.  Thus $p_0$ is contained in the closure of $R(e_{k+1})$.
\end{proof}

\begin{claim*}
Every $d_l$ is northeast of every $c_k$. 
\end{claim*}

\begin{proof}
The previous claim implies that the northeast (southwest) corner of $R(e_k)$ is northeast (southwest) of $p_0$. Thus all $d_l$ are northeast of $p_0$, and all $c_k$ are southwest of $p_0$.  Now \reflem{PartialOrders} gives the claim.
\end{proof}

\begin{claim*}
The strip $P$ is finite.
\end{claim*}

\begin{proof}
Suppose that $P$ is infinite.  We produce a sequence of strips $P_n$ below $P = P_0$, where $P_{i+1}$ is obtained from $P_i$ by flipping down through the first pair of triangles in $P_i$ that form the top of a tetrahedron.  Such a pair must exist by \refcor{CannotTurnLeftForever}.  Note that the vertices and boundary edges of $P_{i+1}$ are the same as those of $P_i$, and hence the same as those of $P_0 = P$.  Let $P_K$ be the first strip that contains an interior blue edge, $e_\triangledown$ say. Note that $K$ exists by \refcor{BranchLinesToggle} (for lower branch lines), An example sequence of strips is shown in \reffig{RedStripTrainTracks}. 

The edge $e_\triangledown$ connects vertices on opposite sides of the strip $P_K$.  Suppose that the endpoints of $e_\triangledown$ are $c_k$ and $d_l$.  Applying \reflem{EdgeRectSlope}, we find that one of $c_k$ and $d_l$ is southeast of the other.  By \reflem{PartialOrders} this contradicts the previous claim.
\end{proof}

We finish the proof of \reflem{SpannedSNOrWE} as follows. Let $e_N$ be the last edge of $P$.
Let $f'$ be the face adjacent to $e_N$ that is not in $P$.  
Either $f'$ is majority blue (see \reffig{Southeast}) or majority red (see \reffig{Northeast} or \reffig{Southwest}).  In the majority blue case, $R$ contains the median of $R(f')$, so by \reflem{MedianSpan} we are done.  In the majority red case, let $e'$ be the other red edge of $f'$.  Since $P$ is maximal, $R$ does not contain the southeastern corner of $R(e')$.  Thus $R$ is spanned by $R(e')$ (south-north in the case of \reffig{Northeast}, and west-east in the case of \reffig{Southwest}) and we are done.  
\end{proof}

\begin{figure}[htbp]
\subfloat[]{
\includegraphics[width = 0.45\textwidth]{Figures/red_strip_train_track1}
\label{Fig:RedStripTrainTrack1}
}
\quad
\subfloat[]{
\includegraphics[width = 0.45\textwidth]{Figures/red_strip_train_track1p5}
\label{Fig:RedStripTrainTrack1p5}
}

\subfloat[]{
\includegraphics[width = 0.45\textwidth]{Figures/red_strip_train_track2}
\label{Fig:RedStripTrainTrack2}
}
\quad
\subfloat[]{
\includegraphics[width = 0.45\textwidth]{Figures/red_strip_train_track3}
\label{Fig:RedStripTrainTrack3}
}

\subfloat[]{
\includegraphics[width = 0.45\textwidth]{Figures/red_strip_train_track4}
\label{Fig:RedStripTrainTrack4}
}
\caption{On the left in each subfigure we see a strip of triangles from above.  On the right, we see a side view, showing the height of the mid-curve through the strip. In particular, the top edge of a tetrahedron appears as a local maximum of the curve.}
\label{Fig:RedStripTrainTracks}
\end{figure}

\begin{proof}[Proof of \refthm{MaximalRectangle}]
Let $R$ be a maximal rectangle.  Fix $K$, a layer of a layering.  By \reflem{SpannedSNOrWE}, there is an edge of $K$ whose edge rectangle spans $R$.  Breaking symmetry, we suppose that $R$ is south-north spanned.  By \reflem{FinitelySpanned}, the set $E$ of edges in $K$, whose rectangles south-north span $R$, is finite.  For $e, e' \in E$ we say that $R(e)$ is \emph{to the west} of $R(e')$ if 
\begin{itemize}
\item the west side of $R(e)$ is to the west of the west side of $R(e')$ or, in the case of a tie, 
\item the east side of $R(e)$ is to the west of the east side of $R(e')$.
\end{itemize}

\noindent 
Let $e_\triangleleft$ be any edge whose rectangle $R(e_\triangleleft)$ is the furthest to the west. 

Breaking symmetry, suppose that $e_\triangleleft$ is red.  Let $f_\triangleleft$ be the face containing $e_\triangleleft$ and so that the (material) southeastern corners of $R(e_\triangleleft)$ and $R(f_\triangleleft)$ coincide. 

There are four possible face rectangles, depending on which corner is ideal. See \reffig{FourFaces}. 

\begin{claim*}
The rectangle $R(f_\triangleleft)$ has its ideal corner to its northeast. 
\end{claim*}

\begin{proof}
We rule out three of the four cases.
\begin{itemize}
\item 
By assumption, the southeastern corner of $R(f_\triangleleft)$ is material.
\item 
Suppose that the ideal corner of $R(f_\triangleleft)$ is in the southwest, as shown in \reffig{Southwest}.  Let $e$ be the other red edge of $f_\triangleleft$. Note that $R(e)$ extends further to the north than $R(e_\triangleleft)$. Since $R$ is maximal, it meets the west side of $R(e_\triangleleft)$. Thus $R(e)$ also spans $R$; we deduce that $e \in E$. This contradicts our assumption that $e_\triangleleft$ is the westmost edge in $E$. 
\item 
An argument similar to that for the southwest ideal corner rules out the ideal corner being in the northwest, as in \reffig{Northwest}.  \qedhere
\end{itemize}
\end{proof}

Since $R$ is maximal, its west side contains the singularity at the west side of $R(f_\triangleleft)$.  We have a similar situation on the east of $R$, with an eastmost edge $e_\triangleright$ and face $f_\triangleright$.  See \reffig{MaxRectangleCaps}.

\begin{figure}[htbp]
\centering
\includegraphics[width=0.6\textwidth]{Figures/max_rectangle_caps}
\caption{Above: An example of a maximal rectangle $R$ (shaded).  It has a westmost south-north spanning edge rectangle that is in the east of a face rectangle $R(f_\triangleleft)$ as in \reffig{Northeast}.  It also has an eastmost south-north spanning edge rectangle that is in the west of a face rectangle $R(f_\triangleright)$ as in \reffig{Northwest}.  Below: a sketch of the corresponding strip of triangles $P$ in $K$, connecting $f_\triangleleft$ with $f_\triangleright$.} 
\label{Fig:MaxRectangleCaps}
\end{figure}

Since $K$ is a copy of the Farey triangulation (\reflem{Disk}), there is a unique strip of triangles $P$ connecting $f_\triangleleft$ with $f_\triangleright$.  By \refthm{LinkSpace}\refitm{LinkSpaceDense}, we may take $m$ to be a non-singular leaf of $\calF_\alpha$ meeting $R$. By \reflem{Laminations}\refitm{LeavesAreCarried}, there is a corresponding train line $m_K \subset \tau_K$. Note that $m_K \cap \tau_P$ contains a train interval which runs from the track-cusp in $f_\triangleleft$ to the track-cusp in $f_\triangleright$.  These track-cusps point towards each other, so there is at least one large switch of $\tau_P$ at an edge in the interior of $P$.  The two faces incident to such an edge are the upper two faces of a tetrahedron. Moving $P$ down through such a tetrahedron has the effect of performing a split on $\tau_P$.  After finitely many such moves, $P$ consists of two faces only. These are thus the top two faces of a tetrahedron $t$. Thus $R$ is contained in, and therefore equal to, $R(t)$.
\end{proof}

We end with the following, in the spirit of \cite[Theorem~1.1]{Gueritaud16}.

\begin{theorem}
\label{Thm:Gueritaud}
The link space $\calL(\alpha)$, as equipped with
\begin{itemize}
\item
the upper and lower foliations $\calF^\alpha$ and $\calF_\alpha$ and 
\item
the action of $\pi_1(M)$,
\end{itemize}
recovers the veering triangulation $(\calT, \alpha)$. 
\end{theorem}

\begin{proof}
By \reflem{TetRectangle} and \refthm{MaximalRectangle} the tetrahedra of $\cover{\calT}$ are in bijective correspondence with the maximal rectangles of $\calL(\alpha)$.  

\begin{claim*}
The face rectangles in $\calL(\alpha)$ are those rectangles that meet three ideal points, exactly one of which is at a corner.
\end{claim*}

\begin{proof}
Face rectangles are of this type by \reflem{FaceRectangle}\refitm{FaceRectBdy}.  Suppose $R$ is such a rectangle.  Let $R(t)$ be a maximal (and thus tetrahedron) rectangle containing $R$.  Now \refdef{TetRectangle}, \reflem{TetRectangleLower}, and \reflem{TetRectangle} imply that $R$ is a face rectangle.  See \reffig{LinkSpaceFaceFlowChart}(e). 
\end{proof}

\begin{claim*}
The edge rectangles in $\calL(\alpha)$ are those rectangles that meet exactly two ideal points, at opposite corners.
\end{claim*}

\begin{proof}
The proof is similar to the previous claim.
\end{proof}

As in \reflem{EdgeRectSlope}, the colour of an edge $e$ is recovered from the location of the ideal corners of $R(e)$. 
Also, the dihedral angle of an edge $e$ in a tetrahedron $t$ is $\pi$ or zero as $R(e)$ does or does not span $R(t)$.
By \reflem{FaceInTet} two tetrahedra $t$ and $t'$ are glued across a face $f$ if and only if $R(t) \cap R(t') = R(f)$.  Again, see \reffig{LinkSpaceFaceFlowChart}(e).

Using the action of the fundamental group, and passing to orbits, we recover the combinatorics of $(\calT, \alpha)$. 
\end{proof}



















\bibliographystyle{hyperplain}
\begin{thebibliography}{10}

\bibitem{Agol11}
Ian Agol.
\newblock Ideal triangulations of pseudo-{A}nosov mapping tori.
\newblock In {\em Topology and Geometry in Dimension Three: Triangulations,
  Invariants, and Geometric Structures (Proceedings of the Jacofest
  conference)}, volume 560 of {\em AMS Contemporary Mathematics}, pages 1--18,
  2011,  \href{http://arxiv.org/abs/1008.1606v2}{{arXiv:1008.1606v2}}.

\bibitem{Agol15}
Ian Agol.
\newblock Veering triangulations and pseudo-{A}nosov flows, 2015.
\newblock Talk at IAS.

\bibitem{Bell15}
Mark~C. Bell.
\newblock {\em Recognising Mapping Classes}.
\newblock PhD thesis, The University of Warwick, 2015.

\bibitem{Calegari07}
Danny Calegari.
\newblock {\em Foliations and the geometry of 3-manifolds}.
\newblock Oxford Mathematical Monographs. Oxford University Press, Oxford,
  2007.

\bibitem{CalegariDunfield03}
Danny Calegari and Nathan~M. Dunfield.
\newblock Laminations and groups of homeomorphisms of the circle.
\newblock {\em Invent. Math.}, 152(1):149--204, 2003.

\bibitem{CannonThurston07}
James~W. Cannon and William~P. Thurston.
\newblock Group invariant {P}eano curves.
\newblock {\em Geom. Topol.}, 11:1315--1355, 2007.

\bibitem{CassonBleiler88}
Andrew~J. Casson and Steven~A. Bleiler.
\newblock {\em Automorphisms of surfaces after {N}ielsen and {T}hurston},
  volume~9 of {\em London Mathematical Society Student Texts}.
\newblock Cambridge University Press, Cambridge, 1988.

\bibitem{snappy}
Marc Culler, Nathan Dunfield, and Jeffrey~R. Weeks.
\newblock {SnapPy}.
\newblock A computer program for studying the geometry and topology of
  3-manifolds, http://snappy.computop.org.

\bibitem{FarbLeiningerMargalit11}
Benson Farb, Christopher~J. Leininger, and Dan Margalit.
\newblock Small dilatation pseudo-{A}nosov homeomorphisms and 3-manifolds.
\newblock {\em Adv. Math.}, 228(3):1466--1502, 2011.

\bibitem{Fenley12}
S\'ergio Fenley.
\newblock Ideal boundaries of pseudo-{A}nosov flows and uniform convergence
  groups with connections and applications to large scale geometry.
\newblock {\em Geom. Topol.}, 16(1):1--110, 2012.

\bibitem{Frankel18}
Ian {Frankel}.
\newblock {CAT}(-1)-type properties for {Teichm\"u}ller space.
\newblock August 2018,
  \href{http://arxiv.org/abs/1808.10022}{{arXiv:1808.10022}}.

\bibitem{Frankel13}
Steven Frankel.
\newblock {\em Quasigeodesic flows from infinity}.
\newblock PhD thesis, University of Cambridge, May 2013.
\newblock Available from \url{http://users.math.yale.edu/~sf472/thesis.pdf}.

\bibitem{FuterGueritaud13}
David Futer and Fran\c{c}ois Gu\'{e}ritaud.
\newblock Explicit angle structures for veering triangulations.
\newblock {\em Algebr. Geom. Topol.}, 13(1):205--235, 2013.

\bibitem{FuterTaylorWorden18}
David {Futer}, Samuel~J. {Taylor}, and William {Worden}.
\newblock {Random veering triangulations are not geometric}.
\newblock August 2018,
  \href{http://arxiv.org/abs/1808.05586}{{arXiv:1808.05586}}.

\bibitem{Gabai83}
David Gabai.
\newblock Foliations and the topology of {$3$}-manifolds.
\newblock {\em J. Differential Geom.}, 18(3):445--503, 1983.

\bibitem{GabaiOertel89}
David Gabai and Ulrich Oertel.
\newblock Essential laminations in {$3$}-manifolds.
\newblock {\em Ann. of Math. (2)}, 130(1):41--73, 1989.

\bibitem{GSS19}
Andreas Giannopolous, Saul Schleimer, and Henry Segerman.
\newblock A census of veering structures.
\newblock \url{https://math.okstate.edu/people/segerman/veering.html}
  {2019/02/15}.

\bibitem{Gueritaud06}
Fran\c{c}ois Gu\'{e}ritaud.
\newblock On canonical triangulations of once-punctured torus bundles and
  two-bridge link complements.
\newblock {\em Geom. Topol.}, 10:1239--1284, 2006.
\newblock With an appendix by David Futer.

\bibitem{Gueritaud16}
Fran\c{c}ois Gu\'{e}ritaud.
\newblock Veering triangulations and {C}annon-{T}hurston maps.
\newblock {\em J. Topol.}, 9(3):957--983, 2016.

\bibitem{Guirardel05}
Vincent Guirardel.
\newblock C\oe ur et nombre d'intersection pour les actions de groupes sur les
  arbres.
\newblock {\em Ann. Sci. \'{E}cole Norm. Sup. (4)}, 38(6):847--888, 2005.

\bibitem{Hamenstadt09}
Ursula Hamenst\"{a}dt.
\newblock Geometry of the mapping class groups. {I}. {B}oundary amenability.
\newblock {\em Invent. Math.}, 175(3):545--609, 2009.

\bibitem{HodgsonIssaSegerman16}
Craig~D. Hodgson, Ahmad Issa, and Henry Segerman.
\newblock Non-geometric veering triangulations.
\newblock {\em Exp. Math.}, 25(1):17--45, 2016.

\bibitem{HRST11}
Craig~D. Hodgson, J.~Hyam Rubinstein, Henry Segerman, and Stephan Tillmann.
\newblock Veering triangulations admit strict angle structures.
\newblock {\em Geom. Topol.}, 15(4):2073--2089, 2011.

\bibitem{Hodgson15}
Craig~D. Hodgson, J.~Hyam Rubinstein, Henry Segerman, and Stephan Tillmann.
\newblock Triangulations of 3-manifolds with essential edges.
\newblock {\em Ann. Fac. Sci. Toulouse Math. (6)}, 24(5):1103--1145, 2015.

\bibitem{Kapovich09}
Michael Kapovich.
\newblock {\em Hyperbolic manifolds and discrete groups}.
\newblock Modern Birkh\"{a}user Classics. Birkh\"{a}user Boston, Inc., Boston,
  MA, 2009.
\newblock Reprint of the 2001 edition.

\bibitem{Kozai13}
Kenji Kozai.
\newblock {\em Singular hyperbolic structures on pseudo-{A}nosov mapping tori}.
\newblock PhD thesis, Stanford University, June 2013.

\bibitem{Lackenby00}
Marc Lackenby.
\newblock Taut ideal triangulations of 3-manifolds.
\newblock {\em Geom. Topol.}, 4:369--395, 2000.

\bibitem{Landry18}
Michael Landry.
\newblock Taut branched surfaces from veering triangulations.
\newblock {\em Algebr. Geom. Topol.}, 18(2):1089--1114, 2018.

\bibitem{Li02}
Tao Li.
\newblock Laminar branched surfaces in 3-manifolds.
\newblock {\em Geom. Topol.}, 6:153--194, 2002.

\bibitem{MinskyTaylor17}
Yair~N. Minsky and Samuel~J. Taylor.
\newblock Fibered faces, veering triangulations, and the arc complex.
\newblock {\em Geom. Funct. Anal.}, 27(6):1450--1496, 2017.

\bibitem{Moore25}
Robert~L. Moore.
\newblock Concerning upper semi-continuous collections of continua.
\newblock {\em Trans. Amer. Math. Soc.}, 27(4):416--428, 1925.

\bibitem{Morgan84}
John~W. Morgan.
\newblock On {T}hurston's uniformization theorem for three-dimensional
  manifolds.
\newblock In {\em The {S}mith conjecture ({N}ew {Y}ork, 1979)}, volume 112 of
  {\em Pure Appl. Math.}, pages 37--125. Academic Press, Orlando, FL, 1984.

\bibitem{Mosher96}
Lee Mosher.
\newblock Laminations and flows transverse to finite depth foliations.
\newblock pages 1--224, 1996.
\newblock Preprint.

\bibitem{NeumannZagier85}
Walter~D. Neumann and Don Zagier.
\newblock Volumes of hyperbolic three-manifolds.
\newblock {\em Topology}, 24(3):307--332, 1985.

\bibitem{Richards63}
Ian Richards.
\newblock On the classification of noncompact surfaces.
\newblock {\em Trans. Amer. Math. Soc.}, 106:259--269, 1963.

\bibitem{Sakata16}
Naoki Sakata.
\newblock Veering structures of the canonical decompositions of hyperbolic
  fibered two-bridge link complements.
\newblock {\em J. Knot Theory Ramifications}, 25(4):1650015, 34, 2016.

\bibitem{veering_dehn_surgery}
Saul Schleimer and Henry Segerman.
\newblock Veering {D}ehn surgery.
\newblock Preprint.

\bibitem{SchleimerSegerman19}
Saul {Schleimer} and Henry {Segerman}.
\newblock {Essential loops in taut ideal triangulations}.
\newblock Feb 2019,
  \href{http://arxiv.org/abs/1902.03206}{{arXiv:1902.03206}}.

\bibitem{Smale67}
Stephen~J. Smale.
\newblock Differentiable dynamical systems.
\newblock {\em Bull. Amer. Math. Soc.}, 73:747--817, 1967.

\bibitem{Strenner18}
Bal{\'a}zs {Strenner}.
\newblock {Fibrations of 3-manifolds and asymptotic translation length in the
  arc complex}.
\newblock October 2018,
  \href{http://arxiv.org/abs/1810.07236}{{arXiv:1810.07236}}.

\bibitem{thurston_notes}
William Thurston.
\newblock Geometry and topology of 3-manifolds.
\newblock Available from \url{http://msri.org/publications/books/gt3m/}, 1978.

\bibitem{Thurston98}
William Thurston.
\newblock Three-manifolds, foliations and circles {II}: the transverse
  asymptotic geometry of foliations.
\newblock Preprint, 1998.

\bibitem{Thurston82}
William~P. Thurston.
\newblock Three-dimensional manifolds, {K}leinian groups and hyperbolic
  geometry.
\newblock {\em Bull. Amer. Math. Soc. (N.S.)}, 6(3):357--381, 1982.

\bibitem{Tillmann12}
Stephan Tillmann.
\newblock {Degenerations of ideal hyperbolic triangulations}.
\newblock {\em Mathematische Zeitschrift}, 272(3-4):793--823, 2012,
  \href{http://arxiv.org/abs/math/0508295v4}{{arXiv:math/0508295v4}}.

\bibitem{Kerekjarto23}
Bela {von Ker\'ekj\'art\'o}.
\newblock {\em {Vorlesungen \"uber Topologie. I.: Fl\"achentopologie. Mit 80
  Textfiguren.}}, volume~8.
\newblock Springer, Berlin, 1923.

\bibitem{Worden18}
William Worden.
\newblock Experimental statistics of veering triangulations.
\newblock {\em Experiment. Math.}, 2018.

\end{thebibliography}
 
\end{document} 
