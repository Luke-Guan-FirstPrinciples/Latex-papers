\documentclass[useAMS,usenatbib]{mnras}
\usepackage{natbib}
\bibliographystyle{mnras}
\usepackage[english]{babel}
\usepackage{graphicx,amsmath,amsfonts,amssymb}\usepackage{colortbl}
\usepackage{tabu,booktabs}
\usepackage{xcolor}
\usepackage{xspace}
\usepackage{txfonts}
\usepackage{mathrsfs,bm}





\newcommand{\C}[1]{\textbf{#1}}
\newcommand{\cp}[1]{\textcolor{blue}{#1}}
\newcommand{\T}[1]{\textcolor{red}{{\bf{#1}}}}

\def \etal {et~al.~}

\newcommand{\Rmax}{$R_{\rm max}$}
\newcommand{\Vmax}{$V_{\rm max}$}
\newcommand{\nPS}{$n_{\rm P\&S}$}
\newcommand{\nEin}{$n_{\rm E}$}
\newcommand{\amiga}{\texttt{AMIGA}}
\newcommand{\ahf}{\texttt{AHF}}
\newcommand{\mlapm}{\texttt{MLAPM}}
\newcommand{\hMpc}{{\ifmmode{h^{-1}{\rm Mpc}}\else{$h^{-1}$Mpc}\fi}}
\newcommand{\Mpc}{{\ifmmode{{\rm Mpc}}\else{Mpc}\fi}}
\newcommand{\hkpc}{{\ifmmode{h^{-1}{\rm kpc}}\else{$h^{-1}$kpc}\fi}}
\newcommand{\kpc}{{\ifmmode{ {\rm kpc} }\else{{\rm kpc}}\fi}}
\newcommand{\kms}{{\ifmmode{ {\rm km\,s^{-1}} }\else{ ${\rm km\,s^{-1}}$ }\fi}}
\newcommand{\hMsun}{{\ifmmode{h^{-1}{\rm {M_{\odot}}}}\else{$h^{-1}{\rm{M_{\odot}}}$}\fi}}
\newcommand{\Msun}{{\ifmmode{{\rm M}_{\odot}}\else{${\rm M}_{\odot}$}\fi}}
\newcommand{\Mhalo}{{\ifmmode{M_{\rm halo}}\else{$M_{\rm halo}$}\fi}}
\newcommand{\Rvir}{{\ifmmode{R_{\rm vir}}\else{$R_{\rm vir}$}\fi}}
\newcommand{\Mstar}{{\ifmmode{M_{\rm star}}\else{$M_{\rm star}$}\fi}}
\newcommand{\Vrot}{{\ifmmode{V_{\rm rot}}\else{$V_{\rm rot}$}\fi}}
\newcommand{\ltsima}{$\; \buildrel < \over \sim \;$}
\newcommand{\gtsima}{$\; \buildrel > \over \sim \;$}
\newcommand{\lsim}{\lower.5ex\hbox{\ltsima}}
\newcommand{\gsim}{\lower.5ex\hbox{\gtsima}}
\def\nbody{$N$-body}
\def\lesssim{\mathrel{\hbox{\rlap{\hbox{\lower4pt\hbox{$\sim$}}}\hbox{$<$}}}}
\def\gtrsim{\mathrel{\hbox{\rlap{\hbox{\lower4pt\hbox{$\sim$}}}\hbox{$>$}}}}
\newcommand{\new}[1]{\textbf{\textcolor{blue}{#1}}}
\newcommand{\problem}[1]{\textbf{\textcolor{red}{#1}}}
\newcommand{\Sec}[1]{Section~\ref{#1}}
\newcommand{\Eq}[1]{Eq.~(\ref{#1})}
\newcommand{\Fig}[1]{Fig.~\ref{#1}}
\newcommand{\beq}{\begin{equation}}
\newcommand{\eeq}{\end{equation}}
\def\beqa{\begin{eqnarray}}
\def\eeqa{\end{eqnarray}}
\def\LCDM{\ensuremath{\Lambda}CDM}
\def\LWDM{\ensuremath{\Lambda}WDM}
\def\head{ \vbox to 0pt{\vss \hbox to 0pt{\hskip 440pt\rm
      LA-UR-10-07069\hss} \vskip 25pt}}

\newcommand{\tensor}[1]{\mathsf{#1}}
\newcommand{\mat}{\ensuremath{\mathbfss}}
\newcommand{\bnabla}{\bm{\nabla}}
\newcommand{\dd}{\rmn{d}}
\newcommand{\eps}{\varepsilon}
\newcommand{\CR}{\mathrm{cr}}
\newcommand{\bvel}{\ensuremath{\bm{\varv}}}
\newcommand{\bB}{\ensuremath{\mathbfit{B}}}
\newcommand{\bb}{\ensuremath{\mathbfit{b}}}
\newcommand{\bcdot}{\ensuremath{\mathchoice {\mskip\thinmuskip\lower0.2ex\hbox{\scalebox{1.5}{$\cdot$}}\mskip\thinmuskip}}{\mskip\thinmuskip\lower0.2ex\hbox{\scalebox{1.5}{$\cdot$}}\mskip\thinmuskip}{\lower0.3ex\hbox{\scalebox{1.2}{$\cdot$}}}{\lower0.3ex\hbox{\scalebox{1.2}{$\cdot$}}}}


\def \kms {\ifmmode  \,\rm km\,s^{-1} \else $\,\rm km\,s^{-1}  $ \fi }
\def \kpc {\ifmmode  {\rm kpc}  \else ${\rm  kpc}$ \fi  }  
\def \hkpc {\ifmmode  {h^{-1}\rm kpc}  \else ${h^{-1}\rm kpc}$ \fi  }  
\def \hMpc {\ifmmode  {h^{-1}\rm Mpc}  \else ${h^{-1}\rm Mpc}$ \fi  }  
\def \Mpch {\ifmmode  {h^{-1}\rm Mpc}  \else ${h^{-1}\rm Mpc}$ \fi  }  
\def \Msun {\ifmmode {\rm M}_{\odot} \else ${\rm M}_{\odot}$ \fi} 
\def \hMsun {\ifmmode h^{-1}\,\rm M_{\odot} \else $h^{-1}\,\rm M_{\odot}$ \fi}

\def \LCDM {\ifmmode \Lambda{\rm CDM} \else $\Lambda{\rm CDM}$ \fi}
\def \sig8 {\ifmmode \sigma_8 \else $\sigma_8$ \fi} 
\def \OmegaM {\ifmmode \Omega_{\rm m} \else $\Omega_{\rm m}$ \fi} 
\def \Omegab {\ifmmode \Omega_{\rm b} \else $\Omega_{\rm b}$ \fi} 
\def \OmegaL {\ifmmode \Omega_{\rm \Lambda} \else $\Omega_{\rm \Lambda}$\fi} 
\def \Deltavir {\ifmmode \Delta_{\rm vir} \else $\Delta_{\rm vir}$ \fi}
\def \rhocrit {\ifmmode \rho_{\rm crit} \else $\rho_{\rm crit}$ \fi}
\def \rhou {\ifmmode \rho_{\rm u} \else $\rho_{\rm u}$ \fi}
\def \zc {\ifmmode z_{\rm c} \else $z_{\rm c}$ \fi}

\def\lcdm{\ensuremath{\Lambda\textrm{CDM}}\xspace}  
\def\omegam{\ensuremath{\Omega_\textrm{m}}\xspace}
\def\omegal{\ensuremath{\Omega_\Lambda}\xspace}
\def\omegab{\ensuremath{\Omega_\textrm{b}}\xspace}
\def\omegar{\ensuremath{\Omega_\textrm{r}}\xspace}




\def\aj{AJ}\def\actaa{Acta Astron.}\def\araa{ARA\&A}\def\apj{ApJ}\def\apjl{ApJ}\def\apjs{ApJS}\def\ao{Appl.~Opt.}\def\apss{Ap\&SS}\def\aap{A\&A}\def\aapr{A\&A~Rev.}\def\aaps{A\&AS}\def\azh{AZh}\def\baas{BAAS}\def\bac{Bull. astr. Inst. Czechosl.}\def\caa{Chinese Astron. Astrophys.}\def\cjaa{Chinese J. Astron. Astrophys.}\def\icarus{Icarus}\def\jcap{J. Cosmology Astropart. Phys.}\def\jrasc{JRASC}\def\mnras{MNRAS}\def\memras{MmRAS}\def\na{New A}\def\nar{New A Rev.}\def\pasa{PASA}\def\pra{Phys.~Rev.~A}\def\prb{Phys.~Rev.~B}\def\prc{Phys.~Rev.~C}\def\prd{Phys.~Rev.~D}\def\pre{Phys.~Rev.~E}\def\prl{Phys.~Rev.~Lett.}\def\pasp{PASP}\def\pasj{PASJ}\def\qjras{QJRAS}\def\rmxaa{Rev. Mexicana Astron. Astrofis.}\def\skytel{S\&T}\def\solphys{Sol.~Phys.}\def\sovast{Soviet~Ast.}\def\ssr{Space~Sci.~Rev.}\def\zap{ZAp}\def\nat{Nature}\def\iaucirc{IAU~Circ.}\def\aplett{Astrophys.~Lett.}\def\apspr{Astrophys.~Space~Phys.~Res.}\def\bain{Bull.~Astron.~Inst.~Netherlands}\def\fcp{Fund.~Cosmic~Phys.}\def\gca{Geochim.~Cosmochim.~Acta}\def\grl{Geophys.~Res.~Lett.}\def\jcp{J.~Chem.~Phys.}\def\jgr{J.~Geophys.~Res.}\def\jqsrt{J.~Quant.~Spec.~Radiat.~Transf.}\def\memsai{Mem.~Soc.~Astron.~Italiana}\def\nphysa{Nucl.~Phys.~A}\def\physrep{Phys.~Rep.}\def\physscr{Phys.~Scr}\def\planss{Planet.~Space~Sci.}\def\procspie{Proc.~SPIE}


\def\head{ .ps \vbox to 0pt{\vss \hbox to 0pt{\hskip 440pt\rm
      LA-UR-10-07069\hss} \vskip 25pt}} 

\def \spose#1{\hbox  to 0pt{#1\hss}}  
\def \lta{\mathrel{\spose{\lower 3pt\hbox{$\sim$}}\raise 2.0pt\hbox{$<$}}}
\def \gta{\mathrel{\spose{\lower 3pt\hbox{$\sim$}}\raise 2.0pt\hbox{$>$}}}

\title[Cosmic rays in cosmological simulations]
{The effects of cosmic rays on the formation of Milky Way-mass galaxies in a cosmological context}

\author[T. Buck \etal] {Tobias Buck$^{1}$\thanks{E-mail:
    tbuck@aip.de},
    Christoph Pfrommer$^{1}$,
    R\"udiger Pakmor$^{2}$,
    Robert J. J. Grand$^{2}$,
    \newauthor{\& Volker Springel$^{2}$}\\
$^1$Leibniz-Institut f\"ur Astrophysik Potsdam (AIP), An der Sternwarte 16, D-14482 Potsdam, Germany\\
$^2$Max-Planck-Institut f\"ur Astrophysik, Karl-Schwarzschild-Str. 1, D-85748, Garching, Germany
}

\setlength{\topmargin}{-1.3cm}

\begin{document}

\date{Accepted 2020 June 30. Received 2020 June 30; in original form 2019 October 29}

\pagerange{\pageref{firstpage}--\pageref{lastpage}} \pubyear{2019}

\maketitle

\label{firstpage}




\begin{abstract}
  We investigate the impact of cosmic rays (CR) and different modes of CR
  transport on the properties of Milky Way-mass galaxies in cosmological
  magneto-hydrodynamical simulations in the context of the AURIGA project. We
  systematically study how advection, anisotropic diffusion and additional
  Alfv\'en-wave cooling affect the galactic disc and the circum-galactic medium
  (CGM). Global properties such as stellar mass and star formation rate vary
  little between simulations with and without various CR transport physics,
  whereas structural properties such as disc sizes, CGM densities or
  temperatures can be strongly affected. In our simulations, CRs affect the
  accretion of gas onto galaxies by modifying the CGM flow structure. This
  alters the angular momentum distribution which manifests itself as a
  difference in stellar and gaseous disc size. The strength of this effect
  depends on the CR transport model: CR advection results in the most compact
  discs while the Alfv\'en-wave model resembles more the AURIGA model. The
  advection and diffusion models exhibit large ($r\sim50$ kpc) CR
  pressure-dominated gas haloes causing a smoother and partly cooler CGM. The
  additional CR pressure smoothes small-scale density peaks and compensates for
  the missing thermal pressure support at lower CGM temperatures. In contrast,
  the Alfv\'en-wave model is only CR pressure dominated at the disc-halo
  interface and only in this model the gamma-ray emission from hadronic
  interactions agrees with observations. In contrast to previous findings, we
  conclude that details of CR transport are critical for accurately predicting
  the impact of CR feedback on galaxy formation.
\end{abstract}

\noindent
\begin{keywords}

MHD - cosmic rays - galaxies: formation - galaxies: evolution - galaxies: structure - methods: numerical

 \end{keywords}

\section{Introduction} \label{sec:intro}


The formation of galaxies is a multi-scale, multi-physics problem and understanding the details of the physical processes involved is one of the most challenging problems in theoretical astrophysics. Cosmological simulations and semi-analytic studies have demonstrated that feedback from stellar winds and radiation fields, supernovae, and active galactic nuclei (AGNs) are key processes in shaping the structure of galaxies \citep[e.g.][]{Brook2012,Stinson2013,Puchwein2013,Marinacci2014,Vogelsberger2014,Henriques2015,Schaye2015,Dubois2016,Kaviraj2017,Pillepich2018,Hopkins2018}. These processes effectively drive galactic winds, move gas and metals out of galaxies into the intergalactic medium, regulate the star formation rate (SFR) down to the observed low rates or completely quench it in elliptical galaxies, and balance radiative cooling in the centers of galaxy clusters \citep{Kravtsov2012,Battaglia2012,Battaglia2012b,Battaglia2013,McCarthy2014,McCarthy2017,Dolag2016,Weinberger2017}.

While the latest galaxy formation models are quite successful in reproducing key observables of realistic galaxies \citep[e.g.][]{Wang2015,Sawala2016,Grand2017,Hopkins2018,Buck2019b}, most feedback prescriptions are modelled empirically, calibrated against observed scaling relations which limits the predictive power of the corresponding calculations. In particular, resolution requirements of hydrodynamical simulations of galaxy formation made it necessary to implement feedback relatively coarsely: simulations base their feedback prescriptions on explicit sub-grid formulations which model the unresolved, multiphase structure of the interstellar medium (ISM) \citep{Springel2003,Schaye2008}.
The details of the driving mechanisms behind galactic winds and outflows are still unknown and implementations remain phenomenological \citep{Oppenheimer2006}. On larger scales, feedback from AGNs has been invoked in order to balance star formation in galaxy clusters. Here, accretion rates onto the black hole are estimated by the Bondi prescription and feedback energy is injected in form of pure thermal energy \citep{DiMatteo2005,Springel2005} or they involve chaotic cold accretion \citep{Gaspari2013} or AGN feedback might be modelled slightly more complex \citep{Weinberger2017,Dave2019}.

Another obvious source of galactic feedback might be due to the energy and momentum deposition of the ultraviolet radiation of the stars. Radiation pressure acting on dust grains and the atomic lines in dense gas has been argued to transfer enough momentum to the gas in order to exceed the escape velocity and drive winds \citep{Murray2005,Thompson2005}. However, direct radiation-hydrodynamical simulations in simplified set-ups \citep{Krumholz2012} or in isolated galaxy simulations \citep{Rosdahl2015} do not produce strong radiation pressure driven winds, suggesting that radiation feedback is less effective and more gentle than widely assumed \citep[but see also][]{Emerick2018}.

On the other hand, cosmological simulations often neglect feedback from relativistic particles, so-called cosmic rays (CRs), which provide another source of non-thermal feedback. Such a particle population can be created by diffusive shock acceleration at expanding supernova remnants \citep[e.g.][]{Blandford1987, Jubelgas2008} or in AGN-powered jets \citep[e.g.][]{Sijacki2008,Ehlert2018}. CRs and magnetic fields are observed to be in pressure equilibrium with the turbulence in the mid-plane of the Milky Way \citep{Boulares1990} and the pressure forces of the CRs might be able to accelerate the ISM and drive powerful galactic outflows as suggested by a number of theoretical works \citep{Ipavich1975,Breitschwerdt1991,Breitschwerdt2002,Zirakashvili1996,Ptuskin1997,Socrates2008,Everett2008,Samui2010,Dorfi2012} and local three-dimensional (3D) simulations of the ISM \citep{Hanasz2013,Girichidis2016,Simpson2016}. 

\begin{table*}
\label{tab:props}
\begin{center}
\caption{Properties of the main galaxies: \normalfont{We state the virial mass, $M_{200}$, virial radius, $R_{200}$, the stellar mass, $M_{\rm star}$, and the gas mass, $M_{\rm gas}$, as well as the disc and bulge masses, $M_{\rm d}$, $M_{\rm b}$, as resulting from a combined exponential  plus \citet{Sersic1963} fit to the azimuthally averaged surface density profile. We further report the disc scale length, $R_{\rm d}$, the bulge effective radius, $R_{\rm eff}$, the bulge sersic index, $n$, and the disc-to-total ratio D/T of that fit.}}
\begin{tabular}{l c c c c c c c c c c}
		\hline\hline
		simulation & $M_{200}$ & $R_{200}$ & $M_{\rm star}$ & $M_{\rm gas}$ & $M_{\rm d}$ & $R_{\rm d}$ & $M_{\rm b}$  & $R_{\rm eff}$ & $n$ & D/T \\
		  & [$10^{12}\Msun$] & [kpc] & [$10^{10}\Msun$] & [$10^{10}\Msun$] & [$10^{10}\Msun$] & [kpc]  & [$10^{10}\Msun$]& [kpc] & & \\
		\hline
		\multicolumn{11}{c}{Auriga-6 (Au6)}\\
		\hline
		noCR & 1.02 & 212.39 & 4.36 & 7.88 & 4.17 & 4.53 & 0.41 & 0.89 & 0.69 & 0.91 \\
		CRdiffalfven & 1.06 & 215.18 & 5.54 & 9.12 & 4.37 & 2.84 & 1.03 & 0.82 & 0.83 & 0.81\\
		CRdiff & 1.07 & 215.43 & 5.81 & 8.91 & 1.03 & 4.37 & 4.64 & 1.14 & 1.11 & 0.18\\
		CRadv & 1.09 & 216.71 & 6.19 & 9.47 & 1.11 & 4.00 & 4.79 & 1.11 & 1.54 & 0.19\\
		\hline
		\multicolumn{11}{c}{Auriga-L8 (AuL8)}\\
		\hline
		noCR & 0.84 & 199.02 & 4.82 & 7.00 & 4.39 & 3.65 & 0.41 & 0.73 & 0.58 & 0.91\\
		CRdiffalfven & 0.83 & 197.96 & 4.72 & 6.06 & 4.08 & 3.76 & 0.67 & 0.90 & 0.91 & 0.86\\
		CRdiff & 0.83 & 198.48 & 4.51 & 6.65 & 0.00 & 0.00 & 4.51 & 4.33 & 1.71 & 0.00\\
		CRadv & 0.85 & 199.33 & 4.58 & 7.76 & 0.00 & 0.00 & 4.58 & 3.79 & 1.49 & 0.00\\
        \hline
\end{tabular}
\end{center}
\end{table*}


In comparison to other feedback mechanisms, CRs have a number of advantageous properties: 
(i) the CR pressure drops less quickly upon adiabatic expansion than the thermal pressure due to their softer equation of state ($P_{\rm CR} \propto \rho^{\gamma_{\rm CR}}$ with $\gamma_{\rm CR} = 4/3$),
(ii) CR cooling is generally less efficient than the radiative cooling of a thermal plasma \citep{Ensslin2007} and thus acts on longer time-scales compared to thermal energy,
(iii) the non-thermal energy of CRs is not detectable through thermal observables or X-ray emission, therefore the (temporary) storage of feedback energy in CRs also avoids problems with the overproduction of these observables
(iv) they can maintain the outflows in a warm/hot state because the resonant production of Alfv\'en waves through the streaming instability \citep{Kulsrud1969} and the dissipation of wave energy with various plasma physical wave damping processes energises galactic winds as they rise in the galactic haloes.

A number of works have implemented CRs into 3D hydrodynamic simulations of galaxy formation and have demonstrated the ability of CRs to drive winds and regulate star formation  \citep{Jubelgas2008,Uhlig2012,Booth2013,Salem2014,Pakmor2016,Ruszkowski2017,Pfrommer2017b,Jacob2018,Chan2019}. CRs do not couple to the thermal gas via particle-particle collisions but via particle-wave interactions as fast streaming CRs along the magnetic field resonantly excite Alfv\'en waves. CRs are then able to scatter off of these waves which isotropise their distribution function in the wave frame, thus transferring energy and momentum to the thermal plasma and exerting a pressure onto the gas. Thereby, CRs not only impart momentum to the ISM at the launching sites but continuously re-power winds via thermal and dynamic coupling of plasma and CRs.
   
As a result CRs might explain the observed low SFRs in giant elliptical galaxies located at the centers of galaxy groups and clusters. In the absence of any heating processes, the hot gaseous atmosphere of these objects is expected to efficiently cool and form stars at very high rates \citep[up to a few hundred \Msun yr$^{-1}$, e.g.][]{Peterson2006}. However, observed SFRs are much below these expected rates which is why AGN feedback has been invoked to balance radiative cooling. While theoretical considerations have shown that AGN feedback energies are sufficient, the exact coupling mechanism is still under debate \citep{McNamara2007}. While several physical processes have been proposed to mediate the heating \citep[amongst others the dissipation of turbulent energy powered by the AGN,][]{Zhuravleva2014} a promising alternative is given by CRs. A net outward flux of streaming CRs can resonantly excite Alfv\'en waves that experience non-linear Landau damping or decay via a cascading process as a result of strong external turbulence and eventually dissipate locally \citep{Loewenstein1991,Guo2008,Ensslin2011,Pfrommer2013,Wiener2013,Jacob2017,Jacob2017b,Ruszkowski2017b,Ehlert2018}.

By now there are several (magneto-)hydrodynamics (MHD) simulation codes capable of solving the details of the CR proton acceleration and transport in galaxies
and galaxy clusters: 
the Eulerian mesh codes \textsc{cosmocr} \citep{Miniati2001}, \textsc{zeus-3d} \citep{Hanasz2003}, the smoothed particle hydrodynamics code \textsc{gadget-2}
\citep{Pfrommer2006,Ensslin2007,Jubelgas2008}, the adaptive mesh refinement codes \textsc{ramses} \citep{Booth2013, Dubois2016}, \textsc{enzo} \citep{Salem2014}, \textsc{flash} \citep{Girichidis2016}, and \textsc{pluto} \citep{Mignone2018}, the moving-mesh code \textsc{arepo} \citep{Pakmor2016b,Pfrommer2017} and the mesh-free Lagrangian finite mass code \textsc{gizmo} \citep{Chan2019}.  Here, we use the \textsc{arepo} code \citep{Springel2010,Pakmor2016c} combined with the numerical implementations of CR physics \citep{Pfrommer2017,Pakmor2016b} to simulate the formation of Milky Way (MW) like galaxies in a cosmological context. 

This paper is organized as follows: In \Sec{sec:sim} we describe the simulation setup and the different implementations of CR treatment. In \Sec{sec:props} we study the central stellar and gaseous discs focussing on the differences and similarities in properties across various CR physics variants. We further investigate here the accretion of gas onto the main galaxy and the successive build-up of angular momentum. In \Sec{sec:CGMprops} we turn to analyse the effects of CRs on the properties and structure of the CGM. We finish our analysis in \Sec{sec:obs} by comparing a direct observable, namely the gamma-ray luminosity of the simulated galaxies, to observations. In \Sec{sec:dis} we conclude this paper with a discussion, compare our results to previous work, and summarize our results in \Sec{sec:conc}.

\section{Cosmological Simulations} \label{sec:sim}


For this work we simulate the formation of two MW-like disc galaxies from cosmological initial conditions taken from the AURIGA project \citep{Grand2017,Grand2019}. Simulations are performed with the second-order accurate, adaptive moving-mesh code AREPO \citep{Springel2010,Pakmor2016c} which includes important physical prescriptions to model galaxy formation in a cosmological setup. For completeness we describe the most important models below but refer the reader to the works of \citet{Grand2017}, \citet{Pakmor2016b} for more technical details on the AURIGA galaxy formation model and to \citet{Pfrommer2017} for details on the CR physics.

The AURIGA model includes primordial and metal-line cooling with self-shielding corrections and the spatially uniform UV background model of \citet{Faucher2009} is included \citep[for more details see][]{Vogelsberger2013}. The interstellar medium (ISM) is modelled with an effective equation of state \citep{Springel2003} and star-forming gas is treated as a two phase medium. Star formation occurs in thermally unstable gas for densities higher than a threshold density of $n_{\rm th} = 0.13\mbox{ cm}^{-3}$ in a stochastic manner where the probability scales exponentially with time in units of the star formation timescale ($t_{\rm sf}=2.2$ Gyr in the AURIGA model).

\begin{figure*}
\vspace{-.5 cm}
\begin{center}
\includegraphics[width=\textwidth]{./plots/auriga_stellar_light_L6_lvl4.pdf}
\includegraphics[width=\textwidth]{./plots/auriga_stellar_light_L8_lvl4.pdf}
\end{center}
\vspace{-.35cm}
\caption{Face-on and edge-on projected stellar density at $z=0$ for the eight simulations. The upper panel shows the galaxy Au6 and the lower panel shows AuL8. From left to right we show the four different variants of physics: (i) the fiducial AURIGA simulations without CRs, (ii) the simulations with CR advection and anisotropic CR diffusion and Alfv\'en cooling/heating enabled, (iii) the CR anisotropic diffusion and CR advection simulations, and (iv) the pure CR advection simulations. The images are synthesized from a projection of the $K$-, $B$- and $U$-band luminosity of stars, which are shown by the red, green and blue colour channels, in logarithmic intervals, respectively. Younger (older) star particles are therefore represented by bluer (redder) colours. In each face-on panel we note structural properties of the stellar disc resulting from a surface density fit of a combined exponential plus \citet{Sersic1963} profile (see Fig.\ \ref{fig:surf_den_fit}). The plot dimensions are 50 kpc $\times$ 50 kpc and 50 kpc $\times$ 25 kpc, respectively.}
\label{fig:rgb}
\end{figure*}


Each star particle in this model represents a single stellar population (SSP) characterised by age and metallicity assuming a \citet{Chabrier2003} initial mass function (IMF). The mass loss and metal return of each star particle is calculated for supernovae of type II (SNII), SNIa and AGB stars at each time-step. The mass and metal yields are taken from \citet{Karakas2010} for AGB stars, and \citet{Portinari1998} for core collapse SNe. Mass and metals are distributed among nearby gas cells with a top-hat kernel. SNIa events are calculated using a delay time distribution function and their mass and metal return follows the yield tables from \citet{Thielemann2003} and \citet{Travaglio2004}. The number of SNII is calculated from the number of stars in the mass range $8-100\Msun$ and feedback is assumed to occur instantaneously. An active gas cell (star-forming) is probabilistically chosen to either form a star particle or become a site for SNII events \citep[see][]{Vogelsberger2013} in which case a wind particle is launched in an isotropically random direction. The wind velocity is coupled to the one-dimensional velocity dispersion of the dark matter halo. The wind particle interacts only gravitationally until either a gas cell with a density below $0.05$ times the threshold density for star formation is reached, or the maximum travel time is exceeded. Meeting either of the criteria, the wind particle re-couples and deposits its mass, metals, momentum and thermal energy into the gas cell in which it is located \citep[see][and references therein for more details]{Grand2017}. Note, CRs are directly injected into neighbouring cells of a supernova and as such wind particles neither transport CR energy nor the magnetic field.

Active galactic nuclei feedback is implemented following \citet{Springel2005}. The mass growth from gas accretion is described by Eddington-limited Bondi-Hoyle-Lyttleton accretion \citep{Bondi1944,Bondi1952} in addition to a term that models the radio mode accretion \citep[see][for more details]{Grand2017}. Magnetic fields are treated with ideal MHD \citep{Pakmor2011,Pakmor2013} and are seeded at redshift $z=127$ with a homogeneous magnetic field of $10^{-14}$ (comoving) Gauss. The divergence cleaning scheme of \citet{Powell1999} is implemented to ensure that the divergence of the magnetic field vanishes.

Finally, CRs are modelled as a relativistic fluid with a constant adiabatic index of $4/3$ in a two-fluid approximation \citep{Pfrommer2017}. CRs are generated at core-collapse supernovae remnants by instantaneously injecting all CR energy produced by the star particle into its surroundings immediately after birth. The energy efficiency of the injection is set to $\zeta_{\rm SN} = 0.1$. Following \citet{Pfrommer2017}, we assume an equilibrium momentum distribution for the CRs to model their cooling via Coulomb and hadronic interaction with the ambient gas. 

To bracket the uncertainties of CR transport, we simulate three different models with different variants of CR physics \citep[similar to][]{Wiener2017} and one model without CRs. To explain the differences of our CR models, we briefly review the main aspects of CR hydrodynamics. While individual CRs move close to the speed of light, frequent resonant CR interactions with Alfv\'en waves causes their distribution function to (nearly) isotropise in the frame of the Alfv\'en waves such that the CR energy is transported as a superposition of CR advection with the gas, anisotropic streaming with Alfv\'en waves along the magnetic field and diffusion with respect to the wave frame so that the time evolution equation of the CR energy density $\eps_\CR$ in the one-moment formulation of CR transport reads as follows:
\begin{align}
  \frac{\partial \eps_\CR}{\partial t}
  &+ \bnabla\bcdot
  \big[
    \underbrace{\eps_\CR \bvel}_{\rmn{advection}}
    + \underbrace{(\eps_\CR+P_\CR)\bvel_{\rmn{st}}}_{\rmn{streaming}}
    - \underbrace{\kappa_\eps \bb \left( \bb \bcdot \bnabla \eps_\CR\right)}_{\rmn{anisotropic\ diffusion}}\big]
  \nonumber\\
    &= - \underbrace{P_\CR\,\bnabla \bcdot \bvel}_{\rmn{adiab.\ changes}}
      - ~\underbrace{\left|\bvel_{\rmn{A}} \bcdot \bnabla P_\CR\right|}_{\rmn{Alfven\ cooling}}
      + \underbrace{\Lambda_\CR + \Gamma_\CR}_{\rmn{losses\ \&\ sources}}.
      \label{eq:ecr}
\end{align}
Here, $\bvel$ denotes the gas velocity, $\bvel_{\rmn{A}}=\bB/\sqrt{4\upi\rho}$ is the Alfv\'en velocity, $\bB$ is the magnetic field, $\rho$ is the gas mass density, $\bvel_{\rmn{st}}$ is the CR streaming velocity, 
\begin{eqnarray}
  \label{eq:vstream}
  \bvel_{\rmn{st}} = -\bvel_{\rmn{A}}\, \rmn{sgn}(\bB\bcdot\bnabla P_\CR)
  = -\frac{\bB}{\sqrt{4\upi\rho}}\,
  \frac{\bB\bcdot\bnabla P_\CR}{\left|\bB\bcdot\bnabla P_\CR\right|},
\end{eqnarray}
implying that the CR streaming velocity is oriented along magnetic fields lines down the CR pressure gradient with a velocity that corresponds in magnitude to $\bvel_{\rmn{A}}$,$P_\CR$ is the CR pressure, $\kappa_\eps$ is the kinetic energy-weighted spatial CR diffusion coefficient, $\bb=\bB/|\bB|$ is the unit vector along the local magnetic field, and $\Lambda_\CR$ and $\Gamma_\CR$ are non-adiabatic CR losses and sources.

Note that CR streaming and diffusion are both anisotropic transport processes along the mean magnetic field and oriented down the CR gradient. While the streaming term advects CRs with the frame of Alfv\'en waves and maintains CR gradients, diffusion is a dispersive process (owing to the second gradient in the bracket of Eq.~\eqref{eq:ecr}) so that the CR gradient weakens over time, implying that the streaming and diffusion fluxes cannot be the same at all times \citep{Wiener2017}. Most importantly, CR diffusion exactly conserves the CR energy while CR streaming drains CR energy at a rate $\left|\bvel_{\rmn{A}} \bcdot \bnabla P_\CR\right|$ due to the excitation of resonant Alfv\'en waves.

\begin{figure*}
\vspace{-.45 cm}
\begin{center}
\includegraphics[width=.33\textwidth]{./plots/auriga_cr_sfr_L6_lvl4.pdf}
\includegraphics[width=.33\textwidth]{./plots/auriga_cr_sfr_L8_lvl4.pdf}
\includegraphics[width=.33\textwidth]{./plots/stellarvstotalmass.pdf}
\end{center}
\vspace{-.35cm}
\caption{Star formation histories (SFH) and the stellar mass-halo mass relation (right panel) for the AURIGA haloes. We show the SFHs for galaxy Au6 (left panel) and for AuL8 (middle panel), respectively. In the right panel, we show the final stellar mass of our simulations at $z=0$ vs their final halo mass. In each panel we compare the SFH/stellar masses of the four different physics variants, the fiducial Auriga model shown in black, the cosmic ray advection run in blue, the run with additional anisotropic CR diffusion in purple and the run which additionally accounts for CR Alfv\'en wave cooling in red. In the right panel, the light blue line shows the abundance matching result taken from \citet{Moster2013} while the gray dotted line shows the cosmic baryon fraction.}
\label{fig:sfr}
\end{figure*}



In all of our models we omit the CR streaming term on the left-hand side of Eqn.~\eqref{eq:ecr}, which can be most accurately solved with the two-moment method of CR transport \citep{Jiang2018,Thomas2019}. In all three CR models, we account for CR advection and adiabatic changes of the CR energy. Our models are defined as follows:
\begin{itemize}
\item[(i)] \textbf{noCR}: fiducial AURIGA galaxy formation model without CRs.
\item[(ii)] \textbf{CRadv}: CR advection model where CRs are only advected with the gas.
\item[(iii)] \textbf{CRdiff}: CR diffusion model where CRs are advected with the gas but are further allowed to anisotropically diffuse relative to the rest frame of the gas with a diffusion coefficient of $\kappa_\parallel = 10^{28}$~cm$^2$~s$^{-1}$ along the magnetic field and no diffusion perpendicular to it \citep{Pakmor2016b}.
\item[(iv)] \textbf{CRdiffalfven}: anisotropic CR diffusion model with the additional inclusion of the Alfv\'en-wave cooling term that arises due to the energy transfer from CRs to Alfv\'en waves that are self-excited through the resonant CR streaming instability \citep{Kulsrud1969}.
\end{itemize}
While CR diffusion is thought to describe the transport of high-energy CRs above $\sim200$ GeV, at lower energies the transport transitions to mainly CR streaming with self-generated Alfv\'en waves \citep{Evoli2018}, although the role of scattering in external turbulence is not yet settled \citep{Zweibel2017}. While the process of CR diffusion conserves CR energy, additionally accounting for the Alfv\'en-wave cooling term emulates and approximates CR streaming. While this approximation is justified in cases for which the diffusion and streaming fluxes match each other, solutions will necessarily deviate if this condition is not fulfilled \citep{Wiener2017}. Future work is needed to clarify how the explicit inclusion of CR streaming in the presence of different wave damping processes changes the picture presented in this work.


\begin{figure*}
\vspace*{-.4cm}
\begin{center}
\raggedleft
\includegraphics[width=.925\textwidth]{./plots/auriga_gas_density_L6_lvl4.pdf}
\includegraphics[width=.925\textwidth]{./plots/auriga_bfld_lic_L6_lvl4.pdf}
\includegraphics[width=.71\textwidth]{./plots/auriga_xcr_L6_lvl4.pdf}
\end{center}
\vspace{-.35cm}
\caption{From top to bottom we show the gas surface density, the volume-weighted magnetic field strength, $\sqrt{\langle\bB^2\rangle_V}$, and the CR-to-thermal pressure ratio, $X_{\rm cr} = \left<P_{\rm cr}\right>_V / \left<P_{\rm th}\right>_V$, of the different physics variants (left to right) in the Au6 simulation in face-on and edge-on projections. The projection depth is $25$ kpc.}
\label{fig:gas}
\end{figure*}


\begin{figure*}
\vspace{-.25 cm}
\begin{center}
\includegraphics[width=1.0\textwidth]{./plots/auriga_radial_profiles_L6_lvl4.pdf}
\includegraphics[width=1.0\textwidth]{./plots/auriga_radial_profiles_L8_lvl4.pdf}
\end{center}
\vspace{-.6cm}
\caption{Radial profiles of gas density, magnetic field strength, CR-to-thermal pressure ratio, and CR pressure and energy density (left to right) in cylindrical shells of height $\vert z\vert=3$~kpc and width $\Delta r_{xy}=1$ kpc. The upper panels show results for the model galaxy Au6 and lower panels for AuL8. Different physics variants are shown with differently colored lines. The fiducial AURIGA run is shown with a black line, the Alfv\'en run in red, the CR diffusion run in magenta and the CR advection run in blue.}
\label{fig:prof}
\end{figure*}


\section{Galaxy disc properties} \label{sec:props}


\subsection{Stellar disc}

In Fig.\ \ref{fig:rgb} we present face-on and edge-on projections of all eight simulations at $z = 0$ where the upper panels show the four different simulations of the Au6 halo and lower panels of AuL8. The images are a composition of the K-, B- and U-band luminosities (mapped to the red, green and blue colour channels), which indicate the distribution of younger (bluer colours) and older (redder colours) star particles, respectively. All simulations reveal a star-forming disc component with additional clear non-axisymmetric structures such as bars and spiral arms. For the fiducial noCR and the CRdiffalfven model the stellar disc is radially extended and thin. In contrast to that, the CRdiff and CRadv models result in more compact stellar discs further indicated by the lower ratio of D/T between the stellar disc mass (D) and the total stellar mass of the galaxy (T) shown in the lower right corner of each panel. Nevertheless, from the edge-on view these simulations are still identifiable as disc galaxies.

Despite the obvious differences in morphology, the total stellar mass in each model is almost the same as shown by Fig.\ \ref{fig:sfr}. Here, we investigate the star formation history (SFH) and the stellar mass-halo mass relation of the galaxies in the four different models. The SFH of halo Au6 is shown in the left most panel, the one of halo AuL8 in the middle panel and the right panel shows the stellar mass as function of the halo mass in comparison to abundance matching results from \citet{Moster2013}. Within their intrinsic scatter, the SFHs of each physics run do not differ much and the total stellar mass of each galaxy varies only within a factor of $\lesssim1.5$ at the end of each run (see also Table \ref{tab:props}). For halo AuL8 we find that the CRs slightly suppress the stellar mass growth already at early times ($z\sim2$) as can be appreciated from the suppressed SFR (middle panel). At redshift zero the CRdiffalfven run has very similar stellar mass compared to the AURIGA run caused by a late time ($z\lesssim0.2$) burst in star formation (see red line in the middle panel) whereas the CRdiff and CRadv run have slightly lower stellar mass. Halo Au6 on the other hand acquires slightly more stellar mass in the CR runs compared to the fiducial AURIGA model, at late times ($z<0.5$) also visible in the enhanced late time SFR (left panel of Fig.\ \ref{fig:sfr}) for the CR runs. 

We further examined the total amount of gas as well as the cold gas mass and found that both quantities do not change much across the different physics variants. We conclude that in our simulations structural disc properties can be significantly changed by CRs, global stellar properties, however, are robust across different CR physics variants and are not much affected by CRs.

\subsection{Gas disc}
\label{subsec:gasdisc}
The differences in stellar morphology are mainly a result of differences in the gaseous properties of the central galaxies. Figure \ref{fig:gas} shows face-on and edge-on projections of the gas surface density (upper panels), the magnetic field strength (middle panels) and the ratio of CR-to-thermal pressure, $X_{\rm cr}$, (lower panels) of the central gaseous disc of Au6. From left to right we show the noCR, the CRdiffalfven, the CRdiff and the CRadv run. The magnetic field is calculated as $B=\sqrt{\left<B^2\right>_V}$ and the value of $X_{\rm cr}$ is calculated as the ratio of volume-weighted pressure terms, $X_{\rm cr} = \left<P_{\rm cr}\right>_V / \left<P_{\rm th}\right>_V$. Figure \ref{fig:prof} complements this by showing the radial profiles in cylindrical bins of radial width $1$ kpc and height $\pm3$ kpc centered on the disc mid-plane for both galaxies Au6 (upper panels) and AuL8 (lower panels). The profiles have been obtained by adding up thermodynamic extensive quantities. This means, that we have volume-weighted \mbox{(energy-)}densities within a given concentric shell, while temperature averages are calculated via mass weighting.

Comparing the surface density maps of the four different physics variants we find that the CR runs show a more centrally concentrated, thicker gas disc with slightly higher surface densities within the disc region. This is further highlighted by the larger central densities in the left panels of Fig.\ \ref{fig:prof}. Furthermore, from Fig. \ref{fig:gas} we see how the CR pressure smoothes out density features in the disc, particularly in the CRadv and CRdiff runs; the CRdiffalfven run most closely resembles the fiducial AURIGA run. The CRadv run exhibits the most compact, thick and smooth gas disc where the additional CR pressure stabilizes and smoothes the gas. The lower panels of Fig.\ \ref{fig:gas} and the third panels from the left of Fig.\ \ref{fig:prof} show that in all three CR models the thermal pressure in the gas disc is sub-dominant in comparison to the CR pressure. This effect is most prominent in the CRadv and CRdiff runs where CRs can only cool adiabatically via Coulomb and hadronic interactions. This results in CR-to-thermal pressure ratios of $P_{\rm cr}/P_{\rm th}=X_{\rm cr}\gtrsim10$ in the disc region.

We observe slightly higher CR-to-thermal pressure ratios in the CRdiff compared to the CRadv run. At first, this result might be surprising because the CRs in the diffusion run are able to diffuse out of the disc into the halo. The reason for this is as follows: if $\vert \bnabla (P_{\rm cr} + P_{\rm th}) \vert > \vert \rho \bnabla \Phi \vert$ (where $P_{\rm th}$ is the thermal pressure and $\Phi$ is the gravitational potential), then the composite of CRs and thermal gas adiabatically expands and as a result the CR pressure will exceed the thermal pressure because of its softer equation of state; the CR pressure decreases at a slower rate in comparison to the thermal pressure. As the CRs diffuse above and below the galaxy midplane, they push gas out of the disc (via their gradient pressure force), thus lowering the gas density in the disc (see left panels of Fig.~\ref{fig:vert_prof}). Because the temperature in the star forming regions is set by the effective equation of state, the thermal pressure in the CRdiff model is lower (see right most panels of Fig.~\ref{fig:vert_prof}), and hence the ratio of $P_{\rm cr}/P_{\rm th}=X_{\rm cr}$ is larger in this model.

In the CRdiffalfven run on the other hand, the CRs are allowed to diffuse and to cool via the Alfv\'en wave cooling mechanism and thus their stabilizing pressure is less dominant compared to the CRadv run. Here we find typical values of $X_{\rm cr}$ ranging from three to ten within the central $5-10$ kpc of the disc (see Fig.\ \ref{fig:prof}). This allows for a shallower radial density profile within the disc that is then able to grow larger stellar discs. The right-most panel of Fig.\ \ref{fig:prof} shows the radial profiles of the disc CR pressure and the energy density, respectively, averaged over a height of $|z|=3$~kpc. The CR energy density is decreasing with radius and follows in general the shape of the gas density profile with breaks at $r_{xy}\sim14$~kpc and 20~kpc for Au6 and AuL8, respectively. At the solar radius ($\sim8$ kpc) the value of the CR energy density is $\epsilon_{\rm cr}\sim4-5$~eV~cm$^{-3}$ for Au6 ($\epsilon_{\rm cr}\sim2$~eV~cm$^{-3}$ for AuL8) in the CRdiffalfven model while the other two models show a factor of $\gsim3$ higher values. A more detailed comparison to observations follows in Section~\ref{sec:obs_comp}.
We caution that some of the drastic differences of the density profiles between the different variants of CR physics may be due to cosmic variance and the different accretion histories. In particular the differing density profiles in Au6 (Fig.\ \ref{fig:prof}) are reduced in AuL8 where the CRdiffalfven run's density profile follows much more closely the fiducial AURIGA run.

The middle panel of Fig.\ \ref{fig:gas} shows the magnetic field strength which looks very similar in the disc for all CR runs but varies drastically from the fiducial AURIGA runs which show a much smoother, ordered magnetic field. Looking at the second panels of Fig.\ \ref{fig:prof} we find that the magnetic field strength at the edge of the stellar disc ($\sim20$ kpc) is around 1.5 to 2.5 $\mu$G (except for the fiducial Au6 run which shows values of $\sim6$ $\mu$G). The magnetic field strength in our models increases towards the central regions of the disc. In the inner disc (at radii of $\lesssim5$ kpc) we find magnetic field strengths of $\gtrsim7-15~\mu$G ($\gtrsim9-20~\mu$G for Au6) depending on the CR model at hand.
These values of the magnetic field strength are in good agreement with estimates for the MW \citep{Haverkorn2006,Haverkorn2015,Sun2012,Pakmor2018} and local disc galaxies \citep{Beck2019}. In detail, the CR runs show a more structured magnetic field which follows closely the structure of the gas disc because the CRs act as a local feedback source while in the noCR run the wind feedback is non local. Thus, CRs are able to inject turbulence in the gas disc, imprinting more small scale structure onto the magnetic field. This feature is absent in the noCR runs and thus the magnetic field appears much more ordered. However, the halo magnetic field looks very different between the three CR runs. Interestingly, the vertical magnetic field extending into the halo in the CRdiff and CRadv is larger compared to the CRdiffalfven and AURIGA runs. Thus, we conclude that CR dynamics alters the dynamo process in comparison to pure MHD simulations and the higher density features in the CR runs lead to a more structured magnetic field in the gas disc (see also middle panels of Fig.\ \ref{fig:prof}). We note that different variants of CR transport seem not to affect the disc magnetic field much.


\begin{figure}
\vspace*{-.5cm}
\hspace*{-.5cm}
\includegraphics[width=1.25\columnwidth]{./plots/auriga_ang_mom_time_profile_double.pdf}
\vspace{-.75cm}
\caption{Median specific angular momentum evolution of the gas that ends up in the stellar disc at redshift $z=0$. Au6 is shown in the left panel and AuL8 in the right. At lookback times of $t_{\rm lookback}\sim6$ Gyr AuL8 is undergoing a major merger causing the dip in the angular momentum evolution.}
\label{fig:ang_mom}
\end{figure}


\begin{figure*}
\vspace*{-.4cm}
\begin{center}
\includegraphics[width=\textwidth]{./plots/auriga_Pcr_Pkin_velocity_streamline_2_z03.pdf}
\end{center}
\vspace{-.35cm}
\caption{CGM gas flow patterns of the halo Au6 at redshift $z=0.3$ where the stellar disc is shown edge-on. Streamlines indicate the direction of gas flow and arrow colors indicate the sign of the radial velocity of the flow. Red colors signal outflowing gas and blue colors in-flowing gas, respectively. The thickness of the gas slice is $20$ kpc in the $y$-direction and the background color-coding shows the pressure ratio $(P_\CR + P_\rmn{th})/P_{\rmn{kin},\,r}$, i.e., the sum of CR and thermal pressure over the radial kinetic flux term in the Euler equation, $P_{\rmn{kin},\,r}=\rho \varv_r^2$.}
\label{fig:flow}
\end{figure*}


\begin{figure*}
\begin{center}
\includegraphics[width=.49\textwidth]{./plots/auriga_velocity_hist_av_L6.pdf}
\includegraphics[width=.49\textwidth]{./plots/auriga_velocity_hist_av_L8.pdf}
\end{center}
\vspace{-.7cm}
\caption{Time averaged mass weighted distribution of the gas radial velocity in two different $100$ kpc wide spherical shells for the Au6 halo in the two left most panels and the Au8 halo in the two right most panels. The CR runs show a suppressed tail towards large inflow velocities in comparison to the noCR runs. Time averaging is done for 10 snapshots in the redshift range $0.2<z<0.3$ corresponding to a time span of $t\sim1.3$ Gyr.}
\label{fig:flow2}
\end{figure*}


\subsection{Gas accretion onto the disc}
\label{sec:accr}

We have seen that the inclusion of CRs lead to more compact stellar and gaseous discs. In this section, we investigate the evolution of the angular momentum of gas that ends up in stars in the central galaxy at present-day. In practice, we make use of Lagrangian ``tracer particles'' \citep{Genel2013,Grand2019} to follow the motion of resolution elements over time. At the beginning of each simulation, each gas cell in the high-resolution region is assigned a tracer particle with a unique ID. A tracer particle in any given cell moves to a neighbouring cell with a probability proportional to the outward mass flux across a cell face. Usually, a tracer particle has the highest probability  to remain in the same cell, because the moving-mesh nature of AREPO means that cells follow the bulk gas flow as closely as possible.  

Following the median angular momentum of gas which is in stars at redshift $z=0$ back in time (Fig.\ \ref{fig:ang_mom}) we find that CRs suppress the acquisition of angular momentum after the time of disc formation ($t_{\rm lookback}\sim5-8$ Gyr). The suppression is strongest for the CRadv and CRdiff runs while the CRdiffalfven run more closely follows the fiducial runs. At present-day, the CR simulations have acquired a factor of $\sim2-5$ times less specific angular momentum which manifests itself in more compact discs. For AuL8, the differences between the noCR run and the CR runs are smaller and the CRdiffalfven run matches the angular momentum of the noCR run. At a lookback time of $6$ Gyr this galaxy undergoes a merger which masks most of the differences in angular momentum distribution between the different physics runs. The evolution of the median angular momentum is indeed representative of the evolution of the entire distribution of gas angular momentum, as can be verified in Fig.~\ref{fig:ang_mom2}.

The angular momentum acquisition of the galaxy is most efficient if the accreted gas from large scales is undisturbed and flows to the central gas disc. When a large gas disc is first forming in the fiducial AURIGA model the wind feedback model develops outflows perpendicular to the stellar/gas disc (e.g. left panel in Fig.\ \ref{fig:flow}), which is an emergent phenomenon that is the result of the outflow taking the path of least resistance away from the galaxy \citep{Grand2019}. As we have discussed earlier (Fig.\ \ref{fig:gas}), in the CR runs the additional CR pressure support inflates the gas disc and thus enhances the gas density above and below the disc. The gas morphology in these runs is thus less discy and the particular implementation of the wind model results in more spherically symmetric gas flows that are less coherent in the perpendicular direction of the disc as we exemplify in Fig.\ \ref{fig:flow}. This figure shows the gas flow pattern in form of stream lines at redshift $z=0.3$ (corresponding to a lookback time of $\sim3.5$ Gyr).

In order to compare the wind properties to the dominant pressure forces and assess whether CRs change the hydrodynamic halo properties, we overlay the streamlines of Fig.\ \ref{fig:flow} on a colour map that shows the pressure ratio $(P_\CR + P_\rmn{th})/P_{\rmn{kin},\,r}$, where $P_{\rmn{kin},\,r}=\rho \varv_r^2$ is the radial kinetic flux term in the Euler equation. This can be seen by looking at at the momentum equation of an ideal fluid in the presence of CRs, which reads as follows:
\begin{equation}
  \frac{\partial \rho \bvel}{\partial t}
  + \bnabla\bcdot \left(\rho \bvel \bvel^\rmn{T} + P\mat{1} - \bB\bB^\rmn{T}\right)
  = -\rho\bnabla\Phi,
  \label{eq:euler}
\end{equation}
where $P=P_\CR + P_{\rmn{th}} + P_\rmn{mag}$ and $\mat{1}$ is the unit rank-two tensor. Converting this equation to spherical coordinates and neglecting the pressure and tension term of subdominant magnetic fields, the radial momentum flux density is given by $\rho \varv_r^2 + P_\CR + P_{\rmn{th}}$ \citep{Mihalas1984}. Figure \ref{fig:flow} shows that the divergence in the stream lines ($\bnabla \bcdot\bvel < 0$) corresponds to a shock where kinetic energy is converted into thermal energy.

The noCR simulation in Fig.\ \ref{fig:flow} shows coherent outflows along the direction of the spin axis of the disc, which enable flow channels to open up in the perpendicular direction along which low-metallicity gas can be coherently accreted to the central disc (Pakmor et al. in prep.). Whereas the gas flow in the CRdiffalfven run more closely resembles the flow pattern of the fiducial AURIGA model, the direction of the outflows is less ordered and not always perpendicular to the disc. The CRdiff and CRadv runs on the other hand show more spherical, slower outflows which shock the inflowing gas at a distance of $r\sim100-200$~kpc, shutting off the coherent gas inflows to the central galaxy while at the same time preventing coherent outflows to large distances. Therefore, these models exhibit a more quiescent, hydrostatic atmosphere in the halo in comparison to the former two models as can be see from the larger ratios of $(P_\CR + P_{\rmn{th}})/P_{\rm{kin,}\, r}$ (yellow colors in Fig. \ref{fig:flow}). We verified that this result remains qualitatively similar over the entire redshift range $0\leq z\lesssim1$ and we quantified the hydrodynamic effect of CRs on the gas flow with the distribution of radial inflow and outflow velocities in Fig.\ \ref{fig:flow2}.

The simulations with CRs show narrower radial velocity distributions indicating reduced inflow and outflow velocities. This is the result of the process described above: the more elliptical and vertically expanded ISM in the CR runs precludes a geometrically preferred path of least resistance and slows down the outflows in all directions. Hence there are no coherent outflows forming along the spin axis in the CR simulations. Because there are low-velocity outflows present in nearly all directions in these CR runs, this shocks the accreting gas and precludes the formation of most inflow channels that deliver gas from larger distances to the star forming disc. The suppression of the inflow velocities is strongest for the CRdiff run because here CRs impact a larger region compared to the other runs. Interestingly, in the CRdiffalfven simulation we observe reduced infall velocities but similar or even larger outflow velocities in comparison to the fiducial AURIGA model due to the additional CR pressure-driven winds.

The immediate manifestation of this process is the suppression of the accretion of gas from larger distances in the CR runs as displayed in Fig.\ \ref{fig:dist}. In this figure we show the distribution of radial distances at a lookback time of $5$ Gyr of the gas which is at present-day converted into stars of the stellar disc. In the fiducial AURIGA runs gas is accreted from farther away in comparison to the CR counterpart simulations. This is the result of the modified gas accretion pattern on large scales mediated by the effects of the CRs on the structure of the gaseous disc on smaller scales. Note, this does not necessarily mean that SFRs in the CRdiff and CRadv run are suppressed as the outflow velocities are similarly reduced in these runs leaving enough gas to fuel star formation.

In summary, in the fiducial AURIGA and the CRdiffalfven simulations gas accretes relatively unimpeded from large distances whereas in the CRadv and CRdiff simulations the more spherically symmetric outflows and the CR pressurised gaseous haloes are able to held up the gas.

\section{Circum-galactic medium} \label{sec:CGMprops}


\begin{figure}
\vspace*{-.5cm}
\begin{center}
\includegraphics[width=1.2\columnwidth]{./plots/auriga_ang_mom_profile_double.pdf}
\end{center}
\vspace{-.75cm}
\caption{Radial distribution at a lookback time of $5$ Gyr of the gas tracers which at present-day are contained in stars. This figure traces the origin of the gas that evolves into stars at the present day. For AuL8 in the right panel, the peaks around $R\sim140$ kpc signal an ongoing merger.}
\label{fig:dist}
\end{figure}


\begin{figure*}
\begin{center}
\raggedleft
\includegraphics[width=\textwidth]{./plots/auriga_gas_density_large_side_L6_lvl4.pdf}
\includegraphics[width=\textwidth]{./plots/auriga_gas_temp_large_side_L6_lvl4.pdf}
\includegraphics[width=.875\textwidth]{./plots/auriga_xcr_large_side2_L6_lvl4.pdf}
\includegraphics[width=.875\textwidth]{./plots/auriga_ecr_large_L6.pdf}
\end{center}
\vspace{-.35cm}
\caption{Maps of CGM properties for the four different physics runs of halo Au6 as indicated in each panel. From top to bottom we show the gas surface density, the gas temperature, the CR-to-thermal pressure ratio and the CR pressure. The orientation of each panel is chosen to view the central disc edge-on and the projection depth of each panel is equal to its width, $200$ kpc. Note the smooth gas distribution in the CR runs owing to the additional pressure of the CRs, which however differs considerably for our different variants of CR transport.}
\label{fig:CGMgas}
\end{figure*}


\begin{figure*}
\vspace*{-.3cm}
\begin{center}
\includegraphics[width=\textwidth]{./plots/auriga_radial_profiles_large_av_L6_lvl4.pdf}\vspace{-.15cm}
\includegraphics[width=\textwidth]{./plots/auriga_radial_profiles_large_av_L8_lvl4.pdf}
\end{center}
\vspace{-.35cm}
\caption{Comparison of the radial profiles of the gas density (left), gas temperature (middle panel) and magnetic field strength (right) in the four different physics variants of Au6 (upper panel) and AuL8 (lower panels). The profiles represent time averages over the last 10 simulation outputs ($z\sim0.1$, $t\sim1$ Gyr).}
\label{fig:CGMprof}
\end{figure*}



We now turn to analyse the effects of CRs on the CGM properties in the different physics variants. In particular, we focus on how CRs shape the gas density distribution and impact the gas temperature profile of CGM gas by providing additional pressure support which manifests itself in high ratios of CR pressure to thermal pressure. To this extent we show in Fig.\ \ref{fig:CGMgas} from top to bottom maps of the gas surface density, the gas temperature, the ratio of CR pressure to thermal pressure and the value of CR pressure for galaxy Au6. The orientation is chosen such that the gas disc is seen edge-on and the projection depth is the same as the horizontal/vertical extend ($200$ kpc). For a more quantitative comparison we accompany the maps by profiles of the same quantities for both galaxies in Fig.\ \ref{fig:CGMprof} as indicated by the panels' titles. For the CGM properties we have chosen logarithmically spaced spherical shells and averaged profiles over the last ten simulation outputs ($\sim1$ Gyr). Color-coding of the different physics variants is the same as in previous figures.

Looking at the first row of Fig.\ \ref{fig:CGMgas} and comparing the four different physics variants, we find that the CGM gas surface density in the CR runs outside the disc region ($R>50$ kpc) is slightly higher in comparison to the fiducial AURIGA model. Most strikingly, the CGM gas density is significantly more spherical within $50$ kpc in the CRdiff and CRadv runs compared to the fiducial AURIGA  and the CRdiffalfven runs. Additionally, the CGM gas density is smoother in the CR runs in comparison to the noCR run. In the next subsections we address these morphological differences and highlight how CRs cause these changes of the CGM structure by investigating each model separately.

\subsection{AURIGA -- no cosmic rays}
The baseline model for our comparison is the fiducial AURIGA model which has a highly structured CGM with cool, high density patches coexisting next to hot low density regions (left panels in Fig.\ \ref{fig:CGMgas}, see also \citealt{Vandevoort2019}). The clumpy CGM morphology is further reflected in a broad gas density distribution with tails to large gas densities as shown in Fig.\ \ref{fig:CGMhist}. Here we show the density distribution of all four runs in two different concentric shells of width 50 kpc as indicated by the panel titles. The left two panels show galaxy Au6, the right two AuL8.

\begin{figure*}
\vspace*{-.4cm}
\hspace{-.4cm}
\includegraphics[width=.515\textwidth]{./plots/auriga_density_hist_L6_lvl4.pdf}
\hspace{-.4cm}
\includegraphics[width=.515\textwidth]{./plots/auriga_density_hist_L8_lvl4.pdf}
\vspace{-.35cm}
\caption{Distribution of gas density in two different $50$ kpc wide spherical shells for the Au6 halo in the two left most panels and the Au8 halo in the two right most ones. The CRadv and CRdiff runs exhibit narrower density distributions in comparison to the other two models indicating a smoother CGM.}
\label{fig:CGMhist}
\end{figure*}


\subsection{Cosmic ray advection}
The most simple approximation for CR transport physics is the advection of CRs, which neglects all active CR transport processes. Therefore, all CRs in the CGM have been transported there by outflows. In the CRadv run the CGM is significantly more spherical within $50$ kpc, smoother and of slightly lower temperature (right panel of Fig.\ \ref{fig:CGMgas}) when compared to the fiducial AURIGA run owing to the additional CR pressure. Especially far away from the disc at $R\sim50$ kpc the density is slightly enhanced in comparison to the fiducial run. The inclusion of CRs leads to cooler gas temperatures even at distances of $R\sim100$ kpc where CR pressure is approximately in equilibrium with the thermal pressure. The additional CR pressure smoothes out almost all small-scale high density peaks in the CGM gas which quantitatively leads to a narrower gas density distribution in Fig.\ \ref{fig:CGMhist}.

\subsection{Cosmic ray diffusion}
Allowing for CR anisotropic diffusion alters the properties of the CGM dramatically but with similarities to the CRadv run. We find that the CGM gas density is even more spherical within $50$ kpc compared to the CRadv run, highly CR pressure dominated and of much cooler temperatures. In this run, CRs are allowed to diffuse and thus are able to affect the CGM at larger distances from the disc, thus the CR pressure dominated halo is larger in size compared to the CRadv run. Furthermore, the CR pressure contribution is higher compared to the CRadv run (see also discussion in Sections \ref{subsec:gasdisc} and \ref{subsec:dis}) leading to an even smoother CGM (see narrow gas distribution in Fig.\ \ref{fig:CGMhist} for this run). The additional CR pressure which in this run dominates the CGM out to radii of $R\sim(50-100)$ kpc supports the gas against gravitational collapse in the absence of thermal pressure support and thus explains the low CGM temperatures which coincide with the regions where CR pressure dominates. Figure~\ref{fig:CGMprof} shows that in a region of $R<50$ kpc the CR pressure is a factor of $\sim10$ larger than the thermal pressure.


\begin{figure*}
\begin{center}
\includegraphics[width=\textwidth]{./plots/auriga_cr_pressure_profiles_log_L6_lvl4.pdf}
\includegraphics[width=\textwidth]{./plots/auriga_cr_relative_pressure_profiles_log_L6_lvl4.pdf}
\end{center}
\vspace{-.35cm}
\caption{Radial profiles of different pressure components for the four different physics variants of halo Au6. The upper panels compare absolute pressure profiles of the magnetic ($P_{B}=\bB^2/(8\pi)$, orange line), ``kinetic'' ($P_{\rm kin}=\rho \bvel^2/2$, red), thermal ($P_{\rm th}=(\gamma-1)\eps_{\rm th}$, blue) and CR ($P_{\rm cr}=(\gamma_{\rm cr}-1)\eps_{\rm cr}$, green) components as well as the total pressure (black). Lower panels compare the relative pressure contributions to the total pressure.}
\label{fig:pressure}
\end{figure*}


\begin{figure*}
\vspace{-.25cm}
\begin{center}
\includegraphics[width=\textwidth]{./plots/auriga_phase_CGM_L6.pdf}
\end{center}
\vspace{-.35cm}
\caption{Present-day temperature-density phase diagrams of CGM gas ($50<R<200$ kpc) for the four different physics runs of galaxy Au6. Color-coding shows the logarithm of the mass weighted probability density where red colors indicate high and blue colors low probability.}
\label{fig:phase}
\end{figure*}


\subsection{Cosmic ray diffusion with Alfv\'en wave cooling}
The CRdiffalfven model in turn reveals a CGM morphology similar to the fiducial AURIGA model (although with noticeable differences) and shows clear differences in comparison to the CRdiff and CRadv run. In comparison to the AURIGArun (CR runs) the CGM features smoother (more structured) density peaks and the central CGM appears discy. Again, the additional CR pressure explains the smoother gas density compared to the AURIGA model while the Alfv\'en cooling explains the weaker damping compared to the CRdiff and CRadv runs. Therefore in Fig.\ \ref{fig:CGMhist}, the gas density distribution of the CRdiffalfven run lies between the AURIGA model and the other two CR models. 

In comparison to the CRdiff run the CGM is much hotter and shows even less cold regions compared to the fiducial run. In fact, the profiles (middle panels in Fig.\ \ref{fig:CGMprof}) show that Au6 has a hotter CGM in the CRdiffalfven run compared to the AURIGA run. AuL8 on the other hand shows a similar CGM temperature. The slightly enhanced CGM temperature in Au6 is presumably due to the additional Alfv\'en heating of the CRs and the fact that Au6 had a recent burst of SF injecting CRs into the CGM. This burst is not present in AuL8. From the lower panels in Fig.\ \ref{fig:CGMgas} we can further see that the CR pressure dominates the central regions (except for the disc) while for most of the CGM gas the CR pressure is in equilibrium with the thermal pressure (see also right panels in Fig.\ \ref{fig:CGMprof}). This is different to the other CR runs and a manifestation of the Alfv\'en wave cooling in which the CRs lose an e-folding of their initial energy as they diffuse a scale height into the CGM. 

\subsection{CGM pressure support}
In Fig.\ \ref{fig:pressure} we compare in detail the different pressure components (upper panels) and their contribution to the total pressure (lower panels) in the CGM. We compare magnetic pressure (orange), ``kinetic pressure'' (red), thermal pressure (blue) and CR pressure (green) to the total pressure (black) in spherical shells as we have explained for Fig.\ \ref{fig:CGMprof}. From left to right we show the fiducial AURIGA noCR, the CRdiffalfven, the CRdiff and the CRadv runs. 

In the region influenced by accretion onto the disc as well as the disc itself (within a few tens of kpc) the gas is rotationally supported (i.e., has a dominating kinetic pressure) in all runs while the other pressure components, i.e. thermal, magnetic and CR pressure, are roughly in equipartition contributing each about $\sim10$\% to the total pressure. In the outskirts at radii larger than $R>20$ kpc we find that all the runs become increasingly thermal pressure dominated except for the CRdiff run where CR pressure dominates and the thermal pressure becomes negligible. Only close to the virial radius the thermal pressure becomes important again which was already noted by the huge CR pressure dominated halo in Fig.\ \ref{fig:CGMgas}. This confirms our previous findings that the cool region in the CGM at these radii is entirely CR pressure dominated.


\subsection{Temperature-density relation}
Our findings for the structure and morphology of the CGM are summarized in Fig.\ \ref{fig:phase} showing the temperature-density distribution of the CGM gas in the radial range $50<R<200$ kpc. While at first glance differences between the fiducial run and the CR runs are small, one notices that the hot phase in the CR runs tends to inhabit regions of lower temperature. In more detail, we find that the CRdiffalfven runs show a larger spread in temperature at any density compared to the noCR run which shows the importance of CR Alfv\'en heating. This figure further shows that for the CRdiff run gas at $\rho\sim(10^{-4} - 10^{-3})$~cm$^{-3}$  piles up at a temperature of $T\sim10^{4.5}$ K. We interpret this as the CR pressure keeping this gas from falling onto the main galaxy. Finally, we find that in the CRadv run the CR pressure causes a different slope of the $\rho-T$ relation for the non-starforming gas at $T\sim10^{4}$ K due to the CR pressure support at the disc-halo interface.

Thus, CRs do not only affect the properties of the gas disc as we have seen in Fig.\ \ref{fig:gas} but also the gas morphology of the CGM even at large distances close to the virial radius of the halo. The analysis in this section reinforces the need to better understand the physics of CR transport, as we have shown here that different variants of approximating it have a strong impact on the stellar structure and especially the properties of the CGM.

\section{Far-infrared$-$gamma-ray relation}
\label{sec:obs}


Finally, after establishing the differences and similarities between the three variants of CR transport in our simulations we connect our results to the most directly observable CR proton properties of galaxies, namely hadronic gamma-ray emission that arises from inelastic collisions of CRs with the ambient ISM. To this end we compare in Fig.\ \ref{fig:gamma} the gamma-ray luminosities in the Fermi band ($0.1-100$GeV) to the SFR for all the main disc galaxies (big black bordered symbols) in comparison to observational data as indicated in the caption. Additionally, we also show gamma-ray luminosities for the dwarf galaxies within the zoom region (small coloured dots). Note that these dwarfs are not satellite galaxies as they are not part of the main halo but proxies of field dwarfs in the Local Volume. Note that our homogeneous observational sample in the Fermi band spans an energy range of $0.1-100$~GeV leading to somewhat higher gamma-ray luminosities in comparison to previous observational data collections \citep[cf.][who use the smaller energy band between $1-100$~GeV]{Lacki2011}.

The total far-infrared (FIR) luminosity ($8-1000~\mu$m) is a well-established tracer of the SFR of spiral galaxies \citep{Kennicutt1998} with a conversion rate \citep{Kennicutt1998ARA+A} 
\begin{equation}
  \label{eq:FIR-SFR}
  \frac{\rmn{SFR}}{\rmn{M}_\odot~\rmn{yr}^{-1}}=\epsilon\,1.7\times10^{-10}\,\frac{L_{8-1000\,\mu\rmn{m}}}{L_\odot}.
\end{equation}
This SFR-FIR conversion assumes that thermal dust emission is a calorimetric measure of the radiation of young stars, and the factor $\epsilon=0.79$ derives from the \citet{Chabrier2003} IMF \citep{Crain2010}. While this conversion is reliable at $L_{8-1000\,\mu\rmn{m}}>10^9~L_\odot$, it becomes progressively worse at smaller FIR luminosities due to the lower metallicity and dust content, which implies a low optical depth to IR photons and invalidates the calorimetric assumption \citep{Bell2003}. We have verified that down to SFRs comparable to those of M31 ($\sim0.3-0.4$~\Msun~\rm{yr}$^{-1}$) our conversion still holds. The SFR of M31 derived here is in good agreement with SFRs derived using a combination of H$\alpha$ and 24~$\umu$m emission, a combination of far-ultraviolet, 24~$\umu$m, and the total infrared emission which yield $\sim(0.35-0.4)\pm0.04$~\Msun~\rm{yr}$^{-1}$ \citep[see][]{Rahmani2016}.

We show the observed FIR luminosity of the LMC and SMC with gray data points in Fig.\ \ref{fig:gamma}, while SFR estimates for the LMC and SMC are shown with solid black data points. Those are derived by combining H$_\alpha$ and FIR emission \citep[assuming a Chabrier~IMF,][]{Wilke2004} or UVBI photometry \citep{Harris2009} and range for the SMC between 0.036 and 0.1 \Msun yr$^{-1}$, while the LMC forms $0.2~\Msun~\rmn{yr}^{-1}$ of stars \citep{Harris2009}. We refer the reader to \citet{Pfrommer2017b} for more details on the FIR-to-SFR conversion and to \citet{Pfrommer2004} for the computation of the gamma-ray emission resulting from hadronic proton interactions with the ambient ISM.


\begin{figure}
\begin{center}
\includegraphics[width=\columnwidth]{./plots/SFR_gamma_all.pdf}
\end{center}
\vspace{-.35cm}
\caption{Correlation of the gamma-ray luminosity ($L_{0.1-100\ \rm GeV}$) with the SFR and with the FIR luminosity ($L_{8-1000\ \rm mm}$) of star-forming galaxies. We compare our six simulated CR models (differently coloured symbols delineate Au6 and AuL8 haloes) and plot central galaxies (thick symbols) and dwarf galaxies (thin symbols). Upper limits on the observable gamma-ray emission by Fermi-LAT (open gray symbols; \citep{Rojas-Bravo2016}) are contrasted with gamma-ray detections from star-forming galaxies only (solid black) and with AGN emission (filled gray); data are taken from \citet{Ackermann2012}, except for NGC 2146 \citep{Tang2014} and Arp 220 \citep{Griffin2016,Peng2016}.
For the two lowest SFR galaxies, the SMC and LMC we use gray data points for the observed FIR luminosities (top axis) and solid black data points for the SFRs derived using H$_\alpha$ and UVBI photometry.
Note that only the CRdiffalfven runs fall on the best-fit observational FIR–gamma-ray correlation (orange).}
\label{fig:gamma}
\end{figure}


We find that at the MW mass scale or vice-versa at a FIR luminosity of $\sim5\times10^{10}$ L$_{\odot}$ the three CR models predict very different gamma-ray luminosities. Whereas the CRdiffalfven runs are in good agreement with the observational data, especially for NGC253 and M82, the other two models lie above the best fit observational relation of \citet{Rojas-Bravo2016} and are barely consistent with upper limits from Fermi-LAT (gray open symbols). This is interesting as the same physical models for CR transport (CRadv and CRdiff) yield hadronic gamma-ray luminosities in agreement with the best fit observational relation for the collapsing halo initial conditions \citep{Pfrommer2017b}. Unlike our cosmological simulations, these idealized setups produce more extended gas and stellar discs. As a result the inner gravitational potential due to the kinematically cold distribution of central stars is shallower in these idealized models in comparison to our cosmological simulations in the CRdiff and CRadv models. Such a massive central stellar distribution in our cosmological simulations injects too many CRs and compresses the gas to a level that the CRs overproduce the hadronic gamma-ray emission. Clearly, different types of feedback (radiation, supernovae) are necessary to act in tandem with CRs to prevent the formation of these dense cores at high redshift in our cosmological simulations.

At lower SFRs and accordingly lower FIR luminosities, the differences between the three models become smaller and both the CRdiffalfven and the CRadv model agree with the observed relation of \citet{Rojas-Bravo2016} while the CRdiff model produces a somewhat elevated level of gamma-ray luminosity. We attribute this behaviour to the lower injection rate of CRs at these SFRs and higher escape fractions of CRs from these low mass galaxies \citep[][]{Zhang2019}. Our CRdiffalfven model is able to explain the distribution of luminosities of the MW and M31. In particular at SFRs comparable to M82 and NGC~253 only this model matches observed gamma-ray luminosities. 

However, the simulations of all CR models in Fig.~\ref{fig:gamma}, including the CRdiffalfven model, do not agree with the observed SFRs and gamma-ray luminosities of the LMC and SMC. This can either be interpreted as an overprediction of the simulated gamma-ray luminosity (by a factor of about $2-2.6$ for the LMC and about $6-8$ for the SMC depending on the specific simulated counterpart), an underprediction of the simulated SFR (by a factor of $2-10$ for the LMC and $2-6$ for SMC), or a combination of both. There are several reasons that could be responsible for this. Among those is the omission of CR streaming, which will be studied in a forth-coming paper. In fact, CR streaming in combination with diffusion might lead to an increased effective CR transport speed which might decrease the gamma-ray luminosity in dwarf galaxies \citep[e.g.][]{Salem2016,Jacob2018,Chan2019}. Moreover, we would like to emphasise that the LMC and SMC are satellite galaxies of the MW, which are most likely at their first infall and in an interacting state \citep{Harris2009}. This might lead to higher SFRs in comparison to the isolated, non-satellite dwarf galaxies studies in Fig.\ \ref{fig:gamma}. While the CR production scales with the SFR and accordingly the gamma-ray luminosity, it is a matter of time-scales involved whether the gamma-ray luminosity closely follows the measured SFR \citep[which is intrinsically uncertain by a factor of two for the LMC, see e.g.,][]{Harris2009}. In fact the LMC SFR shows a strong burst in the last few $10^7$ yr \citep[cf.\ Fig. 11 of][]{Harris2009}. For such a recent burst we do not expect that the gamma-ray production had time to fully react as the lifetime of galactic CRs is of the order of $\gtrsim3\times10^7$ yr \citep[e.g.,][]{Simpson1988,Lipari2014}.

Finally, our simulated points agree with the mean power-law relation. It is unclear whether the true observational relation continues along the power law relation with the LMC and SMC representing outliers or whether the lower gamma-ray luminosities of the LMC and SMC indeed signal a cutoff or change in slope of this relation.

We note that at low SFRs the CRadv model produces the lowest gamma-ray luminosities while at high SFRs this model shows the opposite effect. This outcome can be explained by a different behaviour of the adiabatic processes in halos of different masses. Figure~4 of \citet{Pfrommer2017} shows that adiabatic losses dominate in smaller halos corresponding to lower SFRs while adiabatic gains prevail in larger halos. Especially, at low SFRs the adiabatic losses in the CRadv model are larger compared to those in the CRdiff model which explains the low gamma-ray luminosities of this model at low SFRs.

Note that our CR models reproduce the observed relation for Galactic values of the diffusion coefficient ($\kappa_\parallel=10^{28}\,\rmn{cm^2~s}^{-1}$) or even without CR diffusion (for our CRadv model). This is in stark contrast to the FIRE-2 simulations that require diffusion coefficients of $\kappa_\parallel>3\times10^{29}\,\rmn{cm^2~s}^{-1}$ to be consistent with the gamma-ray observations \citep{Chan2019,Hopkins2019}. These differences can be traced back to the ISM model of AURIGA, which supports a CR transport with a single diffusion coefficient while the highly structured multi-phase ISM of FIRE-2 would require to fully model CR streaming with various wave damping mechanisms that dominate in the cold (ion-neutral damping) and warm/hot phases (non-linear Landau damping), respectively.

In conclusion, observable scaling relations such as the gamma-ray-FIR relation offer promising tools for distinguishing physically valid models and might help to constrain the values of free sub-grid model parameters \citep[e.g.][]{Buck2019}. However, a detailed model comparison to observations like, e.g., the CGM properties derived from the COS-HALOS survey \citep{Tumlinson2013} needs to include careful post-processing of the simulations which is outside the scope of this study and thus left for future research.


\section{Discussion} \label{sec:dis}


\begin{figure*}
\begin{center}
\vspace*{-.55cm}
\includegraphics[width=1.06\textwidth]{./plots/auriga_stellar_light_3e28.pdf}
\includegraphics[width=\textwidth]{./plots/auriga_gas_density_3e28.pdf}
\includegraphics[width=\textwidth]{./plots/auriga_xcr_large_side_3e28.pdf}
\end{center}
\vspace{-.35cm}
\caption{Face-on and edge-on projection of the stellar light at $z=0$ (upper panels), the gas surface density (middle panels), and the CR-to-thermal pressure ratio, $X_{\rmn{cr}}$ (bottom panels), for the CRdiffalfven and the CRdiff model for two different diffusion coefficients as indicated in the panels. See caption of Fig. \ref{fig:rgb} and \ref{fig:gas} for further details.}
\label{fig:high_kappa}
\end{figure*}


\subsection{Comparison to observations}
\label{sec:obs_comp}

For galactic CRs there exist three different observables which can be used for scrutinising our model predictions: the local energy density of CRs, the gamma-ray luminosity and the CR grammage, $X_s=\int_\rmn{source}^\rmn{observer}\rho_{\rm nuclei}{\rm{d}} l_{\rm CR}$. In this paper, we have calculated the former two observables because those can be robustly derived from our models whereas the latter quantity is a gyro-radii averaged integration of the target density along the path of individual CRs and strongly depends on a number of properties. Among those are the exact initial CR energies that are mapped onto the observed CR spectra, the CR source distribution, the variability of CR injection (both in time and space), the exact topology of the magnetic field (at large and small scales, which are necessarily unresolved in current cosmological simulations and would have to be modelled sub-grid) as well as the exact small scale density distribution in the ISM along the CR path. Therefore, deriving a meaningful value for the CR grammage in our simulations is beyond the scope of this paper and we leave such a detailed comparison for future studies.

Comparing our model observables for the local energy density of CRs (right panels of Fig.\ \ref{fig:prof}) we find that the CRdiffalfven model results in the lowest CR energy densities.
At the solar circle ($r_{xy}\sim8$ kpc) we find a value of $\epsilon_{\rm cr}\sim2$ eV~cm$^{-3}$ for AuL8 and $\epsilon_{\rm cr}\sim4-5$ eV~cm$^{-3}$ for Au6. A fair comparison to MW measurements needs to account for the fact that both model galaxies show slightly higher SFRs
of $\sim2$ for AuL8 and $\sim3$ for Au6 (see e.g., Fig.\ \ref{fig:gamma}) 
in comparison to MW rates of $\sim1.65-1.9$ \Msun yr$^{-1}$ \citep[e.g.,][]{Chomiuk2011,Licquia2015}. Because the SFR is directly proportional to the CR energy input in our models, we expect approximately 1.3 to two times lower CR energy densities in those models at SFRs comparable to the MW. This leaves us with CR energy densities of $\sim1.6$ eV~cm$^{-3}$ for AuL8 and $\sim2-2.7$ eV~cm$^{-3}$ for Au6. Given that inferred observational results are uncertain by a factor of $\sim2$ \citep[see e.g. discussion in][]{Cummings2016} we conclude that the rescaled CR energy densities of those models are in reasonable agreement with estimates for the MW of about $\epsilon_{\rm cr}\lesssim1.8$ eV~cm$^{-3}$ at the solar circle \citep[e.g.][]{Boulares1990,Webber1998,Cummings2016}. By contrast, the CRdiff and CRadv models show much higher CR energy densities.

We have extensively compared and discussed the gamma-ray luminosities of our model galaxies in comparison to observations in Fig.\ \ref{fig:gamma}. Above a SFR of $\sim 0.35$ \Msun yr$^{-1}$ (comparable to that of M31) our CRdiffalfven model is in excellent agreement with the Fermi observations in the energy band $0.1-100$~GeV. By contrast, the other two models show much higher gamma-ray luminosities and over-predict the observed values. Below a SFR of $0.35$ \Msun yr$^{-1}$ there are only two data points for the SMC and LMC, which have lower gamma-ray luminosities than our models. As discussed above, the SMC and LMC are satellites of the MW which might either have enhanced SFRs in comparison to non-satellite dwarf galaxies  \citep{Harris2009} effectively shifting them to the right in Fig.\ \ref{fig:gamma} or show lower gamma-ray luminosities in comparison to the observational relation of \citet{Rojas-Bravo2016} due to an increased gamma-ray variance at low SFRs. Therefore, the paucity of gamma-ray data at low SFRs precludes strong conclusions about the correctness of models and shows the importance of obtaining better observational gamma-ray constraints at low SFRs.

We conclude that the CR energy density at the solar circle and the gamma-ray luminosities in our CRdiffalfven model reproduce the observational data well while our CRdiff and CRadv models overpredict both observational constraints.

\subsection{CR diffusion coefficient}
\label{sec:kappa}

The pressure-carrying CR distribution at GeV energies propagates via streaming with Alfv\'en waves and diffusion relative to the frame of those waves \citep{Kulsrud1969,Amato2018,Thomas2020}. If this microscopic CR transport were modelled in the form of CR diffusion, this would imply a spatially and temporarily varying CR diffusion coefficient because of the varying Alfv\'en speed and the various wave damping mechanisms that dominate in different phases of the ISM and modulate the coupling strength of CRs to the ambient plasma \citep{Jiang2018,Thomas2019}: weak wave damping implies a strong coupling and causes the CRs to stream with Alfv\'en waves whereas strong wave damping leaves less waves to scatter CRs so that CR diffusion prevails. If this complex CR transport were modelled with a constant, isotropic CR diffusion coefficient, it assumes typical values for GeV CRs of $\kappa_\rmn{iso}\sim(1-3)\times10^{28}~\rmn{cm}^2~\rmn{s}^{-1}$ as we will argue below.

The relation between $\kappa_\rmn{iso}$ and the CR diffusion coefficient along the magnetic field, $\kappa_\parallel$, depends on the exact magnetic field configuration. For a turbulent field, $\kappa_\rmn{iso} = \kappa_\parallel/3$, whereas if CR transport along the homogeneous magnetic field dominates, we have $\kappa_\rmn{iso} = \kappa_\parallel$. For CR feedback to be active, CRs move along open field lines from the disc into the halo (that are either vertically aligned through an outflow or via the Parker instability) so that $\kappa_\rmn{iso} \sim \kappa_\parallel$ in the regime of interest where galactic winds are accelerated by the CR pressure gradient. There is evidence that CRs escaping from the disc into the halo excite the streaming instability \citep{Kulsrud1969,Evoli2018}, which limits the drift speed to that of the Alfv{\'e}n frame, which is $\sim30\mbox{km s}^{-1}$ at the disc-halo interface and increases to $\sim 300\mbox{km s}^{-1}$ at the virial radius. This coincides with the diffusion velocity $\varv_\rmn{diff}\sim30\mbox{ km s}^{-1} \kappa_{28}\, L_{3\,\rmn{kpc}}$, where $\kappa_{28}=10^{28}\mbox{ cm}^2\mbox{ s}^{-1}$ is the CR diffusion coefficient and $L_{3\,\rmn{kpc}}$ is the CR gradient length, justifying our choice of the CR diffusion coefficient.

A diffusion coefficient for CRs in the Galactic disc can also be estimated from CR propagation models and observations of synchrotron radiation and/or the ratio of secondary to primary nuclei \citep{Strong1998,Ptuskin2006,Ackermann2012,Tabatabaei2013,Amato2018}. In the Galactic halo, CRs have a scale height of $\sim3$ kpc and their residency time in the thick disc is inferred to be $\tau\sim3\times10^7$~yr as obtained from measurements of the ratio of secondary-to-primary CR nuclei \citep{Lipari2014,Evoli2020}. Thus, the diffusion coefficient of GeV CRs is given by $\kappa_\rmn{iso}\sim H^2/(3\tau)\sim3\times10^{28}~\rmn{cm}^2~\rmn{s}^{-1}$. This order of magnitude estimate for $\kappa_\rmn{iso}$ is confirmed by several different studies of CR propagation, including GALPROP simulations that aim to reproduce the Fermi gamma-ray sky \citep{Porter2017,Johannesson2019}. The recently discovered hardening of the momentum power-law slope of the CR proton spectrum at low Galactocentric radii could be a signature of anisotropic diffusion in the complex Galactic magnetic field with $\kappa_\parallel=1\times10^{28}\mbox{ cm}^2\mbox{ s}^{-1}$, as suggested by DRAGON2 simulations \citep{Cerri2017,Evoli2017}. Finally, the flux of unstable secondary CR nuclei in the recent AMS-02 data, produced by spallation processes in the ISM, can be used to constrain the residence time of CR inside the Galaxy, yielding identical values for the diffusion coefficient \citep{Evoli2019,Evoli2020}.

The exact numerical value of $\kappa_\parallel$ determines the CR diffusion timescale in the disc and thus controls the amount of dynamical impact of CRs on the galaxy. A large value of $\kappa_\parallel$ leads to short diffusion timescales and a quick escape of CRs from the disc. We have tested the impact of a three times higher value of the CR diffusion coefficient ($\kappa_\parallel=3\times10^{28}~\rmn{cm}^2~\rmn{s}^{-1}$) on the results obtained here. Figure~\ref{fig:high_kappa} shows from top to bottom face-on and edge-on projections of the stellar disc's light, the gas surface density and edge-on projections of the CGM CR-to-thermal pressure ratio, $X_{\rm cr}$ for the fiducial models (with $\kappa_\parallel=1\times10^{28}~\rmn{cm}^2~\rmn{s}^{-1}$) and those with a three times higher diffusion coefficient. We confirm the expectation that a higher value of $\kappa_\parallel$ leads to a smaller dynamical effect of CRs on galaxy properties such as the reduction in stellar disc size in the CRdiffalfven model (upper panel in Fig.~\ref{fig:high_kappa}). Here the faster CR transport also causes faster Alfv\'en cooling which diminishes the dynamical impact of CRs so that the gas distribution looks similar to the noCR model (cf.\ Figs.~\ref{fig:rgb} and \ref{fig:gas}). This effect is less pronounced in the CRdiff model as here the combination of CR energy conservation (during the diffusion step) and their higher escape speed lead to a larger CR pressure-dominated halo (bottom right panels of Fig.~\ref{fig:high_kappa}). This CR pressure dominated halo in turn prevents the gas from efficiently cooling onto the disc. These simulations reinforce our main results that small variants of CR transport can have substantial impact on the resulting galaxies.

\subsection{Implications for CR transport in galaxies and the CGM}
\label{subsec:dis}

Our analyses of CR feedback have several important implications for CR transport and the excitation of CR driven instabilities. The CR pressure in the CRdiff model dominates over the thermal pressure up to large radii ($r\lesssim80$~kpc). In such a quasi-hydrostatic atmosphere the CGM necessarily attains a comparably smooth distribution. By contrast, the CR pressure distribution in the CRadv model reflects the dominating modes of transport and cooling processes. Advection of CRs with the galactic outflows along streamlines implies a highly structured CR distribution (Fig.~\ref{fig:CGMgas}). Turbulent mixing in the CGM \citep{Pakmor2019} causes a smoother CR distribution, in particular at large radii $r\gtrsim50$~kpc. 

Additionally including CR diffusion smooths the CR distribution considerably as a result of two effects: (i) CR diffusion on cosmological timescales results in a root mean square displacement of $25~\rmn{kpc}~\sqrt{\kappa_{28} t_{10~\rmn{Gyr}}}$ along the magnetic field lines and (ii) perpendicular transport is achieved through field line wandering. Assuming that the velocity differences between neighboring points follow a Gaussian distribution, we obtain explosive Richardson diffusion with a displacement $\langle x^2\rangle \propto t^3$ up to the injection scale of turbulence (and standard diffusion above this scale), which smooths the CR distribution in the CGM considerably (Fig.~\ref{fig:CGMgas}).

Most surprisingly, by additionally accounting for CR Alfv\'en wave losses, the CR pressure distribution becomes highly structured. As CRs diffuse a scale height, they loose an e-folding of their initial energy. This distance can be substantially increased if CRs are predominantly advected with the gas. As their diffusive transport reaches the effective scale height, they cool quickly and subsequent turbulent mixing is greatly suppressed. Hence they should trace out individual streamlines of the gas in their pressure as well as in the CR-to-thermal pressure ratio (see Fig.~\ref{fig:CGMgas}).

We have seen in Fig.~\ref{fig:pressure} that the CR and magnetic pressures vary by four orders of magnitude but trace each other within a factor of five out to the virial radius. This remarkable finding has severe consequences for the existence of current driven CR instabilities. The condition for exciting the hybrid, non-resonant CR instability is $\epsilon_\CR/\epsilon_B \gtrsim 2 c / \varv_\rmn{d}$ \citep{Bell2004}, where $\varv_\rmn{d}$ is the drift speed of CRs that is close to the Alfv\'en speed as explained above. Because our CR and magnetic energy densities closely trace each other, the Bell instability is not excited and hence, no additional growth of the magnetic field is expected from this plasma effect. This also implies that the adopted diffusion coefficient remains valid and is not lowered to the classical Bohm limit due to strong Bell fluctuations, which would scatter CRs off magnetic irregularities at every gyro orbit. This fast CR scattering would manifest itself as a much reduced CR diffusion coefficient by about seven orders of magnitudes, which would effectively imply a transition to the CRadv model.

\subsection{Comparison to previous work}
\subsubsection{Effects on the central galaxy}

In this study we found that CRs have little effect on global galaxy properties such as stellar mass and SFR. All our galaxies exhibit a rotationally supported gas disc dominated by the kinetic pressure (Fig.\ \ref{fig:pressure}) and a mostly thermally supported CGM in the halo ($R\gtrsim75$ kpc). The CRdiff run additionally shows a transition region at the disc-CGM interface ($20<R<75$ kpc) where CRs dominate the pressure budget, permitting a lower CGM temperature \citep[see also][figure 6]{Butsky2018}. On the other hand, in the CRadv run this transition region shows an equilibrium of CR pressure with the thermal and the kinetic pressure. This agrees qualitatively with results obtained by \citet[][figure 2]{Salem2016} and the recent findings of \citet{Hopkins2019} despite the large differences in the diffusion coefficients used (this work: $\kappa_\parallel=1\times10^{28}$ cm$^{2}$ s$^{-1}$ vs. ``best-fit'' $\kappa_\parallel=3\times10^{29}$ cm$^{2}$ s$^{-1}$ in FIRE-2). 

While qualitative agreement between our results and the ones presented by the FIRE group exist, the fundamental differences in the value of the diffusion coefficient is a result of the physical consequences of a different treatment of the ISM in the simulations. As detailed in Sect.~\ref{sec:kappa}, our choice for $\kappa_\parallel$ is justified by CR propagation studies \citep{Porter2017,Cerri2017,Evoli2017,Evoli2019,Evoli2020,Johannesson2019} and is in line with other simulation analyses of galaxies forming in a cosmological environment \citep{Salem2016}. By contrast, the favoured diffusion coefficients in the FIRE-2 simulations are a factor 10-30 larger than those studied here and inferred in the CR literature. 
In fact, we believe that the choice of such a large diffusion coefficient in the FIRE-2 simulations follows from the particular multi-phase ISM model used there and the associated difficulty to accurately model the appropriate wave damping in the different phases of the ISM, as we lay out below.

In order for the CRs to escape the dense gas without significant hadronic losses, the diffusive timescale $\tau_\rmn{diff} \approx L^2/\kappa_\parallel$ needs to be smaller than the hadronic loss time $\tau_\rmn{had} \approx 1 / (n \sigma_\rmn{pp} c)$, where $L$ is the scale height of the disc, $\sigma_\rmn{pp}\approx25$~mbarn is the total inelastic proton-proton cross section at kinetic proton energies of $1-3$~GeV \citep{Kafexhiu2014}, and $c$ is the light speed. By rearranging this inequality we derive a lower bound on the effective diffusion coefficient needed to allow the CRs to leave the galaxy, 
\begin{align}
    \label{eq:kappa}
    \kappa_\parallel>L^2n\sigma_\rmn{pp} c
\end{align}
In the AURIGA model the star forming phase is governed by an equation of state \citep{Springel2003} resulting in a relatively smooth gas distribution where stars form at gas densities above $n_{\rm th}\sim0.13$ cm$^{-3}$. This results in flat and extended gas disks in the AURIGA simulations (top panels of Fig.~\ref{fig:gas}) of scale height $\sim1.5$ kpc (see vertical gas profiles in Fig.~\ref{fig:vert_prof}) with densities in the disk mid-plane of $n\lesssim 0.4$ cm$^{-3}$ (cf. left panels in Fig.~\ref{fig:prof}). Using this density in Eqn.~\eqref{eq:kappa} shows that for the AURIGA model a value of  $\kappa_\parallel\gsim10^{28}$ cm$^{2}$ s$^{-1}$ allows CRs to escape.

On the other hand, in the FIRE-2 simulations the ISM is treated differently and allows for the formation of a multiphase ISM including a cool and dense phase with a significant amount of gas at densities $n\gsim10$ cm$^{-3}$ \citep[see Fig. 9 of][]{Hopkins2019} in their $L_\star$ galaxies. For those densities and assuming similar gas disk scale heights, Eqn.~\eqref{eq:kappa} results in $\kappa_\parallel\gsim2\times10^{29}$ cm$^{2}$ s$^{-1}$ in order to allow the CRs to escape efficiently from the ISM.
Physically, ion-neutral damping would strongly damp the self-excited Alfv\'en waves so that CRs become weakly coupled to the largely neutral gas and could escape almost ballistically at their intrinsic speed of light from these regions \citep{Wiener2017}. Once they enter the warm-hot phase of the ISM, they couple again to the gas (because of the weaker wave damping processes such as non-linear Landau damping) so that they are transported at the Alfv\'en wave speed, which corresponds to an effective parallel diffusion coefficients in the MW today of $\kappa_\parallel=(1-3)\times10^{28}$ cm$^{2}$ s$^{-1}$. A failure to model this transition results in a more diffusively transported CR gas in the warm-hot phases of the ISM and CGM \citep[see also discussion in][]{Salem2014b}. It has been suggested by \citet{Farber2018} that the decoupling of CRs at low temperatures and high densities can be approximated by artificially increasing the diffusion coefficient in low temperature gas by a factor of 30 in comparison to the general CGM gas (their equation~13). This work shows that with this simple decoupling mechanism CRs can efficiently escape the ISM and their dynamical effect is dramatically increased.

The strongest impact of CRs on the central galaxy we can find in our study is the reduction in size of the gaseous and stellar disc. Very similar results regarding the sizes of the stellar discs have been obtained by the FIRE-2 group \citep[see e.g. their Fig.\ 12 in][]{Hopkins2019} who did not investigate the underlying reasons. Here we find that the interplay of CRs and the wind feedback model on the scales of the gas disc mediates large scale effects on the accretion flow of gas. In the CR runs we observe more spherically symmetric outflows blocking the coherent gas accretion in the direction of the disc plane. This in turn modifies the angular momentum acquisition of the galaxy and manifests itself in accretion of gas from a smaller region.

Our results are in stark contrast to the results from \citet{Salem2014b} where the stellar disc grows in size when CRs are considered. The reason for this is not entirely clear but the very different feedback implementations and resolution effects might certainly play a role. These earlier results analysed simulations of worse resolution compared to the ones used here. Furthermore, the fiducial model used in that study results in a very compact stellar disc of unrealistic size. In contrast to this, the AURIGA-CR models start out from galactic discs of realistic size and mass because the wind feedback of the AURIGA noCR runs was tuned to reproduce MW-like galaxies. We discuss the uncertainties of the wind model in more detail in the next section.

\subsubsection{Effects on the circum-galactic medium} 
The most noticeable effect of CRs in the AURIGA simulations is on the structure and morphology of the CGM. Whereas the overall baryonic mass in the CGM is not drastically effected (see density profiles in Fig.\ \ref{fig:CGMprof}), the additional CR pressure affects the small scale density and temperature distribution of the CGM. In particular our CRdiff model has a smoother and cooler CGM which is maintained by the additional CR pressure which smoothes out small-scale high density clumps in the CGM and supports the gas at lower temperatures against gravity. These findings are qualitatively similar to earlier results presented in the literature \citep[e.g.][]{Salem2014b,Salem2016,Chen2016,Butsky2018}.

The findings that the CGM becomes smoother and slightly cooler when diffusing CRs are included is consistent with the results of small scale ISM simulations of the galactic disc using stratified boxes \citep{Girichidis2016,Girichidis2018,Simpson2016}. In particular, \citet{Girichidis2018} finds that the CR pressure inside the disc ($z\lesssim1$ kpc) is largely in equilibrium with the thermal pressure as is the case for all our CR runs (compare Fig.\ \ref{fig:gas}). At distances larger than that ($z\gtrsim1$ kpc), the CR pressure starts to dominate over the thermal pressure with values of $P_{\rm cr}/P_{\rm th}=10-100$ in good agreement with our results. Thus, despite the approximations of our ISM model and the comparatively lower resolution results on the scales studied here, the simulations appear to be converged.

By contrast, our CRdiffalfven model shows a warmer CGM in comparison to the model without CRs (noCR) and in strong contrast to earlier work presented in the literature \citep[e.g.][]{Salem2014b,Salem2016,Chen2016}. As explained above, the reason is the additional CR Alfv\'en wave cooling term that emulates CR energy losses as they are resonantly exciting Alfv\'en waves which scatter their pitch angles (angle between their momentum and mean magnetic field vectors). This causes them to isotropise in the Alfv\'en wave frame and to stream with the Alfv\'en velocity along the local direction of the magnetic fields \citep{Wiener2017}. Whereas this approximation is justified as long as CR streaming and diffusion fluxes match each other, this cannot be guaranteed at all times due to the dispersive mathematical nature of the diffusion operator. Clearly more work is needed to confirm this finding and to better understand the final state of the CGM in the presence of streaming CRs. On the contrary, recent results by the FIRE-2 simulations suggest that CRs are able to reduce the CGM temperature from $\gtrsim10^5$ K to $\sim10^4$ K \citep[see figure 7 in][]{Ji2019} by providing enough pressure support. Our simulations do not support such a drastic change in CGM temperature as we have shown in Fig.\ \ref{fig:phase}. In fact, a complete suppression of the hot phase as in the FIRE-2 simulations is at odds with X-ray observations of the MW hot halo \citep[$\sim10^6$ K,][]{Fang2013,Faerman2017}. 

The most likely reason for this is the implementation of feedback in FIRE-2, which is very explosive and could cause a quenching of their magnetic dynamo. This yields to saturation at a low level with a magnetic energy density that is a factor of 100 below our results. Note that our magnetic field distribution matches Faraday rotation measure data of the MW and external galaxies \citep{Pakmor2018}. The lower magnetic field strength causes the Alfv\'en speed $\bvel_{\rmn{A}}=\bB/\sqrt{4\upi\rho}$ to be ten times smaller and hence, also reduces the CR Alfv\'en wave cooling rate, $|\bvel_{\rmn{A}}\bcdot\bnabla P_\CR|$, by the same factor. Hence, the FIRE-2 runs represent an extreme version of our CRdiff model, in which the CR Alfv\'en wave cooling is nearly absent.

\subsection{Modelling uncertainties}
The results obtained in this paper are subject to a number of physical modelling uncertainties which we discuss below.

\subsubsection{The AURIGA feedback model}
In this study CRs are modelled on top of the AURIGA galaxy formation model which has been calibrated to reproduce MW-like galaxies without the inclusion of CRs. We have kept any ``free'' parameter in the sub-grid model as in the AURIGA model and added the CR physics on top of this. Therefore, the comparably small impact of CR physics on global galaxy properties such as the total stellar mass or SFR (as opposed to previous findings where CRs showed strong impacts) might be due to the already efficient feedback implementation of the AURIGA model without CRs. Here, the biggest uncertainty is the effect of the wind model coupled with the CR feedback. In the AURIGA model the details of the wind model are calibrated to reproduce observed galaxy properties without the additional effects of CRs. In this study, we add CR feedback on top of the already calibrated feedback model of AURIGA without re-tuning any parameters. Whereas this strategy allows us to cleanly single out the effects of CRs, one could imagine that the calibrated AURIGA model might already account for some of the effects CRs might have on galaxy formation. Therefore, the exact choice of parameters for the wind model in combination with the effects of CRs might change the amount of angular momentum losses as observed in our study. There might exist a different combination of wind model parameters and CR feedback model in which the CGM flow is less affected by the CRs and thus the angular momentum losses are reduced. However, the cause of the different angular momentum build-up in the three CRs variants is the modified morphology of the disc-halo interface and we expect the basic effects to be robust. Nevertheless, unless the parameters of the wind model are derived from either observations or theoretical considerations the wind model presents a considerable systematic uncertainty.

\subsubsection{The ISM model}
Our simulations adopt a pressurised ISM which even in the stellar disc is relatively smooth without high density, low temperature peaks \citep[e.g.][figures 9 and 10]{Marinacci2019}. Similarly to the AURIGA feedback model we have not re-tuned any of the parameters of the multiphase model for star formation. This is justified because all our models still reproduce the Kennicutt-Schmidt relation \citep{Kennicutt1998ARA+A} against which this model was calibrated and we have explicitly checked that each of the models reproduces the normalization and slope of the observed relation. Additionally, it is not entirely clear how CRs could be included in the Springel \& Hernquist model in the first place, so that for the purposes of this study it appears most adequate to not modify parameters of the subgrid model.

Transforming to a multi-phase ISM will effect how CRs escape dense star forming regions and thus how they impact the dynamics of the ISM. Most importantly, recent modelling of CR data suggests that CRs below 200~GeV that carry most of the CR pressure are streaming with the Alfv\'en velocity and are diffusively transported at higher energies \citep{Evoli2018}. Hence, we need to model CR streaming in the self-confinement picture where CRs resonantly excite Alfv\'en waves to accurately model their transport in galaxies and the CGM using the two-moment method \citep{Thomas2019}. Following the evolution equation of small-scale resonant Alfv\'en wave energies provides a means to self-consistently model CR diffusion in the Alfv\'en wave frame. This will enable us to simultaneously account for the weaker coupling of CRs in the cold phase ($T<10^4$~K) due to increased ion-neutral damping and the stronger dynamical coupling in the warm-hot phases due to the prevalent non-linear Landau damping \citep[e.g.][]{McKenzie1983} and turbulent damping processes \citep[e.g.][]{Farmer2004,Yan2004}.

\subsubsection{The CR transport models}
Another fundamental uncertainty is given by the details of the CR transport physics and its numerical implementation. To explore the influence of CR transport on galaxy formation we decided to adopt three different variants of CR transport and focused our analysis on the question of how each variant impacts the stellar and gaseous properties. These models result in qualitatively similar global trends, but show that structural properties differ between each of the CR variants. As expected, our CRdiffalfven model that emulates CR streaming gave the most realistic results in terms of stellar and gaseous disc properties as well as for the CGM in agreement with recent findings by \citet{Butsky2018}. In particular, the resulting gamma-ray emission (see Fig. \ref{fig:gamma}) appeared to be an important discriminant of the studied CR models. 


\subsubsection{Cosmological variance of accretion histories}
In this study we have focused on analysing the effects of CR physics in cosmological simulations. Additionally to the different CR transport physics the two galaxies in these kind of simulations are further affected by the different accretion histories. For example, AuL8 undergoes a major merger at a lookback time of $\sim6$ Gyr whereas Au6 has a very quiet merger history at low redshift. Thus, there are natural differences in the evolution and properties between the two haloes complicating the separation of the effects of CRs and cosmological accretion history. On the other hand, CRs do not only affect the main galaxy but also the merging satellites and thus a complete picture of their effects can only be gained by studying a large cosmological volume, which samples the complete galaxy population.

\subsubsection{Numerical resolution study}
Convergence of galaxy properties across different levels of numerical resolution is difficult to achieve in galaxy formation simulations and poses an additional challenge in understanding the physics of galaxy formation. Ideally, the outcome of a simulation should only depend on the modelled physics and not on numerical resolution. In section 6 of \citet{Grand2017} it has been shown that our baseline model, the AURIGA model without CRs, is numerically well converged. We have run additional 8 simulations with a factor of 8 and 16 lower in mass resolution in order to test the numerical robustness of our results. While we detail the resolution dependence of our results in Appendix~\ref{sec:res}, here we summarise the main results: stellar and halo masses of the central galaxies are well converged across different resolution levels (see Table~\ref{tab:res}) and we have verified that all our results and conclusions do not depend on resolution. Especially our main findings are numerically converged: those include the more compact stellar discs (see Fig.~\ref{fig:res_test} in the Appendix) mediated by the modified accretion flow in the CR runs as well as the gas discs inflated by CR pressure and the smoother CGM in the CRadv and CRdiff models in comparison to the other two models (cf.\ Fig.~\ref{fig:gas_res} and the upper row of Fig.~\ref{fig:CGMgas}). Thus, our simulations are well suited to study the effects of CRs in cosmological simulations as the evolution of the galaxies only depends on our physical modelling and not on numerical resolution.


\section{Conclusions} \label{sec:conc}


In this work we set out to study the effects of CRs on the formation of MW-like galaxies in a cosmological context. To this extend we have performed eight magnetohydrodynamical simulations in the context of the AURIGA project \citep{Grand2017} with three different models of varying complexity for the physics of CR transport. All simulations are performed with the second-order accurate moving mesh code Arepo \citep{Springel2010,Pakmor2016c} for magneto-hydrodynamics. The galaxy formation model includes detailed models for gas cooling and heating, star formation as well as stellar and AGN feedback. Additionally, the simulations include the following CR physics: the simplest model advects CRs with the gas flow (CRadv), a more complex variant additionally follows the anisotropic diffusion of CRs parallel to the magnetic field (CRdiff) while in the most complex model CRs are further allowed to cool via the excitation of Alfv\'en waves (CRdiffalfven), attempting to emulate the transport process of self-confined CR streaming.

We have studied in detail the properties of the central galaxy and the CGM and compared model predictions from the CR runs to the fiducial AURIGA model. Bulk galaxy properties are only weakly affected by CRs, whereas the morphology and angular momentum distribution of our galaxies as well as the properties of the CGM are sensitive to the details of the CR physics implementation. Our conclusions are summarized as follows:

\begin{itemize}
\item Galaxy properties like the total stellar mass, SFR or gas mass are largely unaffected by CRs and stable across different physics variants. While previous works have found that CRs are able to suppress star formation in isolated galaxy simulations, our cosmological simulations show that the SFR is largely unaffected by CR feedback (see Fig.\ \ref{fig:sfr}). Note that this could be partially due to the already efficient feedback in AURIGA, which can in principle mask some of the feedback effects CRs would otherwise have.

\item Comparing structural parameters of the galaxies such as disc sizes, disc-to-total stellar mass ratios or gas disc morphology we find strong differences between the simulations that include CRs and the fiducial AURIGA model. The CRadv and CRdiff models result in more compact, bulge dominated discs which show thicker and smoother gas discs, in which the vertical force balance is dominated by the CR pressure (see e.g., Figs.\ \ref{fig:rgb} and \ref{fig:gas}). A similar reduction of stellar disc size is also found by the FIRE-2 group in their simulations including CR feedback. In contrast, the stellar and gaseous discs in the CRdiffalfven model have disc sizes which lie between the fiducial results and the more extreme CR models (e.g., the left panel of Fig.\ \ref{fig:prof}). We find that the magnetic field strength and morphology is similar in all our runs with a value of the order of $\sim10\, \mu$G \citep[that is consistent with MW observations, see][]{Pakmor2018} so that our magnetic energy density is roughly 100 times larger than those obtained with the FIRE-2 simulations \citep[e.g. Figs. 3 and 19 of][]{Hopkins2019}.

\item The interplay of CRs and the wind feedback model strongly affects the gas flow patterns in the CGM (Figs.\ \ref{fig:flow} and \ref{fig:flow2}). The more compact, bulge dominated discs in the CR simulations cause the outflows to become more spherically symmetric in comparison to the fiducial AURIGA run (Fig. \ref{fig:flow}) and thus alter the angular momentum acquisition in the cosmological runs. In this way the action of CR feedback in the star forming disc changes the outflow geometry and suppresses the baryonic accretion of high angular momentum gas, especially at late cosmic times (Fig.\ \ref{fig:ang_mom}). As a consequence, the gas discs in the CR runs are smaller in size as is highlighted in the left panels of Fig.\ \ref{fig:prof}.

\item On larger scales, CRs strongly affect the properties of the CGM. The advection and diffusion models exhibit a smoother and partly cooler CGM (Figs.\ \ref{fig:CGMgas} and \ref{fig:CGMprof}) where the additional CR pressure is able to stabilise the CGM against gravitational collapse compensating for the missing thermal pressure support at lower CGM temperatures. These runs therefore show large ($R\sim50$ kpc) CR pressure contributions in the haloes. In contrast, the Alfv\'en wave model is only CR pressure dominated at the disc-halo interface and the CRs come into equilibrium with the thermal pressure as they are advected into the halo along stream lines of the galactic winds (see also Fig.\ \ref{fig:pressure}). As CRs are actively transported across an effective scale height, they quickly cool, which greatly suppresses further turbulent mixing and causes a highly structured CR pressure distribution in the CGM (Fig. \ref{fig:CGMgas}). This in turn causes a structured density and temperature distribution in the CGM, which maintains large volumes at thermally unstable temperatures of $10^5$~K (Fig. \ref{fig:phase}) which is warmer than the cool ($\sim10^4-10^5$ K) CGM gas found in the CR FIRE-2 simulations \citep[see Fig. 7 in][]{Ji2019}.

\item The magnetic and CR pressures trace each other within a factor of five out to the virial radius (Fig.~\ref{fig:pressure}). This implies that there is not enough free energy available to drive the hybrid, non-resonant CR instability \citep{Bell2004}, which would require the CR-to-magnetic energy density ratio to be larger than $2c/\varv_\rmn{d}\sim10^3-10^4$ where $\varv_d$ is the drift speed of CRs that is close to the Alfv\'en speed. Excitation of the Bell instability would imply fast CR scattering, a much reduced CR diffusion coefficient by about seven orders of magnitudes and effectively transition to the CRadv model.

\item There are active ongoing efforts in developing efficient and accurate CR magneto-hydrodynamical schemes \citep{Jiang2018,Thomas2019} to compute the CR feedback effects in cosmological simulations. To this end, direct observables are invaluable in constraining effective CR transport models, provided the approximations used for CR transport and the ISM are commensurate and not inconsistent. In Fig.\ \ref{fig:gamma} we compare the gamma-ray luminosity from hadronic CR interactions with the ISM of our models to observations of local galaxies. We find that the CRdiffalfven run agrees well with observed relations whereas the CRdiff and CRadv produce higher gamma-ray luminosities at the MW mass scale compared to observations. Our comparison here presents a first step towards understanding the effects of CRs on cosmological galaxy formation, but further work in this direction is needed to constrain valid CR transport coefficient and prevailing transport processes.
\end{itemize}


\section*{Acknowledgments}
We thank the anonymous referee for a careful reading of the manuscript which helped to improve the quality of the paper. TB and CP acknowledge support by the European Research Council under ERC-CoG grant CRAGSMAN-646955. This research was supported in part by the National Science Foundation under Grant No. NSF PHY-1748958. This research made use of the following {\sc{python}} packages: {\sc{matplotlib}} \citep{matplotlib}, {\sc{SciPy}} \citep{scipy}, {\sc{NumPy}} \citep{numpy},  {\sc{IPython and Jupyter}} \citep{ipython,jupyter}.

\section*{Data availability}
The data underlying this article will be shared on reasonable request to the corresponding author.





@ARTICLE{Kafexhiu2014,
       author = {{Kafexhiu}, Ervin and {Aharonian}, Felix and {Taylor}, Andrew M. and
         {Vila}, Gabriela S.},
        title = "{Parametrization of gamma-ray production cross sections for p p interactions in a broad proton energy range from the kinematic threshold to PeV energies}",
      journal = {\prd},
     keywords = {13.75.Cs, 13.85.Ni, 13.85.Tp, Nucleon-nucleon interactions, Inclusive production with identified hadrons, Cosmic-ray interactions, Astrophysics - High Energy Astrophysical Phenomena, Nuclear Experiment},
         year = 2014,
        month = dec,
       volume = {90},
       number = {12},
          eid = {123014},
        pages = {123014},
          doi = {10.1103/PhysRevD.90.123014},
archivePrefix = {arXiv},
       eprint = {1406.7369},
 primaryClass = {astro-ph.HE},
       adsurl = {https://ui.adsabs.harvard.edu/abs/2014PhRvD..90l3014K},
      adsnote = {Provided by the SAO/NASA Astrophysics Data System}
}

@ARTICLE{Farber2018,
       author = {{Farber}, R. and {Ruszkowski}, M. and {Yang}, H. -Y.~K. and
         {Zweibel}, E.~G.},
        title = "{Impact of Cosmic-Ray Transport on Galactic Winds}",
      journal = {\apj},
     keywords = {cosmic rays, galaxies: evolution, galaxies: star formation, Astrophysics - High Energy Astrophysical Phenomena},
         year = 2018,
        month = apr,
       volume = {856},
       number = {2},
          eid = {112},
        pages = {112},
          doi = {10.3847/1538-4357/aab26d},
archivePrefix = {arXiv},
       eprint = {1707.04579},
 primaryClass = {astro-ph.HE},
       adsurl = {https://ui.adsabs.harvard.edu/abs/2018ApJ...856..112F},
      adsnote = {Provided by the SAO/NASA Astrophysics Data System}
}

@ARTICLE{Webber1998,
       author = {{Webber}, W.~R.},
        title = "{A New Estimate of the Local Interstellar Energy Density and Ionization Rate of Galactic Cosmic Cosmic Rays}",
      journal = {\apj},
     keywords = {ISM: COSMIC RAYS, ISM: GENERAL, ISM: Cosmic Rays, ISM: General},
         year = 1998,
        month = oct,
       volume = {506},
       number = {1},
        pages = {329-334},
          doi = {10.1086/306222},
       adsurl = {https://ui.adsabs.harvard.edu/abs/1998ApJ...506..329W},
      adsnote = {Provided by the SAO/NASA Astrophysics Data System}
}

@ARTICLE{Simpson1988,
       author = {{Simpson}, J.~A. and {Garcia-Munoz}, M.},
        title = "{Cosmic-Ray Lifetime in the Galaxy - Experimental Results and Models}",
      journal = {\ssr},
     keywords = {Astronomical Models, Beryllium, Cosmic Rays, Interstellar Matter, Isotope Separation, Milky Way Galaxy, Abundance, Beryllium 10, Diffusion Length, Energy Spectra, Gaseous Diffusion, Half Life, Nuclei (Nuclear Physics), Time Dependence, Space Radiation},
         year = 1988,
        month = sep,
       volume = {46},
       number = {3-4},
        pages = {205-224},
          doi = {10.1007/BF00212240},
       adsurl = {https://ui.adsabs.harvard.edu/abs/1988SSRv...46..205S},
      adsnote = {Provided by the SAO/NASA Astrophysics Data System}
}

@ARTICLE{Lacki2011,
       author = {{Lacki}, Brian C. and {Thompson}, Todd A. and {Quataert}, Eliot and
         {Loeb}, Abraham and {Waxman}, Eli},
        title = "{On the GeV and TeV Detections of the Starburst Galaxies M82 and NGC 253}",
      journal = {\apj},
     keywords = {cosmic rays, galaxies: individual: M82 NGC 253, galaxies: starburst, gamma rays: galaxies, radio continuum: galaxies, Astrophysics - High Energy Astrophysical Phenomena, Astrophysics - Cosmology and Nongalactic Astrophysics},
         year = 2011,
        month = jun,
       volume = {734},
       number = {2},
          eid = {107},
        pages = {107},
          doi = {10.1088/0004-637X/734/2/107},
archivePrefix = {arXiv},
       eprint = {1003.3257},
 primaryClass = {astro-ph.HE},
       adsurl = {https://ui.adsabs.harvard.edu/abs/2011ApJ...734..107L},
      adsnote = {Provided by the SAO/NASA Astrophysics Data System}
}

@ARTICLE{Chomiuk2011,
       author = {{Chomiuk}, Laura and {Povich}, Matthew S.},
        title = "{Toward a Unification of Star Formation Rate Determinations in the Milky Way and Other Galaxies}",
      journal = {\aj},
     keywords = {galaxies: star formation, Galaxy: fundamental parameters, H II regions, ISM: supernova remnants, Astrophysics - Astrophysics of Galaxies, Astrophysics - Cosmology and Nongalactic Astrophysics, Astrophysics - Solar and Stellar Astrophysics},
         year = 2011,
        month = dec,
       volume = {142},
       number = {6},
          eid = {197},
        pages = {197},
          doi = {10.1088/0004-6256/142/6/197},
archivePrefix = {arXiv},
       eprint = {1110.4105},
 primaryClass = {astro-ph.GA},
       adsurl = {https://ui.adsabs.harvard.edu/abs/2011AJ....142..197C},
      adsnote = {Provided by the SAO/NASA Astrophysics Data System}
}

@ARTICLE{Licquia2015,
       author = {{Licquia}, Timothy C. and {Newman}, Jeffrey A.},
        title = "{Improved Estimates of the Milky Way's Stellar Mass and Star Formation Rate from Hierarchical Bayesian Meta-Analysis}",
      journal = {\apj},
     keywords = {Galaxy: bulge, Galaxy: disk, Galaxy: fundamental parameters, Galaxy: stellar content, methods: statistical, stars: formation, Astrophysics - Astrophysics of Galaxies},
         year = 2015,
        month = jun,
       volume = {806},
       number = {1},
          eid = {96},
        pages = {96},
          doi = {10.1088/0004-637X/806/1/96},
archivePrefix = {arXiv},
       eprint = {1407.1078},
 primaryClass = {astro-ph.GA},
       adsurl = {https://ui.adsabs.harvard.edu/abs/2015ApJ...806...96L},
      adsnote = {Provided by the SAO/NASA Astrophysics Data System}
}

@ARTICLE{Cummings2016,
       author = {{Cummings}, A.~C. and {Stone}, E.~C. and {Heikkila}, B.~C. and
         {Lal}, N. and {Webber}, W.~R. and {J{\'o}hannesson}, G. and
         {Moskalenko}, I.~V. and {Orlando}, E. and {Porter}, T.~A.},
        title = "{Galactic Cosmic Rays in the Local Interstellar Medium: Voyager 1 Observations and Model Results}",
      journal = {\apj},
     keywords = {cosmic rays, ISM: abundances, ISM: clouds},
         year = 2016,
        month = nov,
       volume = {831},
       number = {1},
          eid = {18},
        pages = {18},
          doi = {10.3847/0004-637X/831/1/18},
       adsurl = {https://ui.adsabs.harvard.edu/abs/2016ApJ...831...18C},
      adsnote = {Provided by the SAO/NASA Astrophysics Data System}
}

@ARTICLE{Rahmani2016,
       author = {{Rahmani}, S. and {Lianou}, S. and {Barmby}, P.},
        title = "{Star formation laws in the Andromeda galaxy: gas, stars, metals and the surface density of star formation}",
      journal = {\mnras},
     keywords = {methods: observational, methods: statistical, galaxies: ISM, galaxies: spiral, galaxies: star formation, galaxies: stellar content, Astrophysics - Astrophysics of Galaxies},
         year = 2016,
        month = mar,
       volume = {456},
       number = {4},
        pages = {4128-4144},
          doi = {10.1093/mnras/stv2951},
archivePrefix = {arXiv},
       eprint = {1512.06675},
 primaryClass = {astro-ph.GA},
       adsurl = {https://ui.adsabs.harvard.edu/abs/2016MNRAS.456.4128R},
      adsnote = {Provided by the SAO/NASA Astrophysics Data System}
}

@ARTICLE{Pakmor2019,
       author = {{Pakmor}, Ruediger and {van de Voort}, Freeke and {Bieri}, Rebekka and
         {Gomez}, Facundo A. and {Grand}, Robert J.~J. and {Guillet}, Thomas and
         {Marinacci}, Federico and {Pfrommer}, Christoph and
         {Simpson}, Christine M. and {Springel}, Volker},
        title = "{Magnetising the circumgalactic medium of disk galaxies}",
      journal = {arXiv e-prints},
     keywords = {Astrophysics - Astrophysics of Galaxies},
         year = "2019",
        month = "Nov",
          eid = {arXiv:1911.11163},
        pages = {arXiv:1911.11163},
archivePrefix = {arXiv},
       eprint = {1911.11163},
 primaryClass = {astro-ph.GA},
       adsurl = {https://ui.adsabs.harvard.edu/abs/2019arXiv191111163P},
      adsnote = {Provided by the SAO/NASA Astrophysics Data System}
}

@ARTICLE{Amato2018,
       author = {{Amato}, Elena and {Blasi}, Pasquale},
        title = "{Cosmic ray transport in the Galaxy: A review}",
      journal = {Advances in Space Research},
     keywords = {Cosmic rays, ISM, Diffusion, MHD, Astrophysics - High Energy Astrophysical Phenomena},
         year = "2018",
        month = "Nov",
       volume = {62},
       number = {10},
        pages = {2731-2749},
          doi = {10.1016/j.asr.2017.04.019},
archivePrefix = {arXiv},
       eprint = {1704.05696},
 primaryClass = {astro-ph.HE},
       adsurl = {https://ui.adsabs.harvard.edu/abs/2018AdSpR..62.2731A},
      adsnote = {Provided by the SAO/NASA Astrophysics Data System}
}

@ARTICLE{Cerri2017,
       author = {{Cerri}, Silvio Sergio and {Gaggero}, Daniele and {Vittino}, Andrea and
         {Evoli}, Carmelo and {Grasso}, Dario},
        title = "{A signature of anisotropic cosmic-ray transport in the gamma-ray sky}",
      journal = {\jcap},
     keywords = {Astrophysics - High Energy Astrophysical Phenomena, High Energy Physics - Phenomenology},
         year = "2017",
        month = "Oct",
       volume = {2017},
       number = {10},
          eid = {019},
        pages = {019},
          doi = {10.1088/1475-7516/2017/10/019},
archivePrefix = {arXiv},
       eprint = {1707.07694},
 primaryClass = {astro-ph.HE},
       adsurl = {https://ui.adsabs.harvard.edu/abs/2017JCAP...10..019C},
      adsnote = {Provided by the SAO/NASA Astrophysics Data System}
}

@ARTICLE{Porter2017,
       author = {{Porter}, T.~A. and {J{\'o}hannesson}, G. and {Moskalenko}, I.~V.},
        title = "{High-energy Gamma Rays from the Milky Way: Three-dimensional Spatial Models for the Cosmic-Ray and Radiation Field Densities in the Interstellar Medium}",
      journal = {\apj},
     keywords = {astroparticle physics, cosmic rays, galaxy: general, gamma rays: general, gamma rays: ISM, radiation mechanisms: non-thermal, Astrophysics - High Energy Astrophysical Phenomena},
         year = "2017",
        month = "Sep",
       volume = {846},
       number = {1},
          eid = {67},
        pages = {67},
          doi = {10.3847/1538-4357/aa844d},
archivePrefix = {arXiv},
       eprint = {1708.00816},
 primaryClass = {astro-ph.HE},
       adsurl = {https://ui.adsabs.harvard.edu/abs/2017ApJ...846...67P},
      adsnote = {Provided by the SAO/NASA Astrophysics Data System}
}

@ARTICLE{Thomas2020,
       author = {{Thomas}, Timon and {Pfrommer}, Christoph and {En{\ss}lin}, Torsten},
        title = "{Probing Cosmic-Ray Transport with Radio Synchrotron Harps in the Galactic Center}",
      journal = {\apjl},
     keywords = {Astrophysics - High Energy Astrophysical Phenomena, Astrophysics - Astrophysics of Galaxies},
         year = 2020,
        month = feb,
       volume = {890},
       number = {2},
          eid = {L18},
        pages = {L18},
          doi = {10.3847/2041-8213/ab7237},
archivePrefix = {arXiv},
       eprint = {1912.08491},
 primaryClass = {astro-ph.HE},
       adsurl = {https://ui.adsabs.harvard.edu/abs/2020ApJ...890L..18T},
      adsnote = {Provided by the SAO/NASA Astrophysics Data System}
}

@ARTICLE{Faerman2017,
       author = {{Faerman}, Yakov and {Sternberg}, Amiel and {McKee}, Christopher F.},
        title = "{Massive Warm/Hot Galaxy Coronae as Probed by UV/X-Ray Oxygen Absorption and Emission. I. Basic Model}",
      journal = {\apj},
     keywords = {galaxies: formation, galaxies: halos, Galaxy: evolution, Galaxy: formation, intergalactic medium, quasars: absorption lines, Astrophysics - Astrophysics of Galaxies},
         year = "2017",
        month = "Jan",
       volume = {835},
       number = {1},
          eid = {52},
        pages = {52},
          doi = {10.3847/1538-4357/835/1/52},
archivePrefix = {arXiv},
       eprint = {1602.00689},
 primaryClass = {astro-ph.GA},
       adsurl = {https://ui.adsabs.harvard.edu/abs/2017ApJ...835...52F},
      adsnote = {Provided by the SAO/NASA Astrophysics Data System}
}


@INBOOK{Haverkorn2015,
       author = {{Haverkorn}, Marijke},
        title = "{Magnetic Fields in the Milky Way}",
     keywords = {Physics, Astrophysics - Astrophysics of Galaxies},
    booktitle = {Magnetic Fields in Diffuse Media},
         year = "2015",
       editor = {{Lazarian}, Alexander and {de Gouveia Dal Pino}, Elisabete M. and
         {Melioli}, Claudio},
       volume = {407},
       series = {Astrophysics and Space Science Library},
        pages = {483},
          doi = {10.1007/978-3-662-44625-6_17},
       adsurl = {https://ui.adsabs.harvard.edu/abs/2015ASSL..407..483H},
      adsnote = {Provided by the SAO/NASA Astrophysics Data System}
}

@ARTICLE{Sun2012,
       author = {{Sun}, X.~H. and {Reich}, W.},
        title = "{Polarisation properties of Milky-Way-like galaxies}",
      journal = {\aap},
     keywords = {polarization, radio continuum: general, ISM: magnetic fields, Astrophysics - Astrophysics of Galaxies, Astrophysics - Cosmology and Nongalactic Astrophysics},
         year = "2012",
        month = "Jul",
       volume = {543},
          eid = {A127},
        pages = {A127},
          doi = {10.1051/0004-6361/201218802},
archivePrefix = {arXiv},
       eprint = {1206.3343},
 primaryClass = {astro-ph.GA},
       adsurl = {https://ui.adsabs.harvard.edu/abs/2012A&A...543A.127S},
      adsnote = {Provided by the SAO/NASA Astrophysics Data System}
}

@ARTICLE{Beck2019,
       author = {{Beck}, Rainer and {Chamandy}, Luke and {Elson}, Ed and
         {Blackman}, Eric G.},
        title = "{Synthesizing Observations and Theory to Understand Galactic Magnetic Fields: Progress and Challenges}",
      journal = {Galaxies},
     keywords = {Astrophysics - Astrophysics of Galaxies, Astrophysics - High Energy Astrophysical Phenomena, J.2},
         year = "2019",
        month = "Dec",
       volume = {8},
       number = {1},
        pages = {4},
          doi = {10.3390/galaxies8010004},
archivePrefix = {arXiv},
       eprint = {1912.08962},
 primaryClass = {astro-ph.GA},
       adsurl = {https://ui.adsabs.harvard.edu/abs/2019Galax...8....4B},
      adsnote = {Provided by the SAO/NASA Astrophysics Data System}
}


@ARTICLE{Haverkorn2006,
       author = {{Haverkorn}, M. and {Gaensler}, B.~M. and {Brown}, J. -A.~C. and
         {McClure-Griffiths}, N.~M. and {Dickey}, J.~M. and {Green}, A.~J.},
        title = "{Magnetic fields in the Southern Galactic Plane Survey}",
      journal = {Astronomische Nachrichten},
     keywords = {ISM: magnetic fields, ISM: structure, surveys, radio continuum: ISM, turbulence, Astrophysics},
         year = "2006",
        month = "Jun",
       volume = {327},
        pages = {483},
          doi = {10.1002/asna.200610565},
archivePrefix = {arXiv},
       eprint = {astro-ph/0511407},
 primaryClass = {astro-ph},
       adsurl = {https://ui.adsabs.harvard.edu/abs/2006AN....327..483H},
      adsnote = {Provided by the SAO/NASA Astrophysics Data System}
}


@ARTICLE{Harris2004,
       author = {{Harris}, Jason and {Zaritsky}, Dennis},
        title = "{The Star Formation History of the Small Magellanic Cloud}",
      journal = {\aj},
     keywords = {Galaxies: Evolution, Galaxies: Individual: Name: Small Magellanic Cloud, Galaxies: Stellar Content, Galaxies: Magellanic Clouds, Astrophysics},
         year = "2004",
        month = "Mar",
       volume = {127},
       number = {3},
        pages = {1531-1544},
          doi = {10.1086/381953},
archivePrefix = {arXiv},
       eprint = {astro-ph/0312100},
 primaryClass = {astro-ph},
       adsurl = {https://ui.adsabs.harvard.edu/abs/2004AJ....127.1531H},
      adsnote = {Provided by the SAO/NASA Astrophysics Data System}
}

@ARTICLE{Wilke2004,
       author = {{Wilke}, K. and {Klaas}, U. and {Lemke}, D. and {Mattila}, K. and
         {Stickel}, M. and {Haas}, M.},
        title = "{The Small Magellanic Cloud in the far infrared.  II. Global properties}",
      journal = {\aap},
     keywords = {galaxies: Magellanic Clouds, ISM: general, ISM: dust, extinction},
         year = "2004",
        month = "Jan",
       volume = {414},
        pages = {69-78},
          doi = {10.1051/0004-6361:20034113},
       adsurl = {https://ui.adsabs.harvard.edu/abs/2004A&A...414...69W},
      adsnote = {Provided by the SAO/NASA Astrophysics Data System}
}

@ARTICLE{Enzo2014,
       author = {{Bryan}, Greg L. and {Norman}, Michael L. and {O'Shea}, Brian W. and
         {Abel}, Tom and {Wise}, John H. and {Turk}, Matthew J. and
         {Reynolds}, Daniel R. and {Collins}, David C. and {Wang}, Peng and
         {Skillman}, Samuel W. and {Smith}, Britton and {Harkness}, Robert P. and
         {Bordner}, James and {Kim}, Ji-hoon and {Kuhlen}, Michael and
         {Xu}, Hao and {Goldbaum}, Nathan and {Hummels}, Cameron and
         {Kritsuk}, Alexei G. and {Tasker}, Elizabeth and {Skory}, Stephen and
         {Simpson}, Christine M. and {Hahn}, Oliver and {Oishi}, Jeffrey S. and
         {So}, Geoffrey C. and {Zhao}, Fen and {Cen}, Renyue and {Li}, Yuan and
         {Enzo Collaboration}},
        title = "{ENZO: An Adaptive Mesh Refinement Code for Astrophysics}",
      journal = {\apjs},
     keywords = {hydrodynamics, methods: numerical, Astrophysics - Instrumentation and Methods for Astrophysics},
         year = "2014",
        month = "Apr",
       volume = {211},
       number = {2},
          eid = {19},
        pages = {19},
          doi = {10.1088/0067-0049/211/2/19},
archivePrefix = {arXiv},
       eprint = {1307.2265},
 primaryClass = {astro-ph.IM},
       adsurl = {https://ui.adsabs.harvard.edu/abs/2014ApJS..211...19B},
      adsnote = {Provided by the SAO/NASA Astrophysics Data System}
}

@ARTICLE{Butsky2018,
       author = {{Butsky}, Iryna S. and {Quinn}, Thomas R.},
        title = "{The Role of Cosmic-ray Transport in Shaping the Simulated Circumgalactic Medium}",
      journal = {\apj},
     keywords = {cosmic rays, Galaxy: evolution, Galaxy: halo, ISM: jets and outflows, methods: numerical, Astrophysics - Astrophysics of Galaxies},
         year = "2018",
        month = "Dec",
       volume = {868},
       number = {2},
          eid = {108},
        pages = {108},
          doi = {10.3847/1538-4357/aaeac2},
archivePrefix = {arXiv},
       eprint = {1803.06345},
 primaryClass = {astro-ph.GA},
       adsurl = {https://ui.adsabs.harvard.edu/abs/2018ApJ...868..108B},
      adsnote = {Provided by the SAO/NASA Astrophysics Data System}
}

@ARTICLE{Johannesson2019,
       author = {{J{\'o}hannesson}, Gu{\dj}laugur and {Porter}, Troy A. and
         {Moskalenko}, Igor V.},
        title = "{Cosmic-Ray Propagation in Light of the Recent Observation of Geminga}",
      journal = {\apj},
     keywords = {astroparticle physics, cosmic rays, diffusion, Galaxy: structure, gamma rays: diffuse background, gamma rays: ISM, Astrophysics - High Energy Astrophysical Phenomena},
         year = "2019",
        month = "Jul",
       volume = {879},
       number = {2},
          eid = {91},
        pages = {91},
          doi = {10.3847/1538-4357/ab258e},
archivePrefix = {arXiv},
       eprint = {1903.05509},
 primaryClass = {astro-ph.HE},
       adsurl = {https://ui.adsabs.harvard.edu/abs/2019ApJ...879...91J},
      adsnote = {Provided by the SAO/NASA Astrophysics Data System}
}


@ARTICLE{Evoli2019,
       author = {{Evoli}, Carmelo and {Aloisio}, Roberto and {Blasi}, Pasquale},
        title = "{Galactic cosmic rays after the AMS-02 observations}",
      journal = {\prd},
     keywords = {Astrophysics - High Energy Astrophysical Phenomena},
         year = "2019",
        month = "May",
       volume = {99},
       number = {10},
          eid = {103023},
        pages = {103023},
          doi = {10.1103/PhysRevD.99.103023},
archivePrefix = {arXiv},
       eprint = {1904.10220},
 primaryClass = {astro-ph.HE},
       adsurl = {https://ui.adsabs.harvard.edu/abs/2019PhRvD..99j3023E},
      adsnote = {Provided by the SAO/NASA Astrophysics Data System}
}

@ARTICLE{Evoli2020,
       author = {{Evoli}, Carmelo and {Morlino}, Giovanni and {Blasi}, Pasquale and
         {Aloisio}, Roberto},
        title = "{AMS-02 beryllium data and its implication for cosmic ray transport}",
      journal = {\prd},
     keywords = {Astrophysics - High Energy Astrophysical Phenomena},
         year = "2020",
        month = "Jan",
       volume = {101},
       number = {2},
          eid = {023013},
        pages = {023013},
          doi = {10.1103/PhysRevD.101.023013},
archivePrefix = {arXiv},
       eprint = {1910.04113},
 primaryClass = {astro-ph.HE},
       adsurl = {https://ui.adsabs.harvard.edu/abs/2020PhRvD.101b3013E},
      adsnote = {Provided by the SAO/NASA Astrophysics Data System}
}

@ARTICLE{Harris2009,
       author = {{Harris}, Jason and {Zaritsky}, Dennis},
        title = "{The Star Formation History of the Large Magellanic Cloud}",
      journal = {\aj},
     keywords = {galaxies: evolution, galaxies: individual: Large Magellanic Cloud, galaxies: stellar content, Magellanic Clouds, Astrophysics - Cosmology and Nongalactic Astrophysics, Astrophysics - Astrophysics of Galaxies},
         year = "2009",
        month = "Nov",
       volume = {138},
       number = {5},
        pages = {1243-1260},
          doi = {10.1088/0004-6256/138/5/1243},
archivePrefix = {arXiv},
       eprint = {0908.1422},
 primaryClass = {astro-ph.CO},
       adsurl = {https://ui.adsabs.harvard.edu/abs/2009AJ....138.1243H},
      adsnote = {Provided by the SAO/NASA Astrophysics Data System}
}


@ARTICLE{Blandford1987,
       author = {{Blandford}, Roger and {Eichler}, David},
        title = "{Particle acceleration at astrophysical shocks: A theory of cosmic ray origin}",
      journal = {\physrep},
         year = "1987",
        month = "Oct",
       volume = {154},
       number = {1},
        pages = {1-75},
          doi = {10.1016/0370-1573(87)90134-7},
       adsurl = {https://ui.adsabs.harvard.edu/abs/1987PhR...154....1B},
      adsnote = {Provided by the SAO/NASA Astrophysics Data System}
}

@BOOK{Mihalas1984,
       author = {{Mihalas}, D. and {Mihalas}, B.~W.},
        title = "{Foundations of radiation hydrodynamics}",
         year = "1984",
       adsurl = {https://ui.adsabs.harvard.edu/abs/1984oup..book.....M},
      adsnote = {Provided by the SAO/NASA Astrophysics Data System}
}


@ARTICLE{Marinacci2019,
       author = {{Marinacci}, Federico and {Sales}, Laura V. and {Vogelsberger}, Mark and
         {Torrey}, Paul and {Springel}, Volker},
        title = "{Simulating the interstellar medium and stellar feedback on a moving mesh: implementation and isolated galaxies}",
      journal = {\mnras},
     keywords = {ISM: general, galaxies: evolution, galaxies: formation, galaxies: ISM, Astrophysics - Astrophysics of Galaxies},
         year = "2019",
        month = "Nov",
       volume = {489},
       number = {3},
        pages = {4233-4260},
          doi = {10.1093/mnras/stz2391},
archivePrefix = {arXiv},
       eprint = {1905.08806},
 primaryClass = {astro-ph.GA},
       adsurl = {https://ui.adsabs.harvard.edu/abs/2019MNRAS.489.4233M},
      adsnote = {Provided by the SAO/NASA Astrophysics Data System}
}

@ARTICLE{Tumlinson2013,
       author = {{Tumlinson}, Jason and {Thom}, Christopher and {Werk}, Jessica K. and
         {Prochaska}, J. Xavier and {Tripp}, Todd M. and {Katz}, Neal and
         {Dav{\'e}}, Romeel and {Oppenheimer}, Benjamin D. and
         {Meiring}, Joseph D. and {Ford}, Amanda Brady and {O'Meara}, John M. and
         {Peeples}, Molly S. and {Sembach}, Kenneth R. and {Weinberg}, David H.},
        title = "{The COS-Halos Survey: Rationale, Design, and a Census of Circumgalactic Neutral Hydrogen}",
      journal = {\apj},
     keywords = {galaxies: formation, galaxies: halos, intergalactic medium, quasars: absorption lines, Astrophysics - Cosmology and Nongalactic Astrophysics, Astrophysics - Astrophysics of Galaxies},
         year = "2013",
        month = "Nov",
       volume = {777},
       number = {1},
          eid = {59},
        pages = {59},
          doi = {10.1088/0004-637X/777/1/59},
archivePrefix = {arXiv},
       eprint = {1309.6317},
 primaryClass = {astro-ph.CO},
       adsurl = {https://ui.adsabs.harvard.edu/abs/2013ApJ...777...59T},
      adsnote = {Provided by the SAO/NASA Astrophysics Data System}
}

@ARTICLE{Weinberger2017,
       author = {{Weinberger}, Rainer and {Springel}, Volker and {Hernquist}, Lars and
         {Pillepich}, Annalisa and {Marinacci}, Federico and
         {Pakmor}, R{\"u}diger and {Nelson}, Dylan and {Genel}, Shy and
         {Vogelsberger}, Mark and {Naiman}, Jill and {Torrey}, Paul},
        title = "{Simulating galaxy formation with black hole driven thermal and kinetic feedback}",
      journal = {\mnras},
     keywords = {black hole physics, methods: numerical, galaxies: clusters: general, galaxies: evolution, galaxies: formation, cosmology: theory, Astrophysics - Astrophysics of Galaxies},
         year = "2017",
        month = "Mar",
       volume = {465},
       number = {3},
        pages = {3291-3308},
          doi = {10.1093/mnras/stw2944},
archivePrefix = {arXiv},
       eprint = {1607.03486},
 primaryClass = {astro-ph.GA},
       adsurl = {https://ui.adsabs.harvard.edu/abs/2017MNRAS.465.3291W},
      adsnote = {Provided by the SAO/NASA Astrophysics Data System}
}

@ARTICLE{Dave2019,
       author = {{Dav{\'e}}, Romeel and {Angl{\'e}s-Alc{\'a}zar}, Daniel and
         {Narayanan}, Desika and {Li}, Qi and {Rafieferantsoa}, Mika H. and
         {Appleby}, Sarah},
        title = "{SIMBA: Cosmological simulations with black hole growth and feedback}",
      journal = {\mnras},
     keywords = {galaxies: evolution, galaxies: formation, Astrophysics - Astrophysics of Galaxies, Astrophysics - Cosmology and Nongalactic Astrophysics},
         year = "2019",
        month = "Jun",
       volume = {486},
       number = {2},
        pages = {2827-2849},
          doi = {10.1093/mnras/stz937},
archivePrefix = {arXiv},
       eprint = {1901.10203},
 primaryClass = {astro-ph.GA},
       adsurl = {https://ui.adsabs.harvard.edu/abs/2019MNRAS.486.2827D},
      adsnote = {Provided by the SAO/NASA Astrophysics Data System}
}

@ARTICLE{Brook2012,
       author = {{Brook}, C.~B. and {Stinson}, G. and {Gibson}, B.~K. and {Wadsley}, J. and
         {Quinn}, T.},
        title = "{MaGICC discs: matching observed galaxy relationships over a wide stellar mass range}",
      journal = {\mnras},
     keywords = {galaxies: bulges, galaxies: evolution, galaxies: formation, galaxies: spiral, Astrophysics - Cosmology and Nongalactic Astrophysics, Astrophysics - Astrophysics of Galaxies},
         year = "2012",
        month = "Aug",
       volume = {424},
       number = {2},
        pages = {1275-1283},
          doi = {10.1111/j.1365-2966.2012.21306.x},
archivePrefix = {arXiv},
       eprint = {1201.3359},
 primaryClass = {astro-ph.CO},
       adsurl = {https://ui.adsabs.harvard.edu/abs/2012MNRAS.424.1275B},
      adsnote = {Provided by the SAO/NASA Astrophysics Data System}
}

@ARTICLE{Crain2010,
   author = {{Crain}, R.~A. and {McCarthy}, I.~G. and {Frenk}, C.~S. and 
	{Theuns}, T. and {Schaye}, J.},
    title = "{X-ray coronae in simulations of disc galaxy formation}",
  journal = {\mnras},
archivePrefix = "arXiv",
   eprint = {1005.1642},
 primaryClass = "astro-ph.CO",
 keywords = {methods: numerical, galaxies: formation, galaxies: haloes, X-rays: galaxies},
     year = 2010,
    month = sep,
   volume = 407,
    pages = {1403-1422},
      doi = {10.1111/j.1365-2966.2010.16985.x},
   adsurl = {http://esoads.eso.org/abs/2010MNRAS.407.1403C},
  adsnote = {Provided by the SAO/NASA Astrophysics Data System}
}

@ARTICLE{Farmer2004,
       author = {{Farmer}, Alison J. and {Goldreich}, Peter},
        title = "{Wave Damping by Magnetohydrodynamic Turbulence and Its Effect on Cosmic-Ray Propagation in the Interstellar Medium}",
      journal = {\apj},
     keywords = {ISM: Cosmic Rays, Magnetohydrodynamics: MHD, Turbulence, Astrophysics},
         year = "2004",
        month = "Apr",
       volume = {604},
       number = {2},
        pages = {671-674},
          doi = {10.1086/382040},
archivePrefix = {arXiv},
       eprint = {astro-ph/0311400},
 primaryClass = {astro-ph},
       adsurl = {https://ui.adsabs.harvard.edu/abs/2004ApJ...604..671F},
      adsnote = {Provided by the SAO/NASA Astrophysics Data System}
}

@ARTICLE{Yan2004,
   author = {{Yan}, H. and {Lazarian}, A.},
    title = "{Cosmic-Ray Scattering and Streaming in Compressible Magnetohydrodynamic Turbulence}",
  journal = {\apj},
   eprint = {astro-ph/0408172},
 keywords = {Acceleration of Particles, ISM: Cosmic Rays, ISM: Magnetic Fields, Magnetohydrodynamics: MHD, Scattering, Turbulence},
     year = 2004,
    month = oct,
   volume = 614,
    pages = {757-769},
      doi = {10.1086/423733},
   adsurl = {https://ui.adsabs.harvard.edu/abs/2004ApJ...614..757Y},
  adsnote = {Provided by the SAO/NASA Astrophysics Data System}
}

@ARTICLE{Lipari2014,
       author = {{Lipari}, Paolo},
        title = "{The lifetime of cosmic rays in the Milky Way}",
      journal = {arXiv e-prints},
     keywords = {Astrophysics - High Energy Astrophysical Phenomena},
         year = "2014",
        month = "Jul",
          eid = {arXiv:1407.5223},
        pages = {arXiv:1407.5223},
archivePrefix = {arXiv},
       eprint = {1407.5223},
 primaryClass = {astro-ph.HE},
       adsurl = {https://ui.adsabs.harvard.edu/abs/2014arXiv1407.5223L},
      adsnote = {Provided by the SAO/NASA Astrophysics Data System}
}

@ARTICLE{McKenzie1983,
   author = {{McKenzie}, J.~F. and {Bond}, R.~A.~B.},
    title = "{The role of non-linear Landau damping in cosmic ray shock acceleration}",
  journal = {\aap},
 keywords = {Cosmic Rays, Landau Damping, Magnetohydrodynamic Waves, Particle Acceleration, Diffusion Coefficient, Power Spectra, Supernovae},
     year = 1983,
    month = jun,
   volume = 123,
    pages = {111-117},
   adsurl = {https://ui.adsabs.harvard.edu/abs/1983Aadsnote = {Provided by the SAO/NASA Astrophysics Data System}
}

@ARTICLE{Truelove1997,
   author = {{Truelove}, J.~K. and {Klein}, R.~I. and {McKee}, C.~F. and 
	{Holliman}, II, J.~H. and {Howell}, L.~H. and {Greenough}, J.~A.
	},
    title = "{The Jeans Condition: A New Constraint on Spatial Resolution in Simulations of Isothermal Self-gravitational Hydrodynamics}",
  journal = {\apjl},
 keywords = {GRAVITATION, HYDRODYNAMICS, ISM: CLOUDS, METHODS: NUMERICAL, STARS: FORMATION, Gravitation, Hydrodynamics, ISM: Clouds, Methods: Numerical, Stars: Formation},
     year = 1997,
    month = nov,
   volume = 489,
    pages = {L179-L183},
      doi = {10.1086/310975},
   adsurl = {https://ui.adsabs.harvard.edu/abs/1997ApJ...489L.179T},
  adsnote = {Provided by the SAO/NASA Astrophysics Data System}
}

@ARTICLE{Kissmann2014,
   author = {{Kissmann}, R.},
    title = "{PICARD: A novel code for the Galactic Cosmic Ray propagation problem}",
  journal = {Astroparticle Physics},
archivePrefix = "arXiv",
   eprint = {1401.4035},
 primaryClass = "astro-ph.HE",
     year = 2014,
    month = mar,
   volume = 55,
    pages = {37-50},
      doi = {10.1016/j.astropartphys.2014.02.002},
   adsurl = {https://ui.adsabs.harvard.edu/abs/2014APh....55...37K},
  adsnote = {Provided by the SAO/NASA Astrophysics Data System}
}

@ARTICLE{Evoli2017,
   author = {{Evoli}, C. and {Gaggero}, D. and {Vittino}, A. and {Di Bernardo}, G. and 
	{Di Mauro}, M. and {Ligorini}, A. and {Ullio}, P. and {Grasso}, D.
	},
    title = "{Cosmic-ray propagation with DRAGON2: I. numerical solver and astrophysical ingredients}",
  journal = {\jcap},
archivePrefix = "arXiv",
   eprint = {1607.07886},
 primaryClass = "astro-ph.HE",
     year = 2017,
    month = feb,
   volume = 2,
      eid = {015},
    pages = {015},
      doi = {10.1088/1475-7516/2017/02/015},
   adsurl = {https://ui.adsabs.harvard.edu/abs/2017JCAP...02..015E},
  adsnote = {Provided by the SAO/NASA Astrophysics Data System}
}

@ARTICLE{Johannesson2016,
   author = {{J{\'o}hannesson}, G. and {Ruiz de Austri}, R. and {Vincent}, A.~C. and 
	{Moskalenko}, I.~V. and {Orlando}, E. and {Porter}, T.~A. and 
	{Strong}, A.~W. and {Trotta}, R. and {Feroz}, F. and {Graff}, P. and 
	{Hobson}, M.~P.},
    title = "{Bayesian Analysis of Cosmic Ray Propagation: Evidence against Homogeneous Diffusion}",
  journal = {\apj},
archivePrefix = "arXiv",
   eprint = {1602.02243},
 primaryClass = "astro-ph.HE",
 keywords = {astroparticle physics, cosmic rays, diffusion, Galaxy: general, ISM: general, methods: statistical},
     year = 2016,
    month = jun,
   volume = 824,
      eid = {16},
    pages = {16},
      doi = {10.3847/0004-637X/824/1/16},
   adsurl = {https://ui.adsabs.harvard.edu/abs/2016ApJ...824...16J},
  adsnote = {Provided by the SAO/NASA Astrophysics Data System}
}

@ARTICLE{Pfrommer2004,
       author = {{Pfrommer}, C. and {En{\ss}lin}, T.~A.},
        title = "{Constraining the population of cosmic ray protons in cooling flow clusters with {\ensuremath{\gamma}}-ray and radio observations:  Are radio mini-halos of hadronic origin?}",
      journal = {\aap},
     keywords = {galaxies: cooling flows, galaxies: cluster: general, galaxies: cluster: individual: Perseus (A426), galaxies: intergalactic medium, ISM: cosmic rays, radiation mechanisms: non-thermal},
         year = "2004",
        month = "Jan",
       volume = {413},
        pages = {17-36},
          doi = {10.1051/0004-6361:20031464},
       adsurl = {https://ui.adsabs.harvard.edu/abs/2004A&A...413...17P},
      adsnote = {Provided by the SAO/NASA Astrophysics Data System}
}

@ARTICLE{Pakmor2018,
       author = {{Pakmor}, R{\"u}diger and {Guillet}, Thomas and {Pfrommer}, Christoph and
         {G{\'o}mez}, Facundo A. and {Grand}, Robert J.~J. and
         {Marinacci}, Federico and {Simpson}, Christine M. and
         {Springel}, Volker},
        title = "{Faraday rotation maps of disc galaxies}",
      journal = {\mnras},
     keywords = {methods: numerical, galaxy: formation, galaxies: magnetic fields, Astrophysics - Astrophysics of Galaxies},
         year = "2018",
        month = "Dec",
       volume = {481},
       number = {4},
        pages = {4410-4418},
          doi = {10.1093/mnras/sty2601},
archivePrefix = {arXiv},
       eprint = {1807.02113},
 primaryClass = {astro-ph.GA},
       adsurl = {https://ui.adsabs.harvard.edu/abs/2018MNRAS.481.4410P},
      adsnote = {Provided by the SAO/NASA Astrophysics Data System}
}

@ARTICLE{Fang2013,
       author = {{Fang}, Taotao and {Bullock}, James and {Boylan-Kolchin}, Michael},
        title = "{On the Hot Gas Content of the Milky Way Halo}",
      journal = {\apj},
     keywords = {Galaxy: halo, X-rays: diffuse background, Astrophysics - Cosmology and Nongalactic Astrophysics, Astrophysics - Astrophysics of Galaxies, Astrophysics - High Energy Astrophysical Phenomena},
         year = "2013",
        month = "Jan",
       volume = {762},
       number = {1},
          eid = {20},
        pages = {20},
          doi = {10.1088/0004-637X/762/1/20},
archivePrefix = {arXiv},
       eprint = {1211.0758},
 primaryClass = {astro-ph.CO},
       adsurl = {https://ui.adsabs.harvard.edu/abs/2013ApJ...762...20F},
      adsnote = {Provided by the SAO/NASA Astrophysics Data System}
}

@ARTICLE{Buck2019b,
       author = {{Buck}, Tobias and {Obreja}, Aura and {Macci{\`o}}, Andrea V. and
         {Minchev}, Ivan and {Dutton}, Aaron A. and {Ostriker}, Jeremiah P.},
        title = "{NIHAO-UHD: the properties of MW-like stellar discs in high-resolution cosmological simulations}",
      journal = {\mnras},
     keywords = {methods: numerical, galaxies: formation, galaxies: kinematics and dynamics, Galaxy: disc, Galaxy: evolution, Galaxy: structure, Astrophysics - Astrophysics of Galaxies},
         year = "2020",
        month = "Jan",
       volume = {491},
       number = {3},
        pages = {3461-3478},
          doi = {10.1093/mnras/stz3241},
archivePrefix = {arXiv},
       eprint = {1909.05864},
 primaryClass = {astro-ph.GA},
       adsurl = {https://ui.adsabs.harvard.edu/abs/2020MNRAS.491.3461B},
      adsnote = {Provided by the SAO/NASA Astrophysics Data System}
}


@ARTICLE{Girichidis2018,
   author = {{Girichidis}, P. and {Naab}, T. and {Hanasz}, M. and {Walch}, S.
	},
    title = "{Cooler and smoother - the impact of cosmic rays on the phase structure of galactic outflows}",
  journal = {\mnras},
archivePrefix = "arXiv",
   eprint = {1805.09333},
 keywords = {MHD, cosmic rays, ISM: general, ISM: jets and outflows, ISM: structure, galaxies: ISM},
     year = 2018,
    month = sep,
   volume = 479,
    pages = {3042-3067},
      doi = {10.1093/mnras/sty1653},
   adsurl = {https://ui.adsabs.harvard.edu/abs/2018MNRAS.479.3042G},
  adsnote = {Provided by the SAO/NASA Astrophysics Data System}
}

@ARTICLE{Buck2019,
   author = {{Buck}, T. and {Dutton}, A.~A. and {Macci{\`o}}, A.~V.},
    title = "{An observational test for star formation prescriptions in cosmological hydrodynamical simulations}",
  journal = {\mnras},
archivePrefix = "arXiv",
   eprint = {1812.05613},
 keywords = {cosmology: dark matter, galaxies: formation, galaxies: structure, galaxies: star formation, ISM: structure, methods: N-body simulation},
     year = 2019,
    month = jun,
   volume = 486,
    pages = {1481-1487},
      doi = {10.1093/mnras/stz969},
   adsurl = {https://ui.adsabs.harvard.edu/abs/2019MNRAS.486.1481B},
  adsnote = {Provided by the SAO/NASA Astrophysics Data System}
}

@ARTICLE{Chen2016,
   author = {{Chen}, J. and {Bryan}, G.~L. and {Salem}, M.},
    title = "{Cosmological simulations of dwarf galaxies with cosmic ray feedback}",
  journal = {\mnras},
archivePrefix = "arXiv",
   eprint = {1605.06115},
 keywords = {methods: numerical, galaxies: formation},
     year = 2016,
    month = aug,
   volume = 460,
    pages = {3335-3344},
      doi = {10.1093/mnras/stw1197},
   adsurl = {https://ui.adsabs.harvard.edu/abs/2016MNRAS.460.3335C},
  adsnote = {Provided by the SAO/NASA Astrophysics Data System}
}

@ARTICLE{Salem2016,
   author = {{Salem}, M. and {Bryan}, G.~L. and {Corlies}, L.},
    title = "{Role of cosmic rays in the circumgalactic medium}",
  journal = {\mnras},
archivePrefix = "arXiv",
   eprint = {1511.05144},
 keywords = {methods: numerical, cosmic rays, galaxies: formation},
     year = 2016,
    month = feb,
   volume = 456,
    pages = {582-601},
      doi = {10.1093/mnras/stv2641},
   adsurl = {https://ui.adsabs.harvard.edu/abs/2016MNRAS.456..582S},
  adsnote = {Provided by the SAO/NASA Astrophysics Data System}
}

@ARTICLE{Bell2003,
   author = {{Bell}, E.~F.},
    title = "{Estimating Star Formation Rates from Infrared and Radio Luminosities: The Origin of the Radio-Infrared Correlation}",
  journal = {\apj},
   eprint = {astro-ph/0212121},
 keywords = {ISM: Cosmic Rays, ISM: Dust, Extinction, Galaxies: General, Infrared: Galaxies, Radio Continuum: Galaxies, Ultraviolet: Galaxies},
     year = 2003,
    month = apr,
   volume = 586,
    pages = {794-813},
      doi = {10.1086/367829},
   adsurl = {http://adsabs.harvard.edu/abs/2003ApJ...586..794B},
  adsnote = {Provided by the SAO/NASA Astrophysics Data System}
}
@ARTICLE{Kennicutt1998ARA+A,
   author = {{Kennicutt}, Jr., R.~C.},
    title = "{Star Formation in Galaxies Along the Hubble Sequence}",
  journal = {\araa},
   eprint = {astro-ph/9807187},
     year = 1998,
   volume = 36,
    pages = {189-232},
      doi = {10.1146/annurev.astro.36.1.189},
   adsurl = {http://adsabs.harvard.edu/abs/1998ARAadsnote = {Provided by the SAO/NASA Astrophysics Data System}
}

@ARTICLE{Grand2019,
       author = {{Grand}, Robert J.~J. and {van de Voort}, Freeke and {Zjupa}, Jolanta and
         {Fragkoudi}, Francesca and {G{\'o}mez}, Facundo A. and
         {Kauffmann}, Guinevere and {Marinacci}, Federico and
         {Pakmor}, R{\"u}diger and {Springel}, Volker and {White}, Simon D.~M.},
        title = "{Gas accretion and galactic fountain flows in the Auriga cosmological simulations: angular momentum and metal redistribution}",
      journal = {\mnras},
     keywords = {galaxies: evolution, galaxies: formation, galaxies: spiral, galaxies: structure, Astrophysics - Astrophysics of Galaxies, Astrophysics - Cosmology and Nongalactic Astrophysics},
         year = 2019,
        month = dec,
       volume = {490},
       number = {4},
        pages = {4786-4803},
          doi = {10.1093/mnras/stz2928},
archivePrefix = {arXiv},
       eprint = {1909.04038},
 primaryClass = {astro-ph.GA},
       adsurl = {https://ui.adsabs.harvard.edu/abs/2019MNRAS.490.4786G},
      adsnote = {Provided by the SAO/NASA Astrophysics Data System}
}


@ARTICLE{Zweibel2017,
       author = {{Zweibel}, Ellen G.},
        title = "{The basis for cosmic ray feedback: Written on the wind}",
      journal = {Physics of Plasmas},
         year = "2017",
        month = "May",
       volume = {24},
       number = {5},
          eid = {055402},
        pages = {055402},
          doi = {10.1063/1.4984017},
       adsurl = {https://ui.adsabs.harvard.edu/abs/2017PhPl...24e5402Z},
      adsnote = {Provided by the SAO/NASA Astrophysics Data System}
}

@ARTICLE{Vogelsberger2013,
       author = {{Vogelsberger}, Mark and {Genel}, Shy and {Sijacki}, Debora and
         {Torrey}, Paul and {Springel}, Volker and {Hernquist}, Lars},
        title = "{A model for cosmological simulations of galaxy formation physics}",
      journal = {\mnras},
     keywords = {methods: numerical, cosmology: theory, Astrophysics - Cosmology and Nongalactic Astrophysics},
         year = "2013",
        month = "Dec",
       volume = {436},
       number = {4},
        pages = {3031-3067},
          doi = {10.1093/mnras/stt1789},
archivePrefix = {arXiv},
       eprint = {1305.2913},
 primaryClass = {astro-ph.CO},
       adsurl = {https://ui.adsabs.harvard.edu/abs/2013MNRAS.436.3031V},
      adsnote = {Provided by the SAO/NASA Astrophysics Data System}
}

@ARTICLE{Salem2014b,
   author = {{Salem}, M. and {Bryan}, G.~L. and {Hummels}, C.},
    title = "{Cosmological Simulations of Galaxy Formation with Cosmic Rays}",
  journal = {\apjl},
archivePrefix = "arXiv",
   eprint = {1412.0661},
 keywords = {cosmic rays, galaxies: formation, methods: numerical},
     year = 2014,
    month = dec,
   volume = 797,
      eid = {L18},
    pages = {L18},
      doi = {10.1088/2041-8205/797/2/L18},
   adsurl = {https://ui.adsabs.harvard.edu/abs/2014ApJ...797L..18S},
  adsnote = {Provided by the SAO/NASA Astrophysics Data System}
}

@ARTICLE{Tabatabaei2013,
   author = {{Tabatabaei}, F.~S. and {Schinnerer}, E. and {Murphy}, E.~J. and 
	{Beck}, R. and {Groves}, B. and {Meidt}, S. and {Krause}, M. and 
	{Rix}, H.-W. and {Sandstrom}, K. and {Crocker}, A.~F. and {Galametz}, M. and 
	{Helou}, G. and {Wilson}, C.~D. and {Kennicutt}, R. and {Calzetti}, D. and 
	{Draine}, B. and {Aniano}, G. and {Dale}, D. and {Dumas}, G. and 
	{Engelbracht}, C.~W. and {Gordon}, K.~D. and {Hinz}, J. and 
	{Kreckel}, K. and {Montiel}, E. and {Roussel}, H.},
    title = "{A detailed study of the radio-FIR correlation in NGC 6946 with Herschel-PACS/SPIRE from KINGFISH}",
  journal = {\aap},
archivePrefix = "arXiv",
   eprint = {1301.6884},
 keywords = {galaxies: individual: NGC 6946, radio continuum: ISM, ISM: magnetic fields, infrared: ISM, galaxies: star formation},
     year = 2013,
    month = apr,
   volume = 552,
      eid = {A19},
    pages = {A19},
      doi = {10.1051/0004-6361/201220249},
   adsurl = {https://ui.adsabs.harvard.edu/abs/2013Aadsnote = {Provided by the SAO/NASA Astrophysics Data System}
}

@ARTICLE{Ptuskin2006,
   author = {{Ptuskin}, V.~S. and {Moskalenko}, I.~V. and {Jones}, F.~C. and 
	{Strong}, A.~W. and {Zirakashvili}, V.~N.},
    title = "{Dissipation of Magnetohydrodynamic Waves on Energetic Particles: Impact on Interstellar Turbulence and Cosmic-Ray Transport}",
  journal = {\apj},
   eprint = {astro-ph/0510335},
 keywords = {ISM: Cosmic Rays, Diffusion, Elementary Particles, Magnetohydrodynamics: MHD, Turbulence, Waves},
     year = 2006,
    month = may,
   volume = 642,
    pages = {902-916},
      doi = {10.1086/501117},
   adsurl = {https://ui.adsabs.harvard.edu/abs/2006ApJ...642..902P},
  adsnote = {Provided by the SAO/NASA Astrophysics Data System}
}
@ARTICLE{Strong1998,
   author = {{Strong}, A.~W. and {Moskalenko}, I.~V.},
    title = "{Propagation of Cosmic-Ray Nucleons in the Galaxy}",
  journal = {\apj},
   eprint = {astro-ph/9807150},
 keywords = {ACCELERATION OF PARTICLES, ISM: COSMIC RAYS, DIFFUSION, GALAXY: GENERAL, ISM: ABUNDANCES, ISM: GENERAL, Acceleration of Particles, ISM: Cosmic Rays, Diffusion, Galaxy: General, ISM: Abundances, ISM: General},
     year = 1998,
    month = dec,
   volume = 509,
    pages = {212-228},
      doi = {10.1086/306470},
   adsurl = {https://ui.adsabs.harvard.edu/abs/1998ApJ...509..212S},
  adsnote = {Provided by the SAO/NASA Astrophysics Data System}
}

@ARTICLE{Powell1999,
   author = {{Powell}, K.~G. and {Roe}, P.~L. and {Linde}, T.~J. and {Gombosi}, T.~I. and 
	{De Zeeuw}, D.~L.},
    title = "{A Solution-Adaptive Upwind Scheme for Ideal Magnetohydrodynamics}",
  journal = {Journal of Computational Physics},
     year = 1999,
    month = sep,
   volume = 154,
    pages = {284-309},
      doi = {10.1006/jcph.1999.6299},
   adsurl = {https://ui.adsabs.harvard.edu/abs/1999JCoPh.154..284P},
  adsnote = {Provided by the SAO/NASA Astrophysics Data System}
}

@ARTICLE{Bondi1952,
   author = {{Bondi}, H.},
    title = "{On spherically symmetrical accretion}",
  journal = {\mnras},
     year = 1952,
   volume = 112,
    pages = {195},
      doi = {10.1093/mnras/112.2.195},
   adsurl = {https://ui.adsabs.harvard.edu/abs/1952MNRAS.112..195B},
  adsnote = {Provided by the SAO/NASA Astrophysics Data System}
}

@ARTICLE{Bondi1944,
   author = {{Bondi}, H. and {Hoyle}, F.},
    title = "{On the mechanism of accretion by stars}",
  journal = {\mnras},
     year = 1944,
   volume = 104,
    pages = {273},
      doi = {10.1093/mnras/104.5.273},
   adsurl = {https://ui.adsabs.harvard.edu/abs/1944MNRAS.104..273B},
  adsnote = {Provided by the SAO/NASA Astrophysics Data System}
}

@ARTICLE{Faucher2009,
   author = {{Faucher-Gigu{\`e}re}, C.-A. and {Lidz}, A. and {Zaldarriaga}, M. and 
	{Hernquist}, L.},
    title = "{A New Calculation of the Ionizing Background Spectrum and the Effects of He II Reionization}",
  journal = {\apj},
archivePrefix = "arXiv",
   eprint = {0901.4554},
 primaryClass = "astro-ph.CO",
 keywords = {cosmology: theory, diffuse radiation, galaxies: evolution, galaxies: formation, galaxies: high-redshift, quasars: absorption lines},
     year = 2009,
    month = oct,
   volume = 703,
    pages = {1416-1443},
      doi = {10.1088/0004-637X/703/2/1416},
   adsurl = {https://ui.adsabs.harvard.edu/abs/2009ApJ...703.1416F},
  adsnote = {Provided by the SAO/NASA Astrophysics Data System}
}

@ARTICLE{Travaglio2004,
   author = {{Travaglio}, C. and {Hillebrandt}, W. and {Reinecke}, M. and 
	{Thielemann}, F.-K.},
    title = "{Nucleosynthesis in multi-dimensional SN Ia explosions}",
  journal = {\aap},
   eprint = {astro-ph/0406281},
 keywords = {hydrodynamics, nuclear reactions, nucleosynthesis, abundances, stars: supernovae: general},
     year = 2004,
    month = oct,
   volume = 425,
    pages = {1029-1040},
      doi = {10.1051/0004-6361:20041108},
   adsurl = {https://ui.adsabs.harvard.edu/abs/2004Aadsnote = {Provided by the SAO/NASA Astrophysics Data System}
}

@ARTICLE{Thielemann2003,
   author = {{Thielemann}, F.-K. and {Argast}, D. and {Brachwitz}, F. and 
	{Hix}, W.~R. and {H{\"o}flich}, P. and {Liebend{\"o}rfer}, M. and 
	{Martinez-Pinedo}, G. and {Mezzacappa}, A. and {Panov}, I. and 
	{Rauscher}, T.},
    title = "{Nuclear cross sections, nuclear structure and stellar nucleosynthesis}",
  journal = {Nuclear Physics A},
     year = 2003,
    month = may,
   volume = 718,
    pages = {139-146},
      doi = {10.1016/S0375-9474(03)00704-8},
   adsurl = {https://ui.adsabs.harvard.edu/abs/2003NuPhA.718..139T},
  adsnote = {Provided by the SAO/NASA Astrophysics Data System}
}

@ARTICLE{Portinari1998,
   author = {{Portinari}, L. and {Chiosi}, C. and {Bressan}, A.},
    title = "{Galactic chemical enrichment with new metallicity dependent stellar yields}",
  journal = {\aap},
   eprint = {astro-ph/9711337},
 keywords = {NUCLEAR REACTIONS, NUCLEOSYNTHESIS, ABUNDANCES, STARS: MASS LOSS, SUPERNOVAE: GENERAL, GALAXY: EVOLUTION, GALAXY: ABUNDANCES, SOLAR NEIGHBOURHOOD},
     year = 1998,
    month = jun,
   volume = 334,
    pages = {505-539},
   adsurl = {https://ui.adsabs.harvard.edu/abs/1998Aadsnote = {Provided by the SAO/NASA Astrophysics Data System}
}

@ARTICLE{Karakas2010,
   author = {{Karakas}, A.~I.},
    title = "{Updated stellar yields from asymptotic giant branch models}",
  journal = {\mnras},
archivePrefix = "arXiv",
   eprint = {0912.2142},
 primaryClass = "astro-ph.SR",
 keywords = {nuclear reactions, nucleosynthesis, abundances, stars: AGB and post-AGB, stars: Population II, ISM: abundances},
     year = 2010,
    month = apr,
   volume = 403,
    pages = {1413-1425},
      doi = {10.1111/j.1365-2966.2009.16198.x},
   adsurl = {https://ui.adsabs.harvard.edu/abs/2010MNRAS.403.1413K},
  adsnote = {Provided by the SAO/NASA Astrophysics Data System}
}

@ARTICLE{Bell2004,
       author = {{Bell}, A.~R.},
        title = "{Turbulent amplification of magnetic field and diffusive shock acceleration of cosmic rays}",
      journal = {\mnras},
     keywords = {acceleration of particles, magnetic fields, plasmas, shock waves, turbulence, cosmic rays},
         year = "2004",
        month = "Sep",
       volume = {353},
       number = {2},
        pages = {550-558},
          doi = {10.1111/j.1365-2966.2004.08097.x},
       adsurl = {https://ui.adsabs.harvard.edu/abs/2004MNRAS.353..550B},
      adsnote = {Provided by the SAO/NASA Astrophysics Data System}
}

@ARTICLE{Thomas2019,
       author = {{Thomas}, T. and {Pfrommer}, C.},
        title = "{Cosmic-ray hydrodynamics: Alfv{\'e}n-wave regulated transport of cosmic rays}",
      journal = {\mnras},
     keywords = {hydrodynamics, radiative transfer, methods: analytical, methods: numerical, cosmic rays, Astrophysics - High Energy Astrophysical Phenomena, Astrophysics - Cosmology and Nongalactic Astrophysics, Astrophysics - Astrophysics of Galaxies},
         year = "2019",
        month = "May",
       volume = {485},
       number = {3},
        pages = {2977-3008},
          doi = {10.1093/mnras/stz263},
archivePrefix = {arXiv},
       eprint = {1805.11092},
 primaryClass = {astro-ph.HE},
       adsurl = {https://ui.adsabs.harvard.edu/abs/2019MNRAS.485.2977T},
      adsnote = {Provided by the SAO/NASA Astrophysics Data System}
}

@ARTICLE{Jiang2018,
       author = {{Jiang}, Yan-Fei and {Oh}, S. Peng},
        title = "{A New Numerical Scheme for Cosmic-Ray Transport}",
      journal = {\apj},
     keywords = {cosmic rays, galaxies: clusters: intracluster medium, magnetohydrodynamics: MHD, methods: numerical, Astrophysics - High Energy Astrophysical Phenomena},
         year = "2018",
        month = "Feb",
       volume = {854},
       number = {1},
          eid = {5},
        pages = {5},
          doi = {10.3847/1538-4357/aaa6ce},
archivePrefix = {arXiv},
       eprint = {1712.07117},
 primaryClass = {astro-ph.HE},
       adsurl = {https://ui.adsabs.harvard.edu/abs/2018ApJ...854....5J},
      adsnote = {Provided by the SAO/NASA Astrophysics Data System}
}

@Article{Hanasz2003,
  author  = {Hanasz, M. and Lesch, H.},
  title   = {Incorporation of cosmic ray transport into the ZEUS MHD code. Application for studies of Parker instability in the ISM},
  journal = {\aap},
  year    = {2003},
  volume  = {412},
  month   = dec,
  pages   = {331-339},
  doi     = {10.1051/0004-6361:20031433},
  eprint  = {astro-ph/0309660},
  url     = {http://adsabs.harvard.edu/abs/2003A}

@Article{Mignone2018,
  author        = {Mignone, A. and Bodo, G. and Vaidya, B. and Mattia, G.},
  title         = {A Particle Module for the PLUTO Code. I. An Implementation of the MHD-PIC Equations},
  journal       = {\apj},
  year          = {2018},
  volume        = {859},
  eid           = {13},
  month         = may,
  pages         = {13},
  doi           = {10.3847/1538-4357/aabccd},
  eprint        = {1804.01946},
  url           = {http://adsabs.harvard.edu/abs/2018ApJ...859...13M},
  archiveprefix = {arXiv},
  primaryclass  = {astro-ph.HE},
}

@ARTICLE{McNamara2007,
       author = {{McNamara}, B.~R. and {Nulsen}, P.~E.~J.},
        title = "{Heating Hot Atmospheres with Active Galactic Nuclei}",
      journal = {\araa},
     keywords = {Astrophysics},
         year = "2007",
        month = "Sep",
       volume = {45},
       number = {1},
        pages = {117-175},
          doi = {10.1146/annurev.astro.45.051806.110625},
archivePrefix = {arXiv},
       eprint = {0709.2152},
 primaryClass = {astro-ph},
       adsurl = {https://ui.adsabs.harvard.edu/abs/2007ARA&A..45..117M},
      adsnote = {Provided by the SAO/NASA Astrophysics Data System}
}

@ARTICLE{Jacob2018,
       author = {{Jacob}, Svenja and {Pakmor}, R{\"u}diger and {Simpson}, Christine M. and
         {Springel}, Volker and {Pfrommer}, Christoph},
        title = "{The dependence of cosmic ray-driven galactic winds on halo mass}",
      journal = {\mnras},
     keywords = {hydrodynamics, cosmic rays, galaxies: formation, galaxies: starburst, Astrophysics - Astrophysics of Galaxies},
         year = "2018",
        month = "Mar",
       volume = {475},
       number = {1},
        pages = {570-584},
          doi = {10.1093/mnras/stx3221},
archivePrefix = {arXiv},
       eprint = {1712.04947},
 primaryClass = {astro-ph.GA},
       adsurl = {https://ui.adsabs.harvard.edu/abs/2018MNRAS.475..570J},
      adsnote = {Provided by the SAO/NASA Astrophysics Data System}
}

@ARTICLE{Hanasz2013,
       author = {{Hanasz}, M. and {Lesch}, H. and {Naab}, T. and {Gawryszczak}, A. and
         {Kowalik}, K. and {W{\'o}lta{\'n}ski}, D.},
        title = "{Cosmic Rays Can Drive Strong Outflows from Gas-rich High-redshift Disk Galaxies}",
      journal = {\apjl},
     keywords = {cosmic rays, galaxies: general, galaxies: ISM, ISM: magnetic fields, Astrophysics - Astrophysics of Galaxies, Astrophysics - Cosmology and Nongalactic Astrophysics},
         year = "2013",
        month = "Nov",
       volume = {777},
       number = {2},
          eid = {L38},
        pages = {L38},
          doi = {10.1088/2041-8205/777/2/L38},
archivePrefix = {arXiv},
       eprint = {1310.3273},
 primaryClass = {astro-ph.GA},
       adsurl = {https://ui.adsabs.harvard.edu/abs/2013ApJ...777L..38H},
      adsnote = {Provided by the SAO/NASA Astrophysics Data System}
}

@ARTICLE{Simpson2016,
       author = {{Simpson}, Christine M. and {Pakmor}, R{\"u}diger and
         {Marinacci}, Federico and {Pfrommer}, Christoph and {Springel}, Volker and
         {Glover}, Simon C.~O. and {Clark}, Paul C. and {Smith}, Rowan J.},
        title = "{The Role of Cosmic-Ray Pressure in Accelerating Galactic Outflows}",
      journal = {\apjl},
     keywords = {cosmic rays, galaxies: evolution, galaxies: magnetic fields, Astrophysics - Astrophysics of Galaxies},
         year = "2016",
        month = "Aug",
       volume = {827},
       number = {2},
          eid = {L29},
        pages = {L29},
          doi = {10.3847/2041-8205/827/2/L29},
archivePrefix = {arXiv},
       eprint = {1606.02324},
 primaryClass = {astro-ph.GA},
       adsurl = {https://ui.adsabs.harvard.edu/abs/2016ApJ...827L..29S},
      adsnote = {Provided by the SAO/NASA Astrophysics Data System}
}

@ARTICLE{Ehlert2018,
       author = {{Ehlert}, K. and {Weinberger}, R. and {Pfrommer}, C. and {Pakmor}, R. and
         {Springel}, V.},
        title = "{Simulations of the dynamics of magnetized jets and cosmic rays in galaxy clusters}",
      journal = {\mnras},
     keywords = {MHD, methods: numerical, cosmic rays, galaxies: active, galaxies: clusters: intracluster medium, galaxies: jets, Astrophysics - Cosmology and Nongalactic Astrophysics, Astrophysics - Astrophysics of Galaxies},
         year = "2018",
        month = "Dec",
       volume = {481},
       number = {3},
        pages = {2878-2900},
          doi = {10.1093/mnras/sty2397},
archivePrefix = {arXiv},
       eprint = {1806.05679},
 primaryClass = {astro-ph.CO},
       adsurl = {https://ui.adsabs.harvard.edu/abs/2018MNRAS.481.2878E},
      adsnote = {Provided by the SAO/NASA Astrophysics Data System}
}

@ARTICLE{Gaspari2013,
       author = {{Gaspari}, M. and {Ruszkowski}, M. and {Oh}, S. Peng},
        title = "{Chaotic cold accretion on to black holes}",
      journal = {\mnras},
     keywords = {black hole physics, hydrodynamics, instabilities, turbulence, methods: numerical, galaxies: ISM, Astrophysics - Cosmology and Nongalactic Astrophysics, Astrophysics - High Energy Astrophysical Phenomena},
         year = "2013",
        month = "Jul",
       volume = {432},
       number = {4},
        pages = {3401-3422},
          doi = {10.1093/mnras/stt692},
archivePrefix = {arXiv},
       eprint = {1301.3130},
 primaryClass = {astro-ph.CO},
       adsurl = {https://ui.adsabs.harvard.edu/abs/2013MNRAS.432.3401G},
      adsnote = {Provided by the SAO/NASA Astrophysics Data System}
}

@ARTICLE{Zhang2019,
   author = {{Zhang}, Y. and {Peng}, F.-K. and {Wang}, X.-Y.},
    title = "{Interpreting the Relation between the Gamma-Ray and Infrared Luminosities of Star-forming Galaxies}",
  journal = {\apj},
archivePrefix = "arXiv",
   eprint = {1902.09654},
 primaryClass = "astro-ph.HE",
 keywords = {cosmic rays, galaxies: star formation, gamma rays: ISM },
     year = 2019,
    month = apr,
   volume = 874,
      eid = {173},
    pages = {173},
      doi = {10.3847/1538-4357/ab0ae2},
   adsurl = {https://ui.adsabs.harvard.edu/abs/2019ApJ...874..173Z},
  adsnote = {Provided by the SAO/NASA Astrophysics Data System}
}

@ARTICLE{Chan2019,
   author = {{Chan}, T.~K. and {Kere{\v s}}, D. and {Hopkins}, P.~F. and 
	{Quataert}, E. and {Su}, K.-Y. and {Hayward}, C.~C. and {Faucher-Gigu{\`e}re}, C.-A.
	},
    title = "{Cosmic ray feedback in the FIRE simulations: constraining cosmic ray propagation with GeV {$\gamma$}-ray emission}",
  journal = {\mnras},
archivePrefix = "arXiv",
   eprint = {1812.10496},
 keywords = {galaxies: evolution, cosmic rays, gamma-rays: galaxies, galaxies: kinematics and dynamics, galaxies: starburst},
     year = 2019,
    month = sep,
   volume = 488,
    pages = {3716-3744},
      doi = {10.1093/mnras/stz1895},
   adsurl = {https://ui.adsabs.harvard.edu/abs/2019MNRAS.488.3716C},
  adsnote = {Provided by the SAO/NASA Astrophysics Data System}
}

@ARTICLE{Ji2019,
   author = {{Ji}, S. and {Chan}, T.~K. and {Hummels}, C.~B. and {Hopkins}, P.~F. and 
	{Stern}, J. and {Kere{\v s}}, D. and {Quataert}, E. and {Faucher-Gigu{\`e}re}, C.-A. and 
	{Murray}, N.},
    title = "{Properties of the Circumgalactic Medium in Cosmic Ray-Dominated Galaxy Halos}",
  journal = {arXiv e-prints},
archivePrefix = "arXiv",
   eprint = {1909.00003},
 keywords = {Astrophysics - Astrophysics of Galaxies, Astrophysics - Cosmology and Nongalactic Astrophysics, Astrophysics - High Energy Astrophysical Phenomena},
     year = 2019,
    month = aug,
   adsurl = {https://ui.adsabs.harvard.edu/abs/2019arXiv190900003J},
  adsnote = {Provided by the SAO/NASA Astrophysics Data System}
}

@ARTICLE{Hopkins2019,
   author = {{Hopkins}, P.~F. and {Chan}, T.~K. and {Garrison-Kimmel}, S. and 
	{Ji}, S. and {Su}, K.-Y. and {Hummels}, C.~B. and {Keres}, D. and 
	{Quataert}, E. and {Faucher-Giguere}, C.-A.},
    title = "{But What About... Cosmic Rays, Magnetic Fields, Conduction, {\amp} Viscosity in Galaxy Formation}",
  journal = {arXiv e-prints},
archivePrefix = "arXiv",
   eprint = {1905.04321},
 keywords = {Astrophysics - Astrophysics of Galaxies, Astrophysics - Cosmology and Nongalactic Astrophysics, Astrophysics - High Energy Astrophysical Phenomena},
     year = 2019,
    month = may,
   adsurl = {https://ui.adsabs.harvard.edu/abs/2019arXiv190504321H},
  adsnote = {Provided by the SAO/NASA Astrophysics Data System}
}

@ARTICLE{Everett2011,
   author = {{Everett}, J.~E. and {Zweibel}, E.~G.},
    title = "{The Interaction of Cosmic Rays with Diffuse Clouds}",
  journal = {\apj},
archivePrefix = "arXiv",
   eprint = {1107.1243},
 keywords = {cosmic rays, ISM: clouds, ISM: jets and outflows, ISM: magnetic fields},
     year = 2011,
    month = oct,
   volume = 739,
      eid = {60},
    pages = {60},
      doi = {10.1088/0004-637X/739/2/60},
   adsurl = {https://ui.adsabs.harvard.edu/abs/2011ApJ...739...60E},
  adsnote = {Provided by the SAO/NASA Astrophysics Data System}
}


@ARTICLE{Evoli2018,
   author = {{Evoli}, C. and {Blasi}, P. and {Morlino}, G. and {Aloisio}, R.
	},
    title = "{Origin of the Cosmic Ray Galactic Halo Driven by Advected Turbulence and Self-Generated Waves}",
  journal = {Physical Review Letters},
archivePrefix = "arXiv",
   eprint = {1806.04153},
 primaryClass = "astro-ph.HE",
     year = 2018,
    month = jul,
   volume = 121,
   number = 2,
      eid = {021102},
    pages = {021102},
      doi = {10.1103/PhysRevLett.121.021102},
   adsurl = {https://ui.adsabs.harvard.edu/abs/2018PhRvL.121b1102E},
  adsnote = {Provided by the SAO/NASA Astrophysics Data System}
}

@ARTICLE{Kulsrud1969,
   author = {{Kulsrud}, R. and {Pearce}, W.~P.},
    title = "{The Effect of Wave-Particle Interactions on the Propagation of Cosmic Rays}",
  journal = {\apj},
     year = 1969,
    month = may,
   volume = 156,
    pages = {445},
      doi = {10.1086/149981},
   adsurl = {https://ui.adsabs.harvard.edu/abs/1969ApJ...156..445K},
  adsnote = {Provided by the SAO/NASA Astrophysics Data System}
}

@ARTICLE{Tang2014,
   author = {{Tang}, Q.-W. and {Wang}, X.-Y. and {Tam}, P.-H.~T.},
    title = "{Discovery of GeV Emission from the Direction of the Luminous Infrared Galaxy NGC 2146}",
  journal = {\apj},
archivePrefix = "arXiv",
   eprint = {1407.3391},
 primaryClass = "astro-ph.HE",
 keywords = {cosmic rays, galaxies: starburst},
     year = 2014,
    month = oct,
   volume = 794,
      eid = {26},
    pages = {26},
      doi = {10.1088/0004-637X/794/1/26},
   adsurl = {https://ui.adsabs.harvard.edu/abs/2014ApJ...794...26T},
  adsnote = {Provided by the SAO/NASA Astrophysics Data System}
}

@ARTICLE{Peng2016,
   author = {{Peng}, F.-K. and {Wang}, X.-Y. and {Liu}, R.-Y. and {Tang}, Q.-W. and 
	{Wang}, J.-F.},
    title = "{First Detection of GeV Emission from an Ultraluminous Infrared Galaxy: Arp 220 as Seen with the Fermi Large Area Telescope}",
  journal = {\apjl},
archivePrefix = "arXiv",
   eprint = {1603.06355},
 primaryClass = "astro-ph.HE",
 keywords = {gamma rays: galaxies, galaxies: starburst, cosmic rays},
     year = 2016,
    month = apr,
   volume = 821,
      eid = {L20},
    pages = {L20},
      doi = {10.3847/2041-8205/821/2/L20},
   adsurl = {https://ui.adsabs.harvard.edu/abs/2016ApJ...821L..20P},
  adsnote = {Provided by the SAO/NASA Astrophysics Data System}
}

@ARTICLE{Griffin2016,
   author = {{Griffin}, R.~D. and {Dai}, X. and {Thompson}, T.~A.},
    title = "{Constraining Gamma-Ray Emission from Luminous Infrared Galaxies with Fermi-LAT; Tentative Detection of Arp 220}",
  journal = {\apjl},
archivePrefix = "arXiv",
   eprint = {1603.06949},
 primaryClass = "astro-ph.HE",
 keywords = {acceleration of particles, galaxies: individual: Arp 220, galaxies: starburst, gamma rays: galaxies},
     year = 2016,
    month = may,
   volume = 823,
      eid = {L17},
    pages = {L17},
      doi = {10.3847/2041-8205/823/1/L17},
   adsurl = {https://ui.adsabs.harvard.edu/abs/2016ApJ...823L..17G},
  adsnote = {Provided by the SAO/NASA Astrophysics Data System}
}

@ARTICLE{Ackermann2012,
   author = {{Ackermann}, M. and {Ajello}, M. and {Allafort}, A. and {Atwood}, W.~B. and 
	{Baldini}, L. and {Barbiellini}, G. and {Bastieri}, D. and {Bechtol}, K. and 
	{Bellazzini}, R. and {Berenji}, B. and {Blandford}, R.~D. and 
	{Bloom}, E.~D. and {Bonamente}, E. and {Borgland}, A.~W. and 
	{Bouvier}, A. and {Bregeon}, J. and {Brigida}, M. and {Bruel}, P. and 
	{Buehler}, R. and {Buson}, S. and {Caliandro}, G.~A. and {Cameron}, R.~A. and 
	{Caraveo}, P.~A. and {Casandjian}, J.~M. and {Cecchi}, C. and 
	{Charles}, E. and {Chekhtman}, A. and {Cheung}, C.~C. and {Chiang}, J. and 
	{Ciprini}, S. and {Claus}, R. and {Cohen-Tanugi}, J. and {Conrad}, J. and 
	{Cutini}, S. and {de Angelis}, A. and {de Palma}, F. and {Dermer}, C.~D. and 
	{Digel}, S.~W. and {Do Couto E Silva}, E. and {Drell}, P.~S. and 
	{Drlica-Wagner}, A. and {Favuzzi}, C. and {Fegan}, S.~J. and 
	{Ferrara}, E.~C. and {Focke}, W.~B. and {Fortin}, P. and {Fukazawa}, Y. and 
	{Funk}, S. and {Fusco}, P. and {Gargano}, F. and {Gasparrini}, D. and 
	{Germani}, S. and {Giglietto}, N. and {Giommi}, P. and {Giordano}, F. and 
	{Giroletti}, M. and {Glanzman}, T. and {Godfrey}, G. and {Grenier}, I.~A. and 
	{Grove}, J.~E. and {Guiriec}, S. and {Gustafsson}, M. and {Hadasch}, D. and 
	{Harding}, A.~K. and {Hayashida}, M. and {Hughes}, R.~E. and 
	{J{\'o}hannesson}, G. and {Johnson}, A.~S. and {Kamae}, T. and 
	{Katagiri}, H. and {Kataoka}, J. and {Kn{\"o}dlseder}, J. and 
	{Kuss}, M. and {Lande}, J. and {Latronico}, L. and {Lemoine-Goumard}, M. and 
	{Llena Garde}, M. and {Longo}, F. and {Loparco}, F. and {Lovellette}, M.~N. and 
	{Lubrano}, P. and {Madejski}, G.~M. and {Mazziotta}, M.~N. and 
	{McEnery}, J.~E. and {Michelson}, P.~F. and {Mitthumsiri}, W. and 
	{Mizuno}, T. and {Moiseev}, A.~A. and {Monte}, C. and {Monzani}, M.~E. and 
	{Morselli}, A. and {Moskalenko}, I.~V. and {Murgia}, S. and 
	{Nakamori}, T. and {Nolan}, P.~L. and {Norris}, J.~P. and {Nuss}, E. and 
	{Ohno}, M. and {Ohsugi}, T. and {Okumura}, A. and {Omodei}, N. and 
	{Orlando}, E. and {Ormes}, J.~F. and {Ozaki}, M. and {Paneque}, D. and 
	{Parent}, D. and {Pesce-Rollins}, M. and {Pierbattista}, M. and 
	{Piron}, F. and {Pivato}, G. and {Porter}, T.~A. and {Rain{\`o}}, S. and 
	{Rando}, R. and {Razzano}, M. and {Razzaque}, S. and {Reimer}, A. and 
	{Reimer}, O. and {Reposeur}, T. and {Ritz}, S. and {Romani}, R.~W. and 
	{Roth}, M. and {Sadrozinski}, H.~F.-W. and {Sbarra}, C. and 
	{Schalk}, T.~L. and {Sgr{\`o}}, C. and {Siskind}, E.~J. and 
	{Spandre}, G. and {Spinelli}, P. and {Strong}, A.~W. and {Takahashi}, H. and 
	{Takahashi}, T. and {Tanaka}, T. and {Thayer}, J.~G. and {Thayer}, J.~B. and 
	{Tibaldo}, L. and {Tinivella}, M. and {Torres}, D.~F. and {Tosti}, G. and 
	{Troja}, E. and {Uchiyama}, Y. and {Usher}, T.~L. and {Vandenbroucke}, J. and 
	{Vasileiou}, V. and {Vianello}, G. and {Vitale}, V. and {Waite}, A.~P. and 
	{Winer}, B.~L. and {Wood}, K.~S. and {Wood}, M. and {Yang}, Z. and 
	{Zimmer}, S.},
    title = "{Measurement of Separate Cosmic-Ray Electron and Positron Spectra with the Fermi Large Area Telescope}",
  journal = {Physical Review Letters},
archivePrefix = "arXiv",
   eprint = {1109.0521},
 primaryClass = "astro-ph.HE",
 keywords = {Composition energy spectra and interactions, Dark matter, Neutrino muon pion and other elementary particles, cosmic rays, Cosmic rays},
     year = 2012,
    month = jan,
   volume = 108,
   number = 1,
      eid = {011103},
    pages = {011103},
      doi = {10.1103/PhysRevLett.108.011103},
   adsurl = {https://ui.adsabs.harvard.edu/abs/2012PhRvL.108a1103A},
  adsnote = {Provided by the SAO/NASA Astrophysics Data System}
}

@ARTICLE{Rojas-Bravo2016,
   author = {{Rojas-Bravo}, C. and {Araya}, M.},
    title = "{Search for gamma-ray emission from star-forming galaxies with Fermi LAT}",
  journal = {\mnras},
archivePrefix = "arXiv",
   eprint = {1608.04413},
 primaryClass = "astro-ph.HE",
 keywords = {cosmic rays, galaxies: starburst, gamma-rays: galaxies},
     year = 2016,
    month = nov,
   volume = 463,
    pages = {1068-1073},
      doi = {10.1093/mnras/stw2059},
   adsurl = {https://ui.adsabs.harvard.edu/abs/2016MNRAS.463.1068R},
  adsnote = {Provided by the SAO/NASA Astrophysics Data System}
}

@ARTICLE{Wiener2017,
   author = {{Wiener}, J. and {Pfrommer}, C. and {Oh}, S.~P.},
    title = "{Cosmic ray-driven galactic winds: streaming or diffusion?}",
  journal = {\mnras},
archivePrefix = "arXiv",
   eprint = {1608.02585},
 keywords = {galaxies: formation, galaxies: starburst, ISM: jets and outflows, hydrodynamics, cosmic rays, diffusion},
     year = 2017,
    month = may,
   volume = 467,
    pages = {906-921},
      doi = {10.1093/mnras/stx127},
   adsurl = {https://ui.adsabs.harvard.edu/abs/2017MNRAS.467..906W},
  adsnote = {Provided by the SAO/NASA Astrophysics Data System}
}

@ARTICLE{Vandevoort2019,
       author = {{van de Voort}, Freeke and {Springel}, Volker and {Mandelker}, Nir and
         {van den Bosch}, Frank C. and {Pakmor}, R{\"u}diger},
        title = "{Cosmological simulations of the circumgalactic medium with 1 kpc resolution: enhanced H I column densities}",
      journal = {\mnras},
     keywords = {hydrodynamics, methods: numerical, galaxies: evolution, galaxies: formation, galaxies: haloes, intergalactic medium, Astrophysics - Astrophysics of Galaxies},
         year = "2019",
        month = "Jan",
       volume = {482},
       number = {1},
        pages = {L85-L89},
          doi = {10.1093/mnrasl/sly190},
archivePrefix = {arXiv},
       eprint = {1808.04369},
 primaryClass = {astro-ph.GA},
       adsurl = {https://ui.adsabs.harvard.edu/abs/2019MNRAS.482L..85V},
      adsnote = {Provided by the SAO/NASA Astrophysics Data System}
}



@ARTICLE{Pakmor2011,
       author = {{Pakmor}, Ruediger and {Bauer}, Andreas and {Springel}, Volker},
        title = "{Magnetohydrodynamics on an unstructured moving grid}",
      journal = {\mnras},
     keywords = {MHD, turbulence, methods: numerical, stars: formation, Astrophysics - Instrumentation and Methods for Astrophysics},
         year = "2011",
        month = "Dec",
       volume = {418},
       number = {2},
        pages = {1392-1401},
          doi = {10.1111/j.1365-2966.2011.19591.x},
archivePrefix = {arXiv},
       eprint = {1108.1792},
 primaryClass = {astro-ph.IM},
       adsurl = {https://ui.adsabs.harvard.edu/abs/2011MNRAS.418.1392P},
      adsnote = {Provided by the SAO/NASA Astrophysics Data System}
}

@ARTICLE{Pakmor2013,
   author = {{Pakmor}, R. and {Springel}, V.},
    title = "{Simulations of magnetic fields in isolated disc galaxies}",
  journal = {\mnras},
archivePrefix = "arXiv",
   eprint = {1212.1452},
 primaryClass = "astro-ph.CO",
 keywords = {MHD, methods: numerical, galaxies: formation},
     year = 2013,
    month = jun,
   volume = 432,
    pages = {176-193},
      doi = {10.1093/mnras/stt428},
   adsurl = {https://ui.adsabs.harvard.edu/abs/2013MNRAS.432..176P},
  adsnote = {Provided by the SAO/NASA Astrophysics Data System}
}

@ARTICLE{Pakmor2016c,
   author = {{Pakmor}, R. and {Springel}, V. and {Bauer}, A. and {Mocz}, P. and 
	{Munoz}, D.~J. and {Ohlmann}, S.~T. and {Schaal}, K. and {Zhu}, C.
	},
    title = "{Improving the convergence properties of the moving-mesh code AREPO}",
  journal = {\mnras},
archivePrefix = "arXiv",
   eprint = {1503.00562},
 keywords = {hydrodynamics, methods: numerical, galaxy: formation},
     year = 2016,
    month = jan,
   volume = 455,
    pages = {1134-1143},
      doi = {10.1093/mnras/stv2380},
   adsurl = {https://ui.adsabs.harvard.edu/abs/2016MNRAS.455.1134P},
  adsnote = {Provided by the SAO/NASA Astrophysics Data System}
}

@ARTICLE{Pakmor2016b,
   author = {{Pakmor}, R. and {Pfrommer}, C. and {Simpson}, C.~M. and {Kannan}, R. and 
	{Springel}, V.},
    title = "{Semi-implicit anisotropic cosmic ray transport on an unstructured moving mesh}",
  journal = {\mnras},
archivePrefix = "arXiv",
   eprint = {1604.08587},
 keywords = {hydrodynamics, methods: numerical, Galaxy: formation},
     year = 2016,
    month = nov,
   volume = 462,
    pages = {2603-2616},
      doi = {10.1093/mnras/stw1761},
   adsurl = {https://ui.adsabs.harvard.edu/abs/2016MNRAS.462.2603P},
  adsnote = {Provided by the SAO/NASA Astrophysics Data System}
}

@ARTICLE{Pakmor2016,
   author = {{Pakmor}, R. and {Pfrommer}, C. and {Simpson}, C.~M. and {Springel}, V.
	},
    title = "{Galactic Winds Driven by Isotropic and Anisotropic Cosmic-Ray Diffusion in Disk Galaxies}",
  journal = {\apjl},
archivePrefix = "arXiv",
   eprint = {1605.00643},
 keywords = {cosmic rays, galaxies: evolution, galaxies: magnetic fields},
     year = 2016,
    month = jun,
   volume = 824,
      eid = {L30},
    pages = {L30},
      doi = {10.3847/2041-8205/824/2/L30},
   adsurl = {https://ui.adsabs.harvard.edu/abs/2016ApJ...824L..30P},
  adsnote = {Provided by the SAO/NASA Astrophysics Data System}
}

@ARTICLE{Pinzke2017,
   author = {{Pinzke}, A. and {Oh}, S.~P. and {Pfrommer}, C.},
    title = "{Turbulence and particle acceleration in giant radio haloes: the origin of seed electrons}",
  journal = {\mnras},
archivePrefix = "arXiv",
   eprint = {1611.07533},
 primaryClass = "astro-ph.HE",
 keywords = {acceleration of particles, radiation mechanisms: non-thermal, turbulence, cosmic rays, galaxies: clusters: general, gamma-rays: galaxies: clusters},
     year = 2017,
    month = mar,
   volume = 465,
    pages = {4800-4816},
      doi = {10.1093/mnras/stw3024},
   adsurl = {https://ui.adsabs.harvard.edu/abs/2017MNRAS.465.4800P},
  adsnote = {Provided by the SAO/NASA Astrophysics Data System}
}

@ARTICLE{Pinzke2013,
   author = {{Pinzke}, A. and {Oh}, S.~P. and {Pfrommer}, C.},
    title = "{Giant radio relics in galaxy clusters: reacceleration of fossil relativistic electrons?}",
  journal = {\mnras},
archivePrefix = "arXiv",
   eprint = {1301.5644},
 primaryClass = "astro-ph.CO",
 keywords = {elementary particles, magnetic fields, radiation mechanisms: non-thermal, cosmic rays, galaxies: clusters: general},
     year = 2013,
    month = oct,
   volume = 435,
    pages = {1061-1082},
      doi = {10.1093/mnras/stt1308},
   adsurl = {https://ui.adsabs.harvard.edu/abs/2013MNRAS.435.1061P},
  adsnote = {Provided by the SAO/NASA Astrophysics Data System}
}

@ARTICLE{Pinzke2010,
   author = {{Pinzke}, A. and {Pfrommer}, C.},
    title = "{Simulating the {$\gamma$}-ray emission from galaxy clusters: a universal cosmic ray spectrum and spatial distribution}",
  journal = {\mnras},
archivePrefix = "arXiv",
   eprint = {1001.5023},
 keywords = {elementary particles, magnetic fields, radiation mechanisms: non-thermal, cosmic rays, Galaxy: fundamental parameters, galaxies: clusters: general},
     year = 2010,
    month = dec,
   volume = 409,
    pages = {449-480},
      doi = {10.1111/j.1365-2966.2010.17328.x},
   adsurl = {https://ui.adsabs.harvard.edu/abs/2010MNRAS.409..449P},
  adsnote = {Provided by the SAO/NASA Astrophysics Data System}
}

@ARTICLE{Donnert2013,
   author = {{Donnert}, J. and {Dolag}, K. and {Brunetti}, G. and {Cassano}, R.
	},
    title = "{Rise and fall of radio haloes in simulated merging galaxy clusters}",
  journal = {\mnras},
archivePrefix = "arXiv",
   eprint = {1211.3337},
 keywords = {acceleration of particles, galaxies: clusters: general},
     year = 2013,
    month = mar,
   volume = 429,
    pages = {3564-3569},
      doi = {10.1093/mnras/sts628},
   adsurl = {https://ui.adsabs.harvard.edu/abs/2013MNRAS.429.3564D},
  adsnote = {Provided by the SAO/NASA Astrophysics Data System}
}

@ARTICLE{Vazza2012,
   author = {{Vazza}, F. and {Br{\"u}ggen}, M. and {Gheller}, C. and {Brunetti}, G.
	},
    title = "{Modelling injection and feedback of cosmic rays in grid-based cosmological simulations: effects on cluster outskirts}",
  journal = {\mnras},
archivePrefix = "arXiv",
   eprint = {1201.3362},
 keywords = {methods: numerical, galaxies: clusters: general, intergalactic medium, large-scale structure of Universe},
     year = 2012,
    month = apr,
   volume = 421,
    pages = {3375-3398},
      doi = {10.1111/j.1365-2966.2012.20562.x},
   adsurl = {https://ui.adsabs.harvard.edu/abs/2012MNRAS.421.3375V},
  adsnote = {Provided by the SAO/NASA Astrophysics Data System}
}

@ARTICLE{Pfrommer2006,
       author = {{Pfrommer}, C. and {Springel}, V. and {En{\ss}lin}, T.~A. and
         {Jubelgas}, M.},
        title = "{Detecting shock waves in cosmological smoothed particle hydrodynamics simulations}",
      journal = {\mnras},
     keywords = {shock waves-methods: numerical-cosmic rays-galaxies: clusters: general-intergalactic medium-large-scale structure of Universe, Astrophysics},
         year = "2006",
        month = "Mar",
       volume = {367},
       number = {1},
        pages = {113-131},
          doi = {10.1111/j.1365-2966.2005.09953.x},
archivePrefix = {arXiv},
       eprint = {astro-ph/0603483},
 primaryClass = {astro-ph},
       adsurl = {https://ui.adsabs.harvard.edu/abs/2006MNRAS.367..113P},
      adsnote = {Provided by the SAO/NASA Astrophysics Data System}
}

@ARTICLE{Pfrommer2008,
   author = {{Pfrommer}, C. and {En{\ss}lin}, T.~A. and {Springel}, V.},
    title = "{Simulating cosmic rays in clusters of galaxies - II. A unified scheme for radio haloes and relics with predictions of the {$\gamma$}-ray emission}",
  journal = {\mnras},
archivePrefix = "arXiv",
   eprint = {0707.1707},
 keywords = {magnetic fields , radiation mechanisms: non-thermal , cosmic rays , galaxies: clusters: general , large-scale structure of Universe},
     year = 2008,
    month = apr,
   volume = 385,
    pages = {1211-1241},
      doi = {10.1111/j.1365-2966.2008.12956.x},
   adsurl = {https://ui.adsabs.harvard.edu/abs/2008MNRAS.385.1211P},
  adsnote = {Provided by the SAO/NASA Astrophysics Data System}
}

@ARTICLE{Pfrommer2007,
   author = {{Pfrommer}, C. and {En{\ss}lin}, T.~A. and {Springel}, V. and 
	{Jubelgas}, M. and {Dolag}, K.},
    title = "{Simulating cosmic rays in clusters of galaxies - I. Effects on the Sunyaev-Zel'dovich effect and the X-ray emission}",
  journal = {\mnras},
   eprint = {astro-ph/0611037},
 keywords = {radiation mechanisms: general, cosmic rays, galaxies: cluster: general, cooling flows, large-scale structure of Universe, X-rays: galaxies: clusters},
     year = 2007,
    month = jun,
   volume = 378,
    pages = {385-408},
      doi = {10.1111/j.1365-2966.2007.11732.x},
   adsurl = {https://ui.adsabs.harvard.edu/abs/2007MNRAS.378..385P},
  adsnote = {Provided by the SAO/NASA Astrophysics Data System}
}

@ARTICLE{Miniati2001,
   author = {{Miniati}, F. and {Ryu}, D. and {Kang}, H. and {Jones}, T.~W.
	},
    title = "{Cosmic-Ray Protons Accelerated at Cosmological Shocks and Their Impact on Groups and Clusters of Galaxies}",
  journal = {\apj},
   eprint = {astro-ph/0105465},
 keywords = {Acceleration of Particles, Gamma Rays: Theory, Cosmology: Large-Scale Structure of Universe, Methods: Numerical, Shock Waves, X-Rays: Galaxies: Clusters},
     year = 2001,
    month = sep,
   volume = 559,
    pages = {59-69},
      doi = {10.1086/322375},
   adsurl = {https://ui.adsabs.harvard.edu/abs/2001ApJ...559...59M},
  adsnote = {Provided by the SAO/NASA Astrophysics Data System}
}

@ARTICLE{Miniati2001b,
   author = {{Miniati}, F. and {Jones}, T.~W. and {Kang}, H. and {Ryu}, D.
	},
    title = "{Cosmic-Ray Electrons in Groups and Clusters of Galaxies: Primary and Secondary Populations from a Numerical Cosmological Simulation}",
  journal = {\apj},
   eprint = {astro-ph/0108305},
 keywords = {Acceleration of Particles, ISM: Cosmic Rays, Galaxies: Clusters: General, Cosmology: Large-Scale Structure of Universe, Methods: Numerical, X-Rays: Galaxies: Clusters},
     year = 2001,
    month = nov,
   volume = 562,
    pages = {233-253},
      doi = {10.1086/323434},
   adsurl = {https://ui.adsabs.harvard.edu/abs/2001ApJ...562..233M},
  adsnote = {Provided by the SAO/NASA Astrophysics Data System}
}

@ARTICLE{Brunetti2014,
   author = {{Brunetti}, G. and {Jones}, T.~W.},
    title = "{Cosmic Rays in Galaxy Clusters and Their Nonthermal Emission}",
  journal = {International Journal of Modern Physics D},
archivePrefix = "arXiv",
   eprint = {1401.7519},
 keywords = {Galaxies clusters, general radiation mechanisms, nonthermal acceleration of particles, Elementary particle processes, Radiation mechanisms, polarization, Magnetohydrodynamics and plasmas, Galaxy clusters, Galaxy mergers collisions and tidal interactions, Intracluster matter, cooling flows},
     year = 2014,
    month = mar,
   volume = 23,
      eid = {1430007-98},
    pages = {1430007-98},
      doi = {10.1142/S0218271814300079},
   adsurl = {https://ui.adsabs.harvard.edu/abs/2014IJMPD..2330007B},
  adsnote = {Provided by the SAO/NASA Astrophysics Data System}
}

@ARTICLE{Jacob2017,
   author = {{Jacob}, S. and {Pfrommer}, C.},
    title = "{Cosmic ray heating in cool core clusters - I. Diversity of steady state solutions}",
  journal = {\mnras},
archivePrefix = "arXiv",
   eprint = {1609.06321},
 keywords = {cosmic rays, conduction, galaxies: active, galaxies: cooling flows, galaxies: clusters: general, radiation mechanisms: non-thermal},
     year = 2017,
    month = may,
   volume = 467,
    pages = {1449-1477},
      doi = {10.1093/mnras/stx131},
   adsurl = {https://ui.adsabs.harvard.edu/abs/2017MNRAS.467.1449J},
  adsnote = {Provided by the SAO/NASA Astrophysics Data System}
}

@ARTICLE{Jacob2017b,
   author = {{Jacob}, S. and {Pfrommer}, C.},
    title = "{Cosmic ray heating in cool core clusters - II. Self-regulation cycle and non-thermal emission}",
  journal = {\mnras},
archivePrefix = "arXiv",
   eprint = {1609.06322},
 keywords = {cosmic rays, conduction, galaxies: active, galaxies: cooling flows, galaxies: clusters: general, radiation mechanisms: non-thermal},
     year = 2017,
    month = may,
   volume = 467,
    pages = {1478-1495},
      doi = {10.1093/mnras/stx132},
   adsurl = {https://ui.adsabs.harvard.edu/abs/2017MNRAS.467.1478J},
  adsnote = {Provided by the SAO/NASA Astrophysics Data System}
}

@ARTICLE{Wiener2013,
   author = {{Wiener}, J. and {Oh}, S.~P. and {Guo}, F.},
    title = "{Cosmic ray streaming in clusters of galaxies}",
  journal = {\mnras},
archivePrefix = "arXiv",
   eprint = {1303.4746},
 primaryClass = "astro-ph.HE",
 keywords = {radiation mechanisms: non-thermal, turbulence, galaxies: clusters: general, radio continuum: general, X-rays: general},
     year = 2013,
    month = sep,
   volume = 434,
    pages = {2209-2228},
      doi = {10.1093/mnras/stt1163},
   adsurl = {https://ui.adsabs.harvard.edu/abs/2013MNRAS.434.2209W},
  adsnote = {Provided by the SAO/NASA Astrophysics Data System}
}

@ARTICLE{Pfrommer2013,
   author = {{Pfrommer}, C.},
    title = "{Toward a Comprehensive Model for Feedback by Active Galactic Nuclei: New Insights from M87 Observations by LOFAR, Fermi, and H.E.S.S.}",
  journal = {\apj},
archivePrefix = "arXiv",
   eprint = {1303.5443},
 keywords = {cosmic rays, galaxies: active, galaxies: clusters: intracluster medium, galaxies: individual: M87, gamma rays: galaxies: clusters, radio continuum: galaxies },
     year = 2013,
    month = dec,
   volume = 779,
      eid = {10},
    pages = {10},
      doi = {10.1088/0004-637X/779/1/10},
   adsurl = {https://ui.adsabs.harvard.edu/abs/2013ApJ...779...10P},
  adsnote = {Provided by the SAO/NASA Astrophysics Data System}
}

@ARTICLE{Ensslin2011,
   author = {{En{\ss}lin}, T. and {Pfrommer}, C. and {Miniati}, F. and {Subramanian}, K.
	},
    title = "{Cosmic ray transport in galaxy clusters: implications for radio halos, gamma-ray signatures, and cool core heating}",
  journal = {\aap},
archivePrefix = "arXiv",
   eprint = {1008.4717},
 keywords = {acceleration of particles, astroparticle physics, radio continuum: galaxies, magnetic fields, galaxies: clusters: intracluster medium, gamma rays: galaxies: clusters},
     year = 2011,
    month = mar,
   volume = 527,
      eid = {A99},
    pages = {A99},
      doi = {10.1051/0004-6361/201015652},
   adsurl = {https://ui.adsabs.harvard.edu/abs/2011Aadsnote = {Provided by the SAO/NASA Astrophysics Data System}
}

@ARTICLE{Guo2008,
   author = {{Guo}, F. and {Oh}, S.~P.},
    title = "{Feedback heating by cosmic rays in clusters of galaxies}",
  journal = {\mnras},
archivePrefix = "arXiv",
   eprint = {0706.1274},
 keywords = {instabilities , cosmic rays , galaxies: clusters: general , cooling flows , X-rays: galaxies: clusters},
     year = 2008,
    month = feb,
   volume = 384,
    pages = {251-266},
      doi = {10.1111/j.1365-2966.2007.12692.x},
   adsurl = {https://ui.adsabs.harvard.edu/abs/2008MNRAS.384..251G},
  adsnote = {Provided by the SAO/NASA Astrophysics Data System}
}

@ARTICLE{Loewenstein1991,
   author = {{Loewenstein}, M. and {Zweibel}, E.~G. and {Begelman}, M.~C.
	},
    title = "{Cosmic-ray heating of cooling flows - A critical analysis}",
  journal = {\apj},
 keywords = {Cooling Flows (Astrophysics), Cosmic Rays, Intergalactic Media, Magnetohydrodynamic Waves, Radiant Heating, Astronomical Models, Diffusion Coefficient, Temperature Distribution, Wave Propagation},
     year = 1991,
    month = aug,
   volume = 377,
    pages = {392-402},
      doi = {10.1086/170369},
   adsurl = {https://ui.adsabs.harvard.edu/abs/1991ApJ...377..392L},
  adsnote = {Provided by the SAO/NASA Astrophysics Data System}
}

@ARTICLE{Zhuravleva2014,
   author = {{Zhuravleva}, I. and {Churazov}, E. and {Schekochihin}, A.~A. and 
	{Allen}, S.~W. and {Ar{\'e}valo}, P. and {Fabian}, A.~C. and 
	{Forman}, W.~R. and {Sanders}, J.~S. and {Simionescu}, A. and 
	{Sunyaev}, R. and {Vikhlinin}, A. and {Werner}, N.},
    title = "{Turbulent heating in galaxy clusters brightest in X-rays}",
  journal = {\nat},
archivePrefix = "arXiv",
   eprint = {1410.6485},
 primaryClass = "astro-ph.HE",
     year = 2014,
    month = nov,
   volume = 515,
    pages = {85-87},
      doi = {10.1038/nature13830},
   adsurl = {https://ui.adsabs.harvard.edu/abs/2014Natur.515...85Z},
  adsnote = {Provided by the SAO/NASA Astrophysics Data System}
}

@ARTICLE{Peterson2006,
   author = {{Peterson}, J.~R. and {Fabian}, A.~C.},
    title = "{X-ray spectroscopy of cooling clusters}",
  journal = {\physrep},
   eprint = {astro-ph/0512549},
     year = 2006,
    month = apr,
   volume = 427,
    pages = {1-39},
      doi = {10.1016/j.physrep.2005.12.007},
   adsurl = {https://ui.adsabs.harvard.edu/abs/2006PhR...427....1P},
  adsnote = {Provided by the SAO/NASA Astrophysics Data System}
}

@ARTICLE{Ruszkowski2017b,
       author = {{Ruszkowski}, Mateusz and {Yang}, H. -Y. Karen and
         {Reynolds}, Christopher S.},
        title = "{Cosmic-Ray Feedback Heating of the Intracluster Medium}",
      journal = {\apj},
     keywords = {cosmic rays, galaxies: active, galaxies: clusters: intracluster medium, Astrophysics - High Energy Astrophysical Phenomena},
         year = "2017",
        month = "Jul",
       volume = {844},
       number = {1},
          eid = {13},
        pages = {13},
          doi = {10.3847/1538-4357/aa79f8},
archivePrefix = {arXiv},
       eprint = {1701.07441},
 primaryClass = {astro-ph.HE},
       adsurl = {https://ui.adsabs.harvard.edu/abs/2017ApJ...844...13R},
      adsnote = {Provided by the SAO/NASA Astrophysics Data System}
}

@ARTICLE{Ruszkowski2017,
   author = {{Ruszkowski}, M. and {Yang}, H.-Y.~K. and {Zweibel}, E.},
    title = "{Global Simulations of Galactic Winds Including Cosmic-ray Streaming}",
  journal = {\apj},
archivePrefix = "arXiv",
   eprint = {1602.04856},
 keywords = {cosmic rays, galaxies: evolution, cosmic rays, galaxies: star formation},
     year = 2017,
    month = jan,
   volume = 834,
      eid = {208},
    pages = {208},
      doi = {10.3847/1538-4357/834/2/208},
   adsurl = {https://ui.adsabs.harvard.edu/abs/2017ApJ...834..208R},
  adsnote = {Provided by the SAO/NASA Astrophysics Data System}
}

@ARTICLE{Pfrommer2017b,
   author = {{Pfrommer}, C. and {Pakmor}, R. and {Simpson}, C.~M. and {Springel}, V.
	},
    title = "{Simulating Gamma-Ray Emission in Star-forming Galaxies}",
  journal = {\apjl},
archivePrefix = "arXiv",
   eprint = {1709.05343},
 keywords = {cosmic rays, galaxies: formation, gamma rays: galaxies, magnetohydrodynamics: MHD, methods: numerical, radiation mechanisms: non-thermal},
     year = 2017,
    month = oct,
   volume = 847,
      eid = {L13},
    pages = {L13},
      doi = {10.3847/2041-8213/aa8bb1},
   adsurl = {https://ui.adsabs.harvard.edu/abs/2017ApJ...847L..13P},
  adsnote = {Provided by the SAO/NASA Astrophysics Data System}
}

@ARTICLE{Pfrommer2017,
   author = {{Pfrommer}, C. and {Pakmor}, R. and {Schaal}, K. and {Simpson}, C.~M. and 
	{Springel}, V.},
    title = "{Simulating cosmic ray physics on a moving mesh}",
  journal = {\mnras},
archivePrefix = "arXiv",
   eprint = {1604.07399},
 keywords = {MHD, shock waves, cosmic rays, methods: numerical, galaxies: formation, large-scale structure of Universe},
     year = 2017,
    month = mar,
   volume = 465,
    pages = {4500-4529},
      doi = {10.1093/mnras/stw2941},
   adsurl = {https://ui.adsabs.harvard.edu/abs/2017MNRAS.465.4500P},
  adsnote = {Provided by the SAO/NASA Astrophysics Data System}
}

@ARTICLE{Booth2013,
   author = {{Booth}, C.~M. and {Agertz}, O. and {Kravtsov}, A.~V. and {Gnedin}, N.~Y.
	},
    title = "{Simulations of Disk Galaxies with Cosmic Ray Driven Galactic Winds}",
  journal = {\apjl},
archivePrefix = "arXiv",
   eprint = {1308.4974},
 keywords = {cosmic rays, galaxies: formation, methods: numerical },
     year = 2013,
    month = nov,
   volume = 777,
      eid = {L16},
    pages = {L16},
      doi = {10.1088/2041-8205/777/1/L16},
   adsurl = {https://ui.adsabs.harvard.edu/abs/2013ApJ...777L..16B},
  adsnote = {Provided by the SAO/NASA Astrophysics Data System}
}

@ARTICLE{Salem2014,
   author = {{Salem}, M. and {Bryan}, G.~L.},
    title = "{Cosmic ray driven outflows in global galaxy disc models}",
  journal = {\mnras},
archivePrefix = "arXiv",
   eprint = {1307.6215},
 keywords = {methods: numerical, cosmic rays, galaxies: formation},
     year = 2014,
    month = feb,
   volume = 437,
    pages = {3312-3330},
      doi = {10.1093/mnras/stt2121},
   adsurl = {https://ui.adsabs.harvard.edu/abs/2014MNRAS.437.3312S},
  adsnote = {Provided by the SAO/NASA Astrophysics Data System}
}

@ARTICLE{Uhlig2012,
   author = {{Uhlig}, M. and {Pfrommer}, C. and {Sharma}, M. and {Nath}, B.~B. and 
	{En{\ss}lin}, T.~A. and {Springel}, V.},
    title = "{Galactic winds driven by cosmic ray streaming}",
  journal = {\mnras},
archivePrefix = "arXiv",
   eprint = {1203.1038},
 keywords = {cosmic rays, galaxies: dwarf, galaxies: evolution, galaxies: formation, intergalactic medium, galaxies: starburst},
     year = 2012,
    month = jul,
   volume = 423,
    pages = {2374-2396},
      doi = {10.1111/j.1365-2966.2012.21045.x},
   adsurl = {https://ui.adsabs.harvard.edu/abs/2012MNRAS.423.2374U},
  adsnote = {Provided by the SAO/NASA Astrophysics Data System}
}

@ARTICLE{Ensslin2007,
   author = {{En{\ss}lin}, T.~A. and {Pfrommer}, C. and {Springel}, V. and 
	{Jubelgas}, M.},
    title = "{Cosmic ray physics in calculations of cosmological structure formation}",
  journal = {\aap},
   eprint = {astro-ph/0603484},
 keywords = {galaxies: intergalactic medium, galaxies:, clusters: general, acceleration of particles, radiation mechanisms: non-thermal, methods: numerical, cosmic rays},
     year = 2007,
    month = oct,
   volume = 473,
    pages = {41-57},
      doi = {10.1051/0004-6361:20065294},
   adsurl = {https://ui.adsabs.harvard.edu/abs/2007Aadsnote = {Provided by the SAO/NASA Astrophysics Data System}
}

@ARTICLE{Girichidis2016,
   author = {{Girichidis}, P. and {Naab}, T. and {Walch}, S. and {Hanasz}, M. and 
	{Mac Low}, M.-M. and {Ostriker}, J.~P. and {Gatto}, A. and {Peters}, T. and 
	{W{\"u}nsch}, R. and {Glover}, S.~C.~O. and {Klessen}, R.~S. and 
	{Clark}, P.~C. and {Baczynski}, C.},
    title = "{Launching Cosmic-Ray-driven Outflows from the Magnetized Interstellar Medium}",
  journal = {\apjl},
archivePrefix = "arXiv",
   eprint = {1509.07247},
 keywords = {cosmic rays, diffusion, ISM: jets and outflows, ISM: structure, magnetohydrodynamics: MHD},
     year = 2016,
    month = jan,
   volume = 816,
      eid = {L19},
    pages = {L19},
      doi = {10.3847/2041-8205/816/2/L19},
   adsurl = {https://ui.adsabs.harvard.edu/abs/2016ApJ...816L..19G},
  adsnote = {Provided by the SAO/NASA Astrophysics Data System}
}

@ARTICLE{Dorfi2012,
   author = {{Dorfi}, E.~A. and {Breitschwerdt}, D.},
    title = "{Time-dependent galactic winds. I. Structure and evolution of galactic outflows accompanied by cosmic ray acceleration}",
  journal = {\aap},
archivePrefix = "arXiv",
   eprint = {1304.1311},
 primaryClass = "astro-ph.HE",
 keywords = {Galaxy: evolution, ISM: jets and outflows, galaxies: starburst, ISM: supernova remnants, cosmic rays},
     year = 2012,
    month = apr,
   volume = 540,
      eid = {A77},
    pages = {A77},
      doi = {10.1051/0004-6361/201118082},
   adsurl = {https://ui.adsabs.harvard.edu/abs/2012Aadsnote = {Provided by the SAO/NASA Astrophysics Data System}
}

@ARTICLE{Samui2010,
   author = {{Samui}, S. and {Subramanian}, K. and {Srianand}, R.},
    title = "{Cosmic ray driven outflows from high-redshift galaxies}",
  journal = {\mnras},
archivePrefix = "arXiv",
   eprint = {0909.3854},
 keywords = {stars: winds, outflows, galaxies: high-redshift, galaxies: ISM, galaxies: starburst},
     year = 2010,
    month = mar,
   volume = 402,
    pages = {2778-2791},
      doi = {10.1111/j.1365-2966.2009.16099.x},
   adsurl = {https://ui.adsabs.harvard.edu/abs/2010MNRAS.402.2778S},
  adsnote = {Provided by the SAO/NASA Astrophysics Data System}
}

@ARTICLE{Everett2008,
   author = {{Everett}, J.~E. and {Zweibel}, E.~G. and {Benjamin}, R.~A. and 
	{McCammon}, D. and {Rocks}, L. and {Gallagher}, III, J.~S.},
    title = "{The Milky Way's Kiloparsec-Scale Wind: A Hybrid Cosmic-Ray and Thermally Driven Outflow}",
  journal = {\apj},
archivePrefix = "arXiv",
   eprint = {0710.3712},
 keywords = {cosmic rays, Galaxy: evolution, ISM: jets and outflows, ISM: magnetic fields, X-rays: diffuse background},
     year = 2008,
    month = feb,
   volume = 674,
    pages = {258-270},
      doi = {10.1086/524766},
   adsurl = {https://ui.adsabs.harvard.edu/abs/2008ApJ...674..258E},
  adsnote = {Provided by the SAO/NASA Astrophysics Data System}
}

@ARTICLE{Socrates2008,
   author = {{Socrates}, A. and {Davis}, S.~W. and {Ramirez-Ruiz}, E.},
    title = "{The Eddington Limit in Cosmic Rays: An Explanation for the Observed Faintness of Starbursting Galaxies}",
  journal = {\apj},
   eprint = {astro-ph/0609796},
 keywords = {galaxies: formation, galaxies: fundamental parameters, galaxies: starburst},
     year = 2008,
    month = nov,
   volume = 687,
    pages = {202-215},
      doi = {10.1086/590046},
   adsurl = {https://ui.adsabs.harvard.edu/abs/2008ApJ...687..202S},
  adsnote = {Provided by the SAO/NASA Astrophysics Data System}
}

@ARTICLE{Ptuskin1997,
   author = {{Ptuskin}, V.~S. and {Voelk}, H.~J. and {Zirakashvili}, V.~N. and 
	{Breitschwerdt}, D.},
    title = "{Transport of relativistic nucleons in a galactic wind driven by cosmic rays.}",
  journal = {\aap},
 keywords = {COSMIC RAYS, GALAXY: HALO, GALAXIES: HALOS, TURBULENCE, DIFFUSION},
     year = 1997,
    month = may,
   volume = 321,
    pages = {434-443},
   adsurl = {https://ui.adsabs.harvard.edu/abs/1997Aadsnote = {Provided by the SAO/NASA Astrophysics Data System}
}

@ARTICLE{Zirakashvili1996,
   author = {{Zirakashvili}, V.~N. and {Breitschwerdt}, D. and {Ptuskin}, V.~S. and 
	{Voelk}, H.~J.},
    title = "{Magnetohydrodynamic wind driven by cosmic rays in a rotating galaxy.}",
  journal = {\aap},
 keywords = {GALAXIES: HALO, GALAXY: HALO, COSMIC RAYS, GALAXIES: MAGNETIC FIELD, MHD},
     year = 1996,
    month = jul,
   volume = 311,
    pages = {113-126},
   adsurl = {https://ui.adsabs.harvard.edu/abs/1996Aadsnote = {Provided by the SAO/NASA Astrophysics Data System}
}


@ARTICLE{Breitschwerdt1991,
   author = {{Breitschwerdt}, D. and {McKenzie}, J.~F. and {Voelk}, H.~J.
	},
    title = "{Galactic winds. I - Cosmic ray and wave-driven winds from the Galaxy}",
  journal = {\aap},
 keywords = {Cosmic Rays, Halos, Intergalactic Media, Magnetohydrodynamic Waves, Mass Flow, Milky Way Galaxy, Convective Flow, High Temperature Gases, Magnetic Fields},
     year = 1991,
    month = may,
   volume = 245,
    pages = {79-98},
   adsurl = {https://ui.adsabs.harvard.edu/abs/1991Aadsnote = {Provided by the SAO/NASA Astrophysics Data System}
}

@ARTICLE{Breitschwerdt2002,
   author = {{Breitschwerdt}, D. and {Dogiel}, V.~A. and {V{\"o}lk}, H.~J.
	},
    title = "{The gradient of diffuse gamma -ray emission in the Galaxy}",
  journal = {\aap},
   eprint = {astro-ph/0201345},
 keywords = {COSMIC RAYS, MHD, GAMMA RAYS: OBSERVATIONS, ISM: SUPERNOVA REMNANTS \%},
     year = 2002,
    month = apr,
   volume = 385,
    pages = {216-238},
      doi = {10.1051/0004-6361:20020152},
   adsurl = {https://ui.adsabs.harvard.edu/abs/2002Aadsnote = {Provided by the SAO/NASA Astrophysics Data System}
}

@ARTICLE{Kaviraj2017,
   author = {{Kaviraj}, S. and {Laigle}, C. and {Kimm}, T. and {Devriendt}, J.~E.~G. and 
	{Dubois}, Y. and {Pichon}, C. and {Slyz}, A. and {Chisari}, E. and 
	{Peirani}, S.},
    title = "{The Horizon-AGN simulation: evolution of galaxy properties over cosmic time}",
  journal = {\mnras},
archivePrefix = "arXiv",
   eprint = {1605.09379},
 keywords = {methods: numerical, galaxies: evolution, galaxies: formation, galaxies: high-redshift, cosmology: theory, large-scale structure of Universe},
     year = 2017,
    month = jun,
   volume = 467,
    pages = {4739-4752},
      doi = {10.1093/mnras/stx126},
   adsurl = {https://ui.adsabs.harvard.edu/abs/2017MNRAS.467.4739K},
  adsnote = {Provided by the SAO/NASA Astrophysics Data System}
}

@ARTICLE{Dubois2016,
   author = {{Dubois}, Y. and {Peirani}, S. and {Pichon}, C. and {Devriendt}, J. and 
	{Gavazzi}, R. and {Welker}, C. and {Volonteri}, M.},
    title = "{The HORIZON-AGN simulation: morphological diversity of galaxies promoted by AGN feedback}",
  journal = {\mnras},
archivePrefix = "arXiv",
   eprint = {1606.03086},
 keywords = {methods: numerical, galaxies: active, galaxies: evolution, galaxies: formation, galaxies: kinematics and dynamics},
     year = 2016,
    month = dec,
   volume = 463,
    pages = {3948-3964},
      doi = {10.1093/mnras/stw2265},
   adsurl = {https://ui.adsabs.harvard.edu/abs/2016MNRAS.463.3948D},
  adsnote = {Provided by the SAO/NASA Astrophysics Data System}
}

@ARTICLE{Ipavich1975,
   author = {{Ipavich}, F.~M.},
    title = "{Galactic winds driven by cosmic rays}",
  journal = {\apj},
 keywords = {Astronomical Models, Cosmic Rays, Galactic Radiation, Magnetohydrodynamic Waves, Plasma-Particle Interactions, Stellar Winds, Cosmic Plasma, Hydrodynamic Equations, Mass Flow, Mass Transfer, Solar Wind, X Ray Sources},
     year = 1975,
    month = feb,
   volume = 196,
    pages = {107-120},
      doi = {10.1086/153397},
   adsurl = {https://ui.adsabs.harvard.edu/abs/1975ApJ...196..107I},
  adsnote = {Provided by the SAO/NASA Astrophysics Data System}
}

@ARTICLE{Boulares1990,
   author = {{Boulares}, A. and {Cox}, D.~P.},
    title = "{Galactic hydrostatic equilibrium with magnetic tension and cosmic-ray diffusion}",
  journal = {\apj},
 keywords = {Cosmic Rays, Interstellar Gas, Interstellar Magnetic Fields, Milky Way Galaxy, Particle Acceleration, Solar Neighborhood, Gaseous Diffusion, Gravitational Fields, Hydrostatics, Mass Distribution, Pressure Distribution},
     year = 1990,
    month = dec,
   volume = 365,
    pages = {544-558},
      doi = {10.1086/169509},
   adsurl = {https://ui.adsabs.harvard.edu/abs/1990ApJ...365..544B},
  adsnote = {Provided by the SAO/NASA Astrophysics Data System}
}

@ARTICLE{Sijacki2008,
   author = {{Sijacki}, D. and {Pfrommer}, C. and {Springel}, V. and {En{\ss}lin}, T.~A.
	},
    title = "{Simulations of cosmic-ray feedback by active galactic nuclei in galaxy clusters}",
  journal = {\mnras},
archivePrefix = "arXiv",
   eprint = {0801.3285},
 keywords = {black hole physics , methods: numerical , cosmic rays , galaxies: clusters: general , cosmology: theory},
     year = 2008,
    month = jul,
   volume = 387,
    pages = {1403-1415},
      doi = {10.1111/j.1365-2966.2008.13310.x},
   adsurl = {https://ui.adsabs.harvard.edu/abs/2008MNRAS.387.1403S},
  adsnote = {Provided by the SAO/NASA Astrophysics Data System}
}

@ARTICLE{Jubelgas2008,
   author = {{Jubelgas}, M. and {Springel}, V. and {En{\ss}lin}, T. and {Pfrommer}, C.
	},
    title = "{Cosmic ray feedback in hydrodynamical simulations of galaxy formation}",
  journal = {\aap},
   eprint = {astro-ph/0603485},
 keywords = {methods: numerical, acceleration of particles, ISM: general, galaxies: structure, galaxies: clusters: general, intergalactic medium},
     year = 2008,
    month = apr,
   volume = 481,
    pages = {33-63},
      doi = {10.1051/0004-6361:20065295},
   adsurl = {https://ui.adsabs.harvard.edu/abs/2008Aadsnote = {Provided by the SAO/NASA Astrophysics Data System}
}

@ARTICLE{Krumholz2012,
   author = {{Krumholz}, M.~R. and {Thompson}, T.~A.},
    title = "{Direct Numerical Simulation of Radiation Pressure-driven Turbulence and Winds in Star Clusters and Galactic Disks}",
  journal = {\apj},
archivePrefix = "arXiv",
   eprint = {1203.2926},
 keywords = {galaxies: ISM, galaxies: star clusters: general, hydrodynamics, ISM: jets and outflows, radiative transfer },
     year = 2012,
    month = dec,
   volume = 760,
      eid = {155},
    pages = {155},
      doi = {10.1088/0004-637X/760/2/155},
   adsurl = {https://ui.adsabs.harvard.edu/abs/2012ApJ...760..155K},
  adsnote = {Provided by the SAO/NASA Astrophysics Data System}
}

@ARTICLE{Thompson2005,
   author = {{Thompson}, T.~A. and {Quataert}, E. and {Murray}, N.},
    title = "{Radiation Pressure-supported Starburst Disks and Active Galactic Nucleus Fueling}",
  journal = {\apj},
   eprint = {astro-ph/0503027},
 keywords = {Accretion, Accretion Disks, Galaxies: Formation, Galaxies: General, Galaxies: Starburst, Galaxy: Center, Galaxies: Quasars: General},
     year = 2005,
    month = sep,
   volume = 630,
    pages = {167-185},
      doi = {10.1086/431923},
   adsurl = {https://ui.adsabs.harvard.edu/abs/2005ApJ...630..167T},
  adsnote = {Provided by the SAO/NASA Astrophysics Data System}
}

@ARTICLE{Murray2005,
   author = {{Murray}, N. and {Quataert}, E. and {Thompson}, T.~A.},
    title = "{On the Maximum Luminosity of Galaxies and Their Central Black Holes: Feedback from Momentum-driven Winds}",
  journal = {\apj},
   eprint = {astro-ph/0406070},
 keywords = {Galaxies: Formation, Galaxies: Fundamental Parameters, Galaxies: General, Galaxies: Starburst, Galaxies: Intergalactic Medium},
     year = 2005,
    month = jan,
   volume = 618,
    pages = {569-585},
      doi = {10.1086/426067},
   adsurl = {https://ui.adsabs.harvard.edu/abs/2005ApJ...618..569M},
  adsnote = {Provided by the SAO/NASA Astrophysics Data System}
}

@ARTICLE{DiMatteo2005,
   author = {{Di Matteo}, T. and {Springel}, V. and {Hernquist}, L.},
    title = "{Energy input from quasars regulates the growth and activity of black holes and their host galaxies}",
  journal = {\nat},
   eprint = {astro-ph/0502199},
     year = 2005,
    month = feb,
   volume = 433,
    pages = {604-607},
      doi = {10.1038/nature03335},
   adsurl = {https://ui.adsabs.harvard.edu/abs/2005Natur.433..604D},
  adsnote = {Provided by the SAO/NASA Astrophysics Data System}
}

@ARTICLE{Rosdahl2015,
   author = {{Rosdahl}, J. and {Schaye}, J. and {Teyssier}, R. and {Agertz}, O.
	},
    title = "{Galaxies that shine: radiation-hydrodynamical simulations of disc galaxies}",
  journal = {\mnras},
archivePrefix = "arXiv",
   eprint = {1501.04632},
 keywords = {radiative transfer, methods: numerical, galaxies: evolution},
     year = 2015,
    month = jul,
   volume = 451,
    pages = {34-58},
      doi = {10.1093/mnras/stv937},
   adsurl = {https://ui.adsabs.harvard.edu/abs/2015MNRAS.451...34R},
  adsnote = {Provided by the SAO/NASA Astrophysics Data System}
}

@ARTICLE{Oppenheimer2006,
   author = {{Oppenheimer}, B.~D. and {Dav{\'e}}, R.},
    title = "{Cosmological simulations of intergalactic medium enrichment from galactic outflows}",
  journal = {\mnras},
   eprint = {astro-ph/0605651},
 keywords = {methods: numerical, galaxies: formation, galaxies: high-redshift, intergalactic medium, cosmology: theory},
     year = 2006,
    month = dec,
   volume = 373,
    pages = {1265-1292},
      doi = {10.1111/j.1365-2966.2006.10989.x},
   adsurl = {https://ui.adsabs.harvard.edu/abs/2006MNRAS.373.1265O},
  adsnote = {Provided by the SAO/NASA Astrophysics Data System}
}

@ARTICLE{Schaye2008,
   author = {{Schaye}, J. and {Dalla Vecchia}, C.},
    title = "{On the relation between the Schmidt and Kennicutt-Schmidt star formation laws and its implications for numerical simulations}",
  journal = {\mnras},
archivePrefix = "arXiv",
   eprint = {0709.0292},
 keywords = {stars: formation , galaxies: evolution , galaxies: formation , galaxies: ISM},
     year = 2008,
    month = jan,
   volume = 383,
    pages = {1210-1222},
      doi = {10.1111/j.1365-2966.2007.12639.x},
   adsurl = {https://ui.adsabs.harvard.edu/abs/2008MNRAS.383.1210S},
  adsnote = {Provided by the SAO/NASA Astrophysics Data System}
}

@ARTICLE{Springel2003,
   author = {{Springel}, V. and {Hernquist}, L.},
    title = "{Cosmological smoothed particle hydrodynamics simulations: a hybrid multiphase model for star formation}",
  journal = {\mnras},
   eprint = {astro-ph/0206393},
 keywords = {methods: numerical, galaxies: evolution, galaxies: formation},
     year = 2003,
    month = feb,
   volume = 339,
    pages = {289-311},
      doi = {10.1046/j.1365-8711.2003.06206.x},
   adsurl = {https://ui.adsabs.harvard.edu/abs/2003MNRAS.339..289S},
  adsnote = {Provided by the SAO/NASA Astrophysics Data System}
}

@ARTICLE{Dolag2016,
   author = {{Dolag}, K. and {Komatsu}, E. and {Sunyaev}, R.},
    title = "{SZ effects in the Magneticum Pathfinder simulation: comparison with the Planck, SPT, and ACT results}",
  journal = {\mnras},
archivePrefix = "arXiv",
   eprint = {1509.05134},
 keywords = {hydrodynamics, methods: numerical, galaxies: clusters: general, cosmic background radiation, cosmology: theory},
     year = 2016,
    month = dec,
   volume = 463,
    pages = {1797-1811},
      doi = {10.1093/mnras/stw2035},
   adsurl = {https://ui.adsabs.harvard.edu/abs/2016MNRAS.463.1797D},
  adsnote = {Provided by the SAO/NASA Astrophysics Data System}
}

@ARTICLE{McCarthy2017,
   author = {{McCarthy}, I.~G. and {Schaye}, J. and {Bird}, S. and {Le Brun}, A.~M.~C.
	},
    title = "{The BAHAMAS project: calibrated hydrodynamical simulations for large-scale structure cosmology}",
  journal = {\mnras},
archivePrefix = "arXiv",
   eprint = {1603.02702},
 keywords = {galaxies: clusters: general, galaxies: haloes, cosmology: theory, large-scale structure of Universe},
     year = 2017,
    month = mar,
   volume = 465,
    pages = {2936-2965},
      doi = {10.1093/mnras/stw2792},
   adsurl = {https://ui.adsabs.harvard.edu/abs/2017MNRAS.465.2936M},
  adsnote = {Provided by the SAO/NASA Astrophysics Data System}
}

@ARTICLE{McCarthy2014,
   author = {{McCarthy}, I.~G. and {Le Brun}, A.~M.~C. and {Schaye}, J. and 
	{Holder}, G.~P.},
    title = "{The thermal Sunyaev-Zel'dovich effect power spectrum in light of Planck}",
  journal = {\mnras},
archivePrefix = "arXiv",
   eprint = {1312.5341},
 keywords = {galaxies: clusters: general, galaxies: clusters: intracluster medium, cosmic background radiation, cosmological parameters, cosmology: theory},
     year = 2014,
    month = jun,
   volume = 440,
    pages = {3645-3657},
      doi = {10.1093/mnras/stu543},
   adsurl = {https://ui.adsabs.harvard.edu/abs/2014MNRAS.440.3645M},
  adsnote = {Provided by the SAO/NASA Astrophysics Data System}
}

@ARTICLE{Battaglia2013,
   author = {{Battaglia}, N. and {Bond}, J.~R. and {Pfrommer}, C. and {Sievers}, J.~L.
	},
    title = "{On the Cluster Physics of Sunyaev-Zel'dovich and X-Ray Surveys. III. Measurement Biases and Cosmological Evolution of Gas and Stellar Mass Fractions}",
  journal = {\apj},
archivePrefix = "arXiv",
   eprint = {1209.4082},
 keywords = {cosmology: theory, galaxies: clusters: general, large-scale structure of universe, methods: numerical},
     year = 2013,
    month = nov,
   volume = 777,
      eid = {123},
    pages = {123},
      doi = {10.1088/0004-637X/777/2/123},
   adsurl = {https://ui.adsabs.harvard.edu/abs/2013ApJ...777..123B},
  adsnote = {Provided by the SAO/NASA Astrophysics Data System}
}

@ARTICLE{Battaglia2012b,
   author = {{Battaglia}, N. and {Bond}, J.~R. and {Pfrommer}, C. and {Sievers}, J.~L.
	},
    title = "{On the Cluster Physics of Sunyaev-Zel'dovich and X-Ray Surveys. II. Deconstructing the Thermal SZ Power Spectrum}",
  journal = {\apj},
archivePrefix = "arXiv",
   eprint = {1109.3711},
 keywords = {cosmic background radiation, cosmology: theory, galaxies: clusters: general, large-scale structure of universe, methods: numerical},
     year = 2012,
    month = oct,
   volume = 758,
      eid = {75},
    pages = {75},
      doi = {10.1088/0004-637X/758/2/75},
   adsurl = {https://ui.adsabs.harvard.edu/abs/2012ApJ...758...75B},
  adsnote = {Provided by the SAO/NASA Astrophysics Data System}
}

@ARTICLE{Battaglia2012,
   author = {{Battaglia}, N. and {Bond}, J.~R. and {Pfrommer}, C. and {Sievers}, J.~L.
	},
    title = "{On the Cluster Physics of Sunyaev-Zel'dovich and X-Ray Surveys. I. The Influence of Feedback, Non-thermal Pressure, and Cluster Shapes on Y-M Scaling Relations}",
  journal = {\apj},
archivePrefix = "arXiv",
   eprint = {1109.3709},
 keywords = {cosmic background radiation, cosmology: theory, galaxies: clusters: general, large-scale structure of universe, methods: numerical},
     year = 2012,
    month = oct,
   volume = 758,
      eid = {74},
    pages = {74},
      doi = {10.1088/0004-637X/758/2/74},
   adsurl = {https://ui.adsabs.harvard.edu/abs/2012ApJ...758...74B},
  adsnote = {Provided by the SAO/NASA Astrophysics Data System}
}

@ARTICLE{Kravtsov2012,
   author = {{Kravtsov}, A.~V. and {Borgani}, S.},
    title = "{Formation of Galaxy Clusters}",
  journal = {\araa},
archivePrefix = "arXiv",
   eprint = {1205.5556},
     year = 2012,
    month = sep,
   volume = 50,
    pages = {353-409},
      doi = {10.1146/annurev-astro-081811-125502},
   adsurl = {https://ui.adsabs.harvard.edu/abs/2012ARAadsnote = {Provided by the SAO/NASA Astrophysics Data System}
}

@ARTICLE{Pillepich2018,
   author = {{Pillepich}, A. and {Springel}, V. and {Nelson}, D. and {Genel}, S. and 
	{Naiman}, J. and {Pakmor}, R. and {Hernquist}, L. and {Torrey}, P. and 
	{Vogelsberger}, M. and {Weinberger}, R. and {Marinacci}, F.},
    title = "{Simulating galaxy formation with the IllustrisTNG model}",
  journal = {\mnras},
archivePrefix = "arXiv",
   eprint = {1703.02970},
 keywords = {methods: numerical, galaxies: evolution, galaxies: formation},
     year = 2018,
    month = jan,
   volume = 473,
    pages = {4077-4106},
      doi = {10.1093/mnras/stx2656},
   adsurl = {http://adsabs.harvard.edu/abs/2018MNRAS.473.4077P},
  adsnote = {Provided by the SAO/NASA Astrophysics Data System}
}

@ARTICLE{Henriques2015,
   author = {{Henriques}, B.~M.~B. and {White}, S.~D.~M. and {Thomas}, P.~A. and 
	{Angulo}, R. and {Guo}, Q. and {Lemson}, G. and {Springel}, V. and 
	{Overzier}, R.},
    title = "{Galaxy formation in the Planck cosmology - I. Matching the observed evolution of star formation rates, colours and stellar masses}",
  journal = {\mnras},
archivePrefix = "arXiv",
   eprint = {1410.0365},
 keywords = {methods: analytical, methods: statistical, galaxies: evolution, galaxies: formation, galaxies: high-redshift},
     year = 2015,
    month = aug,
   volume = 451,
    pages = {2663-2680},
      doi = {10.1093/mnras/stv705},
   adsurl = {https://ui.adsabs.harvard.edu/abs/2015MNRAS.451.2663H},
  adsnote = {Provided by the SAO/NASA Astrophysics Data System}
}

@article{Marinacci2014,
	Adsnote = {Provided by the SAO/NASA Astrophysics Data System},
	Adsurl = {http://adsabs.harvard.edu/abs/2014MNRAS.437.1750M},
	Archiveprefix = {arXiv},
	Author = {{Marinacci}, F. and {Pakmor}, R. and {Springel}, V.},
	Date-Added = {2018-07-01 15:59:18 +0000},
	Date-Modified = {2018-07-01 15:59:18 +0000},
	Doi = {10.1093/mnras/stt2003},
	Eprint = {1305.5360},
	Journal = {\mnras},
	Keywords = {methods: numerical, galaxies: evolution, galaxies: formation, galaxies: spiral},
	Month = jan,
	Pages = {1750-1775},
	Title = {{The formation of disc galaxies in high-resolution moving-mesh cosmological simulations}},
	Volume = 437,
	Year = 2014,
	Bdsk-Url-1 = {http://dx.doi.org/10.1093/mnras/stt2003}
}
	
@ARTICLE{Puchwein2013,
   author = {{Puchwein}, E. and {Springel}, V.},
    title = "{Shaping the galaxy stellar mass function with supernova- and AGN-driven winds}",
  journal = {\mnras},
archivePrefix = "arXiv",
   eprint = {1205.2694},
 keywords = {methods: numerical, galaxies: formation, cosmology: theory},
     year = 2013,
    month = feb,
   volume = 428,
    pages = {2966-2979},
      doi = {10.1093/mnras/sts243},
   adsurl = {https://ui.adsabs.harvard.edu/abs/2013MNRAS.428.2966P},
  adsnote = {Provided by the SAO/NASA Astrophysics Data System}
}

@ARTICLE{Guedes2011,
   author = {{Guedes}, J. and {Callegari}, S. and {Madau}, P. and {Mayer}, L.
	},
    title = "{Forming Realistic Late-type Spirals in a {$\Lambda$}CDM Universe: The Eris Simulation}",
  journal = {\apj},
archivePrefix = "arXiv",
   eprint = {1103.6030},
 keywords = {galaxies: formation},
     year = 2011,
    month = dec,
   volume = 742,
      eid = {76},
    pages = {76},
      doi = {10.1088/0004-637X/742/2/76},
   adsurl = {http://adsabs.harvard.edu/abs/2011ApJ...742...76G},
  adsnote = {Provided by the SAO/NASA Astrophysics Data System}
}

@ARTICLE{Sersic1963,
   author = {{S{\'e}rsic}, J.~L.},
    title = "{Influence of the atmospheric and instrumental dispersion on the brightness distribution in a galaxy}",
  journal = {Boletin de la Asociacion Argentina de Astronomia La Plata Argentina},
     year = 1963,
   volume = 6,
    pages = {41},
   adsurl = {https://ui.adsabs.harvard.edu/abs/1963BAAA....6...41S},
  adsnote = {Provided by the SAO/NASA Astrophysics Data System}
}

@ARTICLE{Dutton2018b,
   author = {{Dutton}, A.~A. and {Obreja}, A. and {Macci{\`o}}, A.~V.},
    title = "{NIHAO XVII: The diversity of dwarf galaxy kinematics and implications for the HI velocity function}",
  journal = {ArXiv e-prints},
archivePrefix = "arXiv",
   eprint = {1807.10518},
 keywords = {Astrophysics - Astrophysics of Galaxies, Astrophysics - Cosmology and Nongalactic Astrophysics},
     year = 2018,
    month = jul,
   adsurl = {http://adsabs.harvard.edu/abs/2018arXiv180710518D},
  adsnote = {Provided by the SAO/NASA Astrophysics Data System}
}

@Article{matplotlib,
  Author    = {Hunter, J. D.},
  Title     = {Matplotlib: A 2D graphics environment},
  Journal   = {Computing In Science \& Engineering},
  Volume    = {9},
  Number    = {3},
  Pages     = {90--95},
  abstract  = {Matplotlib is a 2D graphics package used for Python
  for application development, interactive scripting, and
  publication-quality image generation across user
  interfaces and operating systems.},
  publisher = {IEEE COMPUTER SOC},
  doi = {10.1109/MCSE.2007.55},
  year      = 2007
}

@Misc{scipy,
  author =    {Eric Jones and Travis Oliphant and Pearu Peterson and others},
  title =     {{SciPy}: Open source scientific tools for {Python}},
  year =      {2001--},
  url = "http://www.scipy.org/",
  note = {[Online; accessed <today>]}
}

@article{numpy,
 author = {Walt, Stefan van der and Colbert, S. Chris and Varoquaux, Gael},
 title = {The NumPy Array: A Structure for Efficient Numerical Computation},
 journal = {Computing in Science and Engg.},
 issue_date = {March 2011},
 volume = {13},
 number = {2},
 month = mar,
 year = {2011},
 issn = {1521-9615},
 pages = {22--30},
 numpages = {9},
 url = {http://dx.doi.org/10.1109/MCSE.2011.37},
 doi = {10.1109/MCSE.2011.37},
 acmid = {1957466},
 publisher = {IEEE Educational Activities Department},
 address = {Piscataway, NJ, USA},
 keywords = {NumPy, Python, Python, NumPy, scientific programming, numerical computations, programming libraries, numerical computations, programming libraries, scientific programming},
} 

@Article{ipython,
  Author    = {P\'erez, Fernando and Granger, Brian E.},
  Title     = {{IP}ython: a System for Interactive Scientific Computing},
  Journal   = {Computing in Science and Engineering},
  Volume    = {9},
  Number    = {3},
  Pages     = {21--29},
  month     = may,
  year      = 2007,
  url       = "http://ipython.org",
  ISSN      = "1521-9615",
  doi       = {10.1109/MCSE.2007.53},
  publisher = {IEEE Computer Society},
}

@ARTICLE{Vogelsberger2014,
   author = {{Vogelsberger}, M. and {Genel}, S. and {Springel}, V. and {Torrey}, P. and 
	{Sijacki}, D. and {Xu}, D. and {Snyder}, G. and {Nelson}, D. and 
	{Hernquist}, L.},
    title = "{Introducing the Illustris Project: simulating the coevolution of dark and visible matter in the Universe}",
  journal = {\mnras},
archivePrefix = "arXiv",
   eprint = {1405.2921},
 keywords = {methods: numerical, cosmology: theory},
     year = 2014,
    month = oct,
   volume = 444,
    pages = {1518-1547},
      doi = {10.1093/mnras/stu1536},
   adsurl = {http://adsabs.harvard.edu/abs/2014MNRAS.444.1518V},
  adsnote = {Provided by the SAO/NASA Astrophysics Data System}
}

@ARTICLE{Sawala2016,
   author = {{Sawala}, T. and {Frenk}, C.~S. and {Fattahi}, A. and {Navarro}, J.~F. and 
	{Bower}, R.~G. and {Crain}, R.~A. and {Dalla Vecchia}, C. and 
	{Furlong}, M. and {Helly}, J.~C. and {Jenkins}, A. and {Oman}, K.~A. and 
	{Schaller}, M. and {Schaye}, J. and {Theuns}, T. and {Trayford}, J. and 
	{White}, S.~D.~M.},
    title = "{The APOSTLE simulations: solutions to the Local Group's cosmic puzzles}",
  journal = {\mnras},
archivePrefix = "arXiv",
   eprint = {1511.01098},
 keywords = {galaxies: evolution, galaxies: formation, cosmology: theory},
     year = 2016,
    month = apr,
   volume = 457,
    pages = {1931-1943},
      doi = {10.1093/mnras/stw145},
   adsurl = {http://adsabs.harvard.edu/abs/2016MNRAS.457.1931S},
  adsnote = {Provided by the SAO/NASA Astrophysics Data System}
}

@ARTICLE{Schaye2015,
   author = {{Schaye}, J. and {Crain}, R.~A. and {Bower}, R.~G. and {Furlong}, M. and 
	{Schaller}, M. and {Theuns}, T. and {Dalla Vecchia}, C. and 
	{Frenk}, C.~S. and {McCarthy}, I.~G. and {Helly}, J.~C. and 
	{Jenkins}, A. and {Rosas-Guevara}, Y.~M. and {White}, S.~D.~M. and 
	{Baes}, M. and {Booth}, C.~M. and {Camps}, P. and {Navarro}, J.~F. and 
	{Qu}, Y. and {Rahmati}, A. and {Sawala}, T. and {Thomas}, P.~A. and 
	{Trayford}, J.},
    title = "{The EAGLE project: simulating the evolution and assembly of galaxies and their environments}",
  journal = {\mnras},
archivePrefix = "arXiv",
   eprint = {1407.7040},
 keywords = {methods: numerical, galaxies: evolution, galaxies: formation, cosmology: theory},
     year = 2015,
    month = jan,
   volume = 446,
    pages = {521-554},
      doi = {10.1093/mnras/stu2058},
   adsurl = {http://adsabs.harvard.edu/abs/2015MNRAS.446..521S},
  adsnote = {Provided by the SAO/NASA Astrophysics Data System}
}

@ARTICLE{Wadsley2017,
   author = {{Wadsley}, J.~W. and {Keller}, B.~W. and {Quinn}, T.~R.},
    title = "{Gasoline2: a modern smoothed particle hydrodynamics code}",
  journal = {\mnras},
 keywords = {hydrodynamics, methods: numerical},
     year = 2017,
    month = oct,
   volume = 471,
    pages = {2357-2369},
      doi = {10.1093/mnras/stx1643},
   adsurl = {http://adsabs.harvard.edu/abs/2017MNRAS.471.2357W},
  adsnote = {Provided by the SAO/NASA Astrophysics Data System}
}

@ARTICLE{Hu2017,
   author = {{Hu}, C.-Y. and {Naab}, T. and {Glover}, S.~C.~O. and {Walch}, S. and 
	{Clark}, P.~C.},
    title = "{Variable interstellar radiation fields in simulated dwarf galaxies: supernovae versus photoelectric heating}",
  journal = {\mnras},
archivePrefix = "arXiv",
   eprint = {1701.08779},
 keywords = {galaxies: dwarf, galaxies: ISM, galaxies: star formation},
     year = 2017,
    month = oct,
   volume = 471,
    pages = {2151-2173},
      doi = {10.1093/mnras/stx1773},
   adsurl = {http://adsabs.harvard.edu/abs/2017MNRAS.471.2151H},
  adsnote = {Provided by the SAO/NASA Astrophysics Data System}
}

@ARTICLE{Semenov2017,
   author = {{Semenov}, V.~A. and {Kravtsov}, A.~V. and {Gnedin}, N.~Y.},
    title = "{The Physical Origin of Long Gas Depletion Times in Galaxies}",
  journal = {\apj},
archivePrefix = "arXiv",
   eprint = {1704.04239},
 keywords = {galaxies: evolution, ISM: kinematics and dynamics, methods: numerical, stars: formation},
     year = 2017,
    month = aug,
   volume = 845,
      eid = {133},
    pages = {133},
      doi = {10.3847/1538-4357/aa8096},
   adsurl = {http://adsabs.harvard.edu/abs/2017ApJ...845..133S},
  adsnote = {Provided by the SAO/NASA Astrophysics Data System}
}

@ARTICLE{Emerick2018,
   author = {{Emerick}, A. and {Bryan}, G.~L. and {Mac Low}, M.-M.},
    title = "{Stellar Radiation Is Critical for Regulating Star Formation and Driving Outflows in Low-mass Dwarf Galaxies}",
  journal = {\apjl},
archivePrefix = "arXiv",
   eprint = {1808.00468},
 keywords = {galaxies: dwarf, galaxies: evolution, galaxies: ISM, hydrodynamics, radiative transfer },
     year = 2018,
    month = oct,
   volume = 865,
      eid = {L22},
    pages = {L22},
      doi = {10.3847/2041-8213/aae315},
   adsurl = {http://adsabs.harvard.edu/abs/2018ApJ...865L..22E},
  adsnote = {Provided by the SAO/NASA Astrophysics Data System}
}

@ARTICLE{Buck2018b,
   author = {{Buck}, T. and {Ness}, M. and {Obreja}, A. and {Macci{\`o}}, A.~V. and 
	{Dutton}, A.~A.},
    title = "{Stars behind bars II: A cosmological formation scenario of the Milky Way's central stellar structure}",
  journal = {ArXiv e-prints},
archivePrefix = "arXiv",
   eprint = {1807.00829},
 keywords = {Astrophysics - Astrophysics of Galaxies},
     year = 2018,
    month = jul,
   adsurl = {http://adsabs.harvard.edu/abs/2018arXiv180700829B},
  adsnote = {Provided by the SAO/NASA Astrophysics Data System}
}

@ARTICLE{Buck2018a,
   author = {{Buck}, T. and {Ness}, M.~K. and {Macci{\`o}}, A.~V. and {Obreja}, A. and 
	{Dutton}, A.~A.},
    title = "{Stars Behind Bars. I. The Milky Way's Central Stellar Populations}",
  journal = {\apj},
archivePrefix = "arXiv",
   eprint = {1711.04765},
 keywords = {dark matter, galaxies: bulges, galaxies: formation, galaxies: individual: Milky Way, galaxies: kinematics and dynamics, methods: numerical},
     year = 2018,
    month = jul,
   volume = 861,
      eid = {88},
    pages = {88},
      doi = {10.3847/1538-4357/aac890},
   adsurl = {http://adsabs.harvard.edu/abs/2018ApJ...861...88B},
  adsnote = {Provided by the SAO/NASA Astrophysics Data System}
}

@ARTICLE{Buck2018c,
   author = {{Buck}, T. and {Macci{\`o}}, A.~V. and {Dutton}, A.~A. and {Obreja}, A. and 
	{Frings}, J.},
    title = "{NIHAO XV: The environmental impact of the host galaxy on galactic satellite and field dwarf galaxies}",
  journal = {\mnras},
archivePrefix = "arXiv",
   eprint = {1804.04667},
 keywords = {Galaxy: formation, galaxies: Local Group, galaxies: dwarfs, galaxies: kinematics and dynamics, dark matter, methods: numerical},
     year = 2018,
    month = oct,
      doi = {10.1093/mnras/sty2913},
   adsurl = {http://adsabs.harvard.edu/abs/2018MNRAS.tmp.2784B},
  adsnote = {Provided by the SAO/NASA Astrophysics Data System}
}

@ARTICLE{Messa2018,
   author = {{Messa}, M. and {Adamo}, A. and {{\"O}stlin}, G. and {Calzetti}, D. and 
	{Grasha}, K. and {Grebel}, E.~K. and {Shabani}, F. and {Chandar}, R. and 
	{Dale}, D.~A. and {Dobbs}, C.~L. and {Elmegreen}, B.~G. and 
	{Fumagalli}, M. and {Gouliermis}, D.~A. and {Kim}, H. and {Smith}, L.~J. and 
	{Thilker}, D.~A. and {Tosi}, M. and {Ubeda}, L. and {Walterbos}, R. and 
	{Whitmore}, B.~C. and {Fedorenko}, K. and {Mahadevan}, S. and 
	{Andrews}, J.~E. and {Bright}, S.~N. and {Cook}, D.~O. and {Kahre}, L. and 
	{Nair}, P. and {Pellerin}, A. and {Ryon}, J.~E. and {Ahmad}, S.~D. and 
	{Beale}, L.~P. and {Brown}, K. and {Clarkson}, D.~A. and {Guidarelli}, G.~C. and 
	{Parziale}, R. and {Turner}, J. and {Weber}, M.},
    title = "{The young star cluster population of M51 with LEGUS - I. A comprehensive study of cluster formation and evolution}",
  journal = {\mnras},
archivePrefix = "arXiv",
   eprint = {1709.06101},
 keywords = {galaxies: individual: M51, NGC 5194, galaxies: star clusters: general, galaxies: star formation},
     year = 2018,
    month = jan,
   volume = 473,
    pages = {996-1018},
      doi = {10.1093/mnras/stx2403},
   adsurl = {http://adsabs.harvard.edu/abs/2018MNRAS.473..996M},
  adsnote = {Provided by the SAO/NASA Astrophysics Data System}
}

@ARTICLE{Calzetti2015,
   author = {{Calzetti}, D. and {Lee}, J.~C. and {Sabbi}, E. and {Adamo}, A. and 
	{Smith}, L.~J. and {Andrews}, J.~E. and {Ubeda}, L. and {Bright}, S.~N. and 
	{Thilker}, D. and {Aloisi}, A. and {Brown}, T.~M. and {Chandar}, R. and 
	{Christian}, C. and {Cignoni}, M. and {Clayton}, G.~C. and {da Silva}, R. and 
	{de Mink}, S.~E. and {Dobbs}, C. and {Elmegreen}, B.~G. and 
	{Elmegreen}, D.~M. and {Evans}, A.~S. and {Fumagalli}, M. and 
	{Gallagher}, III, J.~S. and {Gouliermis}, D.~A. and {Grebel}, E.~K. and 
	{Herrero}, A. and {Hunter}, D.~A. and {Johnson}, K.~E. and {Kennicutt}, R.~C. and 
	{Kim}, H. and {Krumholz}, M.~R. and {Lennon}, D. and {Levay}, K. and 
	{Martin}, C. and {Nair}, P. and {Nota}, A. and {{\"O}stlin}, G. and 
	{Pellerin}, A. and {Prieto}, J. and {Regan}, M.~W. and {Ryon}, J.~E. and 
	{Schaerer}, D. and {Schiminovich}, D. and {Tosi}, M. and {Van Dyk}, S.~D. and 
	{Walterbos}, R. and {Whitmore}, B.~C. and {Wofford}, A.},
    title = "{Legacy Extragalactic UV Survey (LEGUS) With the Hubble Space Telescope. I. Survey Description}",
  journal = {\aj},
archivePrefix = "arXiv",
   eprint = {1410.7456},
 keywords = {galaxies: general, galaxies: star clusters: general, galaxies: star formation, galaxies: stellar content, ultraviolet: galaxies, ultraviolet: stars},
     year = 2015,
    month = feb,
   volume = 149,
      eid = {51},
    pages = {51},
      doi = {10.1088/0004-6256/149/2/51},
   adsurl = {http://adsabs.harvard.edu/abs/2015AJ....149...51C},
  adsnote = {Provided by the SAO/NASA Astrophysics Data System}
}

@ARTICLE{Federrath2009,
   author = {{Federrath}, C. and {Klessen}, R.~S. and {Schmidt}, W.},
    title = "{The Fractal Density Structure in Supersonic Isothermal Turbulence: Solenoidal Versus Compressive Energy Injection}",
  journal = {\apj},
archivePrefix = "arXiv",
   eprint = {0710.1359},
 keywords = {hydrodynamics, ISM: clouds, ISM: kinematics and dynamics, ISM: structure, methods: numerical, turbulence},
     year = 2009,
    month = feb,
   volume = 692,
    pages = {364-374},
      doi = {10.1088/0004-637X/692/1/364},
   adsurl = {http://adsabs.harvard.edu/abs/2009ApJ...692..364F},
  adsnote = {Provided by the SAO/NASA Astrophysics Data System}
}

@ARTICLE{Elmegreen2006,
   author = {{Elmegreen}, B.~G. and {Elmegreen}, D.~M. and {Chandar}, R. and 
	{Whitmore}, B. and {Regan}, M.},
    title = "{Hierarchical Star Formation in the Spiral Galaxy NGC 628}",
  journal = {\apj},
   eprint = {astro-ph/0605523},
 keywords = {Galaxies: Individual: NGC Number: NGC 628, Galaxies: Individual: Messier Number: M74},
     year = 2006,
    month = jun,
   volume = 644,
    pages = {879-889},
      doi = {10.1086/503797},
   adsurl = {http://adsabs.harvard.edu/abs/2006ApJ...644..879E},
  adsnote = {Provided by the SAO/NASA Astrophysics Data System}
}

@ARTICLE{Sanchez2008,
   author = {{S{\'a}nchez}, N. and {Alfaro}, E.~J.},
    title = "{The Fractal Distribution of H II Regions in Disk Galaxies}",
  journal = {\apjs},
archivePrefix = "arXiv",
   eprint = {0804.4554},
 keywords = {catalogs, galaxies: structure, H II regions, stars: formation},
     year = 2008,
    month = sep,
   volume = 178,
    pages = {1-19},
      doi = {10.1086/589653},
   adsurl = {http://adsabs.harvard.edu/abs/2008ApJS..178....1S},
  adsnote = {Provided by the SAO/NASA Astrophysics Data System}
}

@ARTICLE{Sanchez2005,
   author = {{S{\'a}nchez}, N. and {Alfaro}, E.~J. and {P{\'e}rez}, E.},
    title = "{The Fractal Dimension of Projected Clouds}",
  journal = {\apj},
   eprint = {astro-ph/0501573},
 keywords = {ISM: Clouds, ISM: General, ISM: Structure},
     year = 2005,
    month = jun,
   volume = 625,
    pages = {849-856},
      doi = {10.1086/429553},
   adsurl = {http://adsabs.harvard.edu/abs/2005ApJ...625..849S},
  adsnote = {Provided by the SAO/NASA Astrophysics Data System}
}

@ARTICLE{Elmegreen1996,
   author = {{Elmegreen}, B.~G. and {Falgarone}, E.},
    title = "{A Fractal Origin for the Mass Spectrum of Interstellar Clouds}",
  journal = {\apj},
 keywords = {ISM: CLOUDS, ISM: STRUCTURE, TURBULENCE},
     year = 1996,
    month = nov,
   volume = 471,
    pages = {816},
      doi = {10.1086/178009},
   adsurl = {http://adsabs.harvard.edu/abs/1996ApJ...471..816E},
  adsnote = {Provided by the SAO/NASA Astrophysics Data System}
}

@ARTICLE{Falgarone1991,
   author = {{Falgarone}, E. and {Phillips}, T.~G. and {Walker}, C.~K.},
    title = "{The edges of molecular clouds - Fractal boundaries and density structure}",
  journal = {\apj},
 keywords = {Carbon Monoxide, Interstellar Matter, Magnetohydrodynamic Turbulence, Molecular Clouds, Rotational Spectra, Angular Resolution, Carbon 12, Carbon 13, Fractals, Nebulae, Stellar Luminosity},
     year = 1991,
    month = sep,
   volume = 378,
    pages = {186-201},
      doi = {10.1086/170419},
   adsurl = {http://adsabs.harvard.edu/abs/1991ApJ...378..186F},
  adsnote = {Provided by the SAO/NASA Astrophysics Data System}
}

@BOOK{Mandelbrot1982,
   author = {{Mandelbrot}, B.~B.},
    title = "{The Fractal Geometry of Nature}",
booktitle = {The Fractal Geometry of Nature, San Francisco: Freeman, 1982},
     year = 1982,
   adsurl = {http://adsabs.harvard.edu/abs/1982fgn..book.....M},
  adsnote = {Provided by the SAO/NASA Astrophysics Data System}
}

@BOOK{Peebles1980,
   author = {{Peebles}, P.~J.~E.},
    title = "{The large-scale structure of the universe}",
 keywords = {Cosmology, Galactic Clusters, Galactic Evolution, Universe, Astronomical Models, Correlation, Mass Distribution, Particle Motion, Relativistic Theory, Statistical Distributions},
booktitle = {Research supported by the National Science Foundation.~Princeton, N.J., Princeton University Press, 1980.~435 p.},
     year = 1980,
   adsurl = {http://adsabs.harvard.edu/abs/1980lssu.book.....P},
  adsnote = {Provided by the SAO/NASA Astrophysics Data System}
}

@ARTICLE{Larson1995,
   author = {{Larson}, R.~B.},
    title = "{Star formation in groups}",
  journal = {\mnras},
 keywords = {BINARIES, GENERAL-STARS, FORMATION-ISM, CLOUDS-OPEN CLUSTERS AND ASSOCIATIONS, GENERAL.},
     year = 1995,
    month = jan,
   volume = 272,
    pages = {213-220},
      doi = {10.1093/mnras/272.1.213},
   adsurl = {http://adsabs.harvard.edu/abs/1995MNRAS.272..213L},
  adsnote = {Provided by the SAO/NASA Astrophysics Data System}
}

@ARTICLE{Calzetti1989,
   author = {{Calzetti}, D. and {Giavalisco}, M. and {Ruffini}, R.},
    title = "{The angular two-point correlation function and the cellular fractal structure of the universe}",
  journal = {\aap},
 keywords = {Angular Correlation, Cosmology, Fractals, Galactic Clusters, Galactic Structure, Spatial Distribution, Universe},
     year = 1989,
    month = dec,
   volume = 226,
    pages = {1-8},
   adsurl = {http://adsabs.harvard.edu/abs/1989Aadsnote = {Provided by the SAO/NASA Astrophysics Data System}
}

@ARTICLE{Grasha2015,
   author = {{Grasha}, K. and {Calzetti}, D. and {Adamo}, A. and {Kim}, H. and 
	{Elmegreen}, B.~G. and {Gouliermis}, D.~A. and {Aloisi}, A. and 
	{Bright}, S.~N. and {Christian}, C. and {Cignoni}, M. and {Dale}, D.~A. and 
	{Dobbs}, C. and {Elmegreen}, D.~M. and {Fumagalli}, M. and {Gallagher}, III, J.~S. and 
	{Grebel}, E.~K. and {Johnson}, K.~E. and {Lee}, J.~C. and {Messa}, M. and 
	{Smith}, L.~J. and {Ryon}, J.~E. and {Thilker}, D. and {Ubeda}, L. and 
	{Wofford}, A.},
    title = "{The Spatial Distribution of the Young Stellar Clusters in the Star-forming Galaxy NGC 628}",
  journal = {\apj},
archivePrefix = "arXiv",
   eprint = {1511.02233},
 keywords = {galaxies: individual: NGC 628, galaxies: star formation, ultraviolet: galaxies},
     year = 2015,
    month = dec,
   volume = 815,
      eid = {93},
    pages = {93},
      doi = {10.1088/0004-637X/815/2/93},
   adsurl = {http://adsabs.harvard.edu/abs/2015ApJ...815...93G},
  adsnote = {Provided by the SAO/NASA Astrophysics Data System}
}

@ARTICLE{Grasha2017b,
   author = {{Grasha}, K. and {Elmegreen}, B.~G. and {Calzetti}, D. and {Adamo}, A. and 
	{Aloisi}, A. and {Bright}, S.~N. and {Cook}, D.~O. and {Dale}, D.~A. and 
	{Fumagalli}, M. and {Gallagher}, III, J.~S. and {Gouliermis}, D.~A. and 
	{Grebel}, E.~K. and {Kahre}, L. and {Kim}, H. and {Krumholz}, M.~R. and 
	{Lee}, J.~C. and {Messa}, M. and {Ryon}, J.~E. and {Ubeda}, L.
	},
    title = "{Hierarchical Star Formation in Turbulent Media: Evidence from Young Star Clusters}",
  journal = {\apj},
archivePrefix = "arXiv",
   eprint = {1705.06281},
 keywords = {galaxies: star clusters: general, galaxies: star formation, galaxies: stellar content, galaxies: structure, stars: formation, ultraviolet: galaxies},
     year = 2017,
    month = jun,
   volume = 842,
      eid = {25},
    pages = {25},
      doi = {10.3847/1538-4357/aa740b},
   adsurl = {http://adsabs.harvard.edu/abs/2017ApJ...842...25G},
  adsnote = {Provided by the SAO/NASA Astrophysics Data System}
}

@ARTICLE{Grasha2017,
   author = {{Grasha}, K. and {Calzetti}, D. and {Adamo}, A. and {Kim}, H. and 
	{Elmegreen}, B.~G. and {Gouliermis}, D.~A. and {Dale}, D.~A. and 
	{Fumagalli}, M. and {Grebel}, E.~K. and {Johnson}, K.~E. and 
	{Kahre}, L. and {Kennicutt}, R.~C. and {Messa}, M. and {Pellerin}, A. and 
	{Ryon}, J.~E. and {Smith}, L.~J. and {Shabani}, F. and {Thilker}, D. and 
	{Ubeda}, L.},
    title = "{The Hierarchical Distribution of the Young Stellar Clusters in Six Local Star-forming Galaxies}",
  journal = {\apj},
archivePrefix = "arXiv",
   eprint = {1704.06321},
 keywords = {galaxies: star clusters: general, galaxies: star formation, galaxies: stellar content, galaxies: structure, stars: formation, ultraviolet: galaxies},
     year = 2017,
    month = may,
   volume = 840,
      eid = {113},
    pages = {113},
      doi = {10.3847/1538-4357/aa6f15},
   adsurl = {http://adsabs.harvard.edu/abs/2017ApJ...840..113G},
  adsnote = {Provided by the SAO/NASA Astrophysics Data System}
}

@ARTICLE{Hopkins2018,
   author = {{Hopkins}, P.~F. and {Wetzel}, A. and {Kere{\v s}}, D. and {Faucher-Gigu{\`e}re}, C.-A. and 
	{Quataert}, E. and {Boylan-Kolchin}, M. and {Murray}, N. and 
	{Hayward}, C.~C. and {Garrison-Kimmel}, S. and {Hummels}, C. and 
	{Feldmann}, R. and {Torrey}, P. and {Ma}, X. and {Angl{\'e}s-Alc{\'a}zar}, D. and 
	{Su}, K.-Y. and {Orr}, M. and {Schmitz}, D. and {Escala}, I. and 
	{Sanderson}, R. and {Grudi{\'c}}, M.~Y. and {Hafen}, Z. and 
	{Kim}, J.-H. and {Fitts}, A. and {Bullock}, J.~S. and {Wheeler}, C. and 
	{Chan}, T.~K. and {Elbert}, O.~D. and {Narayanan}, D.},
    title = "{FIRE-2 simulations: physics versus numerics in galaxy formation}",
  journal = {\mnras},
archivePrefix = "arXiv",
   eprint = {1702.06148},
 keywords = {methods: numerical, stars: formation, galaxies: active, galaxies: evolution, galaxies: formation, cosmology: theory},
     year = 2018,
    month = oct,
   volume = 480,
    pages = {800-863},
      doi = {10.1093/mnras/sty1690},
   adsurl = {http://adsabs.harvard.edu/abs/2018MNRAS.480..800H},
  adsnote = {Provided by the SAO/NASA Astrophysics Data System}
}

@ARTICLE{Grand2017,
   author = {{Grand}, R.~J.~J. and {G{\'o}mez}, F.~A. and {Marinacci}, F. and 
	{Pakmor}, R. and {Springel}, V. and {Campbell}, D.~J.~R. and 
	{Frenk}, C.~S. and {Jenkins}, A. and {White}, S.~D.~M.},
    title = "{The Auriga Project: the properties and formation mechanisms of disc galaxies across cosmic time}",
  journal = {\mnras},
archivePrefix = "arXiv",
   eprint = {1610.01159},
 keywords = {galaxies: evolution, galaxies: kinematics and dynamics, galaxies: spiral, galaxies: structure},
     year = 2017,
    month = may,
   volume = 467,
    pages = {179-207},
      doi = {10.1093/mnras/stx071},
   adsurl = {http://adsabs.harvard.edu/abs/2017MNRAS.467..179G},
  adsnote = {Provided by the SAO/NASA Astrophysics Data System}
}

@ARTICLE{Bose2018,
   author = {{Bose}, S. and {Frenk}, C.~S. and {Jenkins}, A. and {Fattahi}, A. and 
	{Gomez}, F.~A. and {Grand}, R.~J.~J. and {Marinacci}, F. and 
	{Navarro}, J.~F. and {Oman}, K.~A. and {Pakmor}, R. and {Schaye}, J. and 
	{Simpson}, C.~M. and {Springel}, V.},
    title = "{No cores in dark matter-dominated dwarf galaxies with bursty star formation histories}",
  journal = {ArXiv e-prints},
archivePrefix = "arXiv",
   eprint = {1810.03635},
 keywords = {Astrophysics - Astrophysics of Galaxies, Astrophysics - Cosmology and Nongalactic Astrophysics},
     year = 2018,
    month = oct,
   adsurl = {http://adsabs.harvard.edu/abs/2018arXiv181003635B},
  adsnote = {Provided by the SAO/NASA Astrophysics Data System}
}

@ARTICLE{Benitez2018,
   author = {{Benitez-Llambay}, A. and {Frenk}, C.~S. and {Ludlow}, A.~D. and 
	{Navarro}, J.~F.},
    title = "{Baryon-induced dark matter cores in the EAGLE simulations}",
  journal = {ArXiv e-prints},
archivePrefix = "arXiv",
   eprint = {1810.04186},
 keywords = {Astrophysics - Astrophysics of Galaxies},
     year = 2018,
    month = oct,
   adsurl = {http://adsabs.harvard.edu/abs/2018arXiv181004186B},
  adsnote = {Provided by the SAO/NASA Astrophysics Data System}
}

@ARTICLE{Hopkins2013,
   author = {{Hopkins}, P.~F. and {Narayanan}, D. and {Murray}, N.},
    title = "{The meaning and consequences of star formation criteria in galaxy models with resolved stellar feedback}",
  journal = {\mnras},
archivePrefix = "arXiv",
   eprint = {1303.0285},
 primaryClass = "astro-ph.CO",
 keywords = {galaxies: active, galaxies: evolution, galaxies: formation, cosmology: theory},
     year = 2013,
    month = jul,
   volume = 432,
    pages = {2647-2653},
      doi = {10.1093/mnras/stt723},
   adsurl = {http://adsabs.harvard.edu/abs/2013MNRAS.432.2647H},
  adsnote = {Provided by the SAO/NASA Astrophysics Data System}
}

@ARTICLE{Kennicutt2012,
   author = {{Kennicutt}, R.~C. and {Evans}, N.~J.},
    title = "{Star Formation in the Milky Way and Nearby Galaxies}",
  journal = {\araa},
archivePrefix = "arXiv",
   eprint = {1204.3552},
     year = 2012,
    month = sep,
   volume = 50,
    pages = {531-608},
      doi = {10.1146/annurev-astro-081811-125610},
   adsurl = {http://adsabs.harvard.edu/abs/2012ARAadsnote = {Provided by the SAO/NASA Astrophysics Data System}
}

@ARTICLE{Kennicutt1998,
   author = {{Kennicutt}, Jr., R.~C.},
    title = "{Star Formation in Galaxies Along the Hubble Sequence}",
  journal = {\araa},
   eprint = {astro-ph/9807187},
     year = 1998,
   volume = 36,
    pages = {189-232},
      doi = {10.1146/annurev.astro.36.1.189},
   adsurl = {http://adsabs.harvard.edu/abs/1998ARAadsnote = {Provided by the SAO/NASA Astrophysics Data System}
}

@conference{jupyter,
	Author = {Thomas Kluyver and Benjamin Ragan-Kelley and Fernando P{\'e}rez and Brian Granger and Matthias Bussonnier and Jonathan Frederic and Kyle Kelley and Jessica Hamrick and Jason Grout and Sylvain Corlay and Paul Ivanov and Dami{\'a}n Avila and Safia Abdalla and Carol Willing},
	Booktitle = {Positioning and Power in Academic Publishing: Players, Agents and Agendas},
	Editor = {F. Loizides and B. Schmidt},
	Organization = {IOS Press},
	Pages = {87 - 90},
	Title = {Jupyter Notebooks -- a publishing format for reproducible computational workflows},
	Year = {2016}}

@Article{SDSS,
  Title                    = {{The Seventh Data Release of the Sloan Digital Sky Survey}},
  Author                   = {{Abazajian}, K.~N. and {Adelman-McCarthy}, J.~K. and {Ag{\"u}eros}, M.~A. and {Allam}, S.~S. and {Allende Prieto}, C. and {An}, D. and {Anderson}, K.~S.~J. and {Anderson}, S.~F. and {Annis}, J. and {Bahcall}, N.~A. and et al.},
  Journal                  = {\apjs},
  Year                     = {2009},

  Month                    = jun,
  Pages                    = {543-558},
  Volume                   = {182},

  Adsnote                  = {Provided by the SAO/NASA Astrophysics Data System},
  Adsurl                   = {http://adsabs.harvard.edu/abs/2009ApJS..182..543A},
  Archiveprefix            = {arXiv},
  Doi                      = {10.1088/0067-0049/182/2/543},
  Eid                      = {543},
  Eprint                   = {0812.0649},
  Keywords                 = {atlases, catalogs, surveys}
}

@Article{Aceves2013,
  Title                    = {{The radiation energy component of the Hubble function and a LCDM cosmological simulation}},
  Author                   = {{Aceves}, H.},
  Journal                  = {ArXiv e-prints},
  Year                     = {2013},

  Month                    = feb,

  Adsnote                  = {Provided by the SAO/NASA Astrophysics Data System},
  Adsurl                   = {http://adsabs.harvard.edu/abs/2013arXiv1302.4830A},
  Archiveprefix            = {arXiv},
  Eprint                   = {1302.4830},
  Keywords                 = {Astrophysics - Cosmology and Extragalactic Astrophysics},
  Primaryclass             = {astro-ph.CO}
}

@ARTICLE{Agertz2009,
   author = {{Agertz}, O. and {Teyssier}, R. and {Moore}, B.},
    title = "{Disc formation and the origin of clumpy galaxies at high redshift}",
  journal = {\mnras},
archivePrefix = "arXiv",
   eprint = {0901.2536},
 keywords = {galaxies: evolution, galaxies: formation, galaxies: haloes},
     year = 2009,
    month = jul,
   volume = 397,
    pages = {L64-L68},
      doi = {10.1111/j.1745-3933.2009.00685.x},
   adsurl = {http://adsabs.harvard.edu/abs/2009MNRAS.397L..64A},
  adsnote = {Provided by the SAO/NASA Astrophysics Data System}
}

@Article{Anderhalden2012,
  Title                    = {{The galactic halo in mixed dark matter cosmologies}},
  Author                   = {{Anderhalden}, D. and {Diemand}, J. and {Bertone}, G. and {Macci{\`o}}, A.~V. and {Schneider}, A.},
  Journal                  = {\jcap},
  Year                     = {2012},

  Month                    = oct,
  Pages                    = {47},
  Volume                   = {10},

  Adsnote                  = {Provided by the SAO/NASA Astrophysics Data System},
  Adsurl                   = {http://adsabs.harvard.edu/abs/2012JCAP...10..047A},
  Archiveprefix            = {arXiv},
  Doi                      = {10.1088/1475-7516/2012/10/047},
  Eid                      = {047},
  Eprint                   = {1206.3788},
  Primaryclass             = {astro-ph.CO}
}

@Article{Anderhalden2013,
  Title                    = {{Hints on the nature of dark matter from the properties of Milky Way satellites}},
  Author                   = {{Anderhalden}, D. and {Schneider}, A. and {Macci{\`o}}, A.~V. and {Diemand}, J. and {Bertone}, G.},
  Journal                  = {\jcap},
  Year                     = {2013},

  Month                    = mar,
  Pages                    = {14},
  Volume                   = {3},

  Adsnote                  = {Provided by the SAO/NASA Astrophysics Data System},
  Adsurl                   = {http://adsabs.harvard.edu/abs/2013JCAP...03..014A},
  Archiveprefix            = {arXiv},
  Doi                      = {10.1088/1475-7516/2013/03/014},
  Eid                      = {014},
  Eprint                   = {1212.2967}
}

@ARTICLE{Aubert2004,
   author = {{Aubert}, D. and {Pichon}, C. and {Colombi}, S.},
    title = "{The origin and implications of dark matter anisotropic cosmic infall on \~{}L$_{*}$ haloes}",
  journal = {\mnras},
   eprint = {astro-ph/0402405},
 keywords = {galaxies: formation, galaxies: haloes, dark matter},
     year = 2004,
    month = aug,
   volume = 352,
    pages = {376-398},
      doi = {10.1111/j.1365-2966.2004.07883.x},
   adsurl = {http://cdsads.u-strasbg.fr/abs/2004MNRAS.352..376A},
  adsnote = {Provided by the SAO/NASA Astrophysics Data System}
}

@Article{Barkana1999,
  Title                    = {{The Photoevaporation of Dwarf Galaxies during Reionization}},
  Author                   = {{Barkana}, R. and {Loeb}, A.},
  Journal                  = {\apj},
  Year                     = {1999},

  Month                    = sep,
  Pages                    = {54-65},
  Volume                   = {523},

  Adsnote                  = {Provided by the SAO/NASA Astrophysics Data System},
  Adsurl                   = {http://adsabs.harvard.edu/abs/1999ApJ...523...54B},
  Doi                      = {10.1086/307724},
  Eprint                   = {astro-ph/9901114},
  Keywords                 = {COSMOLOGY: THEORY, GALAXIES: FORMATION, GALAXIES: HALOS, RADIATIVE TRANSFER, Cosmology: Theory, Galaxies: Formation, Galaxies: Halos, Radiative Transfer}
}

@Article{Barnes1986,
  Title                    = {{A hierarchical O(N log N) force-calculation algorithm}},
  Author                   = {{Barnes}, J. and {Hut}, P.},
  Journal                  = {\nat},
  Year                     = {1986},

  Month                    = dec,
  Pages                    = {446-449},
  Volume                   = {324},

  Adsnote                  = {Provided by the SAO/NASA Astrophysics Data System},
  Adsurl                   = {http://adsabs.harvard.edu/abs/1986Natur.324..446B},
  Doi                      = {10.1038/324446a0},
  Keywords                 = {COMPUTATIONAL ASTROPHYSICS, MANY BODY PROBLEM, NUMERICAL INTEGRATION, STELLAR MOTIONS, ALGORITHMS, HIERARCHIES}
}

@Article{NGC31092013,
  Title                    = {{Dwarfs walking in a row. The filamentary nature of the NGC 3109 association}},
  Author                   = {{Bellazzini}, M. and {Oosterloo}, T. and {Fraternali}, F. and {Beccari}, G.},
  Journal                  = {\aap},
  Year                     = {2013},

  Month                    = nov,
  Pages                    = {L11},
  Volume                   = {559},

  Adsnote                  = {Provided by the SAO/NASA Astrophysics Data System},
  Adsurl                   = {http://adsabs.harvard.edu/abs/2013AArchiveprefix            = {arXiv},
  Doi                      = {10.1051/0004-6361/201322744},
  Eid                      = {L11},
  Eprint                   = {1310.6365},
  Keywords                 = {Local Group, galaxies: interactions, galaxies: individual: NGC 3109}
}

@Article{Benson2007,
  Title                    = {{Luminosity and stellar mass functions of discs and spheroids in the SDSS and the supermassive black hole mass function}},
  Author                   = {{Benson}, A.~J. and {D{\v z}anovi{\'c}}, D. and {Frenk}, C.~S. and {Sharples}, R.},
  Journal                  = {\mnras},
  Year                     = {2007},

  Month                    = aug,
  Pages                    = {841-866},
  Volume                   = {379},

  Adsnote                  = {Provided by the SAO/NASA Astrophysics Data System},
  Adsurl                   = {http://adsabs.harvard.edu/abs/2007MNRAS.379..841B},
  Doi                      = {10.1111/j.1365-2966.2007.11923.x},
  Eprint                   = {astro-ph/0612719},
  Keywords                 = {galaxies: abundances, galaxies: bulges, galaxies: luminosity function, mass function, galaxies: statistics, galaxies: structure}
}

@ARTICLE{Behroozi2010,
   author = {{Behroozi}, P.~S. and {Conroy}, C. and {Wechsler}, R.~H.},
    title = "{A Comprehensive Analysis of Uncertainties Affecting the Stellar Mass-Halo Mass Relation for 0 $\lt$ z $\lt$ 4}",
  journal = {\apj},
archivePrefix = "arXiv",
   eprint = {1001.0015},
 primaryClass = "astro-ph.CO",
 keywords = {dark matter, galaxies: abundances, galaxies: evolution, galaxies: stellar content},
     year = 2010,
    month = jul,
   volume = 717,
    pages = {379-403},
      doi = {10.1088/0004-637X/717/1/379},
   adsurl = {http://adsabs.harvard.edu/abs/2010ApJ...717..379B},
  adsnote = {Provided by the SAO/NASA Astrophysics Data System}
}

@ARTICLE{Behroozi2013,
   author = {{Behroozi}, P.~S. and {Wechsler}, R.~H. and {Conroy}, C.},
    title = "{The Average Star Formation Histories of Galaxies in Dark Matter Halos from z = 0-8}",
  journal = {\apj},
archivePrefix = "arXiv",
   eprint = {1207.6105},
 primaryClass = "astro-ph.CO",
 keywords = {dark matter, galaxies: abundances, galaxies: evolution, methods: numerical},
     year = 2013,
    month = jun,
   volume = 770,
      eid = {57},
    pages = {57},
      doi = {10.1088/0004-637X/770/1/57},
   adsurl = {http://adsabs.harvard.edu/abs/2013ApJ...770...57B},
  adsnote = {Provided by the SAO/NASA Astrophysics Data System}
}

@ARTICLE{Behroozi2014,
   author = {{Behroozi}, P.~S. and {Wechsler}, R.~H. and {Lu}, Y. and {Hahn}, O. and 
	{Busha}, M.~T. and {Klypin}, A. and {Primack}, J.~R.},
    title = "{Mergers and Mass Accretion for Infalling Halos Both End Well Outside Cluster Virial Radii}",
  journal = {\apj},
archivePrefix = "arXiv",
   eprint = {1310.2239},
 keywords = {dark matter, galaxies: evolution},
     year = 2014,
    month = jun,
   volume = 787,
      eid = {156},
    pages = {156},
      doi = {10.1088/0004-637X/787/2/156},
   adsurl = {http://adsabs.harvard.edu/abs/2014ApJ...787..156B},
  adsnote = {Provided by the SAO/NASA Astrophysics Data System}
}

@Article{DM_detection,
  Title                    = {{Particle dark matter: evidence, candidates and constraints}},
  Author                   = {{Bertone}, G. and {Hooper}, D. and {Silk}, J.},
  Journal                  = {\physrep},
  Year                     = {2005},

  Month                    = jan,
  Pages                    = {279-390},
  Volume                   = {405},

  Adsnote                  = {Provided by the SAO/NASA Astrophysics Data System},
  Adsurl                   = {http://adsabs.harvard.edu/abs/2005PhR...405..279B},
  Doi                      = {10.1016/j.physrep.2004.08.031},
  Eprint                   = {hep-ph/0404175}
}

@ARTICLE{Bernardi2013,
   author = {{Bernardi}, M. and {Meert}, A. and {Sheth}, R.~K. and {Vikram}, V. and 
	{Huertas-Company}, M. and {Mei}, S. and {Shankar}, F.},
    title = "{The massive end of the luminosity and stellar mass functions: dependence on the fit to the light profile}",
  journal = {\mnras},
archivePrefix = "arXiv",
   eprint = {1304.7778},
 keywords = {galaxies: fundamental parameters, galaxies: luminosity function, mass function, galaxies: photometry},
     year = 2013,
    month = nov,
   volume = 436,
    pages = {697-704},
      doi = {10.1093/mnras/stt1607},
   adsurl = {http://adsabs.harvard.edu/abs/2013MNRAS.436..697B},
  adsnote = {Provided by the SAO/NASA Astrophysics Data System}
}

@Article{Bertschinger2001,
  Title                    = {{Multiscale Gaussian Random Fields and Their Application to Cosmological Simulations}},
  Author                   = {{Bertschinger}, E.},
  Journal                  = {\apjs},
  Year                     = {2001},

  Month                    = nov,
  Pages                    = {1-20},
  Volume                   = {137},

  Adsnote                  = {Provided by the SAO/NASA Astrophysics Data System},
  Adsurl                   = {http://adsabs.harvard.edu/abs/2001ApJS..137....1B},
  Doi                      = {10.1086/322526},
  Eprint                   = {astro-ph/0103301},
  Keywords                 = {Cosmology: Theory, Methods: Numerical}
}

@Book{Binney2008,
  Title                    = {Galactic dynamics},
  Author                   = {Binney, James and Tremaine, Scott},
  Publisher                = {Princeton University Press},
  Year                     = {2008},

  Address                  = {Princeton, NJ ; Oxford},
  Edition                  = {2. ed.},
  Series                   = {Princeton series in astrophysics},

  ISBN                     = {978-0-691-13027-9 ; 978-0-691-13026-2},
  Keywords                 = {(s)Galaxie / (s)Dynamik},
  Language                 = {eng},
  Library                  = {UB [Signatur:LN-T 11-14229::(2)] ; AR [Signatur:AS/BIN:J-2008]},
  Pages                    = {XVI, 885 S., [8] Bl.}
}

@ARTICLE{Bournaud2007,
   author = {{Bournaud}, F. and {Elmegreen}, B.~G. and {Elmegreen}, D.~M.
	},
    title = "{Rapid Formation of Exponential Disks and Bulges at High Redshift from the Dynamical Evolution of Clump-Cluster and Chain Galaxies}",
  journal = {\apj},
archivePrefix = "arXiv",
   eprint = {0708.0306},
 keywords = {Galaxies: Evolution, Galaxies: Formation, Galaxies: High-Redshift},
     year = 2007,
    month = nov,
   volume = 670,
    pages = {237-248},
      doi = {10.1086/522077},
   adsurl = {http://adsabs.harvard.edu/abs/2007ApJ...670..237B},
  adsnote = {Provided by the SAO/NASA Astrophysics Data System}
}

@ARTICLE{Bournaud2008,
   author = {{Bournaud}, F. and {Daddi}, E. and {Elmegreen}, B.~G. and {Elmegreen}, D.~M. and 
	{Nesvadba}, N. and {Vanzella}, E. and {Di Matteo}, P. and {Le Tiran}, L. and 
	{Lehnert}, M. and {Elbaz}, D.},
    title = "{Observations and modeling of a clumpy galaxy at z = 1.6. Spectroscopic clues to the origin and evolution of chain galaxies}",
  journal = {\aap},
archivePrefix = "arXiv",
   eprint = {0803.3831},
 keywords = {galaxies: formation, galaxies: kinematics and dynamics, galaxies: evolution, galaxies: interactions},
     year = 2008,
    month = aug,
   volume = 486,
    pages = {741-753},
      doi = {10.1051/0004-6361:20079250},
   adsurl = {http://adsabs.harvard.edu/abs/2008Aadsnote = {Provided by the SAO/NASA Astrophysics Data System}
}

@ARTICLE{Bournaud2009,
   author = {{Bournaud}, F. and {Elmegreen}, B.~G.},
    title = "{Unstable Disks at High Redshift: Evidence for Smooth Accretion in Galaxy Formation}",
  journal = {\apjl},
archivePrefix = "arXiv",
   eprint = {0902.2806},
 primaryClass = "astro-ph.CO",
 keywords = {galaxies: formation, galaxies: high-redshift, instabilities},
     year = 2009,
    month = apr,
   volume = 694,
    pages = {L158-L161},
      doi = {10.1088/0004-637X/694/2/L158},
   adsurl = {http://adsabs.harvard.edu/abs/2009ApJ...694L.158B},
  adsnote = {Provided by the SAO/NASA Astrophysics Data System}
}

@ARTICLE{Bournaud2014,
   author = {{Bournaud}, F. and {Perret}, V. and {Renaud}, F. and {Dekel}, A. and 
	{Elmegreen}, B.~G. and {Elmegreen}, D.~M. and {Teyssier}, R. and 
	{Amram}, P. and {Daddi}, E. and {Duc}, P.-A. and {Elbaz}, D. and 
	{Epinat}, B. and {Gabor}, J.~M. and {Juneau}, S. and {Kraljic}, K. and 
	{Le Floch'}, E.},
    title = "{The Long Lives of Giant Clumps and the Birth of Outflows in Gas-rich Galaxies at High Redshift}",
  journal = {\apj},
archivePrefix = "arXiv",
   eprint = {1307.7136},
 keywords = {galaxies: bulges, galaxies: evolution, galaxies: formation, galaxies: high-redshift, galaxies: structure },
     year = 2014,
    month = jan,
   volume = 780,
      eid = {57},
    pages = {57},
      doi = {10.1088/0004-637X/780/1/57},
   adsurl = {http://adsabs.harvard.edu/abs/2014ApJ...780...57B},
  adsnote = {Provided by the SAO/NASA Astrophysics Data System}
}

@Article{Boylan2011,
  Title                    = {{Too big to fail? The puzzling darkness of massive Milky Way subhaloes}},
  Author                   = {{Boylan-Kolchin}, M. and {Bullock}, J.~S. and {Kaplinghat}, M.},
  Journal                  = {\mnras},
  Year                     = {2011},

  Month                    = jul,
  Pages                    = {L40-L44},
  Volume                   = {415},

  Adsnote                  = {Provided by the SAO/NASA Astrophysics Data System},
  Adsurl                   = {http://adsabs.harvard.edu/abs/2011MNRAS.415L..40B},
  Archiveprefix            = {arXiv},
  Doi                      = {10.1111/j.1745-3933.2011.01074.x},
  Eprint                   = {1103.0007},
  Keywords                 = {Galaxy: halo, galaxies: abundances, cosmology: theory, dark matter},
  Primaryclass             = {astro-ph.CO}
}

@Article{millenium,
  Title                    = {{Resolving cosmic structure formation with the Millennium-II Simulation}},
  Author                   = {{Boylan-Kolchin}, M. and {Springel}, V. and {White}, S.~D.~M. and {Jenkins}, A. and {Lemson}, G.},
  Journal                  = {\mnras},
  Year                     = {2009},

  Month                    = sep,
  Pages                    = {1150-1164},
  Volume                   = {398},

  Adsnote                  = {Provided by the SAO/NASA Astrophysics Data System},
  Adsurl                   = {http://adsabs.harvard.edu/abs/2009MNRAS.398.1150B},
  Archiveprefix            = {arXiv},
  Doi                      = {10.1111/j.1365-2966.2009.15191.x},
  Eprint                   = {0903.3041},
  Keywords                 = {methods: N-body simulations , galaxies: haloes , cosmology: theory},
  Primaryclass             = {astro-ph.CO}
}

@Article{Brooks2012,
  Title                    = {{A Baryonic Solution to the Missing Satellites Problem}},
  Author                   = {{Brooks}, A.~M. and {Kuhlen}, M. and {Zolotov}, A. and {Hooper}, D. },
  Journal                  = {ArXiv e-prints},
  Year                     = {2012},

  Month                    = sep,

  Abstract                 = {{It has been demonstrated that the inclusion of baryonic physics can alter the dark matter densities in the centers of low-mass galaxies, making the central dark matter slope more shallow than predicted in pure cold dark matter simulations. This flattening of the dark matter profile can occur in the most luminous subhalos around Milky Way-mass galaxies. Zolotov et al. (2012) have suggested a correction to be applied to the central masses of dark matter-only satellites in order to mimic the affect of (1) the flattening of the dark matter cusp due to supernova feedback in luminous satellites, and (2) enhanced tidal stripping due to the presence of a baryonic disk. In this paper, we apply this correction to the z=0 subhalo masses from the high resolution, dark matter-only Via Lactea II (VL2) simulation, and find that the number of massive subhalos is dramatically reduced. After adopting a stellar mass to halo mass relationship for the VL2 halos, and identifying subhalos that are (1) likely to be destroyed by stripping and (2) likely to have star formation suppressed by photo-heating, we find that the number of massive, luminous satellites around a Milky Way-mass galaxy is in agreement with the number of observed satellites around the Milky Way or M31. We conclude that baryonic processes have the potential to solve the missing satellites problem. }},
  Adsnote                  = {Provided by the SAO/NASA Astrophysics Data System},
  Adsurl                   = {http://adsabs.harvard.edu/abs/2012arXiv1209.5394B},
  Archiveprefix            = {arXiv},
  Eprint                   = {1209.5394},
  Keywords                 = {Astrophysics - Cosmology and Extragalactic Astrophysics, Astrophysics - Galaxy Astrophysics},
  Primaryclass             = {astro-ph.CO}
}

@Article{Brooks2014,
  Title                    = {{Why Baryons Matter: The Kinematics of Dwarf Spheroidal Satellites}},
  Author                   = {{Brooks}, A.~M. and {Zolotov}, A.},
  Journal                  = {\apj},
  Year                     = {2014},

  Month                    = may,
  Pages                    = {87},
  Volume                   = {786},

  Adsnote                  = {Provided by the SAO/NASA Astrophysics Data System},
  Adsurl                   = {http://adsabs.harvard.edu/abs/2014ApJ...786...87B},
  Archiveprefix            = {arXiv},
  Doi                      = {10.1088/0004-637X/786/2/87},
  Eid                      = {87},
  Eprint                   = {1207.2468},
  Keywords                 = {galaxies: dwarf, galaxies: evolution, galaxies: interactions, galaxies: kinematics and dynamics}
}

@Article{Bryan2001,
  Title                    = {{Achieving Extreme Resolution in Numerical Cosmology Using Adaptive Mesh Refinement: Resolving Primordial Star Formation}},
  Author                   = {{Bryan}, G.~L. and {Abel}, T. and {Norman}, M.~L.},
  Journal                  = {ArXiv Astrophysics e-prints},
  Year                     = {2001},

  Month                    = dec,

  Adsnote                  = {Provided by the SAO/NASA Astrophysics Data System},
  Adsurl                   = {http://adsabs.harvard.edu/abs/2001astro.ph.12089B},
  Eprint                   = {arXiv:astro-ph/0112089},
  Keywords                 = {Astrophysics}
}

@ARTICLE{Buck2015,
   author = {{Buck}, T. and {Macci{\`o}}, A.~V. and {Dutton}, A.~A.},
    title = "{Evidence for Early Filamentary Accretion from the Andromeda Galaxy{\rsquo}s Thin Plane of Satellites}",
  journal = {\apj},
archivePrefix = "arXiv",
   eprint = {1504.05193},
 keywords = {dark matter, galaxies: dwarf, galaxies: formation, galaxies: individual: M31 Andromeda, galaxies: kinematics and dynamics, methods: numerical},
     year = 2015,
    month = aug,
   volume = 809,
      eid = {49},
    pages = {49},
      doi = {10.1088/0004-637X/809/1/49},
   adsurl = {http://adsabs.harvard.edu/abs/2015ApJ...809...49B},
  adsnote = {Provided by the SAO/NASA Astrophysics Data System}
}

@ARTICLE{Buck2017,
   author = {{Buck}, T. and {Macci{\`o}}, A.~V. and {Obreja}, A. and {Dutton}, A.~A. and 
	{Dom{\'{\i}}nguez-Tenreiro}, R. and {Granato}, G.~L.},
    title = "{NIHAO XIII: Clumpy discs or clumpy light in high-redshift galaxies?}",
  journal = {\mnras},
archivePrefix = "arXiv",
   eprint = {1612.05277},
 keywords = {methods: numerical, Galaxy: formation, galaxies: bulges, galaxies: evolution, galaxies: high-redshift, galaxies: ISM},
     year = 2017,
    month = jul,
   volume = 468,
    pages = {3628-3649},
      doi = {10.1093/mnras/stx685},
   adsurl = {http://adsabs.harvard.edu/abs/2017MNRAS.468.3628B},
  adsnote = {Provided by the SAO/NASA Astrophysics Data System}
}

@Article{Bullock2010,
  Title                    = {{Notes on the Missing Satellites Problem}},
  Author                   = {{Bullock}, J.~S.},
  Journal                  = {ArXiv e-prints},
  Year                     = {2010},

  Month                    = sep,

  Adsnote                  = {Provided by the SAO/NASA Astrophysics Data System},
  Adsurl                   = {http://adsabs.harvard.edu/abs/2010arXiv1009.4505B},
  Archiveprefix            = {arXiv},
  Eprint                   = {1009.4505},
  Keywords                 = {Astrophysics - Cosmology and Extragalactic Astrophysics, Astrophysics - Galaxy Astrophysics},
  Primaryclass             = {astro-ph.CO}
}

@Article{Bullock2000,
  Title                    = {{Reionization and the Abundance of Galactic Satellites}},
  Author                   = {{Bullock}, J.~S. and {Kravtsov}, A.~V. and {Weinberg}, D.~H. },
  Journal                  = {\apj},
  Year                     = {2000},

  Month                    = aug,
  Pages                    = {517-521},
  Volume                   = {539},

  Adsnote                  = {Provided by the SAO/NASA Astrophysics Data System},
  Adsurl                   = {http://adsabs.harvard.edu/abs/2000ApJ...539..517B},
  Doi                      = {10.1086/309279},
  Eprint                   = {arXiv:astro-ph/0002214},
  Keywords                 = {Cosmology: Theory, Galaxies: Formation}
}

@ARTICLE{Bureau2006,
   author = {{Bureau}, M. and {Aronica}, G. and {Athanassoula}, E. and {Dettmar}, R.-J. and 
	{Bosma}, A. and {Freeman}, K.~C.},
    title = "{K-band observations of boxy bulges - I. Morphology and surface brightness profiles}",
  journal = {\mnras},
   eprint = {astro-ph/0606056},
 keywords = {galaxies: bulges: galaxies: evolution: galaxies: formation: galaxies: photometry: galaxies: spiral: galaxies: structure, galaxies: bulges, galaxies: evolution, galaxies: formation, galaxies: photometry, galaxies: spiral, galaxies: structure},
     year = 2006,
    month = aug,
   volume = 370,
    pages = {753-772},
      doi = {10.1111/j.1365-2966.2006.10471.x},
   adsurl = {http://adsabs.harvard.edu/abs/2006MNRAS.370..753B},
  adsnote = {Provided by the SAO/NASA Astrophysics Data System}
}

@ARTICLE{Bruzual2003,
   author = {{Bruzual}, G. and {Charlot}, S.},
    title = "{Stellar population synthesis at the resolution of 2003}",
  journal = {\mnras},
   eprint = {astro-ph/0309134},
 keywords = {stars: evolution, galaxies: evolution, galaxies: formation, galaxies: stellar content},
     year = 2003,
    month = oct,
   volume = 344,
    pages = {1000-1028},
      doi = {10.1046/j.1365-8711.2003.06897.x},
   adsurl = {http://adsabs.harvard.edu/abs/2003MNRAS.344.1000B},
  adsnote = {Provided by the SAO/NASA Astrophysics Data System}
}

@ARTICLE{Cacciato2012,
   author = {{Cacciato}, M. and {Dekel}, A. and {Genel}, S.},
    title = "{Evolution of violent gravitational disc instability in galaxies: late stabilization by transition from gas to stellar dominance}",
  journal = {\mnras},
archivePrefix = "arXiv",
   eprint = {1110.2412},
 keywords = {methods: analytical, galaxies: evolution, galaxies: formation, galaxies: haloes, galaxies: spiral, galaxies: star formation},
     year = 2012,
    month = mar,
   volume = 421,
    pages = {818-831},
      doi = {10.1111/j.1365-2966.2011.20359.x},
   adsurl = {http://adsabs.harvard.edu/abs/2012MNRAS.421..818C},
  adsnote = {Provided by the SAO/NASA Astrophysics Data System}
}

@Article{Cautun2015,
  Title                    = {{Planes of satellite galaxies: when exceptions are the rule}},
  Author                   = {{Cautun}, M. and {Bose}, S. and {Frenk}, C.~S. and {Guo}, Q. and {Han}, J. and {Hellwing}, W.~A. and {Sawala}, T. and {Wang}, W. },
  Journal                  = {ArXiv e-prints},
  Year                     = {2015},

  Month                    = jun,

  Adsnote                  = {Provided by the SAO/NASA Astrophysics Data System},
  Adsurl                   = {http://adsabs.harvard.edu/abs/2015arXiv150604151C},
  Archiveprefix            = {arXiv},
  Eprint                   = {1506.04151},
  Keywords                 = {Astrophysics - Astrophysics of Galaxies, Astrophysics - Cosmology and Nongalactic Astrophysics}
}

@ARTICLE{Ceverino2010,
   author = {{Ceverino}, D. and {Dekel}, A. and {Bournaud}, F.},
    title = "{High-redshift clumpy discs and bulges in cosmological simulations}",
  journal = {\mnras},
archivePrefix = "arXiv",
   eprint = {0907.3271},
 keywords = {stars: formation, galaxies: evolution, galaxies: formation, galaxies: kinematics and dynamics, galaxies: spiral},
     year = 2010,
    month = jun,
   volume = 404,
    pages = {2151-2169},
      doi = {10.1111/j.1365-2966.2010.16433.x},
   adsurl = {http://adsabs.harvard.edu/abs/2010MNRAS.404.2151C},
  adsnote = {Provided by the SAO/NASA Astrophysics Data System}
}

@ARTICLE{Ceverino2012,
   author = {{Ceverino}, D. and {Dekel}, A. and {Mandelker}, N. and {Bournaud}, F. and 
	{Burkert}, A. and {Genzel}, R. and {Primack}, J.},
    title = "{Rotational support of giant clumps in high-z disc galaxies}",
  journal = {\mnras},
archivePrefix = "arXiv",
   eprint = {1106.5587},
 keywords = {stars: formation, galaxies: evolution, galaxies: formation, galaxies: kinematics and dynamics, galaxies: spiral, cosmology: observations},
     year = 2012,
    month = mar,
   volume = 420,
    pages = {3490-3520},
      doi = {10.1111/j.1365-2966.2011.20296.x},
   adsurl = {http://adsabs.harvard.edu/abs/2012MNRAS.420.3490C},
  adsnote = {Provided by the SAO/NASA Astrophysics Data System}
}

@ARTICLE{Ceverino2014,
   author = {{Ceverino}, D. and {Klypin}, A. and {Klimek}, E.~S. and {Trujillo-Gomez}, S. and 
	{Churchill}, C.~W. and {Primack}, J. and {Dekel}, A.},
    title = "{Radiative feedback and the low efficiency of galaxy formation in low-mass haloes at high redshift}",
  journal = {\mnras},
archivePrefix = "arXiv",
   eprint = {1307.0943},
 keywords = {galaxies: formation},
     year = 2014,
    month = aug,
   volume = 442,
    pages = {1545-1559},
      doi = {10.1093/mnras/stu956},
   adsurl = {http://adsabs.harvard.edu/abs/2014MNRAS.442.1545C},
  adsnote = {Provided by the SAO/NASA Astrophysics Data System}
}

@ARTICLE{Ceverino2015,
   author = {{Ceverino}, D. and {Dekel}, A. and {Tweed}, D. and {Primack}, J.
	},
    title = "{Early formation of massive, compact, spheroidal galaxies with classical profiles by violent disc instability or mergers}",
  journal = {\mnras},
archivePrefix = "arXiv",
   eprint = {1409.2622},
 keywords = {galaxies: evolution, galaxies: formation, cosmology: theory},
     year = 2015,
    month = mar,
   volume = 447,
    pages = {3291-3310},
      doi = {10.1093/mnras/stu2694},
   adsurl = {http://adsabs.harvard.edu/abs/2015MNRAS.447.3291C},
  adsnote = {Provided by the SAO/NASA Astrophysics Data System}
}

@ARTICLE{Chabrier2003,
   author = {{Chabrier}, G.},
    title = "{Galactic Stellar and Substellar Initial Mass Function}",
  journal = {\pasp},
   eprint = {astro-ph/0304382},
 keywords = {Galaxies: Luminosity Function, Mass Function, Invited Reviews},
     year = 2003,
    month = jul,
   volume = 115,
    pages = {763-795},
      doi = {10.1086/376392},
   adsurl = {http://adsabs.harvard.edu/abs/2003PASP..115..763C},
  adsnote = {Provided by the SAO/NASA Astrophysics Data System}
}

@Article{M812013,
  Title                    = {{Confirmation of Faint Dwarf Galaxies in the M81 Group}},
  Author                   = {{Chiboucas}, K. and {Jacobs}, B.~A. and {Tully}, R.~B. and {Karachentsev}, I.~D. },
  Journal                  = {\aj},
  Year                     = {2013},

  Month                    = nov,
  Pages                    = {126},
  Volume                   = {146},

  Adsnote                  = {Provided by the SAO/NASA Astrophysics Data System},
  Adsurl                   = {http://adsabs.harvard.edu/abs/2013AJ....146..126C},
  Archiveprefix            = {arXiv},
  Doi                      = {10.1088/0004-6256/146/5/126},
  Eid                      = {126},
  Eprint                   = {1309.4130},
  Keywords                 = {galaxies: dwarf, galaxies: fundamental parameters, galaxies: groups: individual: M81, galaxies: luminosity function, mass function, galaxies: photometry}
}

@ARTICLE{Chan2015,
   author = {{Chan}, T.~K. and {Kere{\v s}}, D. and {O{\~n}orbe}, J. and 
	{Hopkins}, P.~F. and {Muratov}, A.~L. and {Faucher-Gigu{\`e}re}, C.-A. and 
	{Quataert}, E.},
    title = "{The impact of baryonic physics on the structure of dark matter haloes: the view from the FIRE cosmological simulations}",
  journal = {\mnras},
archivePrefix = "arXiv",
   eprint = {1507.02282},
 keywords = {galaxies: evolution, galaxies: haloes, galaxies: kinematics and dynamics, galaxies: structure, dark matter},
     year = 2015,
    month = dec,
   volume = 454,
    pages = {2981-3001},
      doi = {10.1093/mnras/stv2165},
   adsurl = {http://adsabs.harvard.edu/abs/2015MNRAS.454.2981C},
  adsnote = {Provided by the SAO/NASA Astrophysics Data System}
}

@Article{Colin2000,
  Title                    = {{Substructure and Halo Density Profiles in a Warm Dark Matter Cosmology}},
  Author                   = {{Col{\'{\i}}n}, P. and {Avila-Reese}, V. and {Valenzuela}, O. },
  Journal                  = {\apj},
  Year                     = {2000},

  Month                    = oct,
  Pages                    = {622-630},
  Volume                   = {542},

  Adsnote                  = {Provided by the SAO/NASA Astrophysics Data System},
  Adsurl                   = {http://adsabs.harvard.edu/abs/2000ApJ...542..622C},
  Doi                      = {10.1086/317057},
  Eprint                   = {astro-ph/0004115},
  Keywords                 = {Cosmology: Dark Matter, Galaxies: Formation, Galaxies: Halos, Methods: n-Body Simulations}
}

@Article{Collins2015,
  Title                    = {{Comparing the Observable Properties of Dwarf Galaxies on and off the Andromeda Plane}},
  Author                   = {{Collins}, M.~L.~M. and {Martin}, N.~F. and {Rich}, R.~M. and {Ibata}, R.~A. and {Chapman}, S.~C. and {McConnachie}, A.~W. and {Ferguson}, A.~M. and {Irwin}, M.~J. and {Lewis}, G.~F.},
  Journal                  = {\apjl},
  Year                     = {2015},

  Month                    = jan,
  Pages                    = {L13},
  Volume                   = {799},

  Adsnote                  = {Provided by the SAO/NASA Astrophysics Data System},
  Adsurl                   = {http://adsabs.harvard.edu/abs/2015ApJ...799L..13C},
  Archiveprefix            = {arXiv},
  Doi                      = {10.1088/2041-8205/799/1/L13},
  Eid                      = {L13},
  Eprint                   = {1411.3324},
  Keywords                 = {dark matter, galaxies: dwarf, galaxies: fundamental parameters, galaxies: kinematics and dynamics, Local Group}
}

@ARTICLE{Conn2012,
   author = {{Conn}, A.~R. and {Ibata}, R.~A. and {Lewis}, G.~F. and {Parker}, Q.~A. and 
	{Zucker}, D.~B. and {Martin}, N.~F. and {McConnachie}, A.~W. and 
	{Irwin}, M.~J. and {Tanvir}, N. and {Fardal}, M.~A. and {Ferguson}, A.~M.~N. and 
	{Chapman}, S.~C. and {Valls-Gabaud}, D.},
    title = "{A Bayesian Approach to Locating the Red Giant Branch Tip Magnitude. II. Distances to the Satellites of M31}",
  journal = {\apj},
archivePrefix = "arXiv",
   eprint = {1209.4952},
 keywords = {galaxies: general, galaxies: stellar content, Local Group},
     year = 2012,
    month = oct,
   volume = 758,
      eid = {11},
    pages = {11},
      doi = {10.1088/0004-637X/758/1/11},
   adsurl = {http://adsabs.harvard.edu/abs/2012ApJ...758...11C},
  adsnote = {Provided by the SAO/NASA Astrophysics Data System}
}

@Article{Conn2013,
  Title                    = {{The Three-dimensional Structure of the M31 Satellite System; Strong Evidence for an Inhomogeneous Distribution of Satellites}},
  Author                   = {{Conn}, A.~R. and {Lewis}, G.~F. and {Ibata}, R.~A. and {Parker}, Q.~A. and {Zucker}, D.~B. and {McConnachie}, A.~W. and {Martin}, N.~F. and {Valls-Gabaud}, D. and {Tanvir}, N. and {Irwin}, M.~J. and {Ferguson}, A.~M.~N. and {Chapman}, S.~C.},
  Journal                  = {\apj},
  Year                     = {2013},

  Month                    = apr,
  Pages                    = {120},
  Volume                   = {766},

  Adsnote                  = {Provided by the SAO/NASA Astrophysics Data System},
  Adsurl                   = {http://adsabs.harvard.edu/abs/2013ApJ...766..120C},
  Archiveprefix            = {arXiv},
  Doi                      = {10.1088/0004-637X/766/2/120},
  Eid                      = {120},
  Eprint                   = {1301.7131},
  Keywords                 = {galaxies: dwarf, galaxies: halos, galaxies: individual: M31, galaxies: structure },
  Primaryclass             = {astro-ph.CO}
}

@Article{Conroy2006,
  Title                    = {{Modeling Luminosity-dependent Galaxy Clustering through Cosmic Time}},
  Author                   = {{Conroy}, C. and {Wechsler}, R.~H. and {Kravtsov}, A.~V.},
  Journal                  = {\apj},
  Year                     = {2006},

  Month                    = aug,
  Pages                    = {201-214},
  Volume                   = {647},

  Adsnote                  = {Provided by the SAO/NASA Astrophysics Data System},
  Adsurl                   = {http://adsabs.harvard.edu/abs/2006ApJ...647..201C},
  Doi                      = {10.1086/503602},
  Eprint                   = {astro-ph/0512234},
  Keywords                 = {Cosmology: Theory, Cosmology: Dark Matter, Galaxies: Clusters: General, Galaxies: Evolution, Galaxies: Halos, Cosmology: Large-Scale Structure of Universe}
}

@ARTICLE{Conroy2012,
   author = {{Conroy}, C. and {van Dokkum}, P.~G.},
    title = "{The Stellar Initial Mass Function in Early-type Galaxies From Absorption Line Spectroscopy. II. Results}",
  journal = {\apj},
archivePrefix = "arXiv",
   eprint = {1205.6473},
 keywords = {galaxies: abundances, galaxies: elliptical and lenticular, cD, galaxies: stellar content},
     year = 2012,
    month = nov,
   volume = 760,
      eid = {71},
    pages = {71},
      doi = {10.1088/0004-637X/760/1/71},
   adsurl = {http://adsabs.harvard.edu/abs/2012ApJ...760...71C},
  adsnote = {Provided by the SAO/NASA Astrophysics Data System}
}

@ARTICLE{Conselice2004,
   author = {{Conselice}, C.~J. and {Grogin}, N.~A. and {Jogee}, S. and {Lucas}, R.~A. and 
	{Dahlen}, T. and {de Mello}, D. and {Gardner}, J.~P. and {Mobasher}, B. and 
	{Ravindranath}, S.},
    title = "{Observing the Formation of the Hubble Sequence in the Great Observatories Origins Deep Survey}",
  journal = {\apjl},
   eprint = {astro-ph/0309039},
 keywords = {Galaxies: Elliptical and Lenticular, cD, Galaxies: Evolution, Galaxies: Formation, Galaxies: Spiral},
     year = 2004,
    month = jan,
   volume = 600,
    pages = {L139-L142},
      doi = {10.1086/378556},
   adsurl = {http://adsabs.harvard.edu/abs/2004ApJ...600L.139C},
  adsnote = {Provided by the SAO/NASA Astrophysics Data System}
}

@ARTICLE{Courteau2015,
   author = {{Courteau}, S. and {Dutton}, A.~A.},
    title = "{On the Global Mass Distribution in Disk Galaxies}",
  journal = {\apjl},
archivePrefix = "arXiv",
   eprint = {1502.04709},
 keywords = {dark matter, galaxies: formation, galaxies: spiral},
     year = 2015,
    month = mar,
   volume = 801,
      eid = {L20},
    pages = {L20},
      doi = {10.1088/2041-8205/801/2/L20},
   adsurl = {http://adsabs.harvard.edu/abs/2015ApJ...801L..20C},
  adsnote = {Provided by the SAO/NASA Astrophysics Data System}
}

@ARTICLE{Daddi2010,
   author = {{Daddi}, E. and {Bournaud}, F. and {Walter}, F. and {Dannerbauer}, H. and 
	{Carilli}, C.~L. and {Dickinson}, M. and {Elbaz}, D. and {Morrison}, G.~E. and 
	{Riechers}, D. and {Onodera}, M. and {Salmi}, F. and {Krips}, M. and 
	{Stern}, D.},
    title = "{Very High Gas Fractions and Extended Gas Reservoirs in z = 1.5 Disk Galaxies}",
  journal = {\apj},
archivePrefix = "arXiv",
   eprint = {0911.2776},
 keywords = {cosmology: observations, galaxies: evolution, galaxies: formation, galaxies: starburst, infrared: galaxies},
     year = 2010,
    month = apr,
   volume = 713,
    pages = {686-707},
      doi = {10.1088/0004-637X/713/1/686},
   adsurl = {http://adsabs.harvard.edu/abs/2010ApJ...713..686D},
  adsnote = {Provided by the SAO/NASA Astrophysics Data System}
}

@Article{Davis1985,
  Title                    = {{The evolution of large-scale structure in a universe dominated by cold dark matter}},
  Author                   = {{Davis}, M. and {Efstathiou}, G. and {Frenk}, C.~S. and {White}, S.~D.~M. },
  Journal                  = {\apj},
  Year                     = {1985},

  Month                    = may,
  Pages                    = {371-394},
  Volume                   = {292},

  Adsnote                  = {Provided by the SAO/NASA Astrophysics Data System},
  Adsurl                   = {http://adsabs.harvard.edu/abs/1985ApJ...292..371D},
  Doi                      = {10.1086/163168},
  Keywords                 = {COMPUTATIONAL ASTROPHYSICS, COSMOLOGY, GALACTIC CLUSTERS, GALACTIC EVOLUTION, MISSING MASS (ASTROPHYSICS), WEAK INTERACTIONS (FIELD THEORY), ASTRONOMICAL MODELS, COMPUTERIZED SIMULATION, CORRELATION, MANY BODY PROBLEM, MASS DISTRIBUTION, MASS TO LIGHT RATIOS, MATTER (PHYSICS), RADIAL VELOCITY, RED SHIFT}
}

@Article{DeBlok2002,
  Title                    = {{High-resolution rotation curves of low surface brightness galaxies}},
  Author                   = {{de Blok}, W.~J.~G. and {Bosma}, A.},
  Journal                  = {\aap},
  Year                     = {2002},

  Month                    = apr,
  Pages                    = {816-846},
  Volume                   = {385},

  Adsnote                  = {Provided by the SAO/NASA Astrophysics Data System},
  Adsurl                   = {http://adsabs.harvard.edu/abs/2002ADoi                      = {10.1051/0004-6361:20020080},
  Eprint                   = {astro-ph/0201276},
  Keywords                 = {GALAXIES: STRUCTURE, GALAXIES: KINEMATICS AND DYNAMICS, GALAXIES: HALOS}
}

@Article{DeBlok2001,
  Title                    = {{High-Resolution Rotation Curves of Low Surface Brightness Galaxies. II. Mass Models}},
  Author                   = {{de Blok}, W.~J.~G. and {McGaugh}, S.~S. and {Rubin}, V.~C.},
  Journal                  = {\aj},
  Year                     = {2001},

  Month                    = nov,
  Pages                    = {2396-2427},
  Volume                   = {122},

  Adsnote                  = {Provided by the SAO/NASA Astrophysics Data System},
  Adsurl                   = {http://adsabs.harvard.edu/abs/2001AJ....122.2396D},
  Doi                      = {10.1086/323450},
  Keywords                 = {Cosmology: Dark Matter, Galaxies: Fundamental Parameters, Galaxies: Kinematics and Dynamics}
}

@Article{Deason2011,
  Title                    = {{Mismatch and misalignment: dark haloes and satellites of disc galaxies}},
  Author                   = {{Deason}, A.~J. and {McCarthy}, I.~G. and {Font}, A.~S. and {Evans}, N.~W. and {Frenk}, C.~S. and {Belokurov}, V. and {Libeskind}, N.~I. and {Crain}, R.~A. and {Theuns}, T.},
  Journal                  = {\mnras},
  Year                     = {2011},

  Month                    = aug,
  Pages                    = {2607-2625},
  Volume                   = {415},

  Adsnote                  = {Provided by the SAO/NASA Astrophysics Data System},
  Adsurl                   = {http://adsabs.harvard.edu/abs/2011MNRAS.415.2607D},
  Archiveprefix            = {arXiv},
  Doi                      = {10.1111/j.1365-2966.2011.18884.x},
  Eprint                   = {1101.0816},
  Keywords                 = {galaxies: general, galaxies: haloes, galaxies: kinematics and dynamics, cosmology: theory, dark matter}
}

@ARTICLE{Debattista2016,
   author = {{Debattista}, V.~P. and {Ness}, M. and {Gonzalez}, O.~A. and 
	{Freeman}, K. and {Zoccali}, M. and {Minniti}, D.},
    title = "{Separation of Stellar Populations by an Evolving Bar: Implications for the Bulge of the Milky Way}",
  journal = {ArXiv e-prints},
archivePrefix = "arXiv",
   eprint = {1611.09023},
 keywords = {Astrophysics - Astrophysics of Galaxies},
     year = 2016,
    month = nov,
   adsurl = {http://adsabs.harvard.edu/abs/2016arXiv161109023D},
  adsnote = {Provided by the SAO/NASA Astrophysics Data System}
}

@ARTICLE{Dehnen2012,
   author = {{Dehnen}, W. and {Aly}, H.},
    title = "{Improving convergence in smoothed particle hydrodynamics simulations without pairing instability}",
  journal = {\mnras},
archivePrefix = "arXiv",
   eprint = {1204.2471},
 primaryClass = "astro-ph.IM",
 keywords = {hydrodynamics, methods: numerical },
     year = 2012,
    month = sep,
   volume = 425,
    pages = {1068-1082},
      doi = {10.1111/j.1365-2966.2012.21439.x},
   adsurl = {http://adsabs.harvard.edu/abs/2012MNRAS.425.1068D},
  adsnote = {Provided by the SAO/NASA Astrophysics Data System}
}

@Article{Dekel2009,
  Title                    = {{Cold streams in early massive hot haloes as the main mode of galaxy formation}},
  Author                   = {{Dekel}, A. and {Birnboim}, Y. and {Engel}, G. and {Freundlich}, J. and {Goerdt}, T. and {Mumcuoglu}, M. and {Neistein}, E. and {Pichon}, C. and {Teyssier}, R. and {Zinger}, E.},
  Journal                  = {\nat},
  Year                     = {2009},

  Month                    = jan,
  Pages                    = {451-454},
  Volume                   = {457},

  Adsnote                  = {Provided by the SAO/NASA Astrophysics Data System},
  Adsurl                   = {http://adsabs.harvard.edu/abs/2009Natur.457..451D},
  Archiveprefix            = {arXiv},
  Doi                      = {10.1038/nature07648},
  Eprint                   = {0808.0553}
}

@ARTICLE{Dekel2009a,
   author = {{Dekel}, A. and {Sari}, R. and {Ceverino}, D.},
    title = "{Formation of Massive Galaxies at High Redshift: Cold Streams, Clumpy Disks, and Compact Spheroids}",
  journal = {\apj},
archivePrefix = "arXiv",
   eprint = {0901.2458},
 primaryClass = "astro-ph.GA",
 keywords = {galaxies: elliptical and lenticular, cD, galaxies: evolution, galaxies: formation, galaxies: halos, galaxies: spiral},
     year = 2009,
    month = sep,
   volume = 703,
    pages = {785-801},
      doi = {10.1088/0004-637X/703/1/785},
   adsurl = {http://adsabs.harvard.edu/abs/2009ApJ...703..785D},
  adsnote = {Provided by the SAO/NASA Astrophysics Data System}
}

@Article{Dekel1986,
  Title                    = {{The origin of dwarf galaxies, cold dark matter, and biased galaxy formation}},
  Author                   = {{Dekel}, A. and {Silk}, J.},
  Journal                  = {\apj},
  Year                     = {1986},

  Month                    = apr,
  Pages                    = {39-55},
  Volume                   = {303},

  Adsnote                  = {Provided by the SAO/NASA Astrophysics Data System},
  Adsurl                   = {http://adsabs.harvard.edu/abs/1986ApJ...303...39D},
  Doi                      = {10.1086/164050},
  Keywords                 = {ABUNDANCE, COLD PLASMAS, DARK MATTER, DWARF GALAXIES, GALACTIC EVOLUTION, STELLAR WINDS, VIRGO GALACTIC CLUSTER, BIG BANG COSMOLOGY, GALACTIC STRUCTURE, LOCAL GROUP (ASTRONOMY), MASS TO LIGHT RATIOS, STELLAR EVOLUTION, SUPERNOVAE}
}

@Article{DiCintio2014,
  Title                    = {{The dependence of dark matter profiles on the stellar-to-halo mass ratio: a prediction for cusps versus cores}},
  Author                   = {{Di Cintio}, A. and {Brook}, C.~B. and {Macci{\`o}}, A.~V. and {Stinson}, G.~S. and {Knebe}, A. and {Dutton}, A.~A. and {Wadsley}, J. },
  Journal                  = {\mnras},
  Year                     = {2014},

  Month                    = jan,
  Pages                    = {415-423},
  Volume                   = {437},

  Adsnote                  = {Provided by the SAO/NASA Astrophysics Data System},
  Adsurl                   = {http://adsabs.harvard.edu/abs/2014MNRAS.437..415D},
  Archiveprefix            = {arXiv},
  Doi                      = {10.1093/mnras/stt1891},
  Eprint                   = {1306.0898},
  Keywords                 = {hydrodynamics, galaxies: evolution, galaxies: formation, dark matter},
  Primaryclass             = {astro-ph.CO}
}

@Article{DiCintio2012,
  Title                    = {{Size matters: the non-universal density profile of subhaloes in SPH simulations and implications for the Milky Way's dSphs}},
  Author                   = {{Di Cintio}, A. and {Knebe}, A. and {Libeskind}, N.~I. and {Brook}, C. and {Yepes}, G. and {Gottloeber}, S. and {Hoffman}, Y.},
  Journal                  = {ArXiv e-prints},
  Year                     = {2012},

  Month                    = apr,

  Abstract                 = {{We use dark matter only and full hydrodynamical Constrained Local UniversE Simulations (CLUES) of the formation of the Local Group to study the density profile of subhaloes of the simulated Milky Way and Andromeda galaxies. We show that the Einasto model provides the best description of the subhaloes' density profile, as opposed to the more commonly used NFW profile or any generalisation of it. We further find that the Einasto shape parameter $\backslash$nEin$\backslash$ is strongly correlated with the total subhalo mass, pointing towards the notion of a non-universality of the subhaloes' density profile. Assuming now that the dSphs of our Galaxy thus follow the Einasto profile and using the maximum and minimum values of $\backslash$nEin$\backslash$ from our SPH simulations as a gauge, we can improve the observational constraints on the $\backslash$Rmax-$\backslash$Vmax$\backslash$ pairs obtained for the brightest satellite galaxies of the Milky Way. When considering only the subhaloes with \$-13.2$\backslash$lesssim M\_V$\backslash$lesssim-8.8\$, i.e. the range of luminosity of the classical dwarfs, we find that all our simulated objects are consistent with the observed dSphs if their haloes follow the Einasto model with \$1.6$\backslash$lesssim n\_$\{$$\backslash$rm E$\}$ $\backslash$lesssim5.3\$. The numerically motivated Einasto profile for the observed dSphs as well as the observationally motivated magnitude cut for the simulated subhaloes will eliminate the ''massive failures'' problem and results in a perfect agreement with observations. }},
  Adsnote                  = {Provided by the SAO/NASA Astrophysics Data System},
  Adsurl                   = {http://adsabs.harvard.edu/abs/2012arXiv1204.0515D},
  Archiveprefix            = {arXiv},
  Eprint                   = {1204.0515},
  Keywords                 = {Astrophysics - Cosmology and Extragalactic Astrophysics, Astrophysics - Galaxy Astrophysics},
  Primaryclass             = {astro-ph.CO}
}

@Article{Diemand2008,
  Title                    = {{Clumps and streams in the local dark matter distribution}},
  Author                   = {{Diemand}, J. and {Kuhlen}, M. and {Madau}, P. and {Zemp}, M. and {Moore}, B. and {Potter}, D. and {Stadel}, J.},
  Journal                  = {\nat},
  Year                     = {2008},

  Month                    = aug,
  Pages                    = {735-738},
  Volume                   = {454},

  Adsnote                  = {Provided by the SAO/NASA Astrophysics Data System},
  Adsurl                   = {http://adsabs.harvard.edu/abs/2008Natur.454..735D},
  Archiveprefix            = {arXiv},
  Doi                      = {10.1038/nature07153},
  Eprint                   = {0805.1244}
}

@Article{Diemand2005,
  Title                    = {{Cusps in cold dark matter haloes}},
  Author                   = {{Diemand}, J. and {Zemp}, M. and {Moore}, B. and {Stadel}, J. and {Carollo}, C.~M.},
  Journal                  = {\mnras},
  Year                     = {2005},

  Month                    = dec,
  Pages                    = {665-673},
  Volume                   = {364},

  Adsnote                  = {Provided by the SAO/NASA Astrophysics Data System},
  Adsurl                   = {http://adsabs.harvard.edu/abs/2005MNRAS.364..665D},
  Doi                      = {10.1111/j.1365-2966.2005.09601.x},
  Eprint                   = {arXiv:astro-ph/0504215},
  Keywords                 = {methods: N-body simulations, methods: numerical, galaxies: clusters: general, galaxies: haloes, dark matter}
}

@ARTICLE{Dominguez2014,
   author = {{Dom{\'{\i}}nguez-Tenreiro}, R. and {Obreja}, A. and {Granato}, G.~L. and 
	{Schurer}, A. and {Alpresa}, P. and {Silva}, L. and {Brook}, C.~B. and 
	{Serna}, A.},
    title = "{GRASIL-3D: an implementation of dust effects in the SEDs of simulated galaxies}",
  journal = {\mnras},
archivePrefix = "arXiv",
   eprint = {1312.0856},
 keywords = {hydrodynamics, radiative transfer, methods: numerical, dust, extinction, galaxies: spiral, infrared: galaxies},
     year = 2014,
    month = apr,
   volume = 439,
    pages = {3868-3889},
      doi = {10.1093/mnras/stu240},
   adsurl = {http://adsabs.harvard.edu/abs/2014MNRAS.439.3868D},
  adsnote = {Provided by the SAO/NASA Astrophysics Data System}
}

@Article{Dubinski1996,
  Title                    = {A parallel tree code},
  Author                   = {Dubinski, John},
  Journal                  = {New Astronomy},
  Year                     = {1996},
  Number                   = {2},
  Pages                    = {133--147},
  Volume                   = {1},

  Publisher                = {Elsevier}
}

@ARTICLE{Dutton2010,
   author = {{Dutton}, A.~A. and {van den Bosch}, F.~C. and {Dekel}, A.},
    title = "{On the origin of the galaxy star-formation-rate sequence: evolution and scatter}",
  journal = {\mnras},
archivePrefix = "arXiv",
   eprint = {0912.2169},
 keywords = {galaxies: evolution, galaxies: formation, galaxies: fundamental parameters, galaxies: haloes, galaxies: high-redshift, galaxies: spiral},
     year = 2010,
    month = jul,
   volume = 405,
    pages = {1690-1710},
      doi = {10.1111/j.1365-2966.2010.16620.x},
   adsurl = {http://adsabs.harvard.edu/abs/2010MNRAS.405.1690D},
  adsnote = {Provided by the SAO/NASA Astrophysics Data System}
}

@ARTICLE{Dutton2011,
   author = {{Dutton}, A.~A. and {Conroy}, C. and {van den Bosch}, F.~C. and 
	{Simard}, L. and {Mendel}, J.~T. and {Courteau}, S. and {Dekel}, A. and 
	{More}, S. and {Prada}, F.},
    title = "{Dark halo response and the stellar initial mass function in early-type and late-type galaxies}",
  journal = {\mnras},
archivePrefix = "arXiv",
   eprint = {1012.5859},
 keywords = {galaxies: elliptical and lenticular, cD, galaxies: fundamental parameters, galaxies: haloes, galaxies: spiral, galaxies: structure, dark matter},
     year = 2011,
    month = sep,
   volume = 416,
    pages = {322-345},
      doi = {10.1111/j.1365-2966.2011.19038.x},
   adsurl = {http://adsabs.harvard.edu/abs/2011MNRAS.416..322D},
  adsnote = {Provided by the SAO/NASA Astrophysics Data System}
}

@Article{Dutton2014,
  Title                    = {{Cold dark matter haloes in the Planck era: evolution of structural parameters for Einasto and NFW profiles}},
  Author                   = {{Dutton}, A.~A. and {Macci{\`o}}, A.~V.},
  Journal                  = {\mnras},
  Year                     = {2014},

  Month                    = jul,
  Pages                    = {3359-3374},
  Volume                   = {441},

  Adsnote                  = {Provided by the SAO/NASA Astrophysics Data System},
  Adsurl                   = {http://adsabs.harvard.edu/abs/2014MNRAS.441.3359D},
  Archiveprefix            = {arXiv},
  Doi                      = {10.1093/mnras/stu742},
  Eprint                   = {1402.7073},
  Keywords                 = {methods: numerical, galaxies: haloes, cosmology: theory, dark matter}
}

@ARTICLE{Dutton2013,
   author = {{Dutton}, A.~A. and {Treu}, T. and {Brewer}, B.~J. and {Marshall}, P.~J. and 
	{Auger}, M.~W. and {Barnab{\`e}}, M. and {Koo}, D.~C. and {Bolton}, A.~S. and 
	{Koopmans}, L.~V.~E.},
    title = "{The SWELLS survey - V. A Salpeter stellar initial mass function in the bulges of massive spiral galaxies}",
  journal = {\mnras},
archivePrefix = "arXiv",
   eprint = {1206.4310},
 keywords = {stars: luminosity function, mass function, galaxies: bulges, galaxies: kinematics and dynamics, galaxies: spiral, dark matter},
     year = 2013,
    month = feb,
   volume = 428,
    pages = {3183-3195},
      doi = {10.1093/mnras/sts262},
   adsurl = {http://adsabs.harvard.edu/abs/2013MNRAS.428.3183D},
  adsnote = {Provided by the SAO/NASA Astrophysics Data System}
}

@ARTICLE{Dutton2013b,
   author = {{Dutton}, A.~A. and {Macci{\`o}}, A.~V. and {Mendel}, J.~T. and 
	{Simard}, L.},
    title = "{Universal IMF versus dark halo response in early-type galaxies: breaking the degeneracy with the Fundamental Plane}",
  journal = {\mnras},
archivePrefix = "arXiv",
   eprint = {1204.2825},
 keywords = {stars: luminosity function, mass function, galaxies: elliptical and lenticular, cD, galaxies: fundamental parameters, galaxies: haloes, galaxies: kinematics and dynamics, dark matter},
     year = 2013,
    month = jul,
   volume = 432,
    pages = {2496-2511},
      doi = {10.1093/mnras/stt608},
   adsurl = {http://adsabs.harvard.edu/abs/2013MNRAS.432.2496D},
  adsnote = {Provided by the SAO/NASA Astrophysics Data System}
}

@ARTICLE{Dutton2016,
   author = {{Dutton}, A.~A. and {Macci{\`o}}, A.~V. and {Frings}, J. and 
	{Wang}, L. and {Stinson}, G.~S. and {Penzo}, C. and {Kang}, X.
	},
    title = "{NIHAO V: too big does not fail - reconciling the conflict between {$\Lambda$}CDM predictions and the circular velocities of nearby field galaxies}",
  journal = {\mnras},
archivePrefix = "arXiv",
   eprint = {1512.00453},
 keywords = {galaxies: dwarf, galaxies: haloes, galaxies: kinematics and dynamics, Local Group, cosmology: theory, dark matter},
     year = 2016,
    month = mar,
   volume = 457,
    pages = {L74-L78},
      doi = {10.1093/mnrasl/slv193},
   adsurl = {http://adsabs.harvard.edu/abs/2016MNRAS.457L..74D},
  adsnote = {Provided by the SAO/NASA Astrophysics Data System}
}

@ARTICLE{Dutton2016b,
   author = {{Dutton}, A.~A. and {Obreja}, A. and {Wang}, L. and {Gutcke}, T.~A. and 
	{Buck}, T. and {Udrescu}, S.~M. and {Frings}, J. and {Stinson}, G.~S. and 
	{Kang}, X. and {Macci{\`o}}, A.~V.},
    title = "{NIHAO XII: galactic uniformity in a {$\Lambda$}CDM universe}",
  journal = {\mnras},
archivePrefix = "arXiv",
   eprint = {1610.06375},
 keywords = {methods: numerical, galaxies: fundamental parameters, galaxies: haloes, galaxies: kinematics and dynamics, dark matter},
     year = 2017,
    month = jun,
   volume = 467,
    pages = {4937-4950},
      doi = {10.1093/mnras/stx458},
   adsurl = {http://adsabs.harvard.edu/abs/2017MNRAS.467.4937D},
  adsnote = {Provided by the SAO/NASA Astrophysics Data System}
}


@ARTICLE{Badry2016,
   author = {{El-Badry}, K. and {Wetzel}, A. and {Geha}, M. and {Hopkins}, P.~F. and 
	{Kere{\v s}}, D. and {Chan}, T.~K. and {Faucher-Gigu{\`e}re}, C.-A.
	},
    title = "{Breathing FIRE: How Stellar Feedback Drives Radial Migration, Rapid Size Fluctuations, and Population Gradients in Low-mass Galaxies}",
  journal = {\apj},
archivePrefix = "arXiv",
   eprint = {1512.01235},
 keywords = {galaxies: dwarf, galaxies: evolution, galaxies: kinematics and dynamics, galaxies: star formation},
     year = 2016,
    month = apr,
   volume = 820,
      eid = {131},
    pages = {131},
      doi = {10.3847/0004-637X/820/2/131},
   adsurl = {http://adsabs.harvard.edu/abs/2016ApJ...820..131E},
  adsnote = {Provided by the SAO/NASA Astrophysics Data System}
}

@ARTICLE{Elmegreen2005,
   author = {{Elmegreen}, B.~G. and {Elmegreen}, D.~M.},
    title = "{Stellar Populations in 10 Clump-Cluster Galaxies of the Hubble Ultra Deep Field}",
  journal = {\apj},
   eprint = {astro-ph/0504032},
 keywords = {Galaxies: Evolution, Galaxies: Formation, Galaxies: High-Redshift, Galaxies: Irregular},
     year = 2005,
    month = jul,
   volume = 627,
    pages = {632-646},
      doi = {10.1086/430514},
   adsurl = {http://adsabs.harvard.edu/abs/2005ApJ...627..632E},
  adsnote = {Provided by the SAO/NASA Astrophysics Data System}
}

@ARTICLE{Elmegreen2007,
   author = {{Elmegreen}, D.~M. and {Elmegreen}, B.~G. and {Ravindranath}, S. and 
	{Coe}, D.~A.},
    title = "{Resolved Galaxies in the Hubble Ultra Deep Field: Star Formation in Disks at High Redshift}",
  journal = {\apj},
   eprint = {astro-ph/0701121},
 keywords = {Galaxies: Evolution, Galaxies: Formation, Galaxies: High-Redshift},
     year = 2007,
    month = apr,
   volume = 658,
    pages = {763-777},
      doi = {10.1086/511667},
   adsurl = {http://adsabs.harvard.edu/abs/2007ApJ...658..763E},
  adsnote = {Provided by the SAO/NASA Astrophysics Data System}
}

@ARTICLE{Elmegreen2009,
   author = {{Elmegreen}, B.~G. and {Elmegreen}, D.~M. and {Fernandez}, M.~X. and 
	{Lemonias}, J.~J.},
    title = "{Bulge and Clump Evolution in Hubble Ultra Deep Field Clump Clusters, Chains and Spiral Galaxies}",
  journal = {\apj},
archivePrefix = "arXiv",
   eprint = {0810.5404},
 keywords = {galaxies: bulges, galaxies: evolution, galaxies: formation, galaxies: high-redshift},
     year = 2009,
    month = feb,
   volume = 692,
    pages = {12-31},
      doi = {10.1088/0004-637X/692/1/12},
   adsurl = {http://adsabs.harvard.edu/abs/2009ApJ...692...12E},
  adsnote = {Provided by the SAO/NASA Astrophysics Data System}
}

@ARTICLE{Ferland1998,
   author = {{Ferland}, G.~J. and {Korista}, K.~T. and {Verner}, D.~A. and 
	{Ferguson}, J.~W. and {Kingdon}, J.~B. and {Verner}, E.~M.},
    title = "{CLOUDY 90: Numerical Simulation of Plasmas and Their Spectra}",
  journal = {\pasp},
     year = 1998,
    month = jul,
   volume = 110,
    pages = {761-778},
      doi = {10.1086/316190},
   adsurl = {http://adsabs.harvard.edu/abs/1998PASP..110..761F},
  adsnote = {Provided by the SAO/NASA Astrophysics Data System}
}

@Article{Flores1994,
  Title                    = {{Observational and theoretical constraints on singular dark matter halos}},
  Author                   = {{Flores}, R.~A. and {Primack}, J.~R.},
  Journal                  = {\apjl},
  Year                     = {1994},

  Month                    = may,
  Pages                    = {L1-L4},
  Volume                   = {427},

  Adsnote                  = {Provided by the SAO/NASA Astrophysics Data System},
  Adsurl                   = {http://adsabs.harvard.edu/abs/1994ApJ...427L...1F},
  Doi                      = {10.1086/187350},
  Eprint                   = {astro-ph/9402004},
  Keywords                 = {Astronomical Models, Astrophysics, Constraints, Dark Matter, Density Distribution, Galaxies, Halos, Star Clusters, Star Formation, Curve Fitting, Distortion, Gravitational Lenses, Isothermal Processes, Rotation, Stellar Cores}
}

@ARTICLE{Forster2006,
   author = {{F{\"o}rster Schreiber}, N.~M. and {Genzel}, R. and {Lehnert}, M.~D. and 
	{Bouch{\'e}}, N. and {Verma}, A. and {Erb}, D.~K. and {Shapley}, A.~E. and 
	{Steidel}, C.~C. and {Davies}, R. and {Lutz}, D. and {Nesvadba}, N. and 
	{Tacconi}, L.~J. and {Eisenhauer}, F. and {Abuter}, R. and {Gilbert}, A. and 
	{Gillessen}, S. and {Sternberg}, A.},
    title = "{SINFONI Integral Field Spectroscopy of z \~{} 2 UV-selected Galaxies: Rotation Curves and Dynamical Evolution}",
  journal = {\apj},
   eprint = {astro-ph/0603559},
 keywords = {Galaxies: Evolution, Galaxies: High-Redshift, Galaxies: Kinematics and Dynamics, Infrared: Galaxies},
     year = 2006,
    month = jul,
   volume = 645,
    pages = {1062-1075},
      doi = {10.1086/504403},
   adsurl = {http://adsabs.harvard.edu/abs/2006ApJ...645.1062F},
  adsnote = {Provided by the SAO/NASA Astrophysics Data System}
}

@ARTICLE{Forster2011,
   author = {{F{\"o}rster Schreiber}, N.~M. and {Shapley}, A.~E. and {Genzel}, R. and 
	{Bouch{\'e}}, N. and {Cresci}, G. and {Davies}, R. and {Erb}, D.~K. and 
	{Genel}, S. and {Lutz}, D. and {Newman}, S. and {Shapiro}, K.~L. and 
	{Steidel}, C.~C. and {Sternberg}, A. and {Tacconi}, L.~J.},
    title = "{Constraints on the Assembly and Dynamics of Galaxies. II. Properties of Kiloparsec-scale Clumps in Rest-frame Optical Emission of z \~{} 2 Star-forming Galaxies}",
  journal = {\apj},
archivePrefix = "arXiv",
   eprint = {1104.0248},
 keywords = {galaxies: evolution, galaxies: high-redshift, galaxies: structure, infrared: galaxies},
     year = 2011,
    month = sep,
   volume = 739,
      eid = {45},
    pages = {45},
      doi = {10.1088/0004-637X/739/1/45},
   adsurl = {http://adsabs.harvard.edu/abs/2011ApJ...739...45F},
  adsnote = {Provided by the SAO/NASA Astrophysics Data System}
}

@ARTICLE{Genel2013,
       author = {{Genel}, Shy and {Vogelsberger}, Mark and {Nelson}, Dylan and
         {Sijacki}, Debora and {Springel}, Volker and {Hernquist}, Lars},
        title = "{Following the flow: tracer particles in astrophysical fluid simulations}",
      journal = {\mnras},
     keywords = {hydrodynamics, turbulence, methods: numerical, methods: statistical, galaxies: formation, cosmology: theory, Astrophysics - Instrumentation and Methods for Astrophysics, Astrophysics - Cosmology and Extragalactic Astrophysics, Physics - Fluid Dynamics},
         year = "2013",
        month = "Oct",
       volume = {435},
       number = {2},
        pages = {1426-1442},
          doi = {10.1093/mnras/stt1383},
archivePrefix = {arXiv},
       eprint = {1305.2195},
 primaryClass = {astro-ph.IM},
       adsurl = {https://ui.adsabs.harvard.edu/abs/2013MNRAS.435.1426G},
      adsnote = {Provided by the SAO/NASA Astrophysics Data System}
}

@ARTICLE{Genel2012,
   author = {{Genel}, S. and {Naab}, T. and {Genzel}, R. and {F{\"o}rster Schreiber}, N.~M. and 
	{Sternberg}, A. and {Oser}, L. and {Johansson}, P.~H. and {Dav{\'e}}, R. and 
	{Oppenheimer}, B.~D. and {Burkert}, A.},
    title = "{Short-lived Star-forming Giant Clumps in Cosmological Simulations of z {\ap} 2 Disks}",
  journal = {\apj},
archivePrefix = "arXiv",
   eprint = {1011.0433},
 primaryClass = "astro-ph.CO",
 keywords = {galaxies: evolution, galaxies: formation, galaxies: high-redshift, galaxies: structure},
     year = 2012,
    month = jan,
   volume = 745,
      eid = {11},
    pages = {11},
      doi = {10.1088/0004-637X/745/1/11},
   adsurl = {http://adsabs.harvard.edu/abs/2012ApJ...745...11G},
  adsnote = {Provided by the SAO/NASA Astrophysics Data System}
}

@ARTICLE{Genel2012a,
   author = {{Genel}, S. and {Dekel}, A. and {Cacciato}, M.},
    title = "{On the effect of cosmological inflow on turbulence and instability in galactic discs}",
  journal = {\mnras},
archivePrefix = "arXiv",
   eprint = {1203.0810},
 keywords = {methods: analytical, galaxies: evolution, galaxies: formation, galaxies: high-redshift, galaxies: kinematics and dynamics, galaxies: star formation },
     year = 2012,
    month = sep,
   volume = 425,
    pages = {788-800},
      doi = {10.1111/j.1365-2966.2012.21652.x},
   adsurl = {http://adsabs.harvard.edu/abs/2012MNRAS.425..788G},
  adsnote = {Provided by the SAO/NASA Astrophysics Data System}
}

@ARTICLE{Genzel2006,
   author = {{Genzel}, R. and {Tacconi}, L.~J. and {Eisenhauer}, F. and {F{\"o}rster Schreiber}, N.~M. and 
	{Cimatti}, A. and {Daddi}, E. and {Bouch{\'e}}, N. and {Davies}, R. and 
	{Lehnert}, M.~D. and {Lutz}, D. and {Nesvadba}, N. and {Verma}, A. and 
	{Abuter}, R. and {Shapiro}, K. and {Sternberg}, A. and {Renzini}, A. and 
	{Kong}, X. and {Arimoto}, N. and {Mignoli}, M.},
    title = "{The rapid formation of a large rotating disk galaxy three billion years after the Big Bang}",
  journal = {\nat},
   eprint = {astro-ph/0608344},
     year = 2006,
    month = aug,
   volume = 442,
    pages = {786-789},
      doi = {10.1038/nature05052},
   adsurl = {http://adsabs.harvard.edu/abs/2006Natur.442..786G},
  adsnote = {Provided by the SAO/NASA Astrophysics Data System}
}

@ARTICLE{Genzel2008,
   author = {{Genzel}, R. and {Burkert}, A. and {Bouch{\'e}}, N. and {Cresci}, G. and 
	{F{\"o}rster Schreiber}, N.~M. and {Shapley}, A. and {Shapiro}, K. and 
	{Tacconi}, L.~J. and {Buschkamp}, P. and {Cimatti}, A. and {Daddi}, E. and 
	{Davies}, R. and {Eisenhauer}, F. and {Erb}, D.~K. and {Genel}, S. and 
	{Gerhard}, O. and {Hicks}, E. and {Lutz}, D. and {Naab}, T. and 
	{Ott}, T. and {Rabien}, S. and {Renzini}, A. and {Steidel}, C.~C. and 
	{Sternberg}, A. and {Lilly}, S.~J.},
    title = "{From Rings to Bulges: Evidence for Rapid Secular Galaxy Evolution at z \~{} 2 from Integral Field Spectroscopy in the SINS Survey}",
  journal = {\apj},
archivePrefix = "arXiv",
   eprint = {0807.1184},
 keywords = {cosmology: observations, galaxies: evolution, galaxies: high-redshift, infrared: galaxies },
     year = 2008,
    month = nov,
   volume = 687,
      eid = {59-77},
    pages = {59-77},
      doi = {10.1086/591840},
   adsurl = {http://adsabs.harvard.edu/abs/2008ApJ...687...59G},
  adsnote = {Provided by the SAO/NASA Astrophysics Data System}
}

@ARTICLE{Genzel2011,
   author = {{Genzel}, R. and {Newman}, S. and {Jones}, T. and {F{\"o}rster Schreiber}, N.~M. and 
	{Shapiro}, K. and {Genel}, S. and {Lilly}, S.~J. and {Renzini}, A. and 
	{Tacconi}, L.~J. and {Bouch{\'e}}, N. and {Burkert}, A. and 
	{Cresci}, G. and {Buschkamp}, P. and {Carollo}, C.~M. and {Ceverino}, D. and 
	{Davies}, R. and {Dekel}, A. and {Eisenhauer}, F. and {Hicks}, E. and 
	{Kurk}, J. and {Lutz}, D. and {Mancini}, C. and {Naab}, T. and 
	{Peng}, Y. and {Sternberg}, A. and {Vergani}, D. and {Zamorani}, G.
	},
    title = "{The Sins Survey of z \~{} 2 Galaxy Kinematics: Properties of the Giant Star-forming Clumps}",
  journal = {\apj},
archivePrefix = "arXiv",
   eprint = {1011.5360},
 keywords = {cosmology: observations, galaxies: evolution, galaxies: high-redshift, infrared: galaxies},
     year = 2011,
    month = jun,
   volume = 733,
      eid = {101},
    pages = {101},
      doi = {10.1088/0004-637X/733/2/101},
   adsurl = {http://adsabs.harvard.edu/abs/2011ApJ...733..101G},
  adsnote = {Provided by the SAO/NASA Astrophysics Data System}
}

@ARTICLE{Genzel2015,
   author = {{Genzel}, R. and {Tacconi}, L.~J. and {Lutz}, D. and {Saintonge}, A. and 
	{Berta}, S. and {Magnelli}, B. and {Combes}, F. and {Garc{\'{\i}}a-Burillo}, S. and 
	{Neri}, R. and {Bolatto}, A. and {Contini}, T. and {Lilly}, S. and 
	{Boissier}, J. and {Boone}, F. and {Bouch{\'e}}, N. and {Bournaud}, F. and 
	{Burkert}, A. and {Carollo}, M. and {Colina}, L. and {Cooper}, M.~C. and 
	{Cox}, P. and {Feruglio}, C. and {F{\"o}rster Schreiber}, N.~M. and 
	{Freundlich}, J. and {Gracia-Carpio}, J. and {Juneau}, S. and 
	{Kovac}, K. and {Lippa}, M. and {Naab}, T. and {Salome}, P. and 
	{Renzini}, A. and {Sternberg}, A. and {Walter}, F. and {Weiner}, B. and 
	{Weiss}, A. and {Wuyts}, S.},
    title = "{Combined CO and Dust Scaling Relations of Depletion Time and Molecular Gas Fractions with Cosmic Time, Specific Star-formation Rate, and Stellar Mass}",
  journal = {\apj},
archivePrefix = "arXiv",
   eprint = {1409.1171},
 keywords = {galaxies: evolution, galaxies: high-redshift, galaxies: kinematics and dynamics, infrared: galaxies},
     year = 2015,
    month = feb,
   volume = 800,
      eid = {20},
    pages = {20},
      doi = {10.1088/0004-637X/800/1/20},
   adsurl = {http://adsabs.harvard.edu/abs/2015ApJ...800...20G},
  adsnote = {Provided by the SAO/NASA Astrophysics Data System}
}

@ARTICLE{Gill2004,
   author = {{Gill}, S.~P.~D. and {Knebe}, A. and {Gibson}, B.~K.},
    title = "{The evolution of substructure - I. A new identification method}",
  journal = {\mnras},
   eprint = {astro-ph/0404258},
 keywords = {methods: N-body simulations, methods: numerical, galaxies: formation, galaxies: haloes},
     year = 2004,
    month = jun,
   volume = 351,
    pages = {399-409},
      doi = {10.1111/j.1365-2966.2004.07786.x},
   adsurl = {http://adsabs.harvard.edu/abs/2004MNRAS.351..399G},
  adsnote = {Provided by the SAO/NASA Astrophysics Data System}
}

@Article{Gillet2015,
  Title                    = {{Vast Planes of Satellites in a High-resolution Simulation of the Local Group: Comparison to Andromeda}},
  Author                   = {{Gillet}, N. and {Ocvirk}, P. and {Aubert}, D. and {Knebe}, A. and {Libeskind}, N. and {Yepes}, G. and {Gottl{\"o}ber}, S. and {Hoffman}, Y.},
  Journal                  = {\apj},
  Year                     = {2015},

  Month                    = feb,
  Pages                    = {34},
  Volume                   = {800},

  Adsnote                  = {Provided by the SAO/NASA Astrophysics Data System},
  Adsurl                   = {http://adsabs.harvard.edu/abs/2015ApJ...800...34G},
  Archiveprefix            = {arXiv},
  Doi                      = {10.1088/0004-637X/800/1/34},
  Eid                      = {34},
  Eprint                   = {1412.3110},
  Keywords                 = {galaxies: dwarf, galaxies: kinematics and dynamics, Local Group }
}

@ARTICLE{Girardi2010,
   author = {{Girardi}, L. and {Williams}, B.~F. and {Gilbert}, K.~M. and 
	{Rosenfield}, P. and {Dalcanton}, J.~J. and {Marigo}, P. and 
	{Boyer}, M.~L. and {Dolphin}, A. and {Weisz}, D.~R. and {Melbourne}, J. and 
	{Olsen}, K.~A.~G. and {Seth}, A.~C. and {Skillman}, E.},
    title = "{The ACS Nearby Galaxy Survey Treasury. IX. Constraining Asymptotic Giant Branch Evolution with Old Metal-poor Galaxies}",
  journal = {\apj},
archivePrefix = "arXiv",
   eprint = {1009.4618},
 primaryClass = "astro-ph.SR",
 keywords = {stars: general},
     year = 2010,
    month = dec,
   volume = 724,
    pages = {1030-1043},
      doi = {10.1088/0004-637X/724/2/1030},
   adsurl = {http://adsabs.harvard.edu/abs/2010ApJ...724.1030G},
  adsnote = {Provided by the SAO/NASA Astrophysics Data System}
}

@ARTICLE{Gonzalez2017,
   author = {{Gonzalez}, O.~A. and {Debattista}, V.~P. and {Ness}, M. and 
	{Erwin}, P. and {Gadotti}, D.~A.},
    title = "{Peanut-shaped metallicity distributions in bulges of edge-on galaxies: the case of NGC 4710}",
  journal = {\mnras},
archivePrefix = "arXiv",
   eprint = {1611.09547},
 keywords = {galaxies: bulges, galaxies: evolution, galaxies: formation, galaxies: stellar content, galaxies: structure},
     year = 2017,
    month = mar,
   volume = 466,
    pages = {L93-L97},
      doi = {10.1093/mnrasl/slw232},
   adsurl = {http://adsabs.harvard.edu/abs/2017MNRAS.466L..93G},
  adsnote = {Provided by the SAO/NASA Astrophysics Data System}
}

@Article{Governato2010,
  Title                    = {{Bulgeless dwarf galaxies and dark matter cores from supernova-driven outflows}},
  Author                   = {{Governato}, F. and {Brook}, C. and {Mayer}, L. and {Brooks}, A. and {Rhee}, G. and {Wadsley}, J. and {Jonsson}, P. and {Willman}, B. and {Stinson}, G. and {Quinn}, T. and {Madau}, P.},
  Journal                  = {\nat},
  Year                     = {2010},

  Month                    = jan,
  Pages                    = {203-206},
  Volume                   = {463},

  Adsnote                  = {Provided by the SAO/NASA Astrophysics Data System},
  Adsurl                   = {http://adsabs.harvard.edu/abs/2010Natur.463..203G},
  Archiveprefix            = {arXiv},
  Doi                      = {10.1038/nature08640},
  Eprint                   = {0911.2237},
  Primaryclass             = {astro-ph.CO}
}

@Article{Governato2012,
  Title                    = {{Cuspy no more: how outflows affect the central dark matter and baryon distribution in {$\Lambda$} cold dark matter galaxies}},
  Author                   = {{Governato}, F. and {Zolotov}, A. and {Pontzen}, A. and {Christensen}, C. and {Oh}, S.~H. and {Brooks}, A.~M. and {Quinn}, T. and {Shen}, S. and {Wadsley}, J.},
  Journal                  = {\mnras},
  Year                     = {2012},

  Month                    = may,
  Pages                    = {1231-1240},
  Volume                   = {422},

  Adsnote                  = {Provided by the SAO/NASA Astrophysics Data System},
  Adsurl                   = {http://adsabs.harvard.edu/abs/2012MNRAS.422.1231G},
  Archiveprefix            = {arXiv},
  Doi                      = {10.1111/j.1365-2966.2012.20696.x},
  Eprint                   = {1202.0554},
  Keywords                 = {hydrodynamics, galaxies: evolution, galaxies: formation, galaxies: star formation, dark matter},
  Primaryclass             = {astro-ph.CO}
}

@ARTICLE{Goz2016,
   author = {{Goz}, D. and {Monaco}, P. and {Granato}, G.~L. and {Murante}, G. and 
	{Dom{\'{\i}}nguez-Tenreiro}, R. and {Obreja}, A. and {Annunziatella}, M. and 
	{Tescari}, E.},
    title = "{Panchromatic Spectral Energy Distributions of simulated galaxies: results at redshift $z=0$}",
  journal = {ArXiv e-prints},
archivePrefix = "arXiv",
   eprint = {1610.09843},
 keywords = {Astrophysics - Astrophysics of Galaxies},
     year = 2016,
    month = oct,
   adsurl = {http://adsabs.harvard.edu/abs/2016arXiv161009843G},
  adsnote = {Provided by the SAO/NASA Astrophysics Data System}
}

@ARTICLE{Granato2000,
   author = {{Granato}, G.~L. and {Lacey}, C.~G. and {Silva}, L. and {Bressan}, A. and 
	{Baugh}, C.~M. and {Cole}, S. and {Frenk}, C.~S.},
    title = "{The Infrared Side of Galaxy Formation. I. The Local Universe in the Semianalytical Framework}",
  journal = {\apj},
   eprint = {astro-ph/0001308},
 keywords = {Galaxies: Evolution, Galaxies: Formation, Galaxies: Fundamental Parameters, Galaxies: Starburst, Infrared: Galaxies, Ultraviolet: Galaxies},
     year = 2000,
    month = oct,
   volume = 542,
    pages = {710-730},
      doi = {10.1086/317032},
   adsurl = {http://adsabs.harvard.edu/abs/2000ApJ...542..710G},
  adsnote = {Provided by the SAO/NASA Astrophysics Data System}
}

@ARTICLE{Granato2015,
   author = {{Granato}, G.~L. and {Ragone-Figueroa}, C. and {Dom{\'{\i}}nguez-Tenreiro}, R. and 
	{Obreja}, A. and {Borgani}, S. and {De Lucia}, G. and {Murante}, G.
	},
    title = "{The early phases of galaxy clusters formation in IR: coupling hydrodynamical simulations with GRASIL-3D}",
  journal = {\mnras},
archivePrefix = "arXiv",
   eprint = {1412.6105},
 keywords = {hydrodynamics, radiative transfer, dust, extinction, galaxies: clusters: general, infrared: galaxies, submillimetre: galaxies},
     year = 2015,
    month = jun,
   volume = 450,
    pages = {1320-1332},
      doi = {10.1093/mnras/stv676},
   adsurl = {http://adsabs.harvard.edu/abs/2015MNRAS.450.1320G},
  adsnote = {Provided by the SAO/NASA Astrophysics Data System}
}

@ARTICLE{Guhathakurta1989,
   author = {{Guhathakurta}, P. and {Draine}, B.~T.},
    title = "{Temperature fluctuations in interstellar grains. I - Computational method and sublimation of small grains}",
  journal = {\apj},
 keywords = {Computational Astrophysics, Cosmic Dust, Interstellar Matter, Radiation Distribution, B Stars, Distribution Functions, High Temperature Gases, Iterative Solution, Monte Carlo Method, Radiative Transfer, Temperature Distribution},
     year = 1989,
    month = oct,
   volume = 345,
    pages = {230-244},
      doi = {10.1086/167899},
   adsurl = {http://adsabs.harvard.edu/abs/1989ApJ...345..230G},
  adsnote = {Provided by the SAO/NASA Astrophysics Data System}
}

@Article{Guo2010,
  Title                    = {{How do galaxies populate dark matter haloes?}},
  Author                   = {{Guo}, Q. and {White}, S. and {Li}, C. and {Boylan-Kolchin}, M. },
  Journal                  = {\mnras},
  Year                     = {2010},

  Month                    = may,
  Pages                    = {1111-1120},
  Volume                   = {404},

  Adsnote                  = {Provided by the SAO/NASA Astrophysics Data System},
  Adsurl                   = {http://adsabs.harvard.edu/abs/2010MNRAS.404.1111G},
  Archiveprefix            = {arXiv},
  Doi                      = {10.1111/j.1365-2966.2010.16341.x},
  Eprint                   = {0909.4305},
  Keywords                 = {galaxies: haloes, galaxies: luminosity function, mass function, cosmology: theory, dark matter, large-scale structure of Universe},
  Primaryclass             = {astro-ph.CO}
}

@ARTICLE{Guo2012,
   author = {{Guo}, Y. and {Giavalisco}, M. and {Ferguson}, H.~C. and {Cassata}, P. and 
	{Koekemoer}, A.~M.},
    title = "{Multi-wavelength View of Kiloparsec-scale Clumps in Star-forming Galaxies at z \~{} 2}",
  journal = {\apj},
archivePrefix = "arXiv",
   eprint = {1110.3800},
 keywords = {cosmology: observations, galaxies: active, galaxies: bulges, galaxies: evolution, galaxies: formation, galaxies: high-redshift, galaxies: stellar content, galaxies: structure},
     year = 2012,
    month = oct,
   volume = 757,
      eid = {120},
    pages = {120},
      doi = {10.1088/0004-637X/757/2/120},
   adsurl = {http://adsabs.harvard.edu/abs/2012ApJ...757..120G},
  adsnote = {Provided by the SAO/NASA Astrophysics Data System}
}

@ARTICLE{Guo2015,
   author = {{Guo}, Y. and {Ferguson}, H.~C. and {Bell}, E.~F. and {Koo}, D.~C. and 
	{Conselice}, C.~J. and {Giavalisco}, M. and {Kassin}, S. and 
	{Lu}, Y. and {Lucas}, R. and {Mandelker}, N. and {McIntosh}, D.~M. and 
	{Primack}, J.~R. and {Ravindranath}, S. and {Barro}, G. and 
	{Ceverino}, D. and {Dekel}, A. and {Faber}, S.~M. and {Fang}, J.~J. and 
	{Koekemoer}, A.~M. and {Noeske}, K. and {Rafelski}, M. and {Straughn}, A.
	},
    title = "{Clumpy Galaxies in CANDELS. I. The Definition of UV Clumps and the Fraction of Clumpy Galaxies at 0.5 $\lt$ z $\lt$ 3}",
  journal = {\apj},
archivePrefix = "arXiv",
   eprint = {1410.7398},
 keywords = {galaxies: evolution, galaxies: formation, galaxies: irregular, galaxies: starburst, galaxies: star formation, galaxies: structure},
     year = 2015,
    month = feb,
   volume = 800,
      eid = {39},
    pages = {39},
      doi = {10.1088/0004-637X/800/1/39},
   adsurl = {http://adsabs.harvard.edu/abs/2015ApJ...800...39G},
  adsnote = {Provided by the SAO/NASA Astrophysics Data System}
}


@ARTICLE{Gutcke2016,
   author = {{Gutcke}, T.~A. and {Stinson}, G.~S. and {Macci{\`o}}, A.~V. and 
	{Wang}, L. and {Dutton}, A.~A.},
    title = "{NIHAO VIII: Circum-galactic medium and outflows - The puzzles of HI and OVI gas distributions}",
  journal = {\mnras},
archivePrefix = "arXiv",
   eprint = {1602.06956},
     year = 2016,
    month = oct,
      doi = {10.1093/mnras/stw2539},
   adsurl = {http://esoads.eso.org/abs/2016MNRAS.tmp.1530G},
  adsnote = {Provided by the SAO/NASA Astrophysics Data System}
}


@article{Haardt2005,
	Adsnote = {Provided by the Smithsonian/NASA Astrophysics Data System},
	Adsurl = {http://adsabs.harvard.edu/cgi-bin/nph-bib_query?bibcode=1996ApJ...461...20H&db_key=AST},
	Author = {{Haardt}, F. and {Madau}, P.},
	Date-Added = {2016-02-02 15:08:42 +0000},
	Date-Modified = {2016-02-02 15:08:42 +0000},
	Doi = {10.1086/177035},
	Journal = {unpublished},
	Month = apr,
	Owner = {stinson},
	Timestamp = {2012.07.27},
	Title = {{Radiative Transfer in a Clumpy Universe. II. The Ultraviolet Extragalactic Background}},
	Year = {2005},
	Bdsk-Url-1 = {http://dx.doi.org/10.1086/177035}}

@ARTICLE{Hammer2013,
   author = {{Hammer}, F. and {Yang}, Y. and {Fouquet}, S. and {Pawlowski}, M.~S. and 
	{Kroupa}, P. and {Puech}, M. and {Flores}, H. and {Wang}, J.
	},
    title = "{The vast thin plane of M31 corotating dwarfs: an additional fossil signature of the M31 merger and of its considerable impact in the whole Local Group}",
  journal = {\mnras},
archivePrefix = "arXiv",
   eprint = {1303.1817},
 keywords = {galaxies: dwarf, galaxies: formation, galaxies: interactions, Local Group},
     year = 2013,
    month = jun,
   volume = 431,
    pages = {3543-3549},
      doi = {10.1093/mnras/stt435},
   adsurl = {http://cdsads.u-strasbg.fr/abs/2013MNRAS.431.3543H},
  adsnote = {Provided by the SAO/NASA Astrophysics Data System}
}

@Article{Hernquist1991,
  Title                    = {{Application of the Ewald method to cosmological N-body simulations}},
  Author                   = {{Hernquist}, L. and {Bouchet}, F.~R. and {Suto}, Y.},
  Journal                  = {\apjs},
  Year                     = {1991},

  Month                    = feb,
  Pages                    = {231-240},
  Volume                   = {75},

  Adsnote                  = {Provided by the SAO/NASA Astrophysics Data System},
  Adsurl                   = {http://adsabs.harvard.edu/abs/1991ApJS...75..231H},
  Doi                      = {10.1086/191530},
  Keywords                 = {COMPUTATIONAL ASTROPHYSICS, GALACTIC STRUCTURE, HUBBLE CONSTANT, MANY BODY PROBLEM, ASTRONOMICAL MODELS, BOUNDARY CONDITIONS, SPATIAL RESOLUTION}
}

@MastersThesis{herpich,
  Title                    = {The Effect of Warm Dark Matter on Hydrodynamical Simulations of Disk Galaxy Formation},
  Author                   = {Jakob Herpich},
  School                   = {Department of Physics and Astronomy University of Heidelberg},
  Year                     = {2013},

  Owner                    = {buck-tobias},
  Timestamp                = {2015.05.01}
}

@ARTICLE{Hopkins2010,
   author = {{Hopkins}, P.~F. and {Croton}, D. and {Bundy}, K. and {Khochfar}, S. and 
	{van den Bosch}, F. and {Somerville}, R.~S. and {Wetzel}, A. and 
	{Keres}, D. and {Hernquist}, L. and {Stewart}, K. and {Younger}, J.~D. and 
	{Genel}, S. and {Ma}, C.-P.},
    title = "{Mergers in {$\Lambda$}CDM: Uncertainties in Theoretical Predictions and Interpretations of the Merger Rate}",
  journal = {\apj},
archivePrefix = "arXiv",
   eprint = {1004.2708},
 primaryClass = "astro-ph.CO",
 keywords = {cosmology: theory, galaxies: active, galaxies: evolution, galaxies: formation},
     year = 2010,
    month = dec,
   volume = 724,
    pages = {915-945},
      doi = {10.1088/0004-637X/724/2/915},
   adsurl = {http://adsabs.harvard.edu/abs/2010ApJ...724..915H},
  adsnote = {Provided by the SAO/NASA Astrophysics Data System}
}

@ARTICLE{Hopkins2012,
   author = {{Hopkins}, P.~F. and {Kere{\v s}}, D. and {Murray}, N. and {Quataert}, E. and 
	{Hernquist}, L.},
    title = "{Stellar feedback and bulge formation in clumpy discs}",
  journal = {\mnras},
archivePrefix = "arXiv",
   eprint = {1111.6591},
 primaryClass = "astro-ph.CO",
 keywords = {stars: formation, galaxies: active, galaxies: evolution, galaxies: formation, cosmology: theory},
     year = 2012,
    month = dec,
   volume = 427,
    pages = {968-978},
      doi = {10.1111/j.1365-2966.2012.21981.x},
   adsurl = {http://adsabs.harvard.edu/abs/2012MNRAS.427..968H},
  adsnote = {Provided by the SAO/NASA Astrophysics Data System}
}

@ARTICLE{Hopkins2014,
   author = {{Hopkins}, P.~F. and {Kere{\v s}}, D. and {O{\~n}orbe}, J. and 
	{Faucher-Gigu{\`e}re}, C.-A. and {Quataert}, E. and {Murray}, N. and 
	{Bullock}, J.~S.},
    title = "{Galaxies on FIRE (Feedback In Realistic Environments): stellar feedback explains cosmologically inefficient star formation}",
  journal = {\mnras},
archivePrefix = "arXiv",
   eprint = {1311.2073},
 keywords = {stars: formation, galaxies: active, galaxies: evolution, galaxies: formation, cosmology: theory},
     year = 2014,
    month = nov,
   volume = 445,
    pages = {581-603},
      doi = {10.1093/mnras/stu1738},
   adsurl = {http://adsabs.harvard.edu/abs/2014MNRAS.445..581H},
  adsnote = {Provided by the SAO/NASA Astrophysics Data System}
}

@Article{Ibata2014,
  Title                    = {{A Thousand Shadows of Andromeda: Rotating Planes of Satellites in the Millennium-II Cosmological Simulation}},
  Author                   = {{Ibata}, R.~A. and {Ibata}, N.~G. and {Lewis}, G.~F. and {Martin}, N.~F. and {Conn}, A. and {Elahi}, P. and {Arias}, V. and {Fernando}, N. },
  Journal                  = {\apjl},
  Year                     = {2014},

  Month                    = mar,
  Pages                    = {L6},
  Volume                   = {784},

  Adsnote                  = {Provided by the SAO/NASA Astrophysics Data System},
  Adsurl                   = {http://adsabs.harvard.edu/abs/2014ApJ...784L...6I},
  Archiveprefix            = {arXiv},
  Doi                      = {10.1088/2041-8205/784/1/L6},
  Eid                      = {L6},
  Eprint                   = {1403.2389},
  Keywords                 = {dark matter, galaxies: halos, galaxies: individual: M31}
}

@Article{Ibata2013,
  Title                    = {{A vast, thin plane of corotating dwarf galaxies orbiting the Andromeda galaxy}},
  Author                   = {{Ibata}, R.~A. and {Lewis}, G.~F. and {Conn}, A.~R. and {Irwin}, M.~J. and {McConnachie}, A.~W. and {Chapman}, S.~C. and {Collins}, M.~L. and {Fardal}, M. and {Ferguson}, A.~M.~N. and {Ibata}, N.~G. and {Mackey}, A.~D. and {Martin}, N.~F. and {Navarro}, J. and {Rich}, R.~M. and {Valls-Gabaud}, D. and {Widrow}, L.~M.},
  Journal                  = {\nat},
  Year                     = {2013},

  Month                    = jan,
  Pages                    = {62-65},
  Volume                   = {493},

  Adsnote                  = {Provided by the SAO/NASA Astrophysics Data System},
  Adsurl                   = {http://adsabs.harvard.edu/abs/2013Natur.493...62I},
  Archiveprefix            = {arXiv},
  Doi                      = {10.1038/nature11717},
  Eprint                   = {1301.0446},
  Primaryclass             = {astro-ph.CO}
}

@ARTICLE{Ibata2015,
   author = {{Ibata}, R.~A. and {Famaey}, B. and {Lewis}, G.~F. and {Ibata}, N.~G. and 
	{Martin}, N.},
    title = "{Eppur si Muove: Positional and Kinematic Correlations of Satellite Pairs in the Low Z Universe}",
  journal = {\apj},
archivePrefix = "arXiv",
   eprint = {1411.3718},
 keywords = {galaxies: formation, galaxies: general, galaxies: kinematics and dynamics},
     year = 2015,
    month = may,
   volume = 805,
      eid = {67},
    pages = {67},
      doi = {10.1088/0004-637X/805/1/67},
   adsurl = {http://adsabs.harvard.edu/abs/2015ApJ...805...67I},
  adsnote = {Provided by the SAO/NASA Astrophysics Data System}
}

@ARTICLE{Inoue2012,
   author = {{Inoue}, S. and {Saitoh}, T.~R.},
    title = "{Natures of a clump-origin bulge: a pseudo-bulge like but old metal-rich bulge}",
  journal = {\mnras},
archivePrefix = "arXiv",
   eprint = {1109.2898},
 keywords = {methods: numerical, Galaxy: bulge, Galaxy: disc, Galaxy: formation, galaxies: bulges, galaxies: formation},
     year = 2012,
    month = may,
   volume = 422,
    pages = {1902-1913},
      doi = {10.1111/j.1365-2966.2011.20338.x},
   adsurl = {http://adsabs.harvard.edu/abs/2012MNRAS.422.1902I},
  adsnote = {Provided by the SAO/NASA Astrophysics Data System}
}

@ARTICLE{Inoue2014,
   author = {{Inoue}, S. and {Saitoh}, T.~R.},
    title = "{Properties of thick discs formed in clumpy galaxies}",
  journal = {\mnras},
archivePrefix = "arXiv",
   eprint = {1402.5986},
 keywords = {methods: numerical, Galaxy: disc, Galaxy: formation, galaxies: formation},
     year = 2014,
    month = jun,
   volume = 441,
    pages = {243-255},
      doi = {10.1093/mnras/stu544},
   adsurl = {http://adsabs.harvard.edu/abs/2014MNRAS.441..243I},
  adsnote = {Provided by the SAO/NASA Astrophysics Data System}
}

@ARTICLE{Inoue2016,
   author = {{Inoue}, S. and {Dekel}, A. and {Mandelker}, N. and {Ceverino}, D. and 
	{Bournaud}, F. and {Primack}, J.},
    title = "{Non-linear violent disc instability with high Toomre's Q in high-redshift clumpy disc galaxies}",
  journal = {\mnras},
archivePrefix = "arXiv",
   eprint = {1510.07695},
 keywords = {instabilities, methods: numerical, galaxies: formation, galaxies: kinematics and dynamics},
     year = 2016,
    month = feb,
   volume = 456,
    pages = {2052-2069},
      doi = {10.1093/mnras/stv2793},
   adsurl = {http://adsabs.harvard.edu/abs/2016MNRAS.456.2052I},
  adsnote = {Provided by the SAO/NASA Astrophysics Data System}
}

@ARTICLE{Jones2010,
   author = {{Jones}, T.~A. and {Swinbank}, A.~M. and {Ellis}, R.~S. and 
	{Richard}, J. and {Stark}, D.~P.},
    title = "{Resolved spectroscopy of gravitationally lensed galaxies: recovering coherent velocity fields in subluminous z \~{} 2-3 galaxies}",
  journal = {\mnras},
archivePrefix = "arXiv",
   eprint = {0910.4488},
 keywords = {galaxies: evolution, galaxies: formation, galaxies: high-redshift},
     year = 2010,
    month = may,
   volume = 404,
    pages = {1247-1262},
      doi = {10.1111/j.1365-2966.2010.16378.x},
   adsurl = {http://adsabs.harvard.edu/abs/2010MNRAS.404.1247J},
  adsnote = {Provided by the SAO/NASA Astrophysics Data System}
}

@Article{6df,
  Title                    = {{The 6dF Galaxy Survey: final redshift release (DR3) and southern large-scale structures}},
  Author                   = {{Jones}, D.~H. and {Read}, M.~A. and {Saunders}, W. and {Colless}, M. and {Jarrett}, T. and {Parker}, Q.~A. and {Fairall}, A.~P. and {Mauch}, T. and {Sadler}, E.~M. and {Watson}, F.~G. and {Burton}, D. and {Campbell}, L.~A. and {Cass}, P. and {Croom}, S.~M. and {Dawe}, J. and {Fiegert}, K. and {Frankcombe}, L. and {Hartley}, M. and {Huchra}, J. and {James}, D. and {Kirby}, E. and {Lahav}, O. and {Lucey}, J. and {Mamon}, G.~A. and {Moore}, L. and {Peterson}, B.~A. and {Prior}, S. and {Proust}, D. and {Russell}, K. and {Safouris}, V. and {Wakamatsu}, K.-I. and {Westra}, E. and {Williams}, M.},
  Journal                  = {\mnras},
  Year                     = {2009},

  Month                    = oct,
  Pages                    = {683-698},
  Volume                   = {399},

  Adsnote                  = {Provided by the SAO/NASA Astrophysics Data System},
  Adsurl                   = {http://adsabs.harvard.edu/abs/2009MNRAS.399..683J},
  Archiveprefix            = {arXiv},
  Doi                      = {10.1111/j.1365-2966.2009.15338.x},
  Eprint                   = {0903.5451},
  Keywords                 = {surveys , galaxies: distances and redshifts , cosmology: observations , large-scale structure of Universe},
  Primaryclass             = {astro-ph.CO}
}

@ARTICLE{Kang2005,
   author = {{Kang}, X. and {Mao}, S. and {Gao}, L. and {Jing}, Y.~P.},
    title = "{Are great disks defined by satellite galaxies in Milky-Way type halos rare in {$\Lambda$}CDM?}",
  journal = {\aap},
   eprint = {astro-ph/0501333},
 keywords = {Galaxy: evolution, Galaxy: halo, galaxies: dwarf, galaxies: structure},
     year = 2005,
    month = jul,
   volume = 437,
    pages = {383-388},
      doi = {10.1051/0004-6361:20052675},
   adsurl = {http://adsabs.harvard.edu/abs/2005Aadsnote = {Provided by the SAO/NASA Astrophysics Data System}
}

@ARTICLE{Karim2011,
   author = {{Karim}, A. and {Schinnerer}, E. and {Mart{\'{\i}}nez-Sansigre}, A. and 
	{Sargent}, M.~T. and {van der Wel}, A. and {Rix}, H.-W. and 
	{Ilbert}, O. and {Smol{\v c}i{\'c}}, V. and {Carilli}, C. and 
	{Pannella}, M. and {Koekemoer}, A.~M. and {Bell}, E.~F. and 
	{Salvato}, M.},
    title = "{The Star Formation History of Mass-selected Galaxies in the COSMOS Field}",
  journal = {\apj},
archivePrefix = "arXiv",
   eprint = {1011.6370},
 keywords = {galaxies: evolution, galaxies: star formation, radio continuum: galaxies, surveys},
     year = 2011,
    month = apr,
   volume = 730,
      eid = {61},
    pages = {61},
      doi = {10.1088/0004-637X/730/2/61},
   adsurl = {http://adsabs.harvard.edu/abs/2011ApJ...730...61K},
  adsnote = {Provided by the SAO/NASA Astrophysics Data System}
}

@ARTICLE{Keller2014,
   author = {{Keller}, B.~W. and {Wadsley}, J. and {Benincasa}, S.~M. and 
	{Couchman}, H.~M.~P.},
    title = "{A superbubble feedback model for galaxy simulations}",
  journal = {\mnras},
archivePrefix = "arXiv",
   eprint = {1405.2625},
 keywords = {methods: numerical, ISM: bubbles, galaxies: evolution, galaxies: formation, galaxies: ISM},
     year = 2014,
    month = aug,
   volume = 442,
    pages = {3013-3025},
      doi = {10.1093/mnras/stu1058},
   adsurl = {http://adsabs.harvard.edu/abs/2014MNRAS.442.3013K},
  adsnote = {Provided by the SAO/NASA Astrophysics Data System}
}

@Article{Klypin2001,
  Title                    = {{Resolving the Structure of Cold Dark Matter Halos}},
  Author                   = {{Klypin}, A. and {Kravtsov}, A.~V. and {Bullock}, J.~S. and {Primack}, J.~R.},
  Journal                  = {\apj},
  Year                     = {2001},

  Month                    = jun,
  Pages                    = {903-915},
  Volume                   = {554},

  Adsnote                  = {Provided by the SAO/NASA Astrophysics Data System},
  Adsurl                   = {http://adsabs.harvard.edu/abs/2001ApJ...554..903K},
  Doi                      = {10.1086/321400},
  Eprint                   = {arXiv:astro-ph/0006343},
  Keywords                 = {Cosmology: Theory, Cosmology: Dark Matter, Galaxies: Halos, Galaxies: Structure, Methods: Numerical}
}

@Article{Klypin1999,
  Title                    = {{Where Are the Missing Galactic Satellites?}},
  Author                   = {{Klypin}, A. and {Kravtsov}, A.~V. and {Valenzuela}, O. and {Prada}, F.},
  Journal                  = {\apj},
  Year                     = {1999},

  Month                    = sep,
  Pages                    = {82-92},
  Volume                   = {522},

  Adsnote                  = {Provided by the SAO/NASA Astrophysics Data System},
  Adsurl                   = {http://adsabs.harvard.edu/abs/1999ApJ...522...82K},
  Doi                      = {10.1086/307643},
  Eprint                   = {arXiv:astro-ph/9901240},
  Keywords                 = {COSMOLOGY: THEORY, GALAXIES: CLUSTERS: GENERAL, GALAXIES: INTERACTIONS, GALAXY: FORMATION, GALAXIES: LOCAL GROUP, METHODS: NUMERICAL}
}

@Article{Knollmann2009,
  Title                    = {{AHF: Amiga's Halo Finder}},
  Author                   = {{Knollmann}, S.~R. and {Knebe}, A.},
  Journal                  = {\apjs},
  Year                     = {2009},

  Month                    = jun,
  Pages                    = {608-624},
  Volume                   = {182},

  Abstract                 = {{Cosmological simulations are the key tool for investigating the different processes involved in the formation of the universe from small initial density perturbations to galaxies and clusters of galaxies observed today. The identification and analysis of bound objects, halos, is one of the most important steps in drawing useful physical information from simulations. In the advent of larger and larger simulations, a reliable and parallel halo finder, able to cope with the ever-increasing data files, is a must. In this work we present the freely available MPI parallel halo finder AHF. We provide a description of the algorithm and the strategy followed to handle large simulation data. We also describe the parameters a user may choose in order to influence the process of halo finding, as well as pointing out which parameters are crucial to ensure untainted results from the parallel approach. Furthermore, we demonstrate the ability of AHF to scale to high-resolution simulations. }},
  Adsnote                  = {Provided by the SAO/NASA Astrophysics Data System},
  Adsurl                   = {http://adsabs.harvard.edu/abs/2009ApJS..182..608K},
  Archiveprefix            = {arXiv},
  Doi                      = {10.1088/0067-0049/182/2/608},
  Eprint                   = {0904.3662},
  Keywords                 = {methods: numerical},
  Primaryclass             = {astro-ph.CO}
}

@Article{Koch2006,
  Title                    = {{The Anisotropic Distribution of M31 Satellite Galaxies: A Polar Great Plane of Early-type Companions}},
  Author                   = {{Koch}, A. and {Grebel}, E.~K.},
  Journal                  = {\aj},
  Year                     = {2006},

  Month                    = mar,
  Pages                    = {1405-1415},
  Volume                   = {131},

  Adsnote                  = {Provided by the SAO/NASA Astrophysics Data System},
  Adsurl                   = {http://adsabs.harvard.edu/abs/2006AJ....131.1405K},
  Doi                      = {10.1086/499534},
  Eprint                   = {astro-ph/0509258},
  Keywords                 = {Galaxies: Dwarf, Galaxies: Evolution, Galaxies: Individual: Messier Number: M31, Galaxies: Individual: Messier Number: M32, Galaxies: Individual: Messier Number: M33, Galaxies: Individual: NGC Number: NGC 147, Galaxies: Individual: NGC Number: NGC 185, Galaxies: Individual: NGC Number: NGC 205, Galaxies: Individual: Name: Andromeda I, galaxies: individual (II), galaxies: individual (III), galaxies: individual (V), galaxies: individual (VI), galaxies: individual (VII), galaxies: individual (IX), galaxies: individual (PegDIG), Galaxies: Interactions, Galaxies: Kinematics and Dynamics, Galaxies: Local Group}
}

@Article{Koposov2008,
  Title                    = {{The Luminosity Function of the Milky Way Satellites}},
  Author                   = {{Koposov}, S. and {Belokurov}, V. and {Evans}, N.~W. and {Hewett}, P.~C. and {Irwin}, M.~J. and {Gilmore}, G. and {Zucker}, D.~B. and {Rix}, H.-W. and {Fellhauer}, M. and {Bell}, E.~F. and {Glushkova}, E.~V.},
  Journal                  = {\apj},
  Year                     = {2008},

  Month                    = oct,
  Pages                    = {279-291},
  Volume                   = {686},

  Adsnote                  = {Provided by the SAO/NASA Astrophysics Data System},
  Adsurl                   = {http://adsabs.harvard.edu/abs/2008ApJ...686..279K},
  Archiveprefix            = {arXiv},
  Doi                      = {10.1086/589911},
  Eprint                   = {0706.2687},
  Keywords                 = {Galaxy: Formation, Galaxy: Halo, Galaxy: Structure, Galaxies: Local Group}
}

@PhdThesis{Kravtsov1999,
  Title                    = {{High-resolution simulations of structure formation in the universe}},
  Author                   = {{Kravtsov}, A.~V.},
  School                   = {NEW MEXICO STATE UNIVERSITY},
  Year                     = {1999},

  Adsnote                  = {Provided by the SAO/NASA Astrophysics Data System},
  Adsurl                   = {http://adsabs.harvard.edu/abs/1999PhDT........25K}
}

@Article{Kravtsov2004,
  Title                    = {{The Tumultuous Lives of Galactic Dwarfs and the Missing Satellites Problem}},
  Author                   = {{Kravtsov}, A.~V. and {Gnedin}, O.~Y. and {Klypin}, A.~A.},
  Journal                  = {\apj},
  Year                     = {2004},

  Month                    = jul,
  Pages                    = {482-497},
  Volume                   = {609},

  Adsnote                  = {Provided by the SAO/NASA Astrophysics Data System},
  Adsurl                   = {http://adsabs.harvard.edu/abs/2004ApJ...609..482K},
  Doi                      = {10.1086/421322},
  Eprint                   = {astro-ph/0401088},
  Keywords                 = {Cosmology: Theory, Galaxies: Dwarf, Galaxies: Evolution, Galaxies: Formation, Galaxies: Halos, Methods: Numerical}
}

@ARTICLE{Kravtsov2014,
   author = {{Kravtsov}, A. and {Vikhlinin}, A. and {Meshscheryakov}, A.},
    title = "{Stellar mass -- halo mass relation and star formation efficiency in high-mass halos}",
  journal = {ArXiv e-prints},
archivePrefix = "arXiv",
   eprint = {1401.7329},
 primaryClass = "astro-ph.CO",
 keywords = {Astrophysics - Cosmology and Extragalactic Astrophysics},
     year = 2014,
    month = jan,
   adsurl = {http://adsabs.harvard.edu/abs/2014arXiv1401.7329K},
  adsnote = {Provided by the SAO/NASA Astrophysics Data System}
}

@Article{Kroupa2012,
  Title                    = {{The Failures of the Standard Model of Cosmology Require a New Paradigm}},
  Author                   = {{Kroupa}, P. and {Pawlowski}, M. and {Milgrom}, M.},
  Journal                  = {International Journal of Modern Physics D},
  Year                     = {2012},

  Month                    = dec,
  Pages                    = {30003},
  Volume                   = {21},

  Adsnote                  = {Provided by the SAO/NASA Astrophysics Data System},
  Adsurl                   = {http://adsabs.harvard.edu/abs/2012IJMPD..2130003K},
  Archiveprefix            = {arXiv},
  Doi                      = {10.1142/S0218271812300030},
  Eid                      = {1230003},
  Eprint                   = {1301.3907},
  Keywords                 = {Cosmology, dark matter, gravitation, MOND, Characteristics and properties of the Milky Way galaxy, Local group, Magellanic Clouds, Characteristics and properties of external galaxies and extragalactic objects, Galaxy groups clusters and superclusters, large scale structure of the Universe, Cosmology, Fundamental problems and general formalism, Experimental tests of gravitational theories, Dark matter, Observational cosmology},
  Primaryclass             = {astro-ph.CO}
}

@Article{Kroupa2005,
  Title                    = {{The great disk of Milky-Way satellites and cosmological sub-structures}},
  Author                   = {{Kroupa}, P. and {Theis}, C. and {Boily}, C.~M.},
  Journal                  = {\aap},
  Year                     = {2005},

  Month                    = feb,
  Pages                    = {517-521},
  Volume                   = {431},

  Adsnote                  = {Provided by the SAO/NASA Astrophysics Data System},
  Adsurl                   = {http://adsabs.harvard.edu/abs/2005ADoi                      = {10.1051/0004-6361:20041122},
  Eprint                   = {astro-ph/0410421},
  Keywords                 = {Galaxy: evolution, Galaxy: halo, galaxies: dwarf, galaxies: kinematics and dynamics, galaxies: Local Group, Galaxy: formation}
}

@Article{Libeskind2005,
  Title                    = {{The distribution of satellite galaxies: the great pancake}},
  Author                   = {{Libeskind}, N.~I. and {Frenk}, C.~S. and {Cole}, S. and {Helly}, J.~C. and {Jenkins}, A. and {Navarro}, J.~F. and {Power}, C.},
  Journal                  = {\mnras},
  Year                     = {2005},

  Month                    = oct,
  Pages                    = {146-152},
  Volume                   = {363},

  Adsnote                  = {Provided by the SAO/NASA Astrophysics Data System},
  Adsurl                   = {http://adsabs.harvard.edu/abs/2005MNRAS.363..146L},
  Doi                      = {10.1111/j.1365-2966.2005.09425.x},
  Eprint                   = {astro-ph/0503400},
  Keywords                 = {cosmology: theory, dark matter, galaxies: haloes}
}

@Article{Libeskind2009,
  Title                    = {{How common is the Milky Way-satellite system alignment?}},
  Author                   = {{Libeskind}, N.~I. and {Frenk}, C.~S. and {Cole}, S. and {Jenkins}, A. and {Helly}, J.~C.},
  Journal                  = {\mnras},
  Year                     = {2009},

  Month                    = oct,
  Pages                    = {550-558},
  Volume                   = {399},

  Adsnote                  = {Provided by the SAO/NASA Astrophysics Data System},
  Adsurl                   = {http://adsabs.harvard.edu/abs/2009MNRAS.399..550L},
  Archiveprefix            = {arXiv},
  Doi                      = {10.1111/j.1365-2966.2009.15315.x},
  Eprint                   = {0905.1696},
  Keywords                 = {galaxies: haloes , galaxies: kinematics and dynamics , Local Group , cosmology: theory , large-scale structure of Universe},
  Primaryclass             = {astro-ph.CO}
}

@Article{Libeskind2015,
  Title                    = {{Planes of satellite galaxies and the cosmic web}},
  Author                   = {{Libeskind}, N.~I. and {Hoffman}, Y. and {Tully}, R.~B. and {Courtois}, H.~M. and {Pomar{\`e}de}, D. and {Gottl{\"o}ber}, S. and {Steinmetz}, M.},
  Journal                  = {\mnras},
  Year                     = {2015},

  Month                    = sep,
  Pages                    = {1052-1059},
  Volume                   = {452},

  Adsnote                  = {Provided by the SAO/NASA Astrophysics Data System},
  Adsurl                   = {http://adsabs.harvard.edu/abs/2015MNRAS.452.1052L},
  Archiveprefix            = {arXiv},
  Doi                      = {10.1093/mnras/stv1302},
  Eprint                   = {1503.05915},
  Keywords                 = {galaxies: haloes, cosmology: theory, dark matter, large-scale structure of Universe}
}

@Article{Libeskind2014,
  Title                    = {{The universal nature of subhalo accretion}},
  Author                   = {{Libeskind}, N.~I. and {Knebe}, A. and {Hoffman}, Y. and {Gottl{\"o}ber}, S.},
  Journal                  = {\mnras},
  Year                     = {2014},

  Month                    = sep,
  Pages                    = {1274-1280},
  Volume                   = {443},

  Adsnote                  = {Provided by the SAO/NASA Astrophysics Data System},
  Adsurl                   = {http://adsabs.harvard.edu/abs/2014MNRAS.443.1274L},
  Archiveprefix            = {arXiv},
  Doi                      = {10.1093/mnras/stu1216},
  Eprint                   = {1407.0394},
  Keywords                 = {galaxies: haloes, cosmology: theory, dark matter, large-scale structure of Universe}
}

@Article{Libeskind2011,
  Title                    = {{The preferred direction of infalling satellite galaxies in the Local Group}},
  Author                   = {{Libeskind}, N.~I. and {Knebe}, A. and {Hoffman}, Y. and {Gottl{\"o}ber}, S. and {Yepes}, G. and {Steinmetz}, M.},
  Journal                  = {\mnras},
  Year                     = {2011},

  Month                    = mar,
  Pages                    = {1525-1535},
  Volume                   = {411},

  Adsnote                  = {Provided by the SAO/NASA Astrophysics Data System},
  Adsurl                   = {http://adsabs.harvard.edu/abs/2011MNRAS.411.1525L},
  Archiveprefix            = {arXiv},
  Doi                      = {10.1111/j.1365-2966.2010.17786.x},
  Eprint                   = {1010.1531},
  Keywords                 = {galaxies: formation, galaxies: haloes, cosmology: theory, dark matter}
}

@Article{Lovell2012,
  Title                    = {{The haloes of bright satellite galaxies in a warm dark matter universe}},
  Author                   = {{Lovell}, M.~R. and {Eke}, V. and {Frenk}, C.~S. and {Gao}, L. and {Jenkins}, A. and {Theuns}, T. and {Wang}, J. and {White}, S.~D.~M. and {Boyarsky}, A. and {Ruchayskiy}, O.},
  Journal                  = {\mnras},
  Year                     = {2012},

  Month                    = mar,
  Pages                    = {2318-2324},
  Volume                   = {420},

  Adsnote                  = {Provided by the SAO/NASA Astrophysics Data System},
  Adsurl                   = {http://adsabs.harvard.edu/abs/2012MNRAS.420.2318L},
  Archiveprefix            = {arXiv},
  Doi                      = {10.1111/j.1365-2966.2011.20200.x},
  Eprint                   = {1104.2929},
  Keywords                 = {galaxies: dwarf, dark matter}
}

@Article{Lovell2011,
  Title                    = {{The link between galactic satellite orbits and subhalo accretion}},
  Author                   = {{Lovell}, M.~R. and {Eke}, V.~R. and {Frenk}, C.~S. and {Jenkins}, A. },
  Journal                  = {\mnras},
  Year                     = {2011},

  Month                    = jun,
  Pages                    = {3013-3021},
  Volume                   = {413},

  Adsnote                  = {Provided by the SAO/NASA Astrophysics Data System},
  Adsurl                   = {http://adsabs.harvard.edu/abs/2011MNRAS.413.3013L},
  Archiveprefix            = {arXiv},
  Doi                      = {10.1111/j.1365-2966.2011.18377.x},
  Eprint                   = {1008.0484},
  Keywords                 = {galaxies: formation, dark matter},
  Primaryclass             = {astro-ph.CO}
}

@Article{LyndenBell1976,
  Title                    = {{Dwarf galaxies and globular clusters in high velocity hydrogen streams}},
  Author                   = {{Lynden-Bell}, D.},
  Journal                  = {\mnras},
  Year                     = {1976},

  Month                    = mar,
  Pages                    = {695-710},
  Volume                   = {174},

  Adsnote                  = {Provided by the SAO/NASA Astrophysics Data System},
  Adsurl                   = {http://adsabs.harvard.edu/abs/1976MNRAS.174..695L},
  Keywords                 = {Galactic Nuclei, Galaxies, Globular Clusters, Hydrogen Clouds, Magellanic Clouds, Astronomical Maps, Gas Flow, Milky Way Galaxy, Radial Velocity, Stellar Evolution}
}

@InProceedings{Maccio2010,
  Title                    = {{Milky Way Satellite Properties in a {$\Lambda$}CDM Universe}},
  Author                   = {{Macci{\`o}}, A.~V.},
  Booktitle                = {American Institute of Physics Conference Series},
  Year                     = {2010},
  Editor                   = {{Debattista}, V.~P. and {Popescu}, C.~C.},
  Month                    = jun,
  Pages                    = {355-358},
  Series                   = {American Institute of Physics Conference Series},
  Volume                   = {1240},

  Adsnote                  = {Provided by the SAO/NASA Astrophysics Data System},
  Adsurl                   = {http://adsabs.harvard.edu/abs/2010AIPC.1240..355M},
  Doi                      = {10.1063/1.3458532},
  Keywords                 = {dark matter, star formation, red shift, accretion, Dark matter, Star formation, Distances redshifts radial velocities, spatial distribution of galaxies, Infall accretion and accretion disks}
}

@Article{Maccio2010a,
  Title                    = {{Luminosity function and radial distribution of Milky Way satellites in a {$\Lambda$}CDM Universe}},
  Author                   = {{Macci{\`o}}, A.~V. and {Kang}, X. and {Fontanot}, F. and {Somerville}, R.~S. and {Koposov}, S. and {Monaco}, P.},
  Journal                  = {\mnras},
  Year                     = {2010},

  Month                    = mar,
  Pages                    = {1995-2008},
  Volume                   = {402},

  Adsnote                  = {Provided by the SAO/NASA Astrophysics Data System},
  Adsurl                   = {http://adsabs.harvard.edu/abs/2010MNRAS.402.1995M},
  Archiveprefix            = {arXiv},
  Doi                      = {10.1111/j.1365-2966.2009.16031.x},
  Eprint                   = {0903.4681},
  Keywords                 = {gravitation, methods: N-body simulation, methods: numerical, galaxies: haloes, cosmology: theory, dark matter},
  Primaryclass             = {astro-ph.CO}
}

@Article{Maccio2012a,
  Title                    = {{Halo Expansion in Cosmological Hydro Simulations: Toward a Baryonic Solution of the Cusp/Core Problem in Massive Spirals}},
  Author                   = {{Macci{\`o}}, A.~V. and {Stinson}, G. and {Brook}, C.~B. and {Wadsley}, J. and {Couchman}, H.~M.~P. and {Shen}, S. and {Gibson}, B.~K. and {Quinn}, T.},
  Journal                  = {\apjl},
  Year                     = {2012},

  Month                    = jan,
  Pages                    = {L9},
  Volume                   = {744},

  Abstract                 = {{A clear prediction of the cold dark matter (CDM) model is the existence of cuspy dark matter halo density profiles on all mass scales. This is not in agreement with the observed rotation curves of spiral galaxies, challenging on small scales the otherwise successful CDM paradigm. In this work we employ high-resolution cosmological hydrodynamical simulations to study the effects of dissipative processes on the inner distribution of dark matter in Milky Way like objects (M {\ap} 10$^{12}$ M $_{&sun;}$). Our simulations include supernova feedback, and the effects of the radiation pressure of massive stars before they explode as supernovae. The increased stellar feedback results in the expansion of the dark matter halo instead of contraction with respect to N-body simulations. Baryons are able to erase the dark matter cuspy distribution, creating a flat, cored, dark matter density profile in the central several kiloparsecs of a massive Milky-Way-like halo. The profile is well fit by a Burkert profile, with fitting parameters consistent with the observations. In addition, we obtain flat rotation curves as well as extended, exponential stellar disk profiles. While the stellar disk we obtain is still partially too thick to resemble the Milky Way thin disk, this pilot study shows that there is enough energy available in the baryonic component to alter the dark matter distribution even in massive disk galaxies, providing a possible solution to the long-standing problem of cusps versus cores. }},
  Adsnote                  = {Provided by the SAO/NASA Astrophysics Data System},
  Adsurl                   = {http://adsabs.harvard.edu/abs/2012ApJ...744L...9M},
  Archiveprefix            = {arXiv},
  Doi                      = {10.1088/2041-8205/744/1/L9},
  Eid                      = {L9},
  Eprint                   = {1111.5620},
  Keywords                 = {cosmology: theory, galaxies: structure, hydrodynamics, methods: numerical},
  Primaryclass             = {astro-ph.CO}
}

@Article{Maccio2012b,
  Title                    = {{Halo Expansion in Cosmological Hydro Simulations: Toward a Baryonic Solution of the Cusp/Core Problem in Massive Spirals}},
  Author                   = {{Macci{\`o}}, A.~V. and {Stinson}, G. and {Brook}, C.~B. and {Wadsley}, J. and {Couchman}, H.~M.~P. and {Shen}, S. and {Gibson}, B.~K. and {Quinn}, T.},
  Journal                  = {\apjl},
  Year                     = {2012},

  Month                    = jan,
  Pages                    = {L9},
  Volume                   = {744},

  Adsnote                  = {Provided by the SAO/NASA Astrophysics Data System},
  Adsurl                   = {http://adsabs.harvard.edu/abs/2012ApJ...744L...9M},
  Archiveprefix            = {arXiv},
  Doi                      = {10.1088/2041-8205/744/1/L9},
  Eid                      = {L9},
  Eprint                   = {1111.5620},
  Keywords                 = {cosmology: theory, galaxies: structure, hydrodynamics, methods: numerical},
  Primaryclass             = {astro-ph.CO}
}

@ARTICLE{Maccio2016,
   author = {{Macci{\`o}}, A.~V. and {Udrescu}, S.~M. and {Dutton}, A.~A. and 
	{Obreja}, A. and {Wang}, L. and {Stinson}, G.~R. and {Kang}, X.
	},
    title = "{NIHAO X: reconciling the local galaxy velocity function with cold dark matter via mock H i observations}",
  journal = {\mnras},
archivePrefix = "arXiv",
   eprint = {1503.04818},
     year = 2016,
    month = nov,
   volume = 463,
    pages = {L69-L73},
      doi = {10.1093/mnrasl/slw147},
   adsurl = {http://esoads.eso.org/abs/2016MNRAS.463L..69M},
  adsnote = {Provided by the SAO/NASA Astrophysics Data System}
}



@ARTICLE{Mandelker2014,
   author = {{Mandelker}, N. and {Dekel}, A. and {Ceverino}, D. and {Tweed}, D. and 
	{Moody}, C.~E. and {Primack}, J.},
    title = "{The population of giant clumps in simulated high-z galaxies: in situ and ex situ migration and survival}",
  journal = {\mnras},
archivePrefix = "arXiv",
   eprint = {1311.0013},
 keywords = {stars: formation, galaxies: evolution, galaxies: formation, galaxies: kinematics and dynamics},
     year = 2014,
    month = oct,
   volume = 443,
    pages = {3675-3702},
      doi = {10.1093/mnras/stu1340},
   adsurl = {http://adsabs.harvard.edu/abs/2014MNRAS.443.3675M},
  adsnote = {Provided by the SAO/NASA Astrophysics Data System}
}

@ARTICLE{Mandelker2016,
   author = {{Mandelker}, N. and {Dekel}, A. and {Ceverino}, D. and {DeGraf}, C. and 
	{Guo}, Y. and {Primack}, J.},
    title = "{Giant Clumps in Simulated High-z Galaxies: Properties, Evolution and Dependence on Feedback}",
  journal = {\mnras},
archivePrefix = "arXiv",
   eprint = {1512.08791},
 keywords = {cosmology, galaxies: evolution, galaxies: formation, galaxies: kinematics and dynamics, stars: formation},
     year = 2016,
    month = sep,
      doi = {10.1093/mnras/stw2358},
   adsurl = {http://adsabs.harvard.edu/abs/2016MNRAS.tmp.1460M},
  adsnote = {Provided by the SAO/NASA Astrophysics Data System}
}

@ARTICLE{Marigo2008,
   author = {{Marigo}, P. and {Girardi}, L. and {Bressan}, A. and {Groenewegen}, M.~A.~T. and 
	{Silva}, L. and {Granato}, G.~L.},
    title = "{Evolution of asymptotic giant branch stars. II. Optical to far-infrared isochrones with improved TP-AGB models}",
  journal = {\aap},
archivePrefix = "arXiv",
   eprint = {0711.4922},
 keywords = {astronomical data bases: miscellaneous, stars: AGB and post-AGB, stars: carbon, stars: evolution, galaxies: Magellanic Clouds, stars: Hertzsprung-Russell (HR) and C-M diagrams},
     year = 2008,
    month = may,
   volume = 482,
    pages = {883-905},
      doi = {10.1051/0004-6361:20078467},
   adsurl = {http://adsabs.harvard.edu/abs/2008Aadsnote = {Provided by the SAO/NASA Astrophysics Data System}
}

@Article{Marti2003,
  Title                    = {{Numerical Hydrodynamics in Special Relativity}},
  Author                   = {{Mart{\'{\i}}}, J.~M. and {M{\"u}ller}, E.},
  Journal                  = {Living Reviews in Relativity},
  Year                     = {2003},

  Month                    = dec,
  Pages                    = {7},
  Volume                   = {6},

  Adsnote                  = {Provided by the SAO/NASA Astrophysics Data System},
  Adsurl                   = {http://adsabs.harvard.edu/abs/2003LRR.....6....7M},
  Doi                      = {10.12942/lrr-2003-7}
}

@ARTICLE{Martig2009,
   author = {{Martig}, M. and {Bournaud}, F. and {Teyssier}, R. and {Dekel}, A.
	},
    title = "{Morphological Quenching of Star Formation: Making Early-Type Galaxies Red}",
  journal = {\apj},
archivePrefix = "arXiv",
   eprint = {0905.4669},
 primaryClass = "astro-ph.CO",
 keywords = {galaxies: elliptical and lenticular, cD, galaxies: evolution, galaxies: formation},
     year = 2009,
    month = dec,
   volume = 707,
    pages = {250-267},
      doi = {10.1088/0004-637X/707/1/250},
   adsurl = {http://adsabs.harvard.edu/abs/2009ApJ...707..250M},
  adsnote = {Provided by the SAO/NASA Astrophysics Data System}
}

@ARTICLE{Martig2016,
   author = {{Martig}, M. and {Fouesneau}, M. and {Rix}, H.-W. and {Ness}, M. and 
	{M{\'e}sz{\'a}ros}, S. and {Garc{\'{\i}}a-Hern{\'a}ndez}, D.~A. and 
	{Pinsonneault}, M. and {Serenelli}, A. and {Silva Aguirre}, V. and 
	{Zamora}, O.},
    title = "{Red giant masses and ages derived from carbon and nitrogen abundances}",
  journal = {\mnras},
archivePrefix = "arXiv",
   eprint = {1511.08203},
 primaryClass = "astro-ph.SR",
 keywords = {stars: abundances, stars: evolution, stars: fundamental parameters},
     year = 2016,
    month = mar,
   volume = 456,
    pages = {3655-3670},
      doi = {10.1093/mnras/stv2830},
   adsurl = {http://adsabs.harvard.edu/abs/2016MNRAS.456.3655M},
  adsnote = {Provided by the SAO/NASA Astrophysics Data System}
}

@Article{Martin2014,
  Title                    = {{Spectroscopy of the Three Distant Andromedan Satellites Cassiopeia III, Lacerta I, and Perseus I}},
  Author                   = {{Martin}, N.~F. and {Chambers}, K.~C. and {Collins}, M.~L.~M. and {Ibata}, R.~A. and {Rich}, R.~M. and {Bell}, E.~F. and {Bernard}, E.~J. and {Ferguson}, A.~M.~N. and {Flewelling}, H. and {Kaiser}, N. and {Magnier}, E.~A. and {Tonry}, J.~L. and {Wainscoat}, R.~J.},
  Journal                  = {\apjl},
  Year                     = {2014},

  Month                    = sep,
  Pages                    = {L14},
  Volume                   = {793},

  Adsnote                  = {Provided by the SAO/NASA Astrophysics Data System},
  Adsurl                   = {http://adsabs.harvard.edu/abs/2014ApJ...793L..14M},
  Archiveprefix            = {arXiv},
  Doi                      = {10.1088/2041-8205/793/1/L14},
  Eid                      = {L14},
  Eprint                   = {1408.5130},
  Keywords                 = {galaxies: individual: Cas III Lac I Per I, galaxies: kinematics and dynamics, Local Group}
}

@Article{Mashchenko2008,
  Title                    = {{Stellar Feedback in Dwarf Galaxy Formation}},
  Author                   = {{Mashchenko}, S. and {Wadsley}, J. and {Couchman}, H.~M.~P.},
  Journal                  = {Science},
  Year                     = {2008},

  Month                    = jan,
  Pages                    = {174-},
  Volume                   = {319},

  Adsnote                  = {Provided by the SAO/NASA Astrophysics Data System},
  Adsurl                   = {http://adsabs.harvard.edu/abs/2008Sci...319..174M},
  Archiveprefix            = {arXiv},
  Doi                      = {10.1126/science.1148666},
  Eprint                   = {0711.4803}
}

@ARTICLE{Mayer2016,
   author = {{Mayer}, L. and {Tamburello}, V. and {Lupi}, A. and {Keller}, B. and 
	{Wadsley}, J. and {Madau}, P.},
    title = "{Clumpy high-z galaxies as a testbed for feedback-regulated galaxy formation}",
  journal = {ArXiv e-prints},
archivePrefix = "arXiv",
   eprint = {1606.06739},
 keywords = {Astrophysics - Astrophysics of Galaxies},
     year = 2016,
    month = jun,
   adsurl = {http://adsabs.harvard.edu/abs/2016arXiv160606739M},
  adsnote = {Provided by the SAO/NASA Astrophysics Data System}
}

@Article{McConnachie2012,
  Title                    = {{The Observed Properties of Dwarf Galaxies in and around the Local Group}},
  Author                   = {{McConnachie}, A.~W.},
  Journal                  = {\aj},
  Year                     = {2012},

  Month                    = jul,
  Pages                    = {4},
  Volume                   = {144},

  Adsnote                  = {Provided by the SAO/NASA Astrophysics Data System},
  Adsurl                   = {http://adsabs.harvard.edu/abs/2012AJ....144....4M},
  Archiveprefix            = {arXiv},
  Doi                      = {10.1088/0004-6256/144/1/4},
  Eid                      = {4},
  Eprint                   = {1204.1562},
  Keywords                 = {catalogs, galaxies: dwarf, galaxies: fundamental parameters, galaxies: general, galaxies: structure, Local Group}
}

@Article{McConnachie2006,
  Title                    = {{The satellite distribution of M31}},
  Author                   = {{McConnachie}, A.~W. and {Irwin}, M.~J.},
  Journal                  = {\mnras},
  Year                     = {2006},

  Month                    = jan,
  Pages                    = {902-914},
  Volume                   = {365},

  Adsnote                  = {Provided by the SAO/NASA Astrophysics Data System},
  Adsurl                   = {http://adsabs.harvard.edu/abs/2006MNRAS.365..902M},
  Doi                      = {10.1111/j.1365-2966.2005.09771.x},
  Eprint                   = {astro-ph/0510654},
  Keywords                 = {galaxies: dwarf, galaxies: general, galaxies: haloes, Local Group}
}

@ARTICLE{McWilliam2010,
   author = {{McWilliam}, A. and {Zoccali}, M.},
    title = "{Two Red Clumps and the X-shaped Milky Way Bulge}",
  journal = {\apj},
archivePrefix = "arXiv",
   eprint = {1008.0519},
 keywords = {Galaxy: bulge, Galaxy: structure, stars: distances, stars: late-type},
     year = 2010,
    month = dec,
   volume = 724,
    pages = {1491-1502},
      doi = {10.1088/0004-637X/724/2/1491},
   adsurl = {http://adsabs.harvard.edu/abs/2010ApJ...724.1491M},
  adsnote = {Provided by the SAO/NASA Astrophysics Data System}
}

@Article{PANDAS,
  Title                    = {{The remnants of galaxy formation from a panoramic survey of the region around M31}},
  Author                   = {{McConnachie}, A.~W. and {Irwin}, M.~J. and {Ibata}, R.~A. and {Dubinski}, J. and {Widrow}, L.~M. and {Martin}, N.~F. and {C{\^o}t{\'e}}, P. and {Dotter}, A.~L. and {Navarro}, J.~F. and {Ferguson}, A.~M.~N. and {Puzia}, T.~H. and {Lewis}, G.~F. and {Babul}, A. and {Barmby}, P. and {Bienaym{\'e}}, O. and {Chapman}, S.~C. and {Cockcroft}, R. and {Collins}, M.~L.~M. and {Fardal}, M.~A. and {Harris}, W.~E. and {Huxor}, A. and {Mackey}, A.~D. and {Pe{\~n}arrubia}, J. and {Rich}, R.~M. and {Richer}, H.~B. and {Siebert}, A. and {Tanvir}, N. and {Valls-Gabaud}, D. and {Venn}, K.~A.},
  Journal                  = {\nat},
  Year                     = {2009},

  Month                    = sep,
  Pages                    = {66-69},
  Volume                   = {461},

  Adsnote                  = {Provided by the SAO/NASA Astrophysics Data System},
  Adsurl                   = {http://adsabs.harvard.edu/abs/2009Natur.461...66M},
  Archiveprefix            = {arXiv},
  Doi                      = {10.1038/nature08327},
  Eprint                   = {0909.0398}
}

@Article{Metz2008,
  Title                    = {{The Orbital Poles of Milky Way Satellite Galaxies: A Rotationally Supported Disk of Satellites}},
  Author                   = {{Metz}, M. and {Kroupa}, P. and {Libeskind}, N.~I.},
  Journal                  = {\apj},
  Year                     = {2008},

  Month                    = jun,
  Pages                    = {287-294},
  Volume                   = {680},

  Adsnote                  = {Provided by the SAO/NASA Astrophysics Data System},
  Adsurl                   = {http://adsabs.harvard.edu/abs/2008ApJ...680..287M},
  Archiveprefix            = {arXiv},
  Doi                      = {10.1086/587833},
  Eprint                   = {0802.3899},
  Keywords                 = {Galaxies: Evolution, Galaxies: Halos, Galaxies: Kinematics and Dynamics, Galaxies: Local Group}
}

@ARTICLE{Moody2014,
   author = {{Moody}, C.~E. and {Guo}, Y. and {Mandelker}, N. and {Ceverino}, D. and 
	{Mozena}, M. and {Koo}, D.~C. and {Dekel}, A. and {Primack}, J.
	},
    title = "{Star formation and clumps in cosmological galaxy simulations with radiation pressure feedback}",
  journal = {\mnras},
archivePrefix = "arXiv",
   eprint = {1405.5266},
 keywords = {galaxies: evolution, galaxies: formation, galaxies: kinematics and dynamics, galaxies: star clusters: general, galaxies: star formation, galaxies: structure},
     year = 2014,
    month = oct,
   volume = 444,
    pages = {1389-1399},
      doi = {10.1093/mnras/stu1534},
   adsurl = {http://adsabs.harvard.edu/abs/2014MNRAS.444.1389M},
  adsnote = {Provided by the SAO/NASA Astrophysics Data System}
}

@Article{Moore1994,
  Title                    = {{Evidence against dissipation-less dark matter from observations of galaxy haloes}},
  Author                   = {{Moore}, B.},
  Journal                  = {\nat},
  Year                     = {1994},

  Month                    = aug,
  Pages                    = {629-631},
  Volume                   = {370},

  Adsnote                  = {Provided by the SAO/NASA Astrophysics Data System},
  Adsurl                   = {http://adsabs.harvard.edu/abs/1994Natur.370..629M},
  Doi                      = {10.1038/370629a0}
}

@Article{Moore1999a,
  Title                    = {{Dark Matter Substructure within Galactic Halos}},
  Author                   = {{Moore}, B. and {Ghigna}, S. and {Governato}, F. and {Lake}, G. and {Quinn}, T. and {Stadel}, J. and {Tozzi}, P.},
  Journal                  = {\apjl},
  Year                     = {1999},

  Month                    = oct,
  Pages                    = {L19-L22},
  Volume                   = {524},

  Abstract                 = {{We use numerical simulations to examine the substructure within galactic and cluster mass halos that form within a hierarchical universe. Clusters are easily reproduced with a steep mass spectrum of thousands of substructure clumps that closely matches the observations. However, the survival of dark matter substructure also occurs on galactic scales, leading to the remarkable result that galaxy halos appear as scaled versions of galaxy clusters. The model predicts that the virialized extent of the Milky Way's halo should contain about 500 satellites with circular velocities larger than the Draco and Ursa Minor systems, i.e., bound masses {\gt}\~{}10\^{}8 M\_solar and tidally limited sizes {\gt}\~{}1 kpc. The substructure clumps are on orbits that take a large fraction of them through the stellar disk, leading to significant resonant and impulsive heating. Their abundance and singular density profiles have important implications for the existence of old thin disks, cold stellar streams, gravitational lensing, and indirect/direct detection experiments. }},
  Adsnote                  = {Provided by the SAO/NASA Astrophysics Data System},
  Adsurl                   = {http://adsabs.harvard.edu/abs/1999ApJ...524L..19M},
  Doi                      = {10.1086/312287},
  Eprint                   = {arXiv:astro-ph/9907411},
  Keywords                 = {COSMOLOGY: OBSERVATIONS, COSMOLOGY: THEORY, COSMOLOGY: DARK MATTER, GALAXIES: CLUSTERS: GENERAL, GALAXIES: FORMATION}
}

@ARTICLE{Morishita2015,
   author = {{Morishita}, T. and {Ichikawa}, T. and {Noguchi}, M. and {Akiyama}, M. and 
	{Patel}, S.~G. and {Kajisawa}, M. and {Obata}, T.},
    title = "{From Diversity to Dichotomy, and Quenching: Milky-Way-like and Massive Galaxy Progenitors at 0.5 $\lt$ z $\lt$3.0}",
  journal = {\apj},
archivePrefix = "arXiv",
   eprint = {1502.05713},
 keywords = {galaxies: evolution, galaxies: formation, galaxies: general, galaxies: high-redshift, galaxies: structure, Galaxy: evolution},
     year = 2015,
    month = may,
   volume = 805,
      eid = {34},
    pages = {34},
      doi = {10.1088/0004-637X/805/1/34},
   adsurl = {http://adsabs.harvard.edu/abs/2015ApJ...805...34M},
  adsnote = {Provided by the SAO/NASA Astrophysics Data System}
}

@Article{Moster2010,
  Title                    = {{Constraints on the Relationship between Stellar Mass and Halo Mass at Low and High Redshift}},
  Author                   = {{Moster}, B.~P. and {Somerville}, R.~S. and {Maulbetsch}, C. and {van den Bosch}, F.~C. and {Macci{\`o}}, A.~V. and {Naab}, T. and {Oser}, L.},
  Journal                  = {\apj},
  Year                     = {2010},

  Month                    = feb,
  Pages                    = {903-923},
  Volume                   = {710},

  Adsnote                  = {Provided by the SAO/NASA Astrophysics Data System},
  Adsurl                   = {http://adsabs.harvard.edu/abs/2010ApJ...710..903M},
  Archiveprefix            = {arXiv},
  Doi                      = {10.1088/0004-637X/710/2/903},
  Eprint                   = {0903.4682},
  Keywords                 = {cosmology: theory, dark matter, galaxies: clusters: general, galaxies: evolution, galaxies: halos, galaxies: high-redshift, galaxies: statistics, galaxies: stellar content, large-scale structure of universe},
  Primaryclass             = {astro-ph.CO}
}

@ARTICLE{Moster2013,
   author = {{Moster}, B.~P. and {Naab}, T. and {White}, S.~D.~M.},
    title = "{Galactic star formation and accretion histories from matching galaxies to dark matter haloes}",
  journal = {\mnras},
archivePrefix = "arXiv",
   eprint = {1205.5807},
 keywords = {galaxies: evolution, galaxies: high-redshift, galaxies: statistics, galaxies: stellar content, cosmology: theory, dark matter},
     year = 2013,
    month = feb,
   volume = 428,
    pages = {3121-3138},
      doi = {10.1093/mnras/sts261},
   adsurl = {http://adsabs.harvard.edu/abs/2013MNRAS.428.3121M},
  adsnote = {Provided by the SAO/NASA Astrophysics Data System}
}

@ARTICLE{Murata2014,
   author = {{Murata}, K.~L. and {Kajisawa}, M. and {Taniguchi}, Y. and {Kobayashi}, M.~A.~R. and 
	{Shioya}, Y. and {Capak}, P. and {Ilbert}, O. and {Koekemoer}, A.~M. and 
	{Salvato}, M. and {Scoville}, N.~Z.},
    title = "{Evolution of the Fraction of Clumpy Galaxies at 0.2 $\lt$ z $\lt$ 1.0 in the COSMOS Field}",
  journal = {\apj},
archivePrefix = "arXiv",
   eprint = {1403.1496},
 keywords = {galaxies: evolution, galaxies: irregular, galaxies: star formation},
     year = 2014,
    month = may,
   volume = 786,
      eid = {15},
    pages = {15},
      doi = {10.1088/0004-637X/786/1/15},
   adsurl = {http://adsabs.harvard.edu/abs/2014ApJ...786...15M},
  adsnote = {Provided by the SAO/NASA Astrophysics Data System}
}

@Article{NFW,
  Title                    = {{A Universal Density Profile from Hierarchical Clustering}},
  Author                   = {{Navarro}, J.~F. and {Frenk}, C.~S. and {White}, S.~D.~M.},
  Journal                  = {\apj},
  Year                     = {1997},

  Month                    = dec,
  Pages                    = {493},
  Volume                   = {490},

  Adsnote                  = {Provided by the SAO/NASA Astrophysics Data System},
  Adsurl                   = {http://adsabs.harvard.edu/abs/1997ApJ...490..493N},
  Doi                      = {10.1086/304888},
  Eprint                   = {arXiv:astro-ph/9611107},
  Keywords                 = {COSMOLOGY: THEORY, COSMOLOGY: DARK MATTER, GALAXIES: HALOS, METHODS: NUMERICAL}
}

@ARTICLE{Nataf2010,
   author = {{Nataf}, D.~M. and {Udalski}, A. and {Gould}, A. and {Fouqu{\'e}}, P. and 
	{Stanek}, K.~Z.},
    title = "{The Split Red Clump of the Galactic Bulge from OGLE-III}",
  journal = {\apjl},
archivePrefix = "arXiv",
   eprint = {1007.5065},
 keywords = {Galaxy: bulge, Galaxy: structure, stars: horizontal-branch},
     year = 2010,
    month = sep,
   volume = 721,
    pages = {L28-L32},
      doi = {10.1088/2041-8205/721/1/L28},
   adsurl = {http://adsabs.harvard.edu/abs/2010ApJ...721L..28N},
  adsnote = {Provided by the SAO/NASA Astrophysics Data System}
}

@Article{Navarro1996,
  Title                    = {{The Structure of Cold Dark Matter Halos}},
  Author                   = {{Navarro}, J.~F. and {Frenk}, C.~S. and {White}, S.~D.~M.},
  Journal                  = {\apj},
  Year                     = {1996},

  Month                    = may,
  Pages                    = {563},
  Volume                   = {462},

  Adsnote                  = {Provided by the SAO/NASA Astrophysics Data System},
  Adsurl                   = {http://adsabs.harvard.edu/abs/1996ApJ...462..563N},
  Doi                      = {10.1086/177173},
  Eprint                   = {astro-ph/9508025},
  Keywords                 = {COSMOLOGY: THEORY, COSMOLOGY: DARK MATTER, GALAXIES: HALOS, METHODS: NUMERICAL}
}

@ARTICLE{Nelson2016,
   author = {{Nelson}, E.~J. and {van Dokkum}, P.~G. and {F{\"o}rster Schreiber}, N.~M. and 
	{Franx}, M. and {Brammer}, G.~B. and {Momcheva}, I.~G. and {Wuyts}, S. and 
	{Whitaker}, K.~E. and {Skelton}, R.~E. and {Fumagalli}, M. and 
	{Hayward}, C.~C. and {Kriek}, M. and {Labb{\'e}}, I. and {Leja}, J. and 
	{Rix}, H.-W. and {Tacconi}, L.~J. and {van der Wel}, A. and 
	{van den Bosch}, F.~C. and {Oesch}, P.~A. and {Dickey}, C. and 
	{Ulf Lange}, J.},
    title = "{Where Stars Form: Inside-out Growth and Coherent Star Formation from HST H{$\alpha$} Maps of 3200 Galaxies across the Main Sequence at 0.7 $\lt$ z $\lt$ 1.5}",
  journal = {\apj},
archivePrefix = "arXiv",
   eprint = {1507.03999},
 keywords = {galaxies: evolution, galaxies: formation, galaxies: high-redshift, galaxies: star formation, galaxies: structure},
     year = 2016,
    month = sep,
   volume = 828,
      eid = {27},
    pages = {27},
      doi = {10.3847/0004-637X/828/1/27},
   adsurl = {http://adsabs.harvard.edu/abs/2016ApJ...828...27N},
  adsnote = {Provided by the SAO/NASA Astrophysics Data System}
}

@ARTICLE{Nelson2016a,
   author = {{Nelson}, E.~J. and {van Dokkum}, P.~G. and {Momcheva}, I.~G. and 
	{Brammer}, G.~B. and {Wuyts}, S. and {Franx}, M. and {F{\"o}rster Schreiber}, N.~M. and 
	{Whitaker}, K.~E. and {Skelton}, R.~E.},
    title = "{Spatially Resolved Dust Maps from Balmer Decrements in Galaxies at z \~{} 1.4}",
  journal = {\apjl},
archivePrefix = "arXiv",
   eprint = {1511.04443},
 keywords = {galaxies: evolution, galaxies: star formation, galaxies: structure},
     year = 2016,
    month = jan,
   volume = 817,
      eid = {L9},
    pages = {L9},
      doi = {10.3847/2041-8205/817/1/L9},
   adsurl = {http://adsabs.harvard.edu/abs/2016ApJ...817L...9N},
  adsnote = {Provided by the SAO/NASA Astrophysics Data System}
}

@ARTICLE{Ness2012,
   author = {{Ness}, M. and {Freeman}, K. and {Athanassoula}, E. and {Wylie-De-Boer}, E. and 
	{Bland-Hawthorn}, J. and {Lewis}, G.~F. and {Yong}, D. and {Asplund}, M. and 
	{Lane}, R.~R. and {Kiss}, L.~L. and {Ibata}, R.},
    title = "{The Origin of the Split Red Clump in the Galactic Bulge of the Milky Way}",
  journal = {\apj},
archivePrefix = "arXiv",
   eprint = {1207.0888},
 keywords = {Galaxy: abundances, Galaxy: bulge, Galaxy: kinematics and dynamics, Galaxy: structure, stars: late-type},
     year = 2012,
    month = sep,
   volume = 756,
      eid = {22},
    pages = {22},
      doi = {10.1088/0004-637X/756/1/22},
   adsurl = {http://adsabs.harvard.edu/abs/2012ApJ...756...22N},
  adsnote = {Provided by the SAO/NASA Astrophysics Data System}
}

@ARTICLE{Ness2016,
   author = {{Ness}, M. and {Hogg}, D.~W. and {Rix}, H.-W. and {Martig}, M. and 
	{Pinsonneault}, M.~H. and {Ho}, A.~Y.~Q.},
    title = "{Spectroscopic Determination of Masses (and Implied Ages) for Red Giants}",
  journal = {\apj},
archivePrefix = "arXiv",
   eprint = {1511.08204},
 primaryClass = "astro-ph.SR",
 keywords = {Galaxy: stellar content, methods: data analysis, methods: statistical, stars: evolution, stars: fundamental parameters, techniques: spectroscopic},
     year = 2016,
    month = jun,
   volume = 823,
      eid = {114},
    pages = {114},
      doi = {10.3847/0004-637X/823/2/114},
   adsurl = {http://adsabs.harvard.edu/abs/2016ApJ...823..114N},
  adsnote = {Provided by the SAO/NASA Astrophysics Data System}
}

@ARTICLE{Noguchi1998,
   author = {{Noguchi}, M.},
    title = "{Clumpy star-forming regions as the origin of the peculiar morphology of high-redshift galaxies}",
  journal = {\nat},
 keywords = {Peculiar Stars, Star Formation, Stellar Structure, Galactic Evolution, Gravitational Collapse, Galactic Bulge, Red Shift, Digital Simulation, Galactic Structure, Hubble Constant, Quasars},
     year = 1998,
    month = mar,
   volume = 392,
    pages = {253},
      doi = {10.1038/32596},
   adsurl = {http://adsabs.harvard.edu/abs/1998Natur.392..253N},
  adsnote = {Provided by the SAO/NASA Astrophysics Data System}
}

@ARTICLE{Noguchi1999,
   author = {{Noguchi}, M.},
    title = "{Early Evolution of Disk Galaxies: Formation of Bulges in Clumpy Young Galactic Disks}",
  journal = {\apj},
   eprint = {astro-ph/9806355},
 keywords = {GALAXIES: EVOLUTION, GALAXIES: FORMATION, GALAXIES: ISM, GALAXIES: KINEMATICS AND DYNAMICS, GALAXIES: STRUCTURE, Galaxies: Evolution, Galaxies: Formation, Galaxies: ISM, Galaxies: Kinematics and Dynamics, Galaxies: Structure},
     year = 1999,
    month = mar,
   volume = 514,
    pages = {77-95},
      doi = {10.1086/306932},
   adsurl = {http://adsabs.harvard.edu/abs/1999ApJ...514...77N},
  adsnote = {Provided by the SAO/NASA Astrophysics Data System}
}

@ARTICLE{Obreja2014,
   author = {{Obreja}, A. and {Brook}, C.~B. and {Stinson}, G. and {Dom{\'{\i}}nguez-Tenreiro}, R. and 
	{Gibson}, B.~K. and {Silva}, L. and {Granato}, G.~L.},
    title = "{The main sequence and the fundamental metallicity relation in MaGICC Galaxies: evolution and scatter}",
  journal = {\mnras},
archivePrefix = "arXiv",
   eprint = {1404.0043},
 keywords = {galaxies: abundances, galaxies: evolution, galaxies: formation, galaxies: star formation, galaxies: stellar content},
     year = 2014,
    month = aug,
   volume = 442,
    pages = {1794-1804},
      doi = {10.1093/mnras/stu891},
   adsurl = {http://adsabs.harvard.edu/abs/2014MNRAS.442.1794O},
  adsnote = {Provided by the SAO/NASA Astrophysics Data System}
}

@ARTICLE{Obreja2016,
   author = {{Obreja}, A. and {Stinson}, G.~S. and {Dutton}, A.~A. and {Macci{\`o}}, A.~V. and 
	{Wang}, L. and {Kang}, X.},
    title = "{NIHAO VI. The hidden discs of simulated galaxies}",
  journal = {\mnras},
archivePrefix = "arXiv",
   eprint = {1603.01703},
 keywords = {methods: numerical, galaxies: fundamental parameters, galaxies: kinematics and dynamics, galaxies: stellar content, galaxies: structure},
     year = 2016,
    month = jun,
   volume = 459,
    pages = {467-486},
      doi = {10.1093/mnras/stw690},
   adsurl = {http://adsabs.harvard.edu/abs/2016MNRAS.459..467O},
  adsnote = {Provided by the SAO/NASA Astrophysics Data System}
}

@ARTICLE{Oklopcic2016,
   author = {{Oklopcic}, A. and {Hopkins}, P.~F. and {Feldmann}, R. and {Keres}, D. and 
	{Faucher-Giguere}, C.-A. and {Murray}, N.},
    title = "{Giant clumps in the FIRE simulations: a case study of a massive high-redshift galaxy}",
  journal = {ArXiv e-prints},
archivePrefix = "arXiv",
   eprint = {1603.03778},
 keywords = {Astrophysics - Astrophysics of Galaxies},
     year = 2016,
    month = mar,
   adsurl = {http://adsabs.harvard.edu/abs/2016arXiv160303778O},
  adsnote = {Provided by the SAO/NASA Astrophysics Data System}
}

@ARTICLE{Overzier2009,
   author = {{Overzier}, R.~A. and {Heckman}, T.~M. and {Tremonti}, C. and 
	{Armus}, L. and {Basu-Zych}, A. and {Gon{\c c}alves}, T. and 
	{Rich}, R.~M. and {Martin}, D.~C. and {Ptak}, A. and {Schiminovich}, D. and 
	{Ford}, H.~C. and {Madore}, B. and {Seibert}, M.},
    title = "{Local Lyman Break Galaxy Analogs: The Impact of Massive Star-Forming Clumps on the Interstellar Medium and the Global Structure of Young, Forming Galaxies}",
  journal = {\apj},
archivePrefix = "arXiv",
   eprint = {0910.1352},
 keywords = {galaxies: active, galaxies: bulges, galaxies: high-redshift, galaxies: peculiar, galaxies: starburst},
     year = 2009,
    month = nov,
   volume = 706,
    pages = {203-222},
      doi = {10.1088/0004-637X/706/1/203},
   adsurl = {http://adsabs.harvard.edu/abs/2009ApJ...706..203O},
  adsnote = {Provided by the SAO/NASA Astrophysics Data System}
}

@ARTICLE{Patel2013,
   author = {{Patel}, S.~G. and {van Dokkum}, P.~G. and {Franx}, M. and {Quadri}, R.~F. and 
	{Muzzin}, A. and {Marchesini}, D. and {Williams}, R.~J. and 
	{Holden}, B.~P. and {Stefanon}, M.},
    title = "{HST/WFC3 Confirmation of the Inside-out Growth of Massive Galaxies at 0 $\lt$ z $\lt$ 2 and Identification of Their Star-forming Progenitors at z \~{} 3}",
  journal = {\apj},
archivePrefix = "arXiv",
   eprint = {1208.0341},
 keywords = {galaxies: structure, galaxies: evolution, galaxies: formation},
     year = 2013,
    month = mar,
   volume = 766,
      eid = {15},
    pages = {15},
      doi = {10.1088/0004-637X/766/1/15},
   adsurl = {http://adsabs.harvard.edu/abs/2013ApJ...766...15P},
  adsnote = {Provided by the SAO/NASA Astrophysics Data System}
}

@ARTICLE{Pawlowski2012,
   author = {{Pawlowski}, M.~S. and {Kroupa}, P. and {Angus}, G. and {de Boer}, K.~S. and 
	{Famaey}, B. and {Hensler}, G.},
    title = "{Filamentary accretion cannot explain the orbital poles of the Milky Way satellites}",
  journal = {\mnras},
archivePrefix = "arXiv",
   eprint = {1204.6039},
 keywords = {galaxies: dwarf, galaxies: formation, galaxies: kinematics and dynamics, Local Group, dark matter},
     year = 2012,
    month = jul,
   volume = 424,
    pages = {80-92},
      doi = {10.1111/j.1365-2966.2012.21169.x},
   adsurl = {http://adsabs.harvard.edu/abs/2012MNRAS.424...80P},
  adsnote = {Provided by the SAO/NASA Astrophysics Data System}
}

@Article{Pawlowski2013,
  Title                    = {{Dwarf galaxy planes: the discovery of symmetric structures in the Local Group}},
  Author                   = {{Pawlowski}, M.~S. and {Kroupa}, P. and {Jerjen}, H.},
  Journal                  = {\mnras},
  Year                     = {2013},

  Month                    = nov,
  Pages                    = {1928-1957},
  Volume                   = {435},

  Adsnote                  = {Provided by the SAO/NASA Astrophysics Data System},
  Adsurl                   = {http://adsabs.harvard.edu/abs/2013MNRAS.435.1928P},
  Archiveprefix            = {arXiv},
  Doi                      = {10.1093/mnras/stt1384},
  Eprint                   = {1307.6210},
  Keywords                 = {Galaxy: halo, galaxies: dwarf, galaxies: individual: M31, galaxies: kinematics and dynamics, Local Group, Magellanic Clouds},
  Primaryclass             = {astro-ph.CO}
}

@Article{Pawlowski2014,
  Title                    = {{The Vast Polar Structure of the Milky Way Attains New Members}},
  Author                   = {{Pawlowski}, M.~S. and {Kroupa}, P.},
  Journal                  = {\apj},
  Year                     = {2014},

  Month                    = jul,
  Pages                    = {74},
  Volume                   = {790},

  Adsnote                  = {Provided by the SAO/NASA Astrophysics Data System},
  Adsurl                   = {http://adsabs.harvard.edu/abs/2014ApJ...790...74P},
  Archiveprefix            = {arXiv},
  Doi                      = {10.1088/0004-637X/790/1/74},
  Eid                      = {74},
  Eprint                   = {1407.2612},
  Keywords                 = {galaxies: individual: Crater, galaxies: kinematics and dynamics, Galaxy: halo, Galaxy: structure, globular clusters: individual: PSO J174.0675-10.8774, Local Group }
}

@Article{Pawlowski2014a,
  Title                    = {{Co-orbiting satellite galaxy structures are still in conflict with the distribution of primordial dwarf galaxies}},
  Author                   = {{Pawlowski}, M.~S. and {Famaey}, B. and {Jerjen}, H. and {Merritt}, D. and {Kroupa}, P. and {Dabringhausen}, J. and {L{\"u}ghausen}, F. and {Forbes}, D.~A. and {Hensler}, G. and {Hammer}, F. and {Puech}, M. and {Fouquet}, S. and {Flores}, H. and {Yang}, Y.},
  Journal                  = {\mnras},
  Year                     = {2014},

  Month                    = aug,
  Pages                    = {2362-2380},
  Volume                   = {442},

  Adsnote                  = {Provided by the SAO/NASA Astrophysics Data System},
  Adsurl                   = {http://adsabs.harvard.edu/abs/2014MNRAS.442.2362P},
  Archiveprefix            = {arXiv},
  Doi                      = {10.1093/mnras/stu1005},
  Eprint                   = {1406.1799},
  Keywords                 = {methods: data analysis, galaxies: dwarf, galaxies: kinematics and dynamics, Local Group, dark matter}
}

@ARTICLE{Pawlowski2014b,
   author = {{Pawlowski}, M.~S. and {McGaugh}, S.~S.},
    title = "{Co-orbiting Planes of Sub-halos are Similarly Unlikely around Paired and Isolated Hosts}",
  journal = {\apjl},
archivePrefix = "arXiv",
   eprint = {1406.6062},
 keywords = {dark matter, galaxies: kinematics and dynamics, Galaxy: halo, Galaxy: structure, Local Group},
     year = 2014,
    month = jul,
   volume = 789,
      eid = {L24},
    pages = {L24},
      doi = {10.1088/2041-8205/789/1/L24},
   adsurl = {http://adsabs.harvard.edu/abs/2014ApJ...789L..24P},
  adsnote = {Provided by the SAO/NASA Astrophysics Data System}
}

@Article{Pawlowski2015,
  Title                    = {{The new Milky Way satellites: alignment with the VPOS and predictions for proper motions and velocity dispersions}},
  Author                   = {{Pawlowski}, M.~S. and {McGaugh}, S.~S. and {Jerjen}, H.},
  Journal                  = {\mnras},
  Year                     = {2015},

  Month                    = oct,
  Pages                    = {1047-1061},
  Volume                   = {453},

  Adsnote                  = {Provided by the SAO/NASA Astrophysics Data System},
  Adsurl                   = {http://adsabs.harvard.edu/abs/2015MNRAS.453.1047P},
  Archiveprefix            = {arXiv},
  Doi                      = {10.1093/mnras/stv1588},
  Eprint                   = {1505.07465},
  Keywords                 = {Galaxy: halo, galaxies: dwarf, galaxies: kinematics and dynamics, Local Group, Magellanic Clouds}
}

@Article{CMB,
  Title                    = {{A Measurement of Excess Antenna Temperature at 4080 Mc/s.}},
  Author                   = {{Penzias}, A.~A. and {Wilson}, R.~W.},
  Journal                  = {\apj},
  Year                     = {1965},

  Month                    = jul,
  Pages                    = {419-421},
  Volume                   = {142},

  Adsnote                  = {Provided by the SAO/NASA Astrophysics Data System},
  Adsurl                   = {http://adsabs.harvard.edu/abs/1965ApJ...142..419P},
  Doi                      = {10.1086/148307},
  Keywords                 = {Microwave Background, Cosmic Background Radiation}
}

@ARTICLE{Penarrubia2014,
   author = {{Pe{\~n}arrubia}, J. and {Ma}, Y.-Z. and {Walker}, M.~G. and 
	{McConnachie}, A.},
    title = "{A dynamical model of the local cosmic expansion}",
  journal = {\mnras},
archivePrefix = "arXiv",
   eprint = {1405.0306},
 keywords = {Galaxy: fundamental parameters, Galaxy: kinematics and dynamics, Local Group, cosmological parameters, dark energy, dark matter},
     year = 2014,
    month = sep,
   volume = 443,
    pages = {2204-2222},
      doi = {10.1093/mnras/stu879},
   adsurl = {http://adsabs.harvard.edu/abs/2014MNRAS.443.2204P},
  adsnote = {Provided by the SAO/NASA Astrophysics Data System}
}

@Article{Penzo2014,
  Title                    = {{Dark MaGICC: the effect of dark energy on disc galaxy formation. Cosmology does matter}},
  Author                   = {{Penzo}, C. and {Macci{\`o}}, A.~V. and {Casarini}, L. and {Stinson}, G.~S. and {Wadsley}, J.},
  Journal                  = {\mnras},
  Year                     = {2014},

  Month                    = jul,
  Pages                    = {176-186},
  Volume                   = {442},

  Adsnote                  = {Provided by the SAO/NASA Astrophysics Data System},
  Adsurl                   = {http://adsabs.harvard.edu/abs/2014MNRAS.442..176P},
  Archiveprefix            = {arXiv},
  Doi                      = {10.1093/mnras/stu857},
  Eprint                   = {1401.3338},
  Keywords                 = {hydrodynamics, methods: numerical, galaxies: formation, galaxies: spiral, galaxies: star formation, dark energy},
  Primaryclass             = {astro-ph.CO}
}

@ARTICLE{Perret2014,
   author = {{Perret}, V. and {Renaud}, F. and {Epinat}, B. and {Amram}, P. and 
	{Bournaud}, F. and {Contini}, T. and {Teyssier}, R. and {Lambert}, J.-C.
	},
    title = "{Evolution of the mass, size, and star formation rate in high redshift merging galaxies. MIRAGE - A new sample of simulations with detailed stellar feedback}",
  journal = {\aap},
archivePrefix = "arXiv",
   eprint = {1307.7130},
 keywords = {galaxies: evolution, galaxies: formation, galaxies: high-redshift, galaxies: star formation, galaxies: interactions, methods: numerical},
     year = 2014,
    month = feb,
   volume = 562,
      eid = {A1},
    pages = {A1},
      doi = {10.1051/0004-6361/201322395},
   adsurl = {http://adsabs.harvard.edu/abs/2014Aadsnote = {Provided by the SAO/NASA Astrophysics Data System}
}

@ARTICLE{Perez2013,
   author = {{Perez}, J. and {Valenzuela}, O. and {Tissera}, P.~B. and {Michel-Dansac}, L.
	},
    title = "{Clumpy disc and bulge formation}",
  journal = {\mnras},
archivePrefix = "arXiv",
   eprint = {1308.4396},
 keywords = {galaxies: bulges, galaxies: evolution, galaxies: formation, galaxies: interactions},
     year = 2013,
    month = nov,
   volume = 436,
    pages = {259-265},
      doi = {10.1093/mnras/stt1563},
   adsurl = {http://adsabs.harvard.edu/abs/2013MNRAS.436..259P},
  adsnote = {Provided by the SAO/NASA Astrophysics Data System}
}

@Article{Perlmutter1999,
  Title                    = {{Measurements of Omega and Lambda from 42 High-Redshift Supernovae}},
  Author                   = {{Perlmutter}, S. and {Aldering}, G. and {Goldhaber}, G. and {Knop}, R.~A. and {Nugent}, P. and {Castro}, P.~G. and {Deustua}, S. and {Fabbro}, S. and {Goobar}, A. and {Groom}, D.~E. and {Hook}, I.~M. and {Kim}, A.~G. and {Kim}, M.~Y. and {Lee}, J.~C. and {Nunes}, N.~J. and {Pain}, R. and {Pennypacker}, C.~R. and {Quimby}, R. and {Lidman}, C. and {Ellis}, R.~S. and {Irwin}, M. and {McMahon}, R.~G. and {Ruiz-Lapuente}, P. and {Walton}, N. and {Schaefer}, B. and {Boyle}, B.~J. and {Filippenko}, A.~V. and {Matheson}, T. and {Fruchter}, A.~S. and {Panagia}, N. and {Newberg}, H.~J.~M. and {Couch}, W.~J. and {Supernova Cosmology Project}},
  Journal                  = {\apj},
  Year                     = {1999},

  Month                    = jun,
  Pages                    = {565-586},
  Volume                   = {517},

  Abstract                 = {{We report measurements of the mass density, Omega\_M, and cosmological-constant energy density, Omega\_Lambda, of the universe based on the analysis of 42 type Ia supernovae discovered by the Supernova Cosmology Project. The magnitude-redshift data for these supernovae, at redshifts between 0.18 and 0.83, are fitted jointly with a set of supernovae from the Cal{\'a}n/Tololo Supernova Survey, at redshifts below 0.1, to yield values for the cosmological parameters. All supernova peak magnitudes are standardized using a SN Ia light-curve width-luminosity relation. The measurement yields a joint probability distribution of the cosmological parameters that is approximated by the relation 0.8Omega\_M-0.6Omega\_Lambda\~{}-0.2+/-0.1 in the region of interest (Omega\_M{\lt}\~{}1.5). For a flat (Omega\_M+Omega\_Lambda=1) cosmology we find Omega\^{}flat\_M=0.28\^{}+0.09\_-0.08 (1 sigma statistical) \^{}+0.05\_-0.04 (identified systematics). The data are strongly inconsistent with a Lambda=0 flat cosmology, the simplest inflationary universe model. An open, Lambda=0 cosmology also does not fit the data well: the data indicate that the cosmological constant is nonzero and positive, with a confidence of P(Lambda{\gt}0)=99\%, including the identified systematic uncertainties. The best-fit age of the universe relative to the Hubble time is t\^{}flat\_0=14.9\^{}+1.4\_-1.1(0.63/h) Gyr for a flat cosmology. The size of our sample allows us to perform a variety of statistical tests to check for possible systematic errors and biases. We find no significant differences in either the host reddening distribution or Malmquist bias between the low-redshift Cal{\'a}n/Tololo sample and our high-redshift sample. Excluding those few supernovae that are outliers in color excess or fit residual does not significantly change the results. The conclusions are also robust whether or not a width-luminosity relation is used to standardize the supernova peak magnitudes. We discuss and constrain, where possible, hypothetical alternatives to a cosmological constant. }},
  Adsnote                  = {Provided by the SAO/NASA Astrophysics Data System},
  Adsurl                   = {http://adsabs.harvard.edu/abs/1999ApJ...517..565P},
  Doi                      = {10.1086/307221},
  Eprint                   = {arXiv:astro-ph/9812133},
  Keywords                 = {COSMOLOGY: OBSERVATIONS, COSMOLOGY: DISTANCE SCALE, STARS: SUPERNOVAE: GENERAL}
}

@InProceedings{Perlmutter2003,
  Title                    = {{Measuring Cosmology with Supernovae}},
  Author                   = {{Perlmutter}, S. and {Schmidt}, B.~P.},
  Booktitle                = {Supernovae and Gamma-Ray Bursters},
  Year                     = {2003},
  Editor                   = {{Weiler}, K.},
  Pages                    = {195-217},
  Series                   = {Lecture Notes in Physics, Berlin Springer Verlag},
  Volume                   = {598},

  Abstract                 = {{Over the past decade, supernovae have emerged as some of the most powerful tools for measuring extragalactic distances. A well developed physical understanding of type II supernovae allow them to be used to measure distances independent of the extragalactic distance scale. Type Ia supernovae are empirical tools whose precision and intrinsic brightness make them sensitive probes of the cosmological expansion. Both types of supernovae are consistent with a Hubble Constant within char 12610\% of H0 = 70 km s-1Mpc-1. Two teams have used type Ia supernovae to trace the expansion of the Universe to a look-back time more than 60\% of the age of the Universe. These observations show an accelerating Universe which is currently best explained by a cosmological constant or other form of dark energy with an equation of state near w = p/rho = -1. While there are many possible remaining systematic effects, none appears large enough to challenge these current results. Future experiments are planned to better character ize the equation of state of the dark energy leading to the observed acceleration by observing hundreds or even thousands of objects. These experiments will need to carefully control systematic errors to ensure future conclusions are not dominated by effects unrelated to cosmology. }},
  Adsnote                  = {Provided by the SAO/NASA Astrophysics Data System},
  Adsurl                   = {http://adsabs.harvard.edu/abs/2003LNP...598..195P},
  Eprint                   = {arXiv:astro-ph/0303428}
}

@ARTICLE{Phillips2015,
   author = {{Phillips}, J.~I. and {Cooper}, M.~C. and {Bullock}, J.~S. and 
	{Boylan-Kolchin}, M.},
    title = "{Are rotating planes of satellite galaxies ubiquitous?}",
  journal = {\mnras},
archivePrefix = "arXiv",
   eprint = {1505.05876},
 keywords = {galaxies: dwarf, galaxies: evolution, galaxies: formation, Local Group, galaxies: star formation},
     year = 2015,
    month = nov,
   volume = 453,
    pages = {3839-3847},
      doi = {10.1093/mnras/stv1770},
   adsurl = {http://adsabs.harvard.edu/abs/2015MNRAS.453.3839P},
  adsnote = {Provided by the SAO/NASA Astrophysics Data System}
}

@Article{Planck,
  Title                    = {{Planck 2013 results. XVI. Cosmological parameters}},
  Author                   = {{Planck Collaboration} and {Ade}, P.~A.~R. and {Aghanim}, N. and {Armitage-Caplan}, C. and {Arnaud}, M. and {Ashdown}, M. and {Atrio-Barandela}, F. and {Aumont}, J. and {Baccigalupi}, C. and {Banday}, A.~J. and et al.},
  Journal                  = {\aap},
  Year                     = {2014},

  Month                    = nov,
  Pages                    = {A16},
  Volume                   = {571},

  Adsnote                  = {Provided by the SAO/NASA Astrophysics Data System},
  Adsurl                   = {http://adsabs.harvard.edu/abs/2014AArchiveprefix            = {arXiv},
  Doi                      = {10.1051/0004-6361/201321591},
  Eid                      = {A16},
  Eprint                   = {1303.5076},
  Keywords                 = {cosmic background radiation, cosmological parameters, early Universe, inflation, primordial nucleosynthesis},
  Primaryclass             = {astro-ph.CO}
}

@Article{Power2003,
  Title                    = {{The inner structure of {$\Lambda$}CDM haloes - I. A numerical convergence study}},
  Author                   = {{Power}, C. and {Navarro}, J.~F. and {Jenkins}, A. and {Frenk}, C.~S. and {White}, S.~D.~M. and {Springel}, V. and {Stadel}, J. and {Quinn}, T. },
  Journal                  = {\mnras},
  Year                     = {2003},

  Month                    = jan,
  Pages                    = {14-34},
  Volume                   = {338},

  Adsnote                  = {Provided by the SAO/NASA Astrophysics Data System},
  Adsurl                   = {http://adsabs.harvard.edu/abs/2003MNRAS.338...14P},
  Doi                      = {10.1046/j.1365-8711.2003.05925.x},
  Eprint                   = {astro-ph/0201544},
  Keywords                 = {gravitation, cosmology: theory, dark matter}
}

@ARTICLE{Puech2010,
   author = {{Puech}, M.},
    title = "{Clumpy galaxies at z \~{} 0.6: kinematics, stability and comparison with analogues at other redshifts}",
  journal = {\mnras},
archivePrefix = "arXiv",
   eprint = {1003.3116},
 keywords = {galaxies: evolution, galaxies: general, galaxies: high-redshift, galaxies: interactions, galaxies: kinematics and dynamics, galaxies: spiral},
     year = 2010,
    month = jul,
   volume = 406,
    pages = {535-547},
      doi = {10.1111/j.1365-2966.2010.16689.x},
   adsurl = {http://adsabs.harvard.edu/abs/2010MNRAS.406..535P},
  adsnote = {Provided by the SAO/NASA Astrophysics Data System}
}

@Article{Press1974,
  Title                    = {{Formation of Galaxies and Clusters of Galaxies by Self-Similar Gravitational Condensation}},
  Author                   = {{Press}, W.~H. and {Schechter}, P.},
  Journal                  = {\apj},
  Year                     = {1974},

  Month                    = feb,
  Pages                    = {425-438},
  Volume                   = {187},

  Abstract                 = {{We consider an expanding Friedmann cosmology containing a ''gas'' of self-gravitating masses. The masses condense into aggregates which (when sufficiently bound) we identify as single particles of a larger mass. We propose that after this process has proceeded through several scales, the mass spectrum of condensations becomes ''self-similar'' and independent of the spectrum initially assumed. Some details of the self-similar distribution, and its evolution in time, can be calculated with the linear perturbation theory. Unlike other authors, we make no ad hoc assumptions about the spectrum of long-wavelength initial perturbatidns: the nonlinear N-body interactions of the mass points randomize their positions and generate a perturbation to all larger scales; this should fix the self-similar distribution almost uniquely. The results of numerical experiments on 1000 bodies are presented; these appear to show new nonlinear effects: condensations can ''bootstrap'' their way up in size faster than the linear theory predicts. Our self-similar model predicts relations between the masses and radii of galaxies and clusters of galaxies, as well as their mass spectra. We compare the predictions with available data, and find some rather striking agreements. If the model is to explain galaxies, then isothermal ''seed'' masses of 3 x 1 0 M0 must have existed at recombination. To explain clusters of galaxies, the only necessary seeds are the galaxies themselves. The size of clusters determines, in principle, the deceleration parameter q0 presently available data give only very broad limits, unfortunately. Subject headings: cosmology - galaxies - galaxies, clusters of }},
  Adsnote                  = {Provided by the SAO/NASA Astrophysics Data System},
  Adsurl                   = {http://adsabs.harvard.edu/abs/1974ApJ...187..425P},
  Doi                      = {10.1086/152650}
}

@ARTICLE{Price2008,
   author = {{Price}, D.~J.},
    title = "{Modelling discontinuities and Kelvin Helmholtz instabilities in SPH}",
  journal = {Journal of Computational Physics},
archivePrefix = "arXiv",
   eprint = {0709.2772},
     year = 2008,
    month = dec,
   volume = 227,
    pages = {10040-10057},
      doi = {10.1016/j.jcp.2008.08.011},
   adsurl = {http://adsabs.harvard.edu/abs/2008JCoPh.22710040P},
  adsnote = {Provided by the SAO/NASA Astrophysics Data System}
}

@misc{pynbody,
  author = {{Pontzen}, A. and {Ro{\v s}kar}, R. and {Stinson}, G.~S. and {Woods},
     R. and {Reed}, D.~M. and {Coles}, J. and {Quinn}, T.~R.},
  title = "{pynbody: Astrophysics Simulation Analysis for Python}",
  note = {Astrophysics Source Code Library, ascl:1305.002},
  year = 2013
}

@Article{Rees1986,
  Title                    = {{Lyman absorption lines in quasar spectra - Evidence for gravitationally-confined gas in dark minihaloes}},
  Author                   = {{Rees}, M.~J.},
  Journal                  = {\mnras},
  Year                     = {1986},

  Month                    = jan,
  Pages                    = {25P-30P},
  Volume                   = {218},

  Adsnote                  = {Provided by the SAO/NASA Astrophysics Data System},
  Adsurl                   = {http://adsabs.harvard.edu/abs/1986MNRAS.218P..25R},
  Keywords                 = {Astronomical Spectroscopy, Interstellar Gas, Line Spectra, Missing Mass (Astrophysics), Quasars, Absorption Spectra, Gravitational Effects, Lyman Spectra, Photoionization}
}

@Article{Riess1998,
  Title                    = {{Observational Evidence from Supernovae for an Accelerating Universe and a Cosmological Constant}},
  Author                   = {{Riess}, A.~G. and {Filippenko}, A.~V. and {Challis}, P. and {Clocchiatti}, A. and {Diercks}, A. and {Garnavich}, P.~M. and {Gilliland}, R.~L. and {Hogan}, C.~J. and {Jha}, S. and {Kirshner}, R.~P. and {Leibundgut}, B. and {Phillips}, M.~M. and {Reiss}, D. and {Schmidt}, B.~P. and {Schommer}, R.~A. and {Smith}, R.~C. and {Spyromilio}, J. and {Stubbs}, C. and {Suntzeff}, N.~B. and {Tonry}, J.},
  Journal                  = {\aj},
  Year                     = {1998},

  Month                    = sep,
  Pages                    = {1009-1038},
  Volume                   = {116},

  Abstract                 = {{We present spectral and photometric observations of 10 Type Ia supernovae (SNe Ia) in the redshift range 0.16 {\lt}= z {\lt}= 0.62. The luminosity distances of these objects are determined by methods that employ relations between SN Ia luminosity and light curve shape. Combined with previous data from our High-z Supernova Search Team and recent results by Riess et al., this expanded set of 16 high-redshift supernovae and a set of 34 nearby supernovae are used to place constraints on the following cosmological parameters: the Hubble constant (H\_0), the mass density (Omega\_M), the cosmological constant (i.e., the vacuum energy density, Omega\_Lambda), the deceleration parameter (q\_0), and the dynamical age of the universe (t\_0). The distances of the high-redshift SNe Ia are, on average, 10\%-15\% farther than expected in a low mass density (Omega\_M = 0.2) universe without a cosmological constant. Different light curve fitting methods, SN Ia subsamples, and prior constraints unanimously favor eternally expanding models with positive cosmological constant (i.e., Omega\_Lambda {\gt} 0) and a current acceleration of the expansion (i.e., q\_0 {\lt} 0). With no prior constraint on mass density other than Omega\_M {\gt}= 0, the spectroscopically confirmed SNe Ia are statistically consistent with q\_0 {\lt} 0 at the 2.8 sigma and 3.9 sigma confidence levels, and with Omega\_Lambda {\gt} 0 at the 3.0 sigma and 4.0 sigma confidence levels, for two different fitting methods, respectively. Fixing a ``minimal'' mass density, Omega\_M = 0.2, results in the weakest detection, Omega\_Lambda {\gt} 0 at the 3.0 sigma confidence level from one of the two methods. For a flat universe prior (Omega\_M + Omega\_Lambda = 1), the spectroscopically confirmed SNe Ia require Omega\_Lambda {\gt} 0 at 7 sigma and 9 sigma formal statistical significance for the two different fitting methods. A universe closed by ordinary matter (i.e., Omega\_M = 1) is formally ruled out at the 7 sigma to 8 sigma confidence level for the two different fitting methods. We estimate the dynamical age of the universe to be 14.2 +/- 1.7 Gyr including systematic uncertainties in the current Cepheid distance scale. We estimate the likely effect of several sources of systematic error, including progenitor and metallicity evolution, extinction, sample selection bias, local perturbations in the expansion rate, gravitational lensing, and sample contamination. Presently, none of these effects appear to reconcile the data with Omega\_Lambda = 0 and q\_0 {\gt}= 0. }},
  Adsnote                  = {Provided by the SAO/NASA Astrophysics Data System},
  Adsurl                   = {http://adsabs.harvard.edu/abs/1998AJ....116.1009R},
  Doi                      = {10.1086/300499},
  Eprint                   = {arXiv:astro-ph/9805201},
  Keywords                 = {COSMOLOGY: OBSERVATIONS, STARS: SUPERNOVAE: GENERAL}
}

@ARTICLE{Ritchie2001,
   author = {{Ritchie}, B.~W. and {Thomas}, P.~A.},
    title = "{Multiphase smoothed-particle hydrodynamics}",
  journal = {\mnras},
   eprint = {astro-ph/0005357},
 keywords = {HYDRODYNAMICS, METHODS: NUMERICAL, COOLING FLOWS, GALAXIES: FORMATION},
     year = 2001,
    month = may,
   volume = 323,
    pages = {743-756},
      doi = {10.1046/j.1365-8711.2001.04268.x},
   adsurl = {http://adsabs.harvard.edu/abs/2001MNRAS.323..743R},
  adsnote = {Provided by the SAO/NASA Astrophysics Data System}
}

@ARTICLE{Romeo2010,
   author = {{Romeo}, A.~B. and {Burkert}, A. and {Agertz}, O.},
    title = "{A Toomre-like stability criterion for the clumpy and turbulent interstellar medium}",
  journal = {\mnras},
archivePrefix = "arXiv",
   eprint = {1001.4732},
 keywords = {instabilities, turbulence, ISM: general, ISM: kinematics and dynamics, ISM: structure, galaxies: ISM},
     year = 2010,
    month = sep,
   volume = 407,
    pages = {1223-1230},
      doi = {10.1111/j.1365-2966.2010.16975.x},
   adsurl = {http://adsabs.harvard.edu/abs/2010MNRAS.407.1223R},
  adsnote = {Provided by the SAO/NASA Astrophysics Data System}
}

@ARTICLE{Romeo2011,
   author = {{Romeo}, A.~B. and {Wiegert}, J.},
    title = "{The effective stability parameter for two-component galactic discs: is Q$^{-1}$ {\ap} Q$^{-1}$$_{stars}$ + Q$^{-1}$$_{gas}$?}",
  journal = {\mnras},
archivePrefix = "arXiv",
   eprint = {1101.4519},
 keywords = {instabilities, stars: kinematics and dynamics, ISM: kinematics and dynamics, galaxies: ISM, galaxies: kinematics and dynamics, galaxies: star formation},
     year = 2011,
    month = sep,
   volume = 416,
    pages = {1191-1196},
      doi = {10.1111/j.1365-2966.2011.19120.x},
   adsurl = {http://adsabs.harvard.edu/abs/2011MNRAS.416.1191R},
  adsnote = {Provided by the SAO/NASA Astrophysics Data System}
}

@ARTICLE{Saitoh2009,
   author = {{Saitoh}, T.~R. and {Makino}, J.},
    title = "{A Necessary Condition for Individual Time Steps in SPH Simulations}",
  journal = {\apjl},
archivePrefix = "arXiv",
   eprint = {0808.0773},
 keywords = {galaxies: evolution, galaxies: formation, galaxies: ISM, methods: numerical},
     year = 2009,
    month = jun,
   volume = 697,
    pages = {L99-L102},
      doi = {10.1088/0004-637X/697/2/L99},
   adsurl = {http://adsabs.harvard.edu/abs/2009ApJ...697L..99S},
  adsnote = {Provided by the SAO/NASA Astrophysics Data System}
}

@ARTICLE{Salim2007,
   author = {{Salim}, S. and {Rich}, R.~M. and {Charlot}, S. and {Brinchmann}, J. and 
	{Johnson}, B.~D. and {Schiminovich}, D. and {Seibert}, M. and 
	{Mallery}, R. and {Heckman}, T.~M. and {Forster}, K. and {Friedman}, P.~G. and 
	{Martin}, D.~C. and {Morrissey}, P. and {Neff}, S.~G. and {Small}, T. and 
	{Wyder}, T.~K. and {Bianchi}, L. and {Donas}, J. and {Lee}, Y.-W. and 
	{Madore}, B.~F. and {Milliard}, B. and {Szalay}, A.~S. and {Welsh}, B.~Y. and 
	{Yi}, S.~K.},
    title = "{UV Star Formation Rates in the Local Universe}",
  journal = {\apjs},
archivePrefix = "arXiv",
   eprint = {0704.3611},
 keywords = {Galaxies: Active, Galaxies: Evolution, Galaxies: Fundamental Parameters, Surveys, Ultraviolet: Galaxies},
     year = 2007,
    month = dec,
   volume = 173,
    pages = {267-292},
      doi = {10.1086/519218},
   adsurl = {http://adsabs.harvard.edu/abs/2007ApJS..173..267S},
  adsnote = {Provided by the SAO/NASA Astrophysics Data System}
}

@ARTICLE{Santos-Santos2017,
   author = {{Santos-Santos}, I.~M. and {Di Cintio}, A. and {Brook}, C.~B. and 
	{Macci{\`o}}, A. and {Dutton}, A. and {Dom{\'{\i}}nguez-Tenreiro}, R.
	},
    title = "{NIHAO XIV: Reproducing the observed diversity of dwarf galaxy rotation curve shapes in LCDM}",
  journal = {ArXiv e-prints},
archivePrefix = "arXiv",
   eprint = {1706.04202},
 keywords = {Astrophysics - Astrophysics of Galaxies},
     year = 2017,
    month = jun,
   adsurl = {http://adsabs.harvard.edu/abs/2017arXiv170604202S},
  adsnote = {Provided by the SAO/NASA Astrophysics Data System}
}

@Article{Sawala2014,
  Title                    = {{The chosen few: the low mass halos that host faint galaxies}},
  Author                   = {{Sawala}, T. and {Frenk}, C.~S. and {Fattahi}, A. and {Navarro}, J.~F. and {Theuns}, T. and {Bower}, R.~G. and {Crain}, R.~A. and {Furlong}, M. and {Jenkins}, A. and {Schaller}, M. and {Schaye}, J.},
  Journal                  = {ArXiv e-prints},
  Year                     = {2014},

  Month                    = jun,

  Adsnote                  = {Provided by the SAO/NASA Astrophysics Data System},
  Adsurl                   = {http://adsabs.harvard.edu/abs/2014arXiv1406.6362S},
  Archiveprefix            = {arXiv},
  Eprint                   = {1406.6362},
  Keywords                 = {Astrophysics - Cosmology and Nongalactic Astrophysics, Astrophysics - Astrophysics of Galaxies}
}

@Article{Schneider2012,
  Title                    = {{Non-linear evolution of cosmological structures in warm dark matter models}},
  Author                   = {{Schneider}, A. and {Smith}, R.~E. and {Macci{\`o}}, A.~V. and {Moore}, B.},
  Journal                  = {\mnras},
  Year                     = {2012},

  Month                    = jul,
  Pages                    = {684-698},
  Volume                   = {424},

  Adsnote                  = {Provided by the SAO/NASA Astrophysics Data System},
  Adsurl                   = {http://adsabs.harvard.edu/abs/2012MNRAS.424..684S},
  Archiveprefix            = {arXiv},
  Doi                      = {10.1111/j.1365-2966.2012.21252.x},
  Eprint                   = {1112.0330},
  Keywords                 = {cosmology: theory, dark matter, large-scale structure of Universe},
  Primaryclass             = {astro-ph.CO}
}

@Book{Schneider,
  Title                    = {{Extragalactic Astronomy and Cosmology}},
  Author                   = {{Schneider}, P.},
  Year                     = {2006},

  Adsnote                  = {Provided by the SAO/NASA Astrophysics Data System},
  Adsurl                   = {http://adsabs.harvard.edu/abs/2006eac..book.....S},
  Booktitle                = {Extragalactic Astronomy and Cosmology, by Peter Schneider.~Berlin: Springer, 2006.}
}

@Article{Shandarin1989,
  Title                    = {{The large-scale structure of the universe: Turbulence, intermittency, structures in a self-gravitating medium}},
  Author                   = {{Shandarin}, S.~F. and {Zeldovich}, Y.~B.},
  Journal                  = {Reviews of Modern Physics},
  Year                     = {1989},

  Month                    = apr,
  Pages                    = {185-220},
  Volume                   = {61},

  Adsnote                  = {Provided by the SAO/NASA Astrophysics Data System},
  Adsurl                   = {http://adsabs.harvard.edu/abs/1989RvMP...61..185S},
  Doi                      = {10.1103/RevModPhys.61.185}
}

@ARTICLE{Shen2010,
   author = {{Shen}, S. and {Wadsley}, J. and {Stinson}, G.},
    title = "{The enrichment of the intergalactic medium with adiabatic feedback - I. Metal cooling and metal diffusion}",
  journal = {\mnras},
archivePrefix = "arXiv",
   eprint = {0910.5956},
 keywords = {diffusion, hydrodynamics, methods: numerical, intergalactic medium, large-scale structure of Universe},
     year = 2010,
    month = sep,
   volume = 407,
    pages = {1581-1596},
      doi = {10.1111/j.1365-2966.2010.17047.x},
   adsurl = {http://adsabs.harvard.edu/abs/2010MNRAS.407.1581S},
  adsnote = {Provided by the SAO/NASA Astrophysics Data System}
}

@ARTICLE{Shibuya2016,
   author = {{Shibuya}, T. and {Ouchi}, M. and {Kubo}, M. and {Harikane}, Y.
	},
    title = "{Morphologies of \~{}190,000 Galaxies at z = 0-10 Revealed with HST Legacy Data. II. Evolution of Clumpy Galaxies}",
  journal = {\apj},
archivePrefix = "arXiv",
   eprint = {1511.07054},
 keywords = {cosmology: observations, early universe, galaxies: formation, galaxies: high-redshift},
     year = 2016,
    month = apr,
   volume = 821,
      eid = {72},
    pages = {72},
      doi = {10.3847/0004-637X/821/2/72},
   adsurl = {http://adsabs.harvard.edu/abs/2016ApJ...821...72S},
  adsnote = {Provided by the SAO/NASA Astrophysics Data System}
}

@Article{Silk2012,
  Title                    = {{The current status of galaxy formation}},
  Author                   = {{Silk}, J. and {Mamon}, G.~A.},
  Journal                  = {Research in Astronomy and Astrophysics},
  Year                     = {2012},

  Month                    = aug,
  Pages                    = {917-946},
  Volume                   = {12},

  Abstract                 = {{Understanding galaxy formation is one of the most pressing issues in cosmology. We review the current status of galaxy formation from both an observational and a theoretical perspective, and summarize the prospects for future advances. }},
  Adsnote                  = {Provided by the SAO/NASA Astrophysics Data System},
  Adsurl                   = {http://adsabs.harvard.edu/abs/2012RAA....12..917S},
  Archiveprefix            = {arXiv},
  Doi                      = {10.1088/1674-4527/12/8/004},
  Eprint                   = {1207.3080},
  Primaryclass             = {astro-ph.CO}
}

@ARTICLE{Silva1998,
   author = {{Silva}, L. and {Granato}, G.~L. and {Bressan}, A. and {Danese}, L.
	},
    title = "{Modeling the Effects of Dust on Galactic Spectral Energy Distributions from the Ultraviolet to the Millimeter Band}",
  journal = {\apj},
 keywords = {ISM: DUST, EXTINCTION, GALAXIES: ISM, GALAXIES: SPIRAL, GALAXIES: STARBURST, INFRARED: GALAXIES, RADIATIVE TRANSFER, ISM: Dust, Extinction, Galaxies: ISM, Galaxies: Spiral, Galaxies: Starburst, Infrared: Galaxies, Radiative Transfer},
     year = 1998,
    month = dec,
   volume = 509,
    pages = {103-117},
      doi = {10.1086/306476},
   adsurl = {http://adsabs.harvard.edu/abs/1998ApJ...509..103S},
  adsnote = {Provided by the SAO/NASA Astrophysics Data System}
}

@ARTICLE{Silva1999,
   author = {{Silva}, L. and {Granato}, G.~L. and {Bressan}, A. and {Lacey}, C.~G. and 
	{Baugh}, C.~M. and {Cole}, S. and {Frenk}, C.~S.},
    title = "{Modeling Dust on Galactic SED: Application to Semi-Analytical Galaxy Formation Models}",
  journal = {ArXiv Astrophysics e-prints},
   eprint = {astro-ph/9903350},
 keywords = {Astrophysics},
     year = 1999,
    month = mar,
   adsurl = {http://adsabs.harvard.edu/abs/1999astro.ph..3350S},
  adsnote = {Provided by the SAO/NASA Astrophysics Data System}
}


@Article{2mass,
  Title                    = {{The Two Micron All Sky Survey (2MASS)}},
  Author                   = {{Skrutskie}, M.~F. and {Cutri}, R.~M. and {Stiening}, R. and {Weinberg}, M.~D. and {Schneider}, S. and {Carpenter}, J.~M. and {Beichman}, C. and {Capps}, R. and {Chester}, T. and {Elias}, J. and {Huchra}, J. and {Liebert}, J. and {Lonsdale}, C. and {Monet}, D.~G. and {Price}, S. and {Seitzer}, P. and {Jarrett}, T. and {Kirkpatrick}, J.~D. and {Gizis}, J.~E. and {Howard}, E. and {Evans}, T. and {Fowler}, J. and {Fullmer}, L. and {Hurt}, R. and {Light}, R. and {Kopan}, E.~L. and {Marsh}, K.~A. and {McCallon}, H.~L. and {Tam}, R. and {Van Dyk}, S. and {Wheelock}, S.},
  Journal                  = {\aj},
  Year                     = {2006},

  Month                    = feb,
  Pages                    = {1163-1183},
  Volume                   = {131},

  Adsnote                  = {Provided by the SAO/NASA Astrophysics Data System},
  Adsurl                   = {http://adsabs.harvard.edu/abs/2006AJ....131.1163S},
  Doi                      = {10.1086/498708},
  Keywords                 = {Catalogs, Infrared: General, Surveys}
}

@ARTICLE{Smith2016,
   author = {{Smith}, R. and {Duc}, P.~A. and {Bournaud}, F. and {Yi}, S.~K.
	},
    title = "{A Formation Scenario for the Disk of Satellites: Accretion of Satellites during Mergers}",
  journal = {\apj},
archivePrefix = "arXiv",
   eprint = {1511.05574},
 keywords = {galaxies: halos, galaxies: interactions, galaxies: kinematics and dynamics, methods: numerical},
     year = 2016,
    month = feb,
   volume = 818,
      eid = {11},
    pages = {11},
      doi = {10.3847/0004-637X/818/1/11},
   adsurl = {http://adsabs.harvard.edu/abs/2016ApJ...818...11S},
  adsnote = {Provided by the SAO/NASA Astrophysics Data System}
}


@Article{Sohn2012,
  Title                    = {{The M31 Velocity Vector. I. Hubble Space Telescope Proper-motion Measurements}},
  Author                   = {{Sohn}, S.~T. and {Anderson}, J. and {van der Marel}, R.~P.},
  Journal                  = {\apj},
  Year                     = {2012},

  Month                    = jul,
  Pages                    = {7},
  Volume                   = {753},

  Adsnote                  = {Provided by the SAO/NASA Astrophysics Data System},
  Adsurl                   = {http://adsabs.harvard.edu/abs/2012ApJ...753....7S},
  Archiveprefix            = {arXiv},
  Doi                      = {10.1088/0004-637X/753/1/7},
  Eid                      = {7},
  Eprint                   = {1205.6863},
  Keywords                 = {galaxies: individual: M31, galaxies: kinematics and dynamics, Local Group, proper motions},
  Primaryclass             = {astro-ph.GA}
}

@ARTICLE{Somerville2001,
   author = {{Somerville}, R.~S. and {Primack}, J.~R. and {Faber}, S.~M.},
    title = "{The nature of high-redshift galaxies}",
  journal = {\mnras},
   eprint = {astro-ph/9806228},
 keywords = {galaxies: evolution, galaxies: formation, galaxies: general, galaxies: high-redshift, galaxies: starburst, cosmology: theory},
     year = 2001,
    month = feb,
   volume = 320,
    pages = {504-528},
      doi = {10.1046/j.1365-8711.2001.03975.x},
   adsurl = {http://adsabs.harvard.edu/abs/2001MNRAS.320..504S},
  adsnote = {Provided by the SAO/NASA Astrophysics Data System}
}

@INPROCEEDINGS{Somerville2014,
   author = {{Somerville}, R.~S.},
    title = "{The Formation of Galaxies and Supermassive Black Holes: Insights and Puzzles}",
booktitle = {American Astronomical Society Meeting Abstracts \#223},
     year = 2014,
   series = {American Astronomical Society Meeting Abstracts},
   volume = 223,
    month = jan,
      eid = {239.01},
    pages = {239.01},
   adsurl = {http://adsabs.harvard.edu/abs/2014AAS...22323901S},
  adsnote = {Provided by the SAO/NASA Astrophysics Data System}
}

@Article{Springel2010,
  Title                    = {{E pur si muove: Galilean-invariant cosmological hydrodynamical simulations on a moving mesh}},
  Author                   = {{Springel}, V.},
  Journal                  = {\mnras},
  Year                     = {2010},

  Month                    = jan,
  Pages                    = {791-851},
  Volume                   = {401},

  Adsnote                  = {Provided by the SAO/NASA Astrophysics Data System},
  Adsurl                   = {http://adsabs.harvard.edu/abs/2010MNRAS.401..791S},
  Archiveprefix            = {arXiv},
  Doi                      = {10.1111/j.1365-2966.2009.15715.x},
  Eprint                   = {0901.4107},
  Keywords                 = {methods: numerical, galaxies: interactions, cosmology: dark matter},
  Primaryclass             = {astro-ph.CO}
}

@Article{Springel2010a,
  Title                    = {{Smoothed Particle Hydrodynamics in Astrophysics}},
  Author                   = {{Springel}, V.},
  Journal                  = {\araa},
  Year                     = {2010},

  Month                    = sep,
  Pages                    = {391-430},
  Volume                   = {48},

  Adsnote                  = {Provided by the SAO/NASA Astrophysics Data System},
  Adsurl                   = {http://adsabs.harvard.edu/abs/2010ARAArchiveprefix            = {arXiv},
  Doi                      = {10.1146/annurev-astro-081309-130914},
  Eprint                   = {1109.2219},
  Primaryclass             = {astro-ph.CO}
}

@Article{Springel2006,
  Title                    = {{The large-scale structure of the Universe}},
  Author                   = {{Springel}, V. and {Frenk}, C.~S. and {White}, S.~D.~M.},
  Journal                  = {\nat},
  Year                     = {2006},

  Month                    = apr,
  Pages                    = {1137-1144},
  Volume                   = {440},

  Adsnote                  = {Provided by the SAO/NASA Astrophysics Data System},
  Adsurl                   = {http://adsabs.harvard.edu/abs/2006Natur.440.1137S},
  Doi                      = {10.1038/nature04805},
  Eprint                   = {astro-ph/0604561}
}

@Article{Springel2008,
  Title                    = {{The Aquarius Project: the subhaloes of galactic haloes}},
  Author                   = {{Springel}, V. and {Wang}, J. and {Vogelsberger}, M. and {Ludlow}, A. and {Jenkins}, A. and {Helmi}, A. and {Navarro}, J.~F. and {Frenk}, C.~S. and {White}, S.~D.~M.},
  Journal                  = {\mnras},
  Year                     = {2008},

  Month                    = dec,
  Pages                    = {1685-1711},
  Volume                   = {391},

  Adsnote                  = {Provided by the SAO/NASA Astrophysics Data System},
  Adsurl                   = {http://adsabs.harvard.edu/abs/2008MNRAS.391.1685S},
  Archiveprefix            = {arXiv},
  Doi                      = {10.1111/j.1365-2966.2008.14066.x},
  Eprint                   = {0809.0898},
  Keywords                 = {methods: numerical , dark matter}
}

@Article{Springel2005,
  Title                    = {{Simulations of the formation, evolution and clustering of galaxies and quasars}},
  Author                   = {{Springel}, V. and {White}, S.~D.~M. and {Jenkins}, A. and {Frenk}, C.~S. and {Yoshida}, N. and {Gao}, L. and {Navarro}, J. and {Thacker}, R. and {Croton}, D. and {Helly}, J. and {Peacock}, J.~A. and {Cole}, S. and {Thomas}, P. and {Couchman}, H. and {Evrard}, A. and {Colberg}, J. and {Pearce}, F.},
  Journal                  = {\nat},
  Year                     = {2005},

  Month                    = jun,
  Pages                    = {629-636},
  Volume                   = {435},

  Abstract                 = {{The cold dark matter model has become the leading theoretical picture for the formation of structure in the Universe. This model, together with the theory of cosmic inflation, makes a clear prediction for the initial conditions for structure formation and predicts that structures grow hierarchically through gravitational instability. Testing this model requires that the precise measurements delivered by galaxy surveys can be compared to robust and equally precise theoretical calculations. Here we present a simulation of the growth of dark matter structure using 2,160$^{3}$ particles, following them from redshift z = 127 to the present in a cube-shaped region 2.230 billion lightyears on a side. In postprocessing, we also follow the formation and evolution of the galaxies and quasars. We show that baryon-induced features in the initial conditions of the Universe are reflected in distorted form in the low-redshift galaxy distribution, an effect that can be used to constrain the nature of dark energy with future generations of observational surveys of galaxies. }},
  Adsnote                  = {Provided by the SAO/NASA Astrophysics Data System},
  Adsurl                   = {http://adsabs.harvard.edu/abs/2005Natur.435..629S},
  Doi                      = {10.1038/nature03597},
  Eprint                   = {arXiv:astro-ph/0504097}
}

@Misc{Stadel2013,
  Title                    = {{PkdGRAV2: Parallel fast-multipole cosmological code}},

  Author                   = {{Stadel}, J.},
  HowPublished             = {Astrophysics Source Code Library},
  Month                    = may,
  Year                     = {2013},

  Adsnote                  = {Provided by the SAO/NASA Astrophysics Data System},
  Adsurl                   = {http://adsabs.harvard.edu/abs/2013ascl.soft05005S},
  Archiveprefix            = {ascl},
  Eprint                   = {1305.005}
}

@PhdThesis{Stadel2001,
  Title                    = {{Cosmological N-body simulations and their analysis}},
  Author                   = {{Stadel}, J.~G.},
  School                   = {UNIVERSITY OF WASHINGTON},
  Year                     = {2001},

  Adsnote                  = {Provided by the SAO/NASA Astrophysics Data System},
  Adsurl                   = {http://adsabs.harvard.edu/abs/2001PhDT........21S}
}

@ARTICLE{Stark2009,
   author = {{Stark}, D.~P. and {Ellis}, R.~S. and {Bunker}, A. and {Bundy}, K. and 
	{Targett}, T. and {Benson}, A. and {Lacy}, M.},
    title = "{The Evolutionary History of Lyman Break Galaxies Between Redshift 4 and 6: Observing Successive Generations of Massive Galaxies in Formation}",
  journal = {\apj},
archivePrefix = "arXiv",
   eprint = {0902.2907},
 primaryClass = "astro-ph.CO",
 keywords = {galaxies: evolution, galaxies: formation, galaxies: high-redshift, galaxies: starburst, surveys, ultraviolet: galaxies},
     year = 2009,
    month = jun,
   volume = 697,
    pages = {1493-1511},
      doi = {10.1088/0004-637X/697/2/1493},
   adsurl = {http://adsabs.harvard.edu/abs/2009ApJ...697.1493S},
  adsnote = {Provided by the SAO/NASA Astrophysics Data System}
}

@ARTICLE{Stewart2009,
   author = {{Stewart}, K.~R. and {Bullock}, J.~S. and {Wechsler}, R.~H. and 
	{Maller}, A.~H.},
    title = "{Gas-rich Mergers in LCDM: Disk Survivability and the Baryonic Assembly of Galaxies}",
  journal = {\apj},
archivePrefix = "arXiv",
   eprint = {0901.4336},
 primaryClass = "astro-ph.CO",
 keywords = {cosmology: theory, dark matter, galaxies: formation, galaxies: halos, methods: N-body simulations},
     year = 2009,
    month = sep,
   volume = 702,
    pages = {307-317},
      doi = {10.1088/0004-637X/702/1/307},
   adsurl = {http://adsabs.harvard.edu/abs/2009ApJ...702..307S},
  adsnote = {Provided by the SAO/NASA Astrophysics Data System}
}

@ARTICLE{Stinson2006,
   author = {{Stinson}, G. and {Seth}, A. and {Katz}, N. and {Wadsley}, J. and 
	{Governato}, F. and {Quinn}, T.},
    title = "{Star formation and feedback in smoothed particle hydrodynamic simulations - I. Isolated galaxies}",
  journal = {\mnras},
   eprint = {astro-ph/0602350},
 keywords = {hydrodynamics, methods: N-body simulations, stars: formation, Galaxy: evolution},
     year = 2006,
    month = dec,
   volume = 373,
    pages = {1074-1090},
      doi = {10.1111/j.1365-2966.2006.11097.x},
   adsurl = {http://esoads.eso.org/abs/2006MNRAS.373.1074S},
  adsnote = {Provided by the SAO/NASA Astrophysics Data System}
}


@ARTICLE{Stinson2013,
   author = {{Stinson}, G.~S. and {Brook}, C. and {Macci{\`o}}, A.~V. and 
	{Wadsley}, J. and {Quinn}, T.~R. and {Couchman}, H.~M.~P.},
    title = "{Making Galaxies In a Cosmological Context: the need for early stellar feedback}",
  journal = {\mnras},
archivePrefix = "arXiv",
   eprint = {1208.0002},
 keywords = {hydrodynamics, galaxies: formation, galaxies: ISM},
     year = 2013,
    month = jan,
   volume = 428,
    pages = {129-140},
      doi = {10.1093/mnras/sts028},
   adsurl = {http://adsabs.harvard.edu/abs/2013MNRAS.428..129S},
  adsnote = {Provided by the SAO/NASA Astrophysics Data System}
}

@ARTICLE{Stinson2015,
   author = {{Stinson}, G.~S. and {Dutton}, A.~A. and {Wang}, L. and {Macci{\`o}}, A.~V. and 
	{Herpich}, J. and {Bradford}, J.~D. and {Quinn}, T.~R. and {Wadsley}, J. and 
	{Keller}, B.},
    title = "{NIHAO III: the constant disc gas mass conspiracy}",
  journal = {\mnras},
archivePrefix = "arXiv",
   eprint = {1506.08785},
 keywords = {hydrodynamics, galaxies: formation, galaxies: ISM},
     year = 2015,
    month = nov,
   volume = 454,
    pages = {1105-1116},
      doi = {10.1093/mnras/stv1985},
   adsurl = {http://adsabs.harvard.edu/abs/2015MNRAS.454.1105S},
  adsnote = {Provided by the SAO/NASA Astrophysics Data System}
}

@ARTICLE{Swinbank2010,
   author = {{Swinbank}, A.~M. and {Smail}, I. and {Longmore}, S. and {Harris}, A.~I. and 
	{Baker}, A.~J. and {De Breuck}, C. and {Richard}, J. and {Edge}, A.~C. and 
	{Ivison}, R.~J. and {Blundell}, R. and {Coppin}, K.~E.~K. and 
	{Cox}, P. and {Gurwell}, M. and {Hainline}, L.~J. and {Krips}, M. and 
	{Lundgren}, A. and {Neri}, R. and {Siana}, B. and {Siringo}, G. and 
	{Stark}, D.~P. and {Wilner}, D. and {Younger}, J.~D.},
    title = "{Intense star formation within resolved compact regions in a galaxy at z = 2.3}",
  journal = {\nat},
archivePrefix = "arXiv",
   eprint = {1003.3674},
 primaryClass = "astro-ph.CO",
     year = 2010,
    month = apr,
   volume = 464,
    pages = {733-736},
      doi = {10.1038/nature08880},
   adsurl = {http://adsabs.harvard.edu/abs/2010Natur.464..733S},
  adsnote = {Provided by the SAO/NASA Astrophysics Data System}
}

@ARTICLE{Tacconi2010,
   author = {{Tacconi}, L.~J. and {Genzel}, R. and {Neri}, R. and {Cox}, P. and 
	{Cooper}, M.~C. and {Shapiro}, K. and {Bolatto}, A. and {Bouch{\'e}}, N. and 
	{Bournaud}, F. and {Burkert}, A. and {Combes}, F. and {Comerford}, J. and 
	{Davis}, M. and {Schreiber}, N.~M.~F. and {Garcia-Burillo}, S. and 
	{Gracia-Carpio}, J. and {Lutz}, D. and {Naab}, T. and {Omont}, A. and 
	{Shapley}, A. and {Sternberg}, A. and {Weiner}, B.},
    title = "{High molecular gas fractions in normal massive star-forming galaxies in the young Universe}",
  journal = {\nat},
archivePrefix = "arXiv",
   eprint = {1002.2149},
     year = 2010,
    month = feb,
   volume = 463,
    pages = {781-784},
      doi = {10.1038/nature08773},
   adsurl = {http://adsabs.harvard.edu/abs/2010Natur.463..781T},
  adsnote = {Provided by the SAO/NASA Astrophysics Data System}
}


@ARTICLE{Tacconi2013,
   author = {{Tacconi}, L.~J. and {Neri}, R. and {Genzel}, R. and {Combes}, F. and 
	{Bolatto}, A. and {Cooper}, M.~C. and {Wuyts}, S. and {Bournaud}, F. and 
	{Burkert}, A. and {Comerford}, J. and {Cox}, P. and {Davis}, M. and 
	{F{\"o}rster Schreiber}, N.~M. and {Garc{\'{\i}}a-Burillo}, S. and 
	{Gracia-Carpio}, J. and {Lutz}, D. and {Naab}, T. and {Newman}, S. and 
	{Omont}, A. and {Saintonge}, A. and {Shapiro Griffin}, K. and 
	{Shapley}, A. and {Sternberg}, A. and {Weiner}, B.},
    title = "{Phibss: Molecular Gas Content and Scaling Relations in z \~{} 1-3 Massive, Main-sequence Star-forming Galaxies}",
  journal = {\apj},
archivePrefix = "arXiv",
   eprint = {1211.5743},
 keywords = {galaxies: evolution, galaxies: high-redshift, galaxies: ISM, ISM: molecules},
     year = 2013,
    month = may,
   volume = 768,
      eid = {74},
    pages = {74},
      doi = {10.1088/0004-637X/768/1/74},
   adsurl = {http://adsabs.harvard.edu/abs/2013ApJ...768...74T},
  adsnote = {Provided by the SAO/NASA Astrophysics Data System}
}

@ARTICLE{Tollet2016,
   author = {{Tollet}, E. and {Macci{\`o}}, A.~V. and {Dutton}, A.~A. and 
	{Stinson}, G.~S. and {Wang}, L. and {Penzo}, C. and {Gutcke}, T.~A. and 
	{Buck}, T. and {Kang}, X. and {Brook}, C. and {Di Cintio}, A. and 
	{Keller}, B.~W. and {Wadsley}, J.},
    title = "{NIHAO - IV: core creation and destruction in dark matter density profiles across cosmic time}",
  journal = {\mnras},
archivePrefix = "arXiv",
   eprint = {1507.03590},
 keywords = {hydrodynamics, galaxies: evolution, dark matter},
     year = 2016,
    month = mar,
   volume = 456,
    pages = {3542-3552},
      doi = {10.1093/mnras/stv2856},
   adsurl = {http://adsabs.harvard.edu/abs/2016MNRAS.456.3542T},
  adsnote = {Provided by the SAO/NASA Astrophysics Data System}
}

@ARTICLE{Tadaki2014,
   author = {{Tadaki}, K.-i. and {Kodama}, T. and {Tanaka}, I. and {Hayashi}, M. and 
	{Koyama}, Y. and {Shimakawa}, R.},
    title = "{The Nature of H{$\alpha$}-selected Galaxies at z $\gt$ 2. II. Clumpy Galaxies and Compact Star-forming Galaxies}",
  journal = {\apj},
archivePrefix = "arXiv",
   eprint = {1311.4260},
 keywords = {galaxies: evolution, galaxies: high-redshift, galaxies: structure },
     year = 2014,
    month = jan,
   volume = 780,
      eid = {77},
    pages = {77},
      doi = {10.1088/0004-637X/780/1/77},
   adsurl = {http://adsabs.harvard.edu/abs/2014ApJ...780...77T},
  adsnote = {Provided by the SAO/NASA Astrophysics Data System}
}

@ARTICLE{Tamburello2015,
   author = {{Tamburello}, V. and {Mayer}, L. and {Shen}, S. and {Wadsley}, J.
	},
    title = "{A lower fragmentation mass scale in high-redshift galaxies and its implications on giant clumps: a systematic numerical study}",
  journal = {\mnras},
archivePrefix = "arXiv",
   eprint = {1412.3319},
 keywords = {Galaxy: evolution, galaxies: bulges, galaxies: evolution, galaxies: high-redshift},
     year = 2015,
    month = nov,
   volume = 453,
    pages = {2490-2514},
      doi = {10.1093/mnras/stv1695},
   adsurl = {http://adsabs.harvard.edu/abs/2015MNRAS.453.2490T},
  adsnote = {Provided by the SAO/NASA Astrophysics Data System}
}

@Article{Tasker2008,
  Title                    = {{A test suite for quantitative comparison of hydrodynamic codes in astrophysics}},
  Author                   = {{Tasker}, E.~J. and {Brunino}, R. and {Mitchell}, N.~L. and {Michielsen}, D. and {Hopton}, S. and {Pearce}, F.~R. and {Bryan}, G.~L. and {Theuns}, T.},
  Journal                  = {\mnras},
  Year                     = {2008},

  Month                    = nov,
  Pages                    = {1267-1281},
  Volume                   = {390},

  Adsnote                  = {Provided by the SAO/NASA Astrophysics Data System},
  Adsurl                   = {http://adsabs.harvard.edu/abs/2008MNRAS.390.1267T},
  Archiveprefix            = {arXiv},
  Doi                      = {10.1111/j.1365-2966.2008.13836.x},
  Eprint                   = {0808.1844},
  Keywords                 = {hydrodynamics , methods: numerical , cosmology: theory}
}

@Article{Tegmark2004,
  Title                    = {{The Three-Dimensional Power Spectrum of Galaxies from the Sloan Digital Sky Survey}},
  Author                   = {{Tegmark}, M. and {Blanton}, M.~R. and {Strauss}, M.~A. and {Hoyle}, F. and {Schlegel}, D. and {Scoccimarro}, R. and {Vogeley}, M.~S. and {Weinberg}, D.~H. and {Zehavi}, I. and {Berlind}, A. and {Budavari}, T. and {Connolly}, A. and {Eisenstein}, D.~J. and {Finkbeiner}, D. and {Frieman}, J.~A. and {Gunn}, J.~E. and {Hamilton}, A.~J.~S. and {Hui}, L. and {Jain}, B. and {Johnston}, D. and {Kent}, S. and {Lin}, H. and {Nakajima}, R. and {Nichol}, R.~C. and {Ostriker}, J.~P. and {Pope}, A. and {Scranton}, R. and {Seljak}, U. and {Sheth}, R.~K. and {Stebbins}, A. and {Szalay}, A.~S. and {Szapudi}, I. and {Verde}, L. and {Xu}, Y. and {Annis}, J. and {Bahcall}, N.~A. and {Brinkmann}, J. and {Burles}, S. and {Castander}, F.~J. and {Csabai}, I. and {Loveday}, J. and {Doi}, M. and {Fukugita}, M. and {Gott}, III, J.~R. and {Hennessy}, G. and {Hogg}, D.~W. and {Ivezi{\'c}}, {\v Z}. and {Knapp}, G.~R. and {Lamb}, D.~Q. and {Lee}, B.~C. and {Lupton}, R.~H. and {McKay}, T.~A. and {Kunszt}, P. and {Munn}, J.~A. and {O'Connell}, L. and {Peoples}, J. and {Pier}, J.~R. and {Richmond}, M. and {Rockosi}, C. and {Schneider}, D.~P. and {Stoughton}, C. and {Tucker}, D.~L. and {Vanden Berk}, D.~E. and {Yanny}, B. and {York}, D.~G. and {SDSS Collaboration}},
  Journal                  = {\apj},
  Year                     = {2004},

  Month                    = may,
  Pages                    = {702-740},
  Volume                   = {606},

  Adsnote                  = {Provided by the SAO/NASA Astrophysics Data System},
  Adsurl                   = {http://adsabs.harvard.edu/abs/2004ApJ...606..702T},
  Doi                      = {10.1086/382125},
  Eprint                   = {astro-ph/0310725},
  Keywords                 = {Galaxies: Statistics, Cosmology: Large-Scale Structure of Universe, Methods: Data Analysis}
}

@Article{Tempel2015,
  Title                    = {{The alignment of satellite galaxies and cosmic filaments: observations and simulations}},
  Author                   = {{Tempel}, E. and {Guo}, Q. and {Kipper}, R. and {Libeskind}, N.~I. },
  Journal                  = {\mnras},
  Year                     = {2015},

  Month                    = jul,
  Pages                    = {2727-2738},
  Volume                   = {450},

  Adsnote                  = {Provided by the SAO/NASA Astrophysics Data System},
  Adsurl                   = {http://adsabs.harvard.edu/abs/2015MNRAS.450.2727T},
  Archiveprefix            = {arXiv},
  Doi                      = {10.1093/mnras/stv919},
  Eprint                   = {1502.02046},
  Keywords                 = {methods: data analysis, methods: statistical, galaxies: statistics, large-scale structure of Universe}
}

@ARTICLE{Toomre1964,
   author = {{Toomre}, A.},
    title = "{On the gravitational stability of a disk of stars}",
  journal = {\apj},
     year = 1964,
    month = may,
   volume = 139,
    pages = {1217-1238},
      doi = {10.1086/147861},
   adsurl = {http://adsabs.harvard.edu/abs/1964ApJ...139.1217T},
  adsnote = {Provided by the SAO/NASA Astrophysics Data System}
}

@Article{Thoul1996,
  Title                    = {{Hydrodynamic Simulations of Galaxy Formation. II. Photoionization and the Formation of Low-Mass Galaxies}},
  Author                   = {{Thoul}, A.~A. and {Weinberg}, D.~H.},
  Journal                  = {\apj},
  Year                     = {1996},

  Month                    = jul,
  Pages                    = {608},
  Volume                   = {465},

  Adsnote                  = {Provided by the SAO/NASA Astrophysics Data System},
  Adsurl                   = {http://adsabs.harvard.edu/abs/1996ApJ...465..608T},
  Doi                      = {10.1086/177446},
  Eprint                   = {astro-ph/9510154},
  Keywords                 = {GALAXIES: FORMATION, GALAXIES: KINEMATICS AND DYNAMICS, HYDRODYNAMICS, RADIATIVE TRANSFER}
}

@Article{Toukmaji1996,
  Title                    = {{Ewald summation techniques in perspective: a survey}},
  Author                   = {{Toukmaji}, A.~Y. and {Board}, Jr., J.~A.},
  Journal                  = {Computer Physics Communications},
  Year                     = {1996},

  Month                    = jun,
  Pages                    = {73-92},
  Volume                   = {95},

  Adsnote                  = {Provided by the SAO/NASA Astrophysics Data System},
  Adsurl                   = {http://adsabs.harvard.edu/abs/1996CoPhC..95...73T},
  Doi                      = {10.1016/0010-4655(96)00016-1}
}

@Article{Tully2015,
  Title                    = {{Two Planes of Satellites in the Centaurus A Group}},
  Author                   = {{Tully}, R.~B. and {Libeskind}, N.~I. and {Karachentsev}, I.~D. and {Karachentseva}, V.~E. and {Rizzi}, L. and {Shaya}, E.~J.},
  Journal                  = {\apjl},
  Year                     = {2015},

  Month                    = apr,
  Pages                    = {L25},
  Volume                   = {802},

  Adsnote                  = {Provided by the SAO/NASA Astrophysics Data System},
  Adsurl                   = {http://adsabs.harvard.edu/abs/2015ApJ...802L..25T},
  Archiveprefix            = {arXiv},
  Doi                      = {10.1088/2041-8205/802/2/L25},
  Eid                      = {L25},
  Eprint                   = {1503.05599},
  Keywords                 = {galaxies: distances and redshifts, galaxies: groups: individual: Cen A, large-scale structure of universe}
}

@Article{DV,
  Title                    = {Implications of the {\sc{force}} in a Darth Matter dominated Universe},
  Author                   = {{Vader}, D. and {Palpatine}, S. and {Doku}, C. and {Stormtrouper 1} and {Skywalker}, A.},
  Journal                  = {Imperial Archives of Dark Scinences},
  Year                     = {1977},
  Volume                   = {1977}
}

@Article{Vale2006,
  Title                    = {{The non-parametric model for linking galaxy luminosity with halo/subhalo mass}},
  Author                   = {{Vale}, A. and {Ostriker}, J.~P.},
  Journal                  = {\mnras},
  Year                     = {2006},

  Month                    = sep,
  Pages                    = {1173-1187},
  Volume                   = {371},

  Adsnote                  = {Provided by the SAO/NASA Astrophysics Data System},
  Adsurl                   = {http://adsabs.harvard.edu/abs/2006MNRAS.371.1173V},
  Doi                      = {10.1111/j.1365-2966.2006.10605.x},
  Eprint                   = {astro-ph/0511816},
  Keywords                 = {galaxies: haloes: cosmology: theory: dark matter: large-scale structure of Universe, galaxies: haloes, cosmology: theory, dark matter, large-scale structure of Universe}
}

@ARTICLE{Dokkum2010,
   author = {{van Dokkum}, P.~G. and {Whitaker}, K.~E. and {Brammer}, G. and 
	{Franx}, M. and {Kriek}, M. and {Labb{\'e}}, I. and {Marchesini}, D. and 
	{Quadri}, R. and {Bezanson}, R. and {Illingworth}, G.~D. and 
	{Muzzin}, A. and {Rudnick}, G. and {Tal}, T. and {Wake}, D.},
    title = "{The Growth of Massive Galaxies Since z = 2}",
  journal = {\apj},
archivePrefix = "arXiv",
   eprint = {0912.0514},
 keywords = {cosmology: observations, galaxies: evolution, galaxies: formation},
     year = 2010,
    month = feb,
   volume = 709,
    pages = {1018-1041},
      doi = {10.1088/0004-637X/709/2/1018},
   adsurl = {http://adsabs.harvard.edu/abs/2010ApJ...709.1018V},
  adsnote = {Provided by the SAO/NASA Astrophysics Data System}
}

@ARTICLE{Wadsley2004,
   author = {{Wadsley}, J.~W. and {Stadel}, J. and {Quinn}, T.},
    title = "{Gasoline: a flexible, parallel implementation of TreeSPH}",
  journal = {\na},
   eprint = {astro-ph/0303521},
     year = 2004,
    month = feb,
   volume = 9,
    pages = {137-158},
      doi = {10.1016/j.newast.2003.08.004},
   adsurl = {http://adsabs.harvard.edu/abs/2004NewA....9..137W},
  adsnote = {Provided by the SAO/NASA Astrophysics Data System}
}

@Article{Wadsley2008,
  Title                    = {{On the treatment of entropy mixing in numerical cosmology}},
  Author                   = {{Wadsley}, J.~W. and {Veeravalli}, G. and {Couchman}, H.~M.~P. },
  Journal                  = {\mnras},
  Year                     = {2008},

  Month                    = jun,
  Pages                    = {427-438},
  Volume                   = {387},

  Adsnote                  = {Provided by the SAO/NASA Astrophysics Data System},
  Adsurl                   = {http://adsabs.harvard.edu/abs/2008MNRAS.387..427W},
  Doi                      = {10.1111/j.1365-2966.2008.13260.x},
  Keywords                 = {diffusion , hydrodynamics , turbulence , methods: numerical , galaxies: clusters: general}
}

@Article{Wang2013,
  Title                    = {{The spatial distribution of galactic satellites in the {$\Lambda$} cold dark matter cosmology}},
  Author                   = {{Wang}, J. and {Frenk}, C.~S. and {Cooper}, A.~P.},
  Journal                  = {\mnras},
  Year                     = {2013},

  Month                    = feb,
  Pages                    = {1502-1513},
  Volume                   = {429},

  Adsnote                  = {Provided by the SAO/NASA Astrophysics Data System},
  Adsurl                   = {http://adsabs.harvard.edu/abs/2013MNRAS.429.1502W},
  Archiveprefix            = {arXiv},
  Doi                      = {10.1093/mnras/sts442},
  Eprint                   = {1206.1340},
  Keywords                 = {methods: numerical, Galaxy: structure, Galaxy: formation, dark matter}
}

@ARTICLE{Wang2015,
   author = {{Wang}, L. and {Dutton}, A.~A. and {Stinson}, G.~S. and {Macci{\`o}}, A.~V. and 
	{Penzo}, C. and {Kang}, X. and {Keller}, B.~W. and {Wadsley}, J.},
    title = "{NIHAO project - I. Reproducing the inefficiency of galaxy formation across cosmic time with a large sample of cosmological hydrodynamical simulations}",
  journal = {\mnras},
archivePrefix = "arXiv",
   eprint = {1503.04818},
 keywords = {methods: numerical, galaxies: dwarf, galaxies: evolution, galaxies: formation, galaxies: spiral, cosmology: theory},
     year = 2015,
    month = nov,
   volume = 454,
    pages = {83-94},
      doi = {10.1093/mnras/stv1937},
   adsurl = {http://adsabs.harvard.edu/abs/2015MNRAS.454...83W},
  adsnote = {Provided by the SAO/NASA Astrophysics Data System}
}

@Article{Wechsler2002,
  Title                    = {{Concentrations of Dark Halos from Their Assembly Histories}},
  Author                   = {{Wechsler}, R.~H. and {Bullock}, J.~S. and {Primack}, J.~R. and {Kravtsov}, A.~V. and {Dekel}, A.},
  Journal                  = {\apj},
  Year                     = {2002},

  Month                    = mar,
  Pages                    = {52-70},
  Volume                   = {568},

  Adsnote                  = {Provided by the SAO/NASA Astrophysics Data System},
  Adsurl                   = {http://adsabs.harvard.edu/abs/2002ApJ...568...52W},
  Doi                      = {10.1086/338765},
  Eprint                   = {astro-ph/0108151},
  Keywords                 = {Cosmology: Theory, Cosmology: Dark Matter, Galaxies: Evolution, Galaxies: Formation, Galaxies: Halos, Galaxies: Structure}
}

@ARTICLE{Wegg2013,
   author = {{Wegg}, C. and {Gerhard}, O.},
    title = "{Mapping the three-dimensional density of the Galactic bulge with VVV red clump stars}",
  journal = {\mnras},
archivePrefix = "arXiv",
   eprint = {1308.0593},
 keywords = {Galaxy: bulge, Galaxy: centre, Galaxy: structure},
     year = 2013,
    month = nov,
   volume = 435,
    pages = {1874-1887},
      doi = {10.1093/mnras/stt1376},
   adsurl = {http://adsabs.harvard.edu/abs/2013MNRAS.435.1874W},
  adsnote = {Provided by the SAO/NASA Astrophysics Data System}
}

@ARTICLE{Wuyts2012,
   author = {{Wuyts}, S. and {F{\"o}rster Schreiber}, N.~M. and {Genzel}, R. and 
	{Guo}, Y. and {Barro}, G. and {Bell}, E.~F. and {Dekel}, A. and 
	{Faber}, S.~M. and {Ferguson}, H.~C. and {Giavalisco}, M. and 
	{Grogin}, N.~A. and {Hathi}, N.~P. and {Huang}, K.-H. and {Kocevski}, D.~D. and 
	{Koekemoer}, A.~M. and {Koo}, D.~C. and {Lotz}, J. and {Lutz}, D. and 
	{McGrath}, E. and {Newman}, J.~A. and {Rosario}, D. and {Saintonge}, A. and 
	{Tacconi}, L.~J. and {Weiner}, B.~J. and {van der Wel}, A.},
    title = "{Smooth(er) Stellar Mass Maps in CANDELS: Constraints on the Longevity of Clumps in High-redshift Star-forming Galaxies}",
  journal = {\apj},
archivePrefix = "arXiv",
   eprint = {1203.2611},
 keywords = {galaxies: high-redshift, galaxies: stellar content, galaxies: structure},
     year = 2012,
    month = jul,
   volume = 753,
      eid = {114},
    pages = {114},
      doi = {10.1088/0004-637X/753/2/114},
   adsurl = {http://adsabs.harvard.edu/abs/2012ApJ...753..114W},
  adsnote = {Provided by the SAO/NASA Astrophysics Data System}
}

@ARTICLE{York2000,
   author = {{York}, D.~G. and {Adelman}, J. and {Anderson}, Jr., J.~E. and 
	{Anderson}, S.~F. and {Annis}, J. and {Bahcall}, N.~A. and {Bakken}, J.~A. and 
	{Barkhouser}, R. and {Bastian}, S. and {Berman}, E. and {Boroski}, W.~N. and 
	{Bracker}, S. and {Briegel}, C. and {Briggs}, J.~W. and {Brinkmann}, J. and 
	{Brunner}, R. and {Burles}, S. and {Carey}, L. and {Carr}, M.~A. and 
	{Castander}, F.~J. and {Chen}, B. and {Colestock}, P.~L. and 
	{Connolly}, A.~J. and {Crocker}, J.~H. and {Csabai}, I. and 
	{Czarapata}, P.~C. and {Davis}, J.~E. and {Doi}, M. and {Dombeck}, T. and 
	{Eisenstein}, D. and {Ellman}, N. and {Elms}, B.~R. and {Evans}, M.~L. and 
	{Fan}, X. and {Federwitz}, G.~R. and {Fiscelli}, L. and {Friedman}, S. and 
	{Frieman}, J.~A. and {Fukugita}, M. and {Gillespie}, B. and 
	{Gunn}, J.~E. and {Gurbani}, V.~K. and {de Haas}, E. and {Haldeman}, M. and 
	{Harris}, F.~H. and {Hayes}, J. and {Heckman}, T.~M. and {Hennessy}, G.~S. and 
	{Hindsley}, R.~B. and {Holm}, S. and {Holmgren}, D.~J. and {Huang}, C.-h. and 
	{Hull}, C. and {Husby}, D. and {Ichikawa}, S.-I. and {Ichikawa}, T. and 
	{Ivezi{\'c}}, {\v Z}. and {Kent}, S. and {Kim}, R.~S.~J. and 
	{Kinney}, E. and {Klaene}, M. and {Kleinman}, A.~N. and {Kleinman}, S. and 
	{Knapp}, G.~R. and {Korienek}, J. and {Kron}, R.~G. and {Kunszt}, P.~Z. and 
	{Lamb}, D.~Q. and {Lee}, B. and {Leger}, R.~F. and {Limmongkol}, S. and 
	{Lindenmeyer}, C. and {Long}, D.~C. and {Loomis}, C. and {Loveday}, J. and 
	{Lucinio}, R. and {Lupton}, R.~H. and {MacKinnon}, B. and {Mannery}, E.~J. and 
	{Mantsch}, P.~M. and {Margon}, B. and {McGehee}, P. and {McKay}, T.~A. and 
	{Meiksin}, A. and {Merelli}, A. and {Monet}, D.~G. and {Munn}, J.~A. and 
	{Narayanan}, V.~K. and {Nash}, T. and {Neilsen}, E. and {Neswold}, R. and 
	{Newberg}, H.~J. and {Nichol}, R.~C. and {Nicinski}, T. and 
	{Nonino}, M. and {Okada}, N. and {Okamura}, S. and {Ostriker}, J.~P. and 
	{Owen}, R. and {Pauls}, A.~G. and {Peoples}, J. and {Peterson}, R.~L. and 
	{Petravick}, D. and {Pier}, J.~R. and {Pope}, A. and {Pordes}, R. and 
	{Prosapio}, A. and {Rechenmacher}, R. and {Quinn}, T.~R. and 
	{Richards}, G.~T. and {Richmond}, M.~W. and {Rivetta}, C.~H. and 
	{Rockosi}, C.~M. and {Ruthmansdorfer}, K. and {Sandford}, D. and 
	{Schlegel}, D.~J. and {Schneider}, D.~P. and {Sekiguchi}, M. and 
	{Sergey}, G. and {Shimasaku}, K. and {Siegmund}, W.~A. and {Smee}, S. and 
	{Smith}, J.~A. and {Snedden}, S. and {Stone}, R. and {Stoughton}, C. and 
	{Strauss}, M.~A. and {Stubbs}, C. and {SubbaRao}, M. and {Szalay}, A.~S. and 
	{Szapudi}, I. and {Szokoly}, G.~P. and {Thakar}, A.~R. and {Tremonti}, C. and 
	{Tucker}, D.~L. and {Uomoto}, A. and {Vanden Berk}, D. and {Vogeley}, M.~S. and 
	{Waddell}, P. and {Wang}, S.-i. and {Watanabe}, M. and {Weinberg}, D.~H. and 
	{Yanny}, B. and {Yasuda}, N. and {SDSS Collaboration}},
    title = "{The Sloan Digital Sky Survey: Technical Summary}",
  journal = {\aj},
   eprint = {astro-ph/0006396},
 keywords = {Cosmology: Observations, Instrumentation: Miscellaneous},
     year = 2000,
    month = sep,
   volume = 120,
    pages = {1579-1587},
      doi = {10.1086/301513},
   adsurl = {http://adsabs.harvard.edu/abs/2000AJ....120.1579Y},
  adsnote = {Provided by the SAO/NASA Astrophysics Data System}
}

@Article{Zeldovich1970,
  Title                    = {{Gravitational instability: An approximate theory for large density perturbations.}},
  Author                   = {{Zel'dovich}, Y.~B.},
  Journal                  = {\aap},
  Year                     = {1970},

  Month                    = mar,
  Pages                    = {84-89},
  Volume                   = {5},

  Adsnote                  = {Provided by the SAO/NASA Astrophysics Data System},
  Adsurl                   = {http://adsabs.harvard.edu/abs/1970A}

@Article{Zentner2005,
  Title                    = {{The Anisotropic Distribution of Galactic Satellites}},
  Author                   = {{Zentner}, A.~R. and {Kravtsov}, A.~V. and {Gnedin}, O.~Y. and {Klypin}, A.~A.},
  Journal                  = {\apj},
  Year                     = {2005},

  Month                    = aug,
  Pages                    = {219-232},
  Volume                   = {629},

  Adsnote                  = {Provided by the SAO/NASA Astrophysics Data System},
  Adsurl                   = {http://adsabs.harvard.edu/abs/2005ApJ...629..219Z},
  Doi                      = {10.1086/431355},
  Eprint                   = {astro-ph/0502496},
  Keywords                 = {Cosmology: Theory, Cosmology: Dark Matter, Galaxies: Formation, Galaxies: Halos, Cosmology: Large-Scale Structure of Universe, Methods: Numerical}
}

@Article{Zwicky1933,
  Title                    = {{Die Rotverschiebung von extragalaktischen Nebeln}},
  Author                   = {{Zwicky}, F.},
  Journal                  = {Helvetica Physica Acta},
  Year                     = {1933},
  Pages                    = {110-127},
  Volume                   = {6},

  Adsnote                  = {Provided by the SAO/NASA Astrophysics Data System},
  Adsurl                   = {http://adsabs.harvard.edu/abs/1933AcHPh...6..110Z}
}

 

\appendix

\section{Surface density fits}

\begin{figure*}
\begin{center}
\vspace{-.35cm}
\includegraphics[width=\textwidth]{./plots/starden_prof127.pdf}
\end{center}
\vspace{-.75cm}
\caption{Face-on stellar surface density profiles for all simulations at $z = 0$ (black dots). The four models of Au6 are shown in the upper row, AuL8 in the bottom row. The profiles are simultaneously fit with a \citet{Sersic1963} (red dashed curve) and exponential (blue curve) profile. The total fitted profile is indicated by the black curve. Resulting fit values for the disc scale length, $R_{\rm d}$, the bulge effective radius, $R_{\rm eff}$ and the bulge S\'ersic index, $n$, are given in each panel. The CRdiff and CRadv models of AuL8 are well fitted by a pure S\'ersic profile.}
\label{fig:surf_den_fit}
\end{figure*}


In Fig. \ref{fig:surf_den_fit} we show azimuthally averaged surface density fits to the stellar disc of the eight simulations at redshift $z=0$. Surface density profiles are created for all the stellar mass within $\pm5$ kpc of the mid plane in the vertical direction. The profiles are simultaneously fit with a \citet{Sersic1963} (red dashed curve) and exponential (blue curve) profile using a non-linear least squares method. Resulting fit values for the disc scale length, $R_{\rm d}$, the bulge effective radius, $R_{\rm eff}$ and the bulge S\'ersic index, $n$, are given in each panel. This figure shows that the CRadv and CRdiff runs result in more compact bulge dominated galaxies whereas the CRdiffalfven run results in a disc dominated galaxy more similar to the fiducial AURIGA run. From the fits we derive disc-to-total mass ratios (D/T) which are given in Table \ref{tab:props} in the main text.

\section{Vertical profiles}

We have created vertical profiles similar to the radial profiles shown in Fig. \ref{fig:prof} for the gas density, the magnetic field strength, the CR pressure and the gas thermal pressure in the central galaxy. We select all Voronoi cells in a cylinder of radius $r=30$ kpc and height $z=\pm10$ kpc and show the data in 30 bins linearly spaced in $z$ in Fig. \ref{fig:vert_prof}.

\begin{figure*}
\begin{center}
\vspace{-.25cm}
\includegraphics[width=1.1\textwidth]{./plots/auriga_vertical_profiles_L6_lvl4.pdf}\vspace{-.25cm}
\includegraphics[width=1.1\textwidth]{./plots/auriga_vertical_profiles_L8_lvl4.pdf}
\end{center}
\vspace{-.45cm}
\caption{Vertical profiles of the gas density (left panel), magnetic field strength (second panel), CR pressure (third panel), and gas thermal pressure
    (forth panel) for the four models of Au6 in the upper row and AuL8 in the lower row.}
\label{fig:vert_prof}
\end{figure*}


\section{Angular momentum distribution}

\begin{figure*}
\vspace{-.35cm}
\includegraphics[width=.9\textwidth]{./plots/auriga_ang_mom_time_hist_L6.pdf}
\includegraphics[width=.9\textwidth]{./plots/auriga_ang_mom_time_hist_L8.pdf}
\vspace{-.25cm}
\caption{Normalized distribution functions of the gas' specific angular momentum for different lookback times as indicated with the colorbar on the right. Upper panels show results for the four different physics variants of Au6 and lower panels for AuL8, respectively. The vertical blue line indicates the maximum of the redshift zero distribution.}
\label{fig:ang_mom2}
\end{figure*}


Figure \ref{fig:ang_mom2} shows the distribution of gas angular momentum at 8 different points in time for the tracer particles ending up in stars at present-day. This highlights how the angular momentum of the accreted gas changes over time for the different CR runs compared to the noCR run. Upper panels show galaxy Au6 and lower panels AuL8, respectively. At early cosmic times ($t_{\rm lookback}\gtrsim12$ Gyr, yellow colors) all simulations show a symmetric distribution of specific angular momenta around $l_{z}=0$ kpc km s$^{-1}$. Then, at lookback times of about $8$ Gyr (greenish colours) all runs have accreted gas with higher angular momentum of values $l_{z}\sim1.5\times10^3$ kpc km s$^{-1}$. The noCR and the CRdiffalfven runs keep acquiring high angular momentum gas also at low redshift (smaller lookback times, blue colours) whereas the angular momentum gain in the CRdiff and CRadv runs  is suppressed. Thus, at redshift zero, the angular momentum distribution in the latter cases peaks around $l_{z}\sim1\times10^3$ kpc km s$^{-1}$ and at $l_{z}\gtrsim2\times10^3$ kpc km s$^{-1}$ in the former cases (see vertical thin lines).

\section{Resolution Study}
\label{sec:res}

\begin{table}
\begin{center}
\caption{Virial mass, $M_{200}$ and stellar mass, $M_{\rm star}$ for the main galaxy across three different resolution levels for all four models.}
\label{tab:res}
\begin{tabular}{l c c c c c}
		\hline\hline
		 & & noCR & CRdiffalfven & CRdiff & CRadv \\
		\hline
		 \multicolumn{6}{c}{level 4} \\
		 \hline
		 $M_{200}$ & [$10^{12}\Msun$] & 1.02 & 1.06 & 1.07 & 1.09 \\
		$M_{\rm star}$ & [$10^{10}\Msun$] & 4.36 & 5.54 & 5.81 & 6.19 \\
		\hline
		 \multicolumn{6}{c}{level 5} \\
		 \hline
		 $M_{200}$ & [$10^{12}\Msun$] & 1.03 & 1.02 & 1.07 & 1.06 \\
		$M_{\rm star}$ & [$10^{10}\Msun$] & 4.24 & 4.19 & 5.87 & 5.80 \\
		 \hline
		 \multicolumn{6}{c}{level 6} \\
		 \hline
		 $M_{200}$ & [$10^{12}\Msun$] & 0.94 & 0.99 & 1.00 & 1.03 \\
		$M_{\rm star}$ & [$10^{10}\Msun$] & 3.78 & 2.84 & 3.44 & 3.71 \\
        \hline
\end{tabular}
\end{center}
\end{table}


\begin{figure*}
\vspace{-.45cm}
\hspace*{-.35 cm}
\includegraphics[width=.95\textwidth]{./plots/auriga_gas_density_large_side_lvl5.pdf}
\caption{Gas surface density maps of Au6 level 5 for all four models as indicated in the panels. Orientation and projection depth are as in Fig. \ref{fig:CGMgas}.}
\label{fig:gas_res}
\end{figure*}


\begin{figure*}
\hspace*{-1.5 cm}
\includegraphics[width=1.15\textwidth]{./plots/starden_prof127_restest.pdf}\vspace{-.5 cm}
\includegraphics[width=1.05\textwidth]{./plots/auriga_gas_density_restest.pdf}\vspace{-.15cm}
\hspace*{3.9 cm}
\includegraphics[width=.8\textwidth]{./plots/auriga_Xcr_restest.pdf}
\vspace{-.75cm}
\caption{Comparison of the profiles of stellar surface density (upper panels), gas density (middle panels) and the CR-to-thermal pressure ratio (bottom panels) of the galaxy Au6 at three different resolution levels (as indicated in the figure legends).}
\label{fig:res_test}
\end{figure*}


Our study shows that CRs strongly affect the CGM and the gaseous and stellar properties of the galactic discs. In combination with the model for the wind feedback, this results in a more hydrostatic gas halo and a modification of the gas accretion onto the central galaxy. This effect is already present at resolution levels 5 and 6 (at a factor of 8 and 16 lower in mass resolution, corresponding to $m_{\rm dm}=2\times10^6\Msun$, $m_{\rm b}=4\times10^5\Msun$, $\epsilon=738$ pc and $m_{\rm dm}=2\times10^7\Msun$, $m_{\rm b}=3\times10^6\Msun$, $\epsilon=1476$ pc) and does not change at our fiducial resolution at level 4 ($m_{\rm dm}=3\times10^5\Msun$, $m_{\rm b}=5\times10^4\Msun$, $\epsilon=369$ pc).

We would like to emphasize that we do not change any subgrid parameters for the ISM, wind feedback and CR physics when we change the numerical resolution. Hence, we do not expect to resolve new physics with increasing resolution but we aim at better resolving the poorly resolved regions at the disc-halo interface and the gas accretion and flow pattern in the CGM (i.e., we study convergence of our numerical model). For example, Fig.\ \ref{fig:gas_res} shows the gas surface density maps of the four models at resolution level 5. These are the same panels as in the upper row of Fig.\ \ref{fig:CGMgas} in the main text. The more compact and inflated gas discs in the vertical direction as well as the smoother CGM in the CRadv and CRdiff simulations are clearly visible. We find that at lower resolution (i.e., at resolution levels 5 and 6) this leads to the same hydrostatic CGM properties we have found for our fiducial resolution (level 4) in Sections \ref{sec:props} and \ref{sec:CGMprops}.

We further quantify the effects of resolution on the properties of the central galaxy such as the size and morphology of the stellar and gaseous disc. To this extent we compare in Fig.\ \ref{fig:res_test} radial profiles of the stellar surface density (upper panels), the gas mass density (middle panels) and the ratio of CR-to-thermal pressure (bottom panels) for all three resolution levels. Stellar and gaseous disc properties are in general well converged across all resolution levels (see also Table \ref{tab:res}). However, we note some differences  of the central stellar and gas density in the lowest resolution simulations (level 6) for the CRadv and CRdiff models. For the fiducial AURIGA model, on the other hand, we see that the radial density profiles of the lowest resolution simulations results are slightly steeper. In the CRdiffalfven model the stellar surface density profile is remarkably similar across all resolution levels while the gas density profiles of level 5 and 6 slightly differ from the highest resolution level. However, we note that these differences at various resolution levels are smaller than the differences found between the two haloes studied in the main text. This argues that cosmic variance causes larger differences and that our models are sufficiently numerically converged, not only for global quantities but also for all radial profiles of interest.

Most importantly for our study is that the implementation of CR physics is converged across different resolution levels. In the bottom panel of Fig.\ \ref{fig:res_test} we compare the ratio of CR-to-thermal pressure across the three resolution levels and find overall good agreement between the results. The biggest differences appear for the CRadv run where the $X_{\rm cr}$ values at large radii are higher for level 5 and level 6 in comparison to the fiducial level 4 run. 

To conclude, we find that the simulations presented here show good numerical convergence of stellar, gaseous and CR properties across three levels of resolution. This suggests that our models are well posed to study the effects of CRs on the evolution of MW-like galaxies because the simulation properties solely depend on physical parameters and not on numerical resolution.

\label{lastpage}

\end{document}
