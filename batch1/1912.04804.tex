\documentclass[prl,reprint,amsmath,amssymb,notitlepage]{revtex4-1}

\newcommand{\Z}{\mathbb{Z}}
\newcommand{\Q}{\mathbb{Q}}
\begin{document}
\textbf{Comment on ``General solution to the $U(1)$ anomaly equations" [PRL 123, 151601 (2019)]}

Ref.~\cite{Costa_Dobrescu_Fox_2019} finds 
solutions to the anomaly equations
\begin{align}\label{anomalyeq}
\sum_{i=1}^n z_i=0,\; \sum_{i=1}^n z_i^3=0
\end{align}
 for $n$ charges $z_i
\in \Z$ in a $U(1)$
gauge theory.
The solutions obtained there are general, except they omit solutions which
do not correspond to chiral representations of $U(1)$, or which are
obtainable by permuting charges, or concatenating solutions of lower
$n$, or by adding zeros.  

Here, we show that the ingenious methods of~\cite{Costa_Dobrescu_Fox_2019} have a simple geometric interpretation,
corresponding to elementary constructions long
known to number theorists~\cite{Mordell_1969}.
Viewing them in this context allows a fully general solution
(i.e.\ {\em without}\/ the exceptions above)
to be written
down in one 
fell swoop. It also allows us to give a variety of other,
qualitatively similar,
parameterisations of the general solution, as well as a qualitatively
different (arguably simpler) form of general solution for even $n$. 

To put the method on a geometric footing, we first observe that, by
clearing denominators, every
solution in integers of (\ref{anomalyeq}) is equivalent to a
solution in the rational number field $\Q$. Eliminating $z_n$, we get
the homogeneous cubic equation
\begin{gather} \label{cubichyper}
\sum_{i=1}^{n-1} z_i^3-\left(\sum_{i=1}^{n-1} z_i\right)^3=0,
\end{gather} 
defining a hypersurface in the projective space
$\mathrm{P}\mathbb{Q}^{n-2}$. As well as identifying solutions
differing by a common multiple, working projectively allows us to
sidestep annoyances associated with points at infinity in affine space
(such as the fact that that affine lines
can be variously intersecting, parallel, or skew, while projective lines either
intersect or are disjoint).

A result of antiquity tells us that a chord
through two given
rational points on a cubic hypersurface intersects the hypersurface in
a third rational point, giving us a way of generating new solutions from
old ones. (This is precisely the obscure construction called a `merger' in~\cite{Costa_Dobrescu_Fox_2019}.)
Further, a rather more recent (though
equally elementary) result of
Mordell~\cite{Mordell_1969} shows that all rational points in a
cubic {\em surface} can be
constructed in this way, starting from two
given disjoint lines in the surface. The generalisation to an arbitrary cubic
{\em hyper}surface $X$ is immediate and gives the following
\newline
{\bf Theorem}: Let $\Gamma_1, \Gamma_2\subset X$ be disjoint
hyperplanes of dimensions
$d_1,d_2=m_o:= (n-3)/2$, if $n$ is odd and of dimensions
$d_1=m_e := (n-2)/2$ and $d_2=m_e-1$ if $n$ is even. Every rational
point $p\in \mathrm{P}\mathbb{Q}^{n-2}$ ({\em ergo} every $p \in X$) lies on a chord joining a point
in $\Gamma_1$ to a point in $\Gamma_2$. 
\newline
{\bf Proof}: The result is obvious if $p \in \Gamma_2$. If $p\notin
\Gamma_2$, then $p$ and $\Gamma_2$ define a ($d_2+1$)-d hyperplane,
which intersects $\Gamma_1$ in a point $p^1$. The line through $p$ and
$p^1$ intersects $\Gamma_2$ in a point $p^2$, yielding a chord. QED.

In the case of interest,
the (projective) line $L=\alpha_1 p^1+\alpha_2 p^2$ through $p^{1,2}$ with homogeneous parameter
$[\alpha_1:\alpha_2] \in \mathrm{P}\mathbb{Q}^1$ 
intersects the cubic hypersurface $X$ defined by (\ref{cubichyper}) when
\begin{align} \label{lineincube}
3\alpha_1 \alpha_2\sum_{i=1}^{n-1}\left(\alpha_1 p^2_i P^1_i+\alpha_2 p^1_i P^2_i\right)=0,\;
 P^a_i:= (p^a_i)^2-\left(\sum_{j=1}^{n-1} p^a_j\right)^2. \nonumber
\end{align}
Thus, along with the points $p^{1,2}$ (corresponding to
$\alpha_{2,1}=0$) we get either a third rational point on $X$ at
\begin{align}
[\alpha_1:\alpha_2]=\Bigg[\sum_{i=1}^{n-1}p^1_i P^2_i:-\sum_{i=1}^{n-1}p^2_iP^1_i\Bigg], \nonumber
\end{align}
or, if the terms on the RHS both vanish, we have that every rational point on
$L$ is on $X$. 

To get a fully general solution, we just need to find suitable
$\Gamma_1,\Gamma_2$. Many choices are possible; let us choose the one
that makes closest contact with
\cite{Costa_Dobrescu_Fox_2019}. To wit,
\begin{align}
\Gamma^{e}_{1}&=[k_1: \cdots: k_{m_e}:-\tilde
            l_1:-k_{1}:\cdots:-k_{m_e}]\nonumber \\
\Gamma^e_{2}&=[0:l_1: \cdots: l_{m_e}:-l_1:\cdots:-l_{m_e}] \nonumber \\
\Gamma^o_1 &=[k_1:\cdots:k_{m_o+1}:-k_1:\cdots:- k_{m_o+1}]\nonumber \\
\Gamma^o_{2}&=[l_2:\cdots: l_{m_o}:\tilde
              k_1:0:-l_1:\cdots:-l_{m_0}:\tilde k_1]. \nonumber
\end{align}
These parameterisations cover $\Gamma_{1,2}$, so by the
Theorem yield all
solutions of (\ref{anomalyeq}). The parameterisations of
\cite{Costa_Dobrescu_Fox_2019}, in contrast, have $\tilde l_1=l_1$
and $\tilde k_1=k_1$; as a result they are unable to reproduce
solutions in which one of the points $p^{1,2}$ in the construction is
at infinity in affine space, e.g.\ $[0:1:-1:0]$ for
$n=4$, which isn't chiral and has zeros.

Two further remarks are in order. Firstly, as we have seen, our parameterization of the
general solution is somewhat distasteful, in that occasionally the
chord joining points on $\Gamma_{1,2}$ lies in $X$, and so yields not
one, but infinitely many solutions. By repeating the construction with
2 different choices of $\Gamma_{1,2}$, one can ensure that every
solution arises as a unique third point of intersection of a chord
with $X$. One can even choose the original $\Gamma_{1,2}$ such that
this can be achieved by simply permuting the co-ordinates
$z_i$. Secondly, in the case where $n$ is even, a completely
different, and arguably even simpler, construction of a general solution is possible. Indeed, in
such cases, the cubic hypersurface has double points, where both the
LHS of  (\ref{cubichyper}) and its partial derivatives vanish
(e.g.\ the rational point [+1:-1:\dots:+1]). A
line through such a double point intersects the cubic in one other
rational point (or the line lies entirely in $X$) and thus all
solutions can be obtained by constructing all lines through just a single
double point, as it were.

This work was supported by STFC consolidated grants ST/P000681/1 and ST/S505316/1. 
\\
\\
\\
B.C.\ Allanach$^{a,1}$, Ben Gripaios$^{b,2}$ and Joseph Tooby-Smith$^{b,3}$\\
\\
$^{a}$ DAMTP, University of Cambridge, Wilberforce Road, Cambridge, 
CB3 0WA, United Kingdom\\
$^{b}$ Cavendish Laboratory, University of Cambridge, J.J. Thomson
  Avenue, Cambridge, CB3 0HE, United Kingdom \\
\\
$^{1}$ B.C.Allanach@damtp.cam.ac.uk\\
$^{2}$ gripaios@hep.phy.cam.ac.uk\\
$^{3}$ jss85@cam.ac.uk
\begin{thebibliography}{2}\makeatletter
\providecommand \@ifxundefined [1]{\@ifx{#1\undefined}
}\providecommand \@ifnum [1]{\ifnum #1\expandafter \@firstoftwo
 \else \expandafter \@secondoftwo
 \fi
}\providecommand \@ifx [1]{\ifx #1\expandafter \@firstoftwo
 \else \expandafter \@secondoftwo
 \fi
}\providecommand \natexlab [1]{#1}\providecommand \enquote  [1]{``#1''}\providecommand \bibnamefont  [1]{#1}\providecommand \bibfnamefont [1]{#1}\providecommand \citenamefont [1]{#1}\providecommand \href@noop [0]{\@secondoftwo}\providecommand \href [0]{\begingroup \@sanitize@url \@href}\providecommand \@href[1]{\@@startlink{#1}\@@href}\providecommand \@@href[1]{\endgroup#1\@@endlink}\providecommand \@sanitize@url [0]{\catcode `\\12\catcode `\$12\catcode
  `\&12\catcode `\#12\catcode `\^12\catcode `\_12\catcode `\%12\relax}\providecommand \@@startlink[1]{}\providecommand \@@endlink[0]{}\providecommand \url  [0]{\begingroup\@sanitize@url \@url }\providecommand \@url [1]{\endgroup\@href {#1}{\urlprefix }}\providecommand \urlprefix  [0]{URL }\providecommand \Eprint [0]{\href }\providecommand \doibase [0]{http://dx.doi.org/}\providecommand \selectlanguage [0]{\@gobble}\providecommand \bibinfo  [0]{\@secondoftwo}\providecommand \bibfield  [0]{\@secondoftwo}\providecommand \translation [1]{[#1]}\providecommand \BibitemOpen [0]{}\providecommand \bibitemStop [0]{}\providecommand \bibitemNoStop [0]{.\EOS\space}\providecommand \EOS [0]{\spacefactor3000\relax}\providecommand \BibitemShut  [1]{\csname bibitem#1\endcsname}\let\auto@bib@innerbib\@empty
\bibitem [{\citenamefont {Costa}\ \emph {et~al.}(2019)\citenamefont {Costa},
  \citenamefont {Dobrescu},\ and\ \citenamefont
  {Fox}}]{Costa_Dobrescu_Fox_2019}\BibitemOpen
  \bibfield  {author} {\bibinfo {author} {\bibfnamefont {D.~B.}\ \bibnamefont
  {Costa}}, \bibinfo {author} {\bibfnamefont {B.~A.}\ \bibnamefont {Dobrescu}},
  \ and\ \bibinfo {author} {\bibfnamefont {P.~J.}\ \bibnamefont {Fox}},\ }\href
  {\doibase 10.1103/PhysRevLett.123.151601} {\bibfield  {journal} {\bibinfo
  {journal} {Physical Review Letters}\ }\textbf {\bibinfo {volume} {123}},\
  \bibinfo {pages} {151601} (\bibinfo {year} {2019})}\BibitemShut {NoStop}\bibitem [{\citenamefont {Mordell}(1969)}]{Mordell_1969}\BibitemOpen
  \bibfield  {author} {\bibinfo {author} {\bibfnamefont {L.}~\bibnamefont
  {Mordell}},\ }\href@noop {} {\emph {\bibinfo {title} {Diophantine
  Equations}}}\ (\bibinfo  {publisher} {Academic Press},\ \bibinfo {year}
  {1969})\BibitemShut {NoStop}\end{thebibliography} 
\bibliographystyle{apsrev4-1}
\end{document}
